\newpage
\chapter{Filtrations on $\HI$}\label{sect_filtration_hi}

The purpose of this chapter is to construct three filtrations of 
$\HI$. The main result of this chapter is that
there is a sequence of coradicals (see Definition 
\ref{def_coradical}) on the category $\HI$ which induces a 
descending strong filtration and an ascending cofiltration (see 
Definition \ref{def_strong_filtration} below) of $\HI$ by the 
associated subcategories (see Theorem \ref{thm_main_result}). The 
key ingredient in the constructions of the filtrations are the 
tensor monoidal structure on $\HI$ and the partial internal hom. 
Both of these structures are induced by the tensor and partial 
internal hom operators on $\DMeff$ introduced in Section 
\ref{sect_TMS_DMeff}. All results in this section are new.

\section{Tensor and partial internal hom structure on $\HI$}

To simplify the definition and the proofs in this chapter and the
next, we invoke Theorem \ref{thm_ALocal_eq_DMeff} and identify the 
category $\DMeff$ with the full triangulated subcategory $\ALocal$ 
of $\A^1$-local complexes from Definition \ref{def_ALocal}. We identify 
objects $M$ in $\DMeff$ with bounded above complexes $F^*$ of 
Nisnevich sheaves with transfers such that $H^nF^*$ is a homotopy 
invariant presheaf with transfers for every $n$. In particular, 
regarding a sheaf with transfers as a chain complex concentrated 
in degree 0, we consider $\HI$ as an additive subcategory of 
$\DMeff$.

Recall the following notions from \cite[1.3]{BBD}:

\begin{defn}\label{def_t_struct}
A \DEF{$t$-category} is a triangulated category $\DCat$ together
with a pair of full subcategories $(\DCat^{\geq 0}, 
\DCat^{\leq 0})$ which satisfies the following:
\begin{enumerate}
\item For all $X$ in $\DCat^{\leq 0}$, and $Y$ in $\DCat^{\geq 1}$, 
$\hom_{\DCat}(X, Y) = 0$.

\item $\DCat^{\leq 0} \subset \DCat^{\leq 1}$ and
$\DCat^{\geq 1} \subset \DCat^{\geq 0}$

\item For all $X$ in $\DCat$, there exists a distinguished 
triangle
\[
A \to X \to B \to A[1]
\]
such that $A$ is in $\DCat^{\leq 0}$ and $B$ is in 
$\DCat^{\geq 1}$.
\end{enumerate}
\noindent Here we write $\DCat^{\geq n}$ and $\DCat^{\leq n}$ for 
$\DCat^{\geq 0}[n]$ and $\DCat^{\leq 0}[n]$ respectively. We call
the pair $(\DCat^{\geq 0}, \DCat^{\leq 0})$ a \DEF{$t$-structure} 
on $\DCat$.

The \DEF{heart} of a $t$-category is the full subcategory
$\Cat{C} \defeq \DCat^{\geq 0} \cap \DCat^{\leq 0}$.
\end{defn}

If $\DCat$ is a $t$-category, then the inclusion of $\DCat^{\leq 
n}$ into $\DCat$ admits a right adjoint $\gltrunc{n}: \DCat \to 
\DCat^{\leq n}$, and the inclusion of $\DCat^{\geq n}$ admits left 
adjoint $\ggtrunc{n}: \DCat \to \DCat^{\geq n}$. Furthermore, for 
all $X$ in $\DCat$, there exists a unique map $d$ in 
$\hom_{\DCat}(\ggtrunc{1}X, \gltrunc{0}X[1])$ such that
\[
\gltrunc{0} X \to X \to \ggtrunc{1} X \stackrel{d}{\to} \gltrunc{0}X[1]
\]
is distinguished (see \cite[1.3.3]{BBD}). For integers $m$ and $n$ 
such that $m < n$, $\gltrunc{m} \gltrunc{n} = 
\gltrunc{n}\gltrunc{m} = \gltrunc{m}$, and $\ggtrunc{m} 
\ggtrunc{n} = \ggtrunc{n} \ggtrunc{m} = \ggtrunc{n}$. Furthermore, 
$\gltrunc{m} \ggtrunc{n} = \ggtrunc{n} \gltrunc{m} = 0$, and
$\gltrunc{n} \ggtrunc{m} = \ggtrunc{m} \gltrunc{n}$ 
(see \cite[1.3.5]{BBD}). When $n = m = 0$, the composition
$\gltrunc{0} \ggtrunc{0}$ defines an additive functor 
$\HH^0 : \DCat \to \Cat{C}$.

Recall from \cite[1.2.5]{BBD} that an abelian subcategory 
$\Cat{C}$ of $\DCat$ is \DEF{admissible} if for all $C$ and $D$ in 
$\Cat{C}$ and $i < 0$, $\hom_{\DCat}(C, D[i]) = 0$, and all exact
sequences in $\Cat{C}$ come from distinguished triangles in 
$\DCat$.

\begin{thm}[\cite{BBD} 1.3.6]\label{thm_heart_is_abel_cat}
Let $\DCat$ be a $t$-category, and let $(\DCat^{\geq 0}, 
\DCat^{\leq 0})$ be its associated $t$ structure. Then the heart 
$\Cat{C}$ is an admissible abelian category, stable under
taking extensions. 
\end{thm}

\begin{ex}[\cite{BBD} 1.3.2]\label{ex_DA_t_struct}
Let $\Cat{A}$ be an abelian category, and $\DCat[D]\Cat{A}$ be its
derived category. There is a natural $t$-structure on 
$\DCat[D]\Cat{A}$. The pair $(\DCat[D]\Cat{A}^{\geq 0}, 
\DCat[D]\Cat{A}^{\leq 0})$ are a pair of full subcategories whose
objects are those with trivial cohomology in the negative and 
positive degrees respectively. In this case, the functors 
$\ggtrunc{n}$ and $\gltrunc{n}$ are given by good truncations. 

The heart of this $t$-structure is precisely $\Cat{A}$, regarded
as a complex concentrated in degree $0$. (See the example 
following the statement of 1.3.6 in \cite{BBD}.)
\end{ex}

\begin{ex}[\cite{BBD} 1.3.16]
If $\DCat'$ is a full triangulated subcategory of a $t$-category 
$\DCat$, then $\DCat'$ is also a $t$-category with the 
$t$-structure given by $({\DCat'}^{\geq 0}, {\DCat'}^{\leq 0})$, 
where ${\DCat'}^{\geq 0} \defeq \DCat^{\geq 0} \cap \DCat$ and 
${\DCat'}^{\leq 0} \defeq \DCat^{\leq 0} \cap \DCat'$. 
\end{ex}

\begin{defn}
Let $\phi: \DCat_1 \to \DCat_2$ be a triangulated functor between
$t$-categories. We say that $\phi$ is \DEF{right $t$-exact} if 
$\phi(\DCat_1^{\leq 0}) \subseteq \DCat_2^{\leq 0}$, and  
\DEF{left $t$-exact} if $\phi(\DCat_1^{\geq 0}) \subseteq 
\DCat_2^{\geq 0}$. We say that $\phi$ is \DEF{$t$-exact} 
if it is both right and left $t$-exact.
\end{defn}

The concept of (left or right) $t$-exactness is a generalization 
of exactness in abelian category. We have the following result
regarding $t$-exact functors and the induced functor on the
hearts.

\begin{prop}[\cite{BBD} 1.3.17]\label{prop_t_exact_implies_exact}
Let $\DCat$ and $\DCat'$ be $t$-categories with hearts 
$\Cat{A}$ and $\Cat{A}'$ respectively. Furthermore, let $F: 
\DCat \to \DCat'$ be a left (resp., right) $t$-exact triangulated 
functor. Then $\HH^0 F$ is a left (resp., right) exact 
functor from $\Cat{A}$ to $\Cat{A}'$.
\end{prop}

If a $t$-category $\DCat$ is equipped with an additive symmetric 
monoidal structure that is right $t$-exact in both factors, then 
so is its heart $\Cat{C}$. The symmetric monoidal structure on
the heart is defined as follows. Suppose $- \tensor -$ is the 
tensor operator on $\DCat$. For $C, C'$ in $\Cat{C}$, we define $C 
\tensor^{\Cat{C}} C'$ by $\HH^0(C \tensor C')$. 
Since $\tensor$ is right $t$-exact in both factors, for all $M$ 
and $N$ in $\DCat^{\leq 0}$,
\[
\HH^0(M \tensor N) = \HH^0(\HH^0(M) \tensor \HH^0(N))
\]
(\cite[5.10]{DegModHom}) and $\tensor^{\Cat{C}}$ is well-defined.
It is now straightforward to verify that $(\Cat{C}, 
\tensor^{\Cat{C}})$ satisfies all the axioms of a symmetric 
monoidal category. 

In addition, if $\DCat$ has a partial internal hom in the sense
that there exists an a bifunctor $\ihom$ defined in the first 
factor for the subcollection $\DCat^{\compact}$ of \SemiInvertible
objects in $\DCat$ such that
\[
\hom_{\DCat}(X \tensor Y, Z) \cong 
   \hom_{\DCat}(X, \ihom(Y, Z))
\]
for all $Y$ in $\DCat^{\compact}$, then $\Cat{C}$ is also equipped
with a partial internal hom. Here, the internal hom bifunctor is given 
by
\[
\ihom_{\Cat{C}}(C, C') \defeq \HH^0(\ihom(C, C')).
\]
Since $\HH^0$ is adjoint to the inclusion of $\Cat{C}$ into
$\DCat$, we have that
\begin{align*}
\hom_{\Cat{C}}(C_1 \tensor^{\Cat{C}} C_2, C_3) &\cong
\hom_{\DCat}(C_1, \ihom(C_2, C_3)) \\
&\cong \hom_{\DCat}(C_1, \HH^0(\ihom(C_2, C_3))) \\
&\cong \hom_{\Cat{C}}(C_1, \ihom_{\Cat{C}}(C_2, C_3))
\end{align*}
for all $C_1, C_2, C_3$ in $\Cat{C}$ and $C_2$ in $\HH^0(\DCat)$,
the bifunctor $\ihom_{\Cat{C}}$ defines a partial internal hom 
with \SemiInvertible objects $\Cat{C} \cap \DCat^{\compact}$ (see
Definition \ref{def_tensor_triang_cat}).

Applying the above discussion to the category $\DMeff$, we see 
that there exists a $t$-structure on $\DMeff$, with the abelian 
category $\HI$ as its heart. We write $\gltrunc{0}$, $\ggtrunc{0}$
and $\HH^0$ for the corresponding functors from $\DMeff$ to the 
respective subcategories defined by its $t$-structure. Moreover, 
the triangulated monoidal structure on $\DMeff$ induces a 
symmetric monoidal bifunctor on $\HI$, which we write as $\tHI$. 
This bifunctor is uniquely characterized by
\[
\hS{X} \tHI \hS{Y} = \hS{X \times Y},
\]
where $\hS{X} = \HH^0(M(X))$ for $X$ in $\Sm$ (see 
\cite[1.8]{DegModHom}).

Moreover, there is a partial internal hom, defined by
\[
\ihomHI(F, G) = \HH^0(\ihomDMf(F, G))
\]
for all $G$ in $\HI$, and all $F$ in $\HI \cap \DMeffgm$. In
particular, since $\Ox \cong \Z(1)[1]$, the partial internal hom on
$\HI$ defines an endofunctor given by $\ihom(\Ox, -)$. 

\begin{defn}
  To emphasize the relationship with corresponding operations in
  $\DMeff$, let us set
\[
\LHIs{F} \defeq F \tHI \Ox \hspace{10pt} \textrm{and}\hspace{10pt} 
   \RHIs{F} \defeq \ihomHI(\Ox, F).
\]
We write $\LHIs[n]{F}$ for $\LHIs{(\LHIs[n - 1]{F})}$ and 
$\RHIs[n]{F}$ for $\RHIs{(\RHIs[n + 1]{F})}$.
\end{defn}

\begin{rmk}\label{rmk_contraction}
To simplify notation, we will drop the ``HI'', and simply write 
$\LHI[n]{F}$ and $\RHI[n]{F}$ for $\LHIs[n]{F}$ and $\RHIs[n]{F}$.
Doing so introduces a number of potential sources of ambiguity. 
The first is that $\LHI[n]{F}$ is already used to represent $F 
\tDM \Z(n)$, where $\Z(n)$ is the motivic complex in $\DMeff$ 
introduced in Section \ref{sect_motivic_complex}. In particular, 
$\LHI[n]{\Z}$ may refer to the motivic complexes as well as the 
objects $\Z \tHI (\Ox)^{\tensor n}$. To clarify this ambiguity, we 
adopt the following convention: For the remainder of the thesis, 
unless otherwise specified, for an object $F$ in $\HI$, 
$\LHI[n]{F}$ will denote $\LHIs[n]{F} \defeq F \tHI 
(\Ox)^{\tensor n}$. All mentions of $\Z(n)$ will refer to the 
motivic complex in $\DMeff$.

The second source of potential ambiguity comes from the fact that 
$\RHI{F}$ is already used to represent the contraction of the 
sheaf $F$ in $\HI$. Recall from Definition \ref{def_contract} that 
$\RHI{F}$ is the sheaf sends $X$ in $\Sm$ to $\cok p^*$, where 
\[
p^* : F(X) \to F(X \times (\A^1 - 0))
\]
is the map induced by the projection $X \times (\A^1 - 0) \to X$.
In fact, there is no ambiguity here, since the contraction of $F$ 
is isomorphic to the sheaf $\ihomHI(\Ox, F)$. This is not 
difficult to see: for $F$ in $\HI$,
\begin{align*}
\RHI{F} &= \mathrm{cok}\left(\ihomDMf(M(\A^1 - 0), F) \to \ihomDMf(\Z, F)\right) \\
&= \ihomDMf(\Z(1)[1], F) = \ihomHI(\Ox, F),
\end{align*}
since $\Z(1)[1] \cong \Ox$ in $\DMeff$ (see \cite[4.1]{MVW}).
\end{rmk}

Since $\Z(1)[1] \cong \O^*$ in $\DMeff$, for any $F$ in $\HI$, 
$\HH^0(F \tDM \Z(1)[1]) = \LHI{F}$ and $\HH^0\ihomDMf(\Z(1)[1], F) 
= \RHI{F}$. These observations allow us to conclude a number of useful
facts. First, the functor $F \mapsto \LHI{F}$ is left adjoint to 
$F \mapsto \RHI{F}$. 

We also have the following result:

\begin{prop}\label{prop_unit_iso}
Let $F$ be a homotopy invariant sheaf with transfers. The unit map 
$F \to \RHI{\LHI{F}}$ is an isomorphism.
\end{prop}
\begin{proof}
For $F$ in $\HI$, by the Cancellation Theorem 
\ref{thm_dm_cancellation}, we have that 
\[
\ihomDMf(\Z(n)[n], F(n)[n]) \cong \ihomDMf(\Z, F) 
\cong F. 
\]
Now apply $\HH^0$ to this chain of isomorphisms, and note 
that $\HH^0(F) = F$. The proposition now follows.
\end{proof}

Finally, we make the following observation that will be useful
in subsequent sections.

\begin{prop}\label{prop_counit_iso_for_HIn}
If $F = \LHI[n]{G}$ for some $G$ in $\HI$, then $\cuHI^n_F : 
\LHI[n]{\RHI[n]{F}} \to F$ is an isomorphism.
\end{prop}
\begin{proof}
Suppose $F = \LHI[n]{G}$ for some $G$ in $\HI$. Writing $L$ for 
the functor 
$F \mapsto \LHI[n]F$, by counit-unit adjunction, the composition
\[
\LHI[n]{G} \xrightarrow{L\eta_G} \LHI[n]{(\RHI[n]{\LHI[n]{G}})}
   \xrightarrow{\cuHI_G L} \LHI[n]{G}
\]
is the identity, where $\eta_{G}$ and $\cuHI_{G}$ are the unit and 
the counit maps respectively. However, by Proposition \ref{prop_unit_iso}, 
$\eta_{G}$ is an isomorphism, and so is $L\eta_{G}$. It follows
that $\cuHI_G L$ is an isomorphism. But $\cuHI_G L$ is the counit
map for $LG = \LHI[n]{G} = F$, and the proposition follows.
\end{proof}

\section{Torsion filtrations on $\HI$}

We now define the first filtration on $\HI$. Let $\LHI[0]{\HI} = 
\HI$ and let $\LHI[n]{\HI}$ denote the full subcategory of objects 
$F$ where $F = \LHI[n]{F'}$ for some $F'$ in $\HI$. It is clear 
that if $m \geq n$, then $\LHI[m]{\HI} \subseteq \LHI[n]{\HI}$. 
In particular, we have a tower of subcategories
\[
\HI = \LHI[0]{\HI} \supset \LHI[1]{\HI} \supset \LHI[2]{\HI} 
\subset \cdots.
%\cdots \subset \LHI[2]{\HI} \subset \LHI[1]{\HI} \subset \LHI[0]{\HI} 
% = \HI 
\]
To see that this filtration is not trivial (i.e. $\LHI[n]{\HI} \neq
\LHI[m]{\HI}$ for all natural numbers $n$ and $m$), notice that
for the constant sheaf $\Z$, it is clear that $\RHI{\Z} = 0$. 
Then $\Z$ is an object in $\HI$ but not in $\LHI{\HI}$. Indeed, if 
$\Z \in \LHI{\HI}$ then $\Z \cong \LHI{F'}$, but then $\RHI{\Z} = F'$ 
by Proposition \ref{prop_unit_iso}, forcing $\Z = 0$. Similarly, 
since $\RHI{\Ox} = \Z$, $\Ox \in \LHI{\HI}$ but $\Ox \notin
\LHI[2]{\HI}$. In general, $\LHI[n - 1]{\Ox}$ is an object of
$\LHI[n]{\HI}$ but not $\LHI[n + 1]{\HI}$.

\begin{rmk}
The subcategories $\LHI[n]{\HI}$ are additive, but \emph{not} 
abelian, except for the case $n = 0$. To see this, consider the 
map
\[
n: \Ox \to \Ox
\]
given by sending $u \in \Ox(X)$ to $u^n$ for each $X$ in $\Sm$.  The
kernel of this map is sheaf of $n$-th roots of unity $\mu_n$.  But
$\RHI{(\mu_n)} = 0$. If $\mu_n$ was in $\LHI[n]{\HI}$, then by
Proposition \ref{prop_counit_iso_for_HIn}, we would have $\mu_n \cong
\LHI[1]{\RHI[1]{(\mu_n)}} = 0$, which is a contradiction. It follows
that $\LHI{\HI}$ is not closed under kernels. Similar arguments show
that $\LHI[n]{\HI}$ is not closed under kernel for any positive
integer $n$.
\end{rmk}

Recall from Definition \ref{def_cat_filtration} that a descending weak
filtration $(\Cat{A}_*, \phi_*)$ is a tower of subcategories
$\Cat{A}_i$ together with reflection or coreflection functors $\phi_i:
\Cat{A} \to \Cat{A}_i$ such that $\phi_i$ restricted to $\Cat{A}_i$ is
naturally isomorphic to the identity.  To show that the full
subcategories $\LHI[n]{\HI}$ define a descending weak filtration, we
need to show that there exist coreflection functors $\sgHI{n}: \HI \to
\LHI[n]{\HI}.$

\begin{prop}\label{prop_HI_upper_slice}
Let $\sgHI{n}$ denote the functor $F \mapsto 
\LHI[n]{(\RHI[n]{F})}$. Then $\sgHI{n}$ is right adjoint to the 
inclusion of $\LHI[n]{\HI}$. In particular, $(\LHI[*]{\HI}, 
\sgHI{*})$ defines a (nontrivial) weak filtration of $\HI$.
\end{prop}
\begin{proof}
Let $f: F \to G$ be a map in $\HI$, with $F$ in $\LHI[n]{\HI}$,
and let $\cuHI^n$ denote the counit $\sgHI{n} \to \id$.
By naturality of $\cuHI^n$, we have the following commutative 
diagram:
\[
\begin{tikzcd}
\LHI[n]{\RHI[n]{F}} \arrow{r}{\cuHI^n f}\arrow{d}{\cuHI^n_F} 
& \LHI[n]{\RHI[n]{G}} \arrow{d}{\cuHI^n_G} \\
F \arrow{r}{f}
& G.
\end{tikzcd}
\]
Since $F \in \LHI[n]{\HI}$, by Proposition 
\ref{prop_counit_iso_for_HIn} the counit map $\cuHI^n_F$ is an 
isomorphism.

Define the map $\chi: \homHI(F, G) \to \hom_{\LHI[n]{\HI}}(F, 
\LHI[n]{\RHI[n]{G}})$ by $f \mapsto 
\cuHI^n f \comp (\cuHI^n_F)^{-1}$. Since $\cuHI^n_G \comp \chi(f) 
= f$, $\chi$ is injective.  Moreover, given a map $g: F 
\to \LHI[n]{\RHI[n]{G}}$, set $f' = \cuHI_G^n \comp g$. Then 
$\chi(f') = g$. Hence $\chi$ is an isomorphism as desired.
From the way $\chi$ is defined, it is clear that $\chi$ is 
functorial in both $F$ and $G$, and therefore $\sgHI{n}$ is
right adjoint to the inclusion of $\LHI[n]{\HI}$ into $\HI$.

To show that $(\LHI[*]{\HI}, \sgHI{*})$ define a weak descending
filtration, the only criterion left to check is that $\sgHI{n}$
restricted to $\LHI[n]{\HI}$ is naturally isomorphic to the
identity. By Proposition \ref{prop_counit_iso_for_HIn}, the
counit map $\cuHI^n: \sgHI{n} F \to F$ is an isomorphism for
all $F$ in $\LHI[n]{\HI}$, and the proposition follows.
\end{proof}

\begin{ex}\label{rmk_sgHI_is_not_strong_filt}
While $(\HI(*), \sgHI{*})$ forms a weak filtration of $\HI$, for
a given sheaf $F$ in $\HI$, the objects $\sgHI{n} F$ are not 
in general subobjects of $F$. Here is an example.

Let $\Oxn$ be the sheaf of $n$-th power of global units 
associated to the presheaf where sections of a smooth finite type 
$k$-scheme $X$ is the abelian subgroup of $\Ox$ given by 
\[
\Oxn(X) = \{x : x=y^n\textrm{ for some }y \textrm{ in } \Ox(X)\}.
\]
It is clear that $\Oxn \in \HI$. Furthermore, there exists the 
following exact sequence 
\[
0 \to \mu_n \to \Ox \to \Oxn \to 0
\]
where $\mu_n$ is the constant sheaf of $n$-th roots of unity.
In particular, $\RHI{(\mu_n)} = 0$. By Proposition 
\ref{prop_contract_is_exact}, the functor $F \mapsto \RHI{F}$ is
exact. Therefore, the map $\RHI{\Ox} \to \RHI{(\Oxn)}$ is an
isomorphism, and
\[
\LHI{\RHI{(\Oxn)}} \cong \LHI{\RHI{\Ox}} = \Ox,
\]
and the counit $\LHI{\RHI{(\Oxn)}} \to \Oxn$ is given precisely by
$x \mapsto x^n$, which has a nontrivial kernel.

We can understand the problem in another way, which is that the 
categories $\HI(*)$ are too small and do not include all
the kernels of counits $\LHI[n]{(\RHI[n]{F})} \to F$. This can be 
fixed by enlarging the filtration at each level, and to do so, we 
turn to torsion theory.
\end{ex}

Motivated by Remark \ref{rmk_sgHI_is_not_strong_filt}, we 
introduce the following more stringent criteria on weak
filtrations.

\begin{defn}\label{def_strong_filtration}
  Let $\Cat{A}$ be an abelian category. We say that an $\Z$-indexed
  descending weak filtration $(\Cat{A}_*, \phi_*)$ is a \DEF{strong
    filtration} if for each $A$ in $\Cat{A}$ and $n$ in $\Z$, $\phi_n
  A \to A$ is a monomorphism of $A$. An ascending weak filtration
  $(\Cat{A}_*, \phi_*)$ is a \DEF{strong cofiltration} if $A \to
  \phi_n A$ is a quotient of $A$ for each $n$ and each $A$ in
  $\Cat{A}$.

  Similarly, we can define ascending strong filtration and descending
  strong cofiltration on $\Cat{A}$.
\end{defn}

\begin{ex}
Here is an example of a strong ascending filtration and a strong
descending filtration on the category $\QCoh$ of quasi-coherent
sheaves on $\P^n$. Let $i_k$ denote the closed immersion of $\P^k$ 
into $\P^n$ as a subscheme identified by the vanishing of the last 
$k$ homogeneous coordinates, and let $j_k$ denote the open 
immersion of $U_k \defeq \P^n - \P^k$ into $\P^n$. 

Let $\QCoh^k$ be the full subcategory of quasi-coherent sheaves
supported in $U_k$ and let $\QCoh_k$ denote the full subcategory
of sheaves supported in $\P^k$. Since 
\[
U_0 \supseteq U_1 \supseteq U_2 \supseteq \cdots \supseteq U_n
\]
and
\[
\P^0 \subseteq \P^1 \subseteq \P^2 \subseteq \cdots \subseteq \P^n
\]
we have the following towers of subcategories:
\[
\QCoh^0 \supseteq \QCoh^1 \supseteq \QCoh^2 \supseteq \cdots
   \supseteq \QCoh^n
\]
and 
\[
\QCoh_0 \subseteq \QCoh_1 \subseteq \QCoh_2 \subseteq \cdots
   \subseteq \QCoh_n.
\]
We will show that the towers of subcategories define a strong 
filtration and strong cofiltration on $\QCoh$. For each positive 
integer $k$ less than $n$ and each $F$ in $\QCoh$, we have the 
following exact sequence of quasi-coherent sheaves on $\P^n$:
\begin{equation}\label{eq_qc_sheaf_ses}
0 \to (j_k)_!(F|_U) \to F \to (i_k)_*(F|_Z) \to 0
\end{equation}
where $(j_k)_!(F|_U)$ is the sheaf associated with the presheaf 
given by
\[
V \mapsto \begin{cases}
F(V) &\textrm{if }V \subseteq U_k\\
0    &\textrm{otherwise}.
\end{cases}
\]
In this case $(j_k)_!(F)$ is in $\QCoh^k$ and $(i_k)_*(F)$ is in 
$\QCoh_k$ (see \cite[Ex. 1.19]{Hart}). In fact, $F \mapsto 
(j_k)_!(F|_{U_k})$ and $F \mapsto (i_k)_*(F|_{\P^k})$ define
functors from $\QCoh$ to $\QCoh^k$ and $\QCoh_k$
respectively. In this case $(j_k)_!$ is right adjoint to 
inclusion, and $(i_k)_*$ is left adjoint to inclusion.

In general, let $Z_1 \subseteq Z_2 \subseteq \cdots Z_n$ be
a sequence of subschemes of some scheme $X$, and let $U_k =
X - Z_k$. Let $\QCoh(X)$ be the abelian category of quasi-coherent 
sheaves on $X$. Then there exists a strong descending filtration 
\begin{equation}\label{eq_qcoh_desc_filt}
\QCoh^0(X) \supseteq \QCoh^1(X) \supseteq \QCoh^2(X) \supseteq 
   \cdots \supseteq \QCoh^n(X)
\end{equation}
where $\QCoh^k(X)$ is the full subcategory of quasi-coherent sheaves
supported on $U_k$, and a strong ascending cofiltration on 
$\QCoh(X)$
\begin{equation}\label{eq_qcoh_asc_filt}
\QCoh_0(X) \subseteq \QCoh_1(X) \subseteq \QCoh_2(X) \subseteq \cdots
\subseteq \QCoh_n(X)
\end{equation}
where $\QCoh_k(X)$ is the full subcategory of quasi-coherent 
sheaves on $X$ supported on $Z_k$. As above, the coreflection 
functors $\phi^k$ from $\QCoh$ to $\QCoh^k$ are given by $F 
\mapsto (j_k)_!(F|U_k)$ where $j_k$ is the open immersion $j_k: 
U_k \to X$; the reflection functors $\phi_k$ from $\QCoh$ to 
$\QCoh_k$ are given by $F \mapsto (i_k)_*(F|_{Z_k})$, where $i_k: 
Z_k \to X$ is the evident closed immersion. Since, for each 
$k$, we have an exact sequence of quasi-coherent sheaves as in
\eqref{eq_qc_sheaf_ses}, $\phi^k(F)$ is a subobject of $F$ and
$\phi_k(F)$ is a quotient of $F$ for each $F$ in $\QCoh(X)$.
The claim that \eqref{eq_qcoh_desc_filt} is a strong filtration
and \eqref{eq_qcoh_asc_filt} now follows.
\end{ex}

We will now state the main theorem. Recall from Theorem 
\ref{thm_corad_equiv_htt} that if $\phi$ is a coradical, the 
associated torsion theory is a pair of full subcategories
$(\Cat{T}, \Cat{F})$ where the torsion subcategory $\Cat{T}$ 
consists of the objects $T$ such that $\phi(T) = 0$, and torsionfree
subcategory $\Cat{F}$ consists of the objects $F$ such that 
$\phi(F) = F$.

\begin{thm}\label{thm_main_result}
There exists a sequence of coradicals $\tlHI{n}, n = 0, 1, 2 
\cdots$ on $\HI$ such that the associated torsionfree 
subcategories $\TFHI{*}$ form a descending strong filtration
of $\HI$ and the associated torsion subcategories $\THI{*}$ 
form a strong cofiltration.
\end{thm}

Theorem \ref{thm_main_result} will be verified by Propositions
\ref{prop_tsubcat_eq_tlHI} and \ref{prop_THI_form_strong_filt} 
below.

We first define the strong cofiltration, and show that the 
reflection functors are coradicals. Let $n$ be a natural number, 
and let $\TLHI{n}{\HI}$ be the full subcategories of objects $F$ 
in $\HI$ such that $\RHI[n]{F} = 0$. Here, we define $\RHI[0]{F}$
to be $F$. Since $\RHI[n - 1]{F} = \RHI{(\RHI[n]{F})}$, we also 
have the following ascending tower of subcategories
\[
0 = \TLHI{0}{\HI} \subset \TLHI{1}{\HI} \subset \TLHI{2}{\HI} 
   \subset \cdots \subset \HI
\]
Since $\RHI{\Ox} = \Z$ and $\RHI{\Z} = 0$, $\Ox$ is in 
$\TLHI{2}{\HI}$ but not in $\TLHI{1}{\HI}$. By Proposition 
\ref{prop_unit_iso} and by induction, $\LHI[n]{\Ox}$ is in 
$\TLHI{n + 1}{\HI}$ but not in $\TLHI{n}{\HI}$.

We now describe the reflection functors $\tlHI{n} : \HI \to 
\TLHI{n}{\HI}$. 

\begin{defn}
Let $n$ be a positive integer, and let $\tlHI{n}(F)$ denote the 
cokernel of the counit $\cuHI^n_F: \LHI[n]{\RHI[n]{F}} \to F$. 
Since $\cuHI^n_F$ is natural in $F$, $\tlHI{n}$ is a functor.
\end{defn}

We will show that $\tlHI{n}$ is the desired reflection functor
from $\HI$ to $\TFHI{n}$. This is established in Proposition
\ref{prop_HI_lower_slice}. We proceed by first considering the
following lemmas:

\begin{lem}\label{lem_tlHI_in_TLHI}
The image of $\HI$ under $\tlHI{n}$ is contained in 
$\TLHI{n}{\HI}$.
\end{lem}

\begin{proof}
Let $F$ be an object of $\HI$. We need to verify that 
$\RHI[n]{\tlHI{n}(F)} = 0$.

By definition, we have
\[
\LHI[n]{\RHI[n]{F}} \to F \to \tlHI{n}(F) \to 0
\]
Since the functor $F \mapsto \RHI[n]{F}$ is exact (see Proposition
\ref{prop_contract_is_exact} and Remark \ref{rmk_contraction}), we 
then have the following exact sequence
\[
\RHI[n]{\LHI[n]{\RHI[n]{F}}} \to \RHI[n]{F} \to
\RHI[n]{\tlHI{n}(F)} \to 0.
\]
By Proposition \ref{prop_unit_iso}, $\RHI[n]{(\LHI[n]{\RHI[n]{F}})}
\to \RHI[n]{F}$ is an isomorphism. Hence,
$\RHI[n]{\tlHI{n}(F)} = 0$ as desired.
\end{proof}

\begin{lem}\label{lem_tlHI_id}
The functor $\tlHI{n}$, restricted to $\TFHI{n}$ is the 
identity. Consequently, the functor $\tlHI{n}$ is idempotent
(see Definition \ref{def_coradical} (2)), and the essential
image of $\HI$ under $\tlHI{n}$ is $\TFHI{n}$.
\end{lem}
\begin{proof}
For each $F$ in $\TFHI{n}$, we have the following exact sequence:
\[
\LHI[n]{\RHI[n]{F}} \to F \to \tlHI{n}(F) \to 0
\]
However, since $F \in \TFHI{n}$, $\RHI[n]{\LHI{n}(F)} = 0$, 
and therefore the counit map is $0$. It follows that $\tlHI{n}(F) = F$ 
as desired.

The first statement follows from the fact that $\tlHI{n}(F)$ is in
$\TLHI{n}{\HI}$, which is established in Lemma \ref{lem_tlHI_in_TLHI}.
\end{proof}

\begin{prop}\label{prop_HI_lower_slice}
For each $n$, the functor $\tlHI{n}$ is left adjoint to the 
inclusion of $\TLHI{n}{\HI}$ into $\HI$.
\end{prop}
\begin{proof}
Let $F$ be a homotopy invariant sheaf with transfers, and let $G$ 
be an object in $\TLHI{n}{\HI}$. For all $f: F \to G$ we have the 
following commutative diagram:
\[
\begin{tikzcd}
F \arrow{r}{\pi_F} \arrow{d}{f}
& \tlHI{n}(F) \arrow{d}{\tlHI{n}(f)} \\
G \arrow{r}{\pi_G}
& \tlHI{n}(G) 
\end{tikzcd}
\]
where $\pi_F$ and $\pi_G$ are surjections. By Lemma \ref{lem_tlHI_id},
the map $G \stackrel{\pi_G}{\to} \tlHI{n}(G)$ is an
isomorphism. Define
\[
\chi : \homHI(F, G) \to \hom_{\TLHI{n}{\HI}}(\tlHI{n}(F), G)
\]
by $f \mapsto \pi_G^{-1} \circ \tlHI{n}(f)$. If $\chi(f) = 0$, 
then $f = 0$. Therefore $\chi$ is injective. For $g: \tlHI{n}(F) 
\to G$, set $f' = \pi \comp g$. Since $\chi(f') = g$, $\chi$ is a 
bijection, as desired.

From the way $\chi$ is defined, it is clear that $\chi$ is 
functorial in both $F$ and $G$. The proposition now follows.
\end{proof}

This shows that $\tlHI{n}$ is an idempotent quotient functor
for each natural number $n$. In fact, we have the following 
result:

\begin{prop}\label{prop_tlHIn_corad}
For each natural number $n$, $\tlHI{n}$ is a coradical.
\end{prop}
\begin{proof}
By Lemma \ref{lem_tlHI_id}, $\tlHI{n}$ is idempotent. By Proposition
\ref{prop_HI_lower_slice}, $\tlHI{n}$ is a left adjoint, and
is therefore right exact. All that remains to show is that for
each $F$ in $\HI$,
\[
\tlHI{n}(\ker (F \to \tlHI{n}(F))) = 0.
\]
Fix a positive integer $n$, and let $K$ denote the kernel of the 
surjection $F \to \tlHI{n}(F)$. We have the following short exact 
sequence 
\[
0 \to K \to F \to \tlHI{n}(F) \to 0
\]
Since $\tlHI{n}(F)$ is in $\TLHI{n}{\HI}$, by definition 
$\RHI[n]{\tlHI{n}(F)} = 0$. Therefore, we have the following
commutative diagram
\[
\begin{tikzcd}
{} &\sgHI{n} K \arrow{d}{\cuHI_F} \arrow{r}
   &\sgHI{n} F \arrow{d}{\cuHI_F} \arrow{r}
   &0 \arrow{d} \arrow{r}
   &0 \\
0 \arrow{r} &
  K \arrow{r}&
  F \arrow{r}&
  \tlHI{n}(F) \arrow{r}&
  0
\end{tikzcd}
\]
By the Snake Lemma, and using the fact that $\cok \cuHI_F = 
\tlHI{n}(F)$, we have the exact sequence
\[
0 \to \tlHI{n}(K) \to \tlHI{n}(F) \stackrel{q}{\to} \tlHI{n}(F) 
   \to 0.
\]
And the map $q$ is the identity. It follows that $\tlHI{n}(K) = 0$ 
as desired.
\end{proof}

Since $\tlHI{n}$ is a coradical, by Theorem \ref{thm_precorad_eq_tt},
there exists a torsion theory $(\Cat{T}_n, \Cat{F}_n)$ associated with
each $\tlHI{n}$. We now give another description of the torsionfree
subcategories.

\begin{prop}\label{prop_tsubcat_eq_tlHI}
  For each positive integer $n$, the full subcategory $\TFHI{n}$ and
  the torsionfree subcategory $\Cat{F}_n$ are the same. Hence, the
  torsionfree subcategories form an ascending strong cofiltration of
  $\HI$.
\end{prop}
\begin{proof}
Recall from Lemma \ref{lem_tlHI_in_TLHI} that 
$\RHI[n]{\tlHI{n}(F)} = 0$ for all $F$ in $\HI$. Hence, if $F$ is 
in $\Cat{F}_n$, $\RHI[n]{F} = \RHI[n]{\tlHI{n}(F)} = 0$.

Conversely, if $\RHI[n]{F} = 0$, then $\tlHI{n}(F) = F$ by Lemma
\ref{lem_tlHI_id}. That is, $F \in \TFHI{n}$. Hence, the
torsion-free subcategory $\Cat{F}_n$ is precisely the full 
subcategory $\TFHI{n}$ of the sheaves $F$ in $\HI$ for which 
$\RHI[n]{F} = 0$.

Since $\TFHI{n}$ form an ascending strong cofiltration, the second
claim now follows.
\end{proof}

We still have to show that the torsion subcategories $\Cat{T}_n$
form a strong descending filtration. Let us first introduce a more
appropriate notation for the torsion subcategory. We will write
$\THI{n}$ for the torsion subcategory $\Cat{T}_n$. 

\begin{defn}\label{def_upper_slice_functor}
Let $\tgHI{n}$ denote the kernel of the natural surjection $\id 
\to \tlHI{n}$. By Proposition \ref{prop_rad_eq_corad} and 
Corollary \ref{cor_tt_ref_and_coref}, $\tgHI{n}$ is an idempotent 
pre-radical, and is right adjoint to the inclusion of $\THI{n}$ in
$\HI$.
\end{defn}

We will now show that $(\THI{*}, \tgHI{*})$ defines a descending 
strong filtration on $\HI$.

\begin{lem}\label{lem_tgHI_reflection}
The essential image of $\tgHI{n}$ is $\THI{n}$, and the 
restriction of $\tgHI{n}$ to $\THI{n}$ is the identity.
\end{lem}
\begin{proof}
Recall from the definition of $\tgHI{n}$ that for each $F$
in $\HI$, there exists a short exact sequence:
\[
0 \to \tgHI{n} F \to F \to \tlHI{n} F \to 0.
\]
Furthermore, recall from Theorem \ref{thm_precorad_eq_tt} that
the for all $F$ in $\THI{n}$, $\tlHI{n} F = 0$. The lemma now
follows.
\end{proof}

\begin{lem}\label{lem_TFHI_properties}
For natural numbers $n$ and $m$ such that $m > n$, $\tlHI{m} 
\tlHI{n} = \tlHI{n}$ and there exist a
natural isomorphism $\tlHI{n}\tlHI{m} \cong \tlHI{n}$. 
\end{lem}
\begin{proof}
Suppose $F$ is in $\HI$. Since $\TFHI{n}$ is a
full subcategory of $\TFHI{m}$, and $\tlHI{m}$ is the identity on 
$\TFHI{m}$ (Lemma \ref{lem_tlHI_id}), we have $\tlHI{m}\tlHI{n} = 
\tlHI{n}$. It remains for us to show that $\tlHI{n}\tlHI{m} \cong
\tlHI{n}$.

We have the following commutative diagram:
\begin{equation}\label{eq_sg_nat_diag}
\begin{tikzcd}
\sgHI{n}\sgHI{m}(F) \arrow{r}\arrow{d}{\cuHI_{\sgHI{m}(F)}} &
\sgHI{n}(F) \arrow{r}\arrow{d}{\cuHI_F} &
\sgHI{n}\tlHI{m}(F) \arrow{r}\arrow{d}{\cuHI_{\tlHI{m}(F)}} &
0 \\
\sgHI{m}(F) \arrow{r} &
F \arrow{r} &
\tlHI{m}(F) \arrow{r}&
0,
\end{tikzcd}
\end{equation}
where the vertical arrows are the counits. Furthermore, by the 
same arguments as in the Snake Lemma, we have the ``snake tail'' 
exact sequence:
\[
\cok \cuHI_{\sgHI{m}(F)} \to \tlHI{n}(F) \to \tlHI{n}\tlHI{m}(F) 
   \to 0.
\]
However, since $\sgHI{m}F \in \LHI[m]{\HI}$, by Proposition
\ref{prop_unit_iso} $\cuHI_{\sgHI{m}(F)}$ is an isomorphism.
Therefore, the natural map $\tlHI{n}(F) \cong \tlHI{n}\tlHI{m}(F)$ is
an isomorphism.
\end{proof}

\begin{prop}\label{prop_THI_form_strong_filt}
The collection $(\THI{*}, \tgHI{*})$ form a descending strong 
filtration of $\HI$, i.e. we have the following descending
tower of subcategories
\[
\HI = \THI{0} \supseteq \THI{1} \supseteq \cdots \supseteq \THI{n} \supseteq \THI{n + 1}
\supseteq \cdots
\]
and coreflection functors $\tgHI{n} : \HI \to \THI{n}$ such
that $\tgHI{n}$ restricted to $\THI{n}$ is the identity, and
$\tgHI{n}(F)$ is a subobject of $F$ for all $n$.
\end{prop}
\begin{proof}
The only claim left to show is that $\THI{m} \subseteq \THI{n}$
for $n \leq m$.

Let $F$ be an object in $\THI{m}$. Then 
$\tlHI{m}(F) = 0$, and by Lemma \ref{lem_TFHI_properties}
\[
0 = \tlHI{n}\tlHI{m}(F) = \tlHI{n}(F).
\]
Thus, $F$ is in $\THI{n}$.
\end{proof}

We introduce the following notion to describe the strong filtration
and cofiltration on $\HI$ and its relationship to the coradicals
$\tlHI{*}$.

\begin{defn}
We call the strong filtration and cofiltration defined by
the torsion theories $(\TGHI{n}{\HI}, \TLHI{n}{\HI})$ for $n = 
0,1,2,\dots$ the \DEF{torsion filtration of $\HI$.}

In general, if $\Cat{A}$ is an abelian category, we say that
$\Cat{A}$ has a torsion filtration if there exists a sequence
of idempotent pre-(co)radicals $\corad{*}$ such that the induced
torsion theories $(\Prerad{n}{\Cat{A}}, \Corad{n}{\Cat{A}})$ (for 
$n$ in $\Z$) fit together to form a descending strong filtration
\[
\Cat{A} \supseteq \cdots \supseteq \Prerad{0}{\Cat{A}} \supseteq
   \Prerad{1}{\Cat{A}} \supseteq \cdots \supseteq \Prerad{n}{\Cat{A}}
   \supseteq \cdots
\]
and an ascending strong cofiltration
\[
0 \subseteq \cdots \subseteq \Corad{0}{\Cat{A}} \subseteq
   \Corad{1}{\Cat{A}} \subseteq \cdots \subseteq \Corad{n}{\Cat{A}}
   \subseteq \cdots.
\]
\end{defn}

We conclude this section by presenting some additional properties of
the torsion subcategories and the functor $\tgHI{n}$. Recall from
\ref{prop_HI_upper_slice} that $\sgHI(F)=\LHI[n]{\RHI[n]{F}}$.

\begin{prop}\label{prop_THI_properties}
For all natural numbers $m$ and $n$ such that $m > n$,

\begin{enumerate}
\item $\THI{n}$ is the full subcategory of objects $F$ such that 
$\sgHI{n}(F) \to F$ is onto.
\tinyskip

\item $\HI(n)$ is a proper full subcategory $\THI{n}$.
\tinyskip

\item $\tlHI{n}\tgHI{m} \cong \tgHI{m}\tlHI{n}$, and $\tgHI{n}\tlHI{m} 
= \tlHI{m}\tgHI{n} = 0$ 
\tinyskip

\item $\tgHI{n}\tgHI{m} \cong \tgHI{m}\tgHI{n} \cong \tgHI{m}$.
\tinyskip
\end{enumerate}
\end{prop}
\begin{proof}
\pfitem{(1)} : For all $F$ in $\HI$ and $n \geq 0$, we have the 
following exact sequence
\[
\sgHI{n}(F) \to F \to \tlHI{n}(F) \to 0.
\]
Therefore, $\tlHI{n}(F) = 0$ if and only if $\sgHI{n}(F) 
\to F$ is a surjection.

\pfitem{(2)} : Let $F$ be an object in $\HI(n)$. Then $F =
\LHI[n]{F'}$ for some $F'$ in $\HI$. By Proposition
\ref{prop_counit_iso_for_HIn}, the counit map $\sgHI{n}(F) \to F$ is
an isomorphism.  By part (1), $F \in \THI{n}$.

\pfitem{(3)} : Let $F$ be a homotopy invariant sheaf with 
transfers. Since $\tgHI{m}(F) \in \THI{n}$, $\tlHI{n} \tgHI{m}(F) 
= 0$ by definition. Furthermore, $\tgHI{m}\tlHI{n}(F) = 0$ since 
it is the kernel of $\tlHI{m} \tlHI{n}(F) \to \tlHI{n}(F)$ 
which is an isomorphism by Lemma \ref{lem_tgHI_reflection}.

To show that $\tlHI{n} \tgHI{m}$ is naturally isomorphic to
$\tgHI{m} \tlHI{n}$, let us first consider the following 
diagram:
\begin{equation}\label{eq_sg_tgHI_nat_diag}
\begin{tikzcd}
{} &
\sgHI{m} \tgHI{n}(F) \arrow{r} \arrow{d} &
\sgHI{m}(F) \arrow{r} \arrow{d} &
\sgHI{m} \tlHI{n}(F) \arrow{r} \arrow{d} &
0 \\
0 \arrow{r} &
\tgHI{n}(F) \arrow{r} &
F \arrow{r} &
\tlHI{n}(F) \arrow{r}&
0
\end{tikzcd}
\end{equation}
where vertical maps are the counits. Notice that the top row is 
exact on the right because $\sgHI{m}$ is right exact. This follows 
from the fact that $\sgHI{m}$ is the composition of the functors 
$F \mapsto \LHI[m]{F}$, which is right exact, and $F \mapsto 
\RHI[m]{F}$, which is exact (Proposition \ref{prop_contract_is_exact}).

Since $m > n$, by Proposition \ref{prop_HI_lower_slice} 
$\tlHI{n}(F)$ is in $\TFHI{n}$, which is a subcategory of $\TFHI{m}$
by Proposition \ref{prop_tsubcat_eq_tlHI}. Since $G$ is in $\TFHI{m}$
if and only if $\RHI[m]{G} = 0$, it follows that 
$\RHI[m]{(\tlHI{n}(F))} = 0$. Hence, $\sgHI{m}\tlHI{n}(F) = 0$. 

Applying Snake Lemma to \eqref{eq_sg_tgHI_nat_diag}, we obtain the 
following exact sequence:
\begin{equation}\label{eq_tlHI_exact_sequence}
0 \to \tlHI{m}\tgHI{n}(F) \to \tlHI{m}(F) \to \tlHI{m} \tlHI{n}(F) 
   \to 0.
\end{equation}
Notice that $\tlHI{n} \tlHI{m}(F) \cong \tlHI{n}(F)$, and the 
composition $\tlHI{m}(F) \to \tlHI{n}\tlHI{m} (F) \to \tlHI{n}(F)$ 
is precisely the natural surjection associated to $\tlHI{n}(F)$. It 
follows that
\[
\tlHI{m} \tgHI{n}(F) \cong \tgHI{n} \tlHI{m}(F).
\]
Since \eqref{eq_sg_tgHI_nat_diag} is natural in $F$, the isomorphism
is natural in $F$ as well.

\pfitem{(4)} : By Proposition \ref{prop_THI_form_strong_filt},
$\THI{m} \subseteq \THI{n}$. Since $\tgHI{n}$ restricted to
$\THI{n}$ is the identity by Lemma \ref{lem_tgHI_reflection}, 
$\tgHI{n} \tgHI{m} = \tgHI{m}$. 

To show that $\tgHI{m} \tgHI{n} \cong \tgHI{m}$, notice that
for a given $F$ in $\HI$ and positive integer $n$, there exists a 
commutative diagram
\begin{equation}\label{eq_tgHI_tlHI_diag}
\begin{tikzcd}
0 \arrow{r} &
\tgHI{n}(F) \arrow{r} \arrow{d}{\eta_{\tgHI{n}(F)}} &
F \arrow{r} \arrow{d}{\eta_F} &
\tlHI{n}(F) \arrow{r} \arrow{d}{\cong} &
0 \\
0 \arrow{r} &
\tlHI{m} \tgHI{n}(F) \arrow{r} &
\tlHI{m}(F) \arrow{r} &
\tlHI{m}\tlHI{n}(F) \arrow{r} &
0.
\end{tikzcd}
\end{equation}
where $\eta$ is the natural surjection $\id \to \tlHI{m}$, and
the bottom row is precisely the short exact sequence 
\eqref{eq_tlHI_exact_sequence}. By Lemma 
\ref{lem_TFHI_properties}, the map $\tlHI{n}(F) \to \tlHI{n} 
\tlHI{m}(F)$ is an isomorphism. Therefore, by the Snake Lemma, we 
have $\tgHI{m} \tgHI{n}(F) \cong \tgHI{m}(F)$. Since 
\eqref{eq_tgHI_tlHI_diag} is natural in $F$, it follows that the 
isomorphism $\tgHI{m}\tgHI{n} \to \tgHI{m}$ is natural as well.
\end{proof}

\section{Slice Filtration on $\DMeff$ and Torsion Filtration on $\HI$}

In this section, we want to relate the filtrations on $\HI$
that we have developed with the slice filtration on $\DMeff$. 
Recall that the slice filtration structure on $\DMeff$ is 
associated with the weak filtration $(\GFiltDM[*]{\DMeff}, 
\sgDM{*})$ and the weak cofiltration $(\LFiltDM[*]{\DMeff}, 
\slDM{*})$ (see Section \ref{sect_slice_filt_dm}). The main result 
that we will verify is the following:

\begin{prop}\label{prop_H_commute_with_filt}
For each positive integer $n$, the following diagram of functors 
commute:
\[
\begin{tikzcd}
\GFiltDM[n]{\DMeff} \arrow{d}{\HH^0} &
\DMeff \arrow{l}{\sgDM{n}} \arrow{r}{\slDM{n}} \arrow{d}{\HH^0} &
\LFiltDM[n]{\DMeff} \arrow{d}{\HH^0} \\
\HI(n) &
\HI \arrow{l}{\sgHI{n}} \arrow{r}{\tlHI{n}} &
\TFHI{n}
\end{tikzcd}
\]
and all vertical arrows are essentially surjective.
\end{prop}

The rest of the section will be devoted to the proof of
Proposition \ref{prop_H_commute_with_filt}. 
First, observe that for every positive integer $n$ and every $F$ 
in $\HI$ (regarded as an object of $\DMeff$), there exists a slice 
triangle:
\[
\sliceTriangle{n}{F}
\]
Applying the cohomological functor $\HH^0$, we obtain the
following long exact sequence
\[
\cdots \stackrel{\delta_{-1}}{\to} \HH^0 \sgDM{n}(F) \to 
   \HH^0 F \to \HH^0 \slDM{n}(F)
   \stackrel{\delta_0}{\to} \HH^1 \sgDM{n}(F) \to \cdots
\]
where $\HH^i F \defeq \HH^0F[i]$. 

Notice that $\HH^0F = F$ and $\sgDM{n}(F) = \ihomDMf(\Z(1)[1],
F)(1)[1]$. By definition, $\HH^0 \sgDM{n}(F) = \LHI[n]{(\RHI[n]{F})}$,
and therefore it is clear that that the functor $\HH^0 \sgDM{n}$
restricted to $\HI$ is $\sgHI{n}$. This shows that the left square of
Proposition \ref{prop_H_commute_with_filt} commutes. $\HI(n)$ is equal
to the essential image of $\GFiltDM[n]{\DMeff}$ under $\HH^0$, and the
coreflection functors from $\HI$ to $\HI(n)$ is compatible with
$\HH^0$.

A comparable statement can be made about the filtration 
$(\TFHI{*}, \tlHI{*})$, but the arguments are more involved.
Notice that, since $\HH^0\sgDM{n}F = \LHI[n]{\RHI[n]{F}}$ we get 
the following exact sequence from \eqref{eq_slice_DM_exact_seq}
\[
\LHI[n]{(\RHI[n]{F})} \to F \to \HH^0{\slDM{n}(F)} 
   \stackrel{\delta_0}{\to} \HH^1 \sgDM{n}(F).
\]
where the map $\LHI[n]{(\RHI[n]{F}} \to F$ is the counit. If we show
that $\HH^1 \sgDM{n}(F) = 0$, then it is clear that $\HH^0 \slDM{n}(F)
\cong \tlHI{n}(F)$. This shows that the right square of Proposition
\ref{prop_H_commute_with_filt} commutes, completing the proof of
Proposition \ref{prop_H_commute_with_filt}. We will prove this
statement in Lemma \ref{lem_H1_sgDM_vanishes}, which will depend on
the following:

\begin{lem}[\cite{DegGenMot} 3.4.5]\label{lem_rhomDM_and_contract}
For $F$ in $\HI$,
\[
\ihomDMf(\Z(1)[1], F) \cong \RHI{F}
\]
as objects in $\DMeff$.
\end{lem}
\begin{proof}
Since $\HH^0 \ihomDMf(\Z(1)[1], F) \cong \RHI{F}$ by definition, 
it suffices to show that $\ihomDMf(\Z(1)[1], F)$ represents a 
chain complex concentrated in degree 0. In particular, we need
to show that the Nisnevich sheaf
\[
\HH^i\ihomDMf(\Z(1)[1], F)
\]
vanishes for all Hensel local schemes $S$.

Let $S$ be a Hensel local scheme. Note that, since $F$ is 
$\A^1$-local,
\begin{align*}
\HH^i\ihomDMf(\Z(1)[1], F)(S) &= H^i\rhomDMf(\CZtr(S) 
   \tDM \Z(1)[1], F) \\
   &\cong H^i\homDSh(\Ztr(S \times 
   \Gm), F).
\end{align*}
Furthermore, there exists a split exact triangle
\[
\Ztr(S \times \A^1) \to \Ztr(S \times (\A^1 - 0)) \to 
   \Ztr(S \times \Gm) \to \Ztr(S \times \A^1)[1],
\]
in $\DShCor$. Applying $\homDSh(-, F)$ to the triangle above, we 
have a long exact sequence in cohomology:
\begin{align*}
\cdots & \rightarrow H^i\homDSh(\Ztr(S \times \A^1), F) 
   \rightarrow H^i\homDSh(\Ztr(S \times (\A^1 - 0)), F) \\
 & \rightarrow H^i\homDSh(\Ztr(S \times \Gm), F) \rightarrow 
   H^i\homDSh(\Ztr(S \times \A^1), F) \rightarrow \cdots
\end{align*}
Note that $H^i\homDSh(\Ztr(X), F) = H_{\Nis}^i(X; F)$. Since
$H^i_{\Nis}(X; F) = 0$ for $i < 0$, and for all $i > 0$, 
\[
H_{\Nis}^i(S \times \A^1; F) = H_{\Nis}^i(S; F) = 0
\] 
and 
\[
H_{\Nis}^i(S \times (\A^1 - 0); F) = 0
\] 
(see \cite[24.5]{MVW}), it follows that 
\[
H^i\homDSh(\Ztr(S \times \Gm), F) = 0 \quad\textrm{for all $i \neq -1, 0$.}
\] 
Thus, we are reduced to showing that $H^{-1}\homDSh(\Ztr(S \times 
\Gm), F) = 0$. However, the map $F(S \times \A^1) = F(S) \to F(S 
\times (\A^1 - 0))$ is a split injection, and the lemma follows.
\end{proof}

\begin{lem}\label{lem_H_com_ihom_DM}
For $M$ in $\DMeff$, 
\[
\HH^i\ihomDMf(\Z(1)[1], M) = \ihomDMf(\Z(1)[1], \HH^i(M)) = 
   \RHI{(\HH^i(M))}
\]
\end{lem}
\begin{proof}
Since $\ihomDMf(\Z(1)[1], -)$ is the right adjoint of $-\tDM 
\Z(1)[1]$, the cohomological functor $\ihomDMf(\Z(1)[1], -)$ is 
left $t$-exact. By Lemma \ref{lem_rhomDM_and_contract} and 
Proposition \ref{prop_contract_is_exact}, the functor 
$\ihomDMf(\Z(1)[1], -)$ is also exact on the heart of $\DMeff$, 
hence it is $t$-exact.

That is, $\ihomDMf(\Z(1)[1], -)$ commutes with $\HH^0$,
and the lemma is established.
\end{proof}

\begin{lem}\label{lem_H1_sgDM_vanishes}
For each natural number $n$ and $F$ in $\HI$, $\HH^i \sgDM{n}(F) 
= 0$ for all positive integers $i$.
\end{lem}
\begin{proof}
Applying Lemma \ref{lem_rhomDM_and_contract}, we have the isomorphism:
\[
\ihomDMf(\Z(n)[n], F) \tDM \Z(n)[n] \cong \RHI[n]{F} \tDM \Z(n)[n].
\] 
Notice that the complex $\ihomDMf(\Z(n)[n], F)$
is concentrated entirely in negative degrees. It follows that 
$\HH^i(\RHI[n]{F} \tDM \Z(n)[n])$ vanishes for $i > 0$. To complete
the proof, notice that $\RHI[n]{F} \tDM \Z(n)[n]$ equals
\begin{align*}
\ihomDMf(\Z(n)[n], F) \tDM \Z(n)[n] &\cong
\ihomDMf(\Z(n), F)[-n] \tDM \Z(n)[n] \\
&\cong \ihomDMf(\Z(n), F) \tDM \Z(n) \\
&= \sgDM{n} F.  &\phantom{aai}\qedhere
\end{align*}
\end{proof}

Using \eqref{eq_slice_DM_exact_seq}, this shows that $\TFHI{n}$ is 
contained in the essential image of $\LFiltDM[n]{\DMeff}$ under 
$\HH^0$. To show the converse, we need to prove that 
$\RHI[n]{(\HH^0 M)} = 0$ for $M$ in $\LFiltDM[n]{\DMeff}$. This 
follows from the definition of $\LFiltDM[n]{\DMeff}$. Indeed, if 
$M \in \LFiltDM[n]{\DMeff}$, then 
\[
\ihomDMf(\Z(n)[n], M) = 0.
\] 
Applying Lemma \ref{lem_H_com_ihom_DM}, it follows that
\[
0 = \HH^0\ihomDMf(\Z(n)[n], M) = \RHI[n]{(\HH^0(M))}
\]
and thus $\HH^0 M \in \TFHI{n}$ by definition. We have just proved 
Proposition \ref{prop_H_commute_with_filt}.

\section{Fundamental Invariants}

As in the case of $\DMeff$, we can also define the structure 
invariants associated to the filtration and cofiltration. In this
case, for every natural number $n$, there exists a functorial 
exact sequence
\[
\sgHI{n} \to \sgHI{n - 1} \to \tlHI{n}\sgHI{n - 1} \to 0.
\]
\begin{defn}\label{defn_sliceHI}
We define \DEF{$n$-th slice functor on $\HI$} to be the functors 
$\slice{n} \defeq \tlHI{n + 1}\sgHI{n}$. 
\end{defn}

Recall from Definition \ref{def_slice_functors_DMeff} that the 
$n$-th slice functor of $(\GFiltDM{\DMeff}, 
\sgDM{*})$ is the triangulated endofunctor $\sliceDM{*}$ that
fits into the following exact triangle
\[
\slDM{n} \to \slDM{n - 1} \to \sliceDM{n - 1} \to \slDM{n}[1].
\]
A consequence of Proposition \ref{prop_H_commute_with_filt} is 
that the slice functors $\slice{n}$ on $\HI$ agree 
with the slice functors $\sliceDM{n}$ on $\DMeff$ in the following
sense:

\begin{cor}\label{cor_H_commute_with_slice}
For all natural numbers $n$,
\[
\HH^0 \sliceDM{n} = \slice{n}.
\]
\end{cor}

Let us first consider the following proposition:

\begin{prop}\label{prop_sg_tl_commute}
For natural numbers $m$ and $n$, $\sgHI{n} \tlHI{m}$ is naturally
isomorphic to $\tlHI{m} \sgHI{n}$, and are both 0 if $m \leq n$.
\end{prop}
\begin{proof}
Let $F$ be an object in $\HI$. By Lemma \ref{lem_LR_commute_LR}, 
we have the commutative square
\begin{equation}\label{eq_prop_sg_tl_com_sq}
\begin{tikzcd}
\sgHI{m}\sgHI{n}(F) \arrow{r}{f} \arrow{d}{\cong} & 
\sgHI{n}(F) \arrow[equal]{d} \\
\sgHI{n}\sgHI{m}(F) \arrow{r}{g} &
\sgHI{n}(F),
\end{tikzcd}
\end{equation}
where $f$ is counit of $\sgHI{m}\sgHI{n}(F) \to \sgHI{n}(F)$ and 
$g$ is obtained by applying $\sgHI{n}$ to the counit $\sgHI{m}(F) 
\to F$. The cokernel of $f$ is precisely $\tlHI{m} \sgHI{n}(F)$.
Since $\sgHI{n}$ is right exact, and the
following sequence is exact
\[
\sgHI{m}(F) \to F \to \tlHI{m}(F) \to 0,
\]
the following sequence is also exact.
\[
\sgHI{n} \sgHI{m}(F) \to \sgHI{n}(F) \to \sgHI{n} \tlHI{m}(F) 
   \to 0.
\]
It follows that the cokernel of $g$ is $\sgHI{n} \tlHI{m}(F)$.  By the
Five Lemma \ref{prop_sg_tl_commute}, $\tlHI{m} \sgHI{n}(F) \cong
\sgHI{n} \tlHI{m}(F)$. Since the square in Lemma
\ref{lem_tlHI_in_TLHI} is functorial, it follows that the isomorphism
identified above is natural in $F$.

Finally, suppose $m \leq n$. Then by Proposition 
\ref{prop_HI_lower_slice} $\RHI[n]{\tlHI{m}(F)} = 0$. It follows 
that $\sgHI{n} \tlHI{m}(F) = 0$, and $\tlHI{m} \sgHI{n}(F) = 0$ as 
well.
\end{proof}

\begin{rmk}
  In case the indexing becomes difficult to keep track, one might wish
  to consider a ``bread'' analogy. Imagine that a half-infinite loaf
  of bread is laid out on a line marked from 0 to $\infty$
  (representing an $F$ in $\HI$), and one is allowed to take cuts at
  the marked points and subsequently pick up all the bread lying
  greater than $n$ or less than $n$. For the functors $\tgHI{n}$ and
  $\sgHI{n}$, the higher the $n$, the less bread one would
  \emph{take}. For the functors $\tlHI{n}$, the greater the $n$, the
  less bread one would \emph{leave}.

If one finds the analogy useful, one might wish to interpret
Lemma \ref{lem_TFHI_properties}, and Propositions
\ref{prop_THI_properties} (3) and (4) with this culinary picture 
in mind.
\end{rmk}

As we did for the filtration $(\HI(*), \sgHI{*})$ in definition
\ref{defn_sliceHI}, we can define the structure invariants for
$(\THI{*}, \tgHI{*})$.
\begin{defn}
For each $F$ in $\HI$ and natural number $n$, write $\tconst{n}$ 
for the functor $\tlHI{n + 1} \tgHI{n}$, which we define to be the 
\DEF{$n$-th fundamental invariant of $F$ associated to $\tgHI{*}$}
or simply the \DEF{$n$-th fundamental invariant}.
\end{defn}

As it turns out, the $n$-th fundamental invariant is \emph{not} 
the same as the $n$-th slice functor on $\HI$. To see this, 
consider the example introduced in Remark 
\ref{rmk_sgHI_is_not_strong_filt}. For $\Oxn$, following the
discussion in \loccit, we have that
\[
\slice{k}(\Oxn) = \slice{k}(\Ox) = \begin{cases}
\Ox &\textrm{if }k = 1\\
0   &\textrm{otherwise}.
\end{cases}
\]
However, a simple calculation reveals that
\[
\tconst{k}(\Oxn) = \begin{cases}
\Oxn &\textrm{if }k = 1\\
0     &\textrm{otherwise}.
\end{cases}
\]
Nonetheless, the $n$-th slice functor is related to the $n$-th
fundamental invariant via the following proposition:

\begin{prop}\label{prop_struct_consts}
Let $m$ and $n$ be natural numbers such that $m > n$. There exists 
a natural surjection from $\tlHI{m} \sgHI{n}$ to $\tlHI{m} 
\tgHI{n}$. In particular, for each $F$ in $\HI$, there exists a 
surjection $\pi_m: \slice{m} F \to \tconst{m} F$.
\end{prop}
\begin{proof}
Let $F$ be an object of $\HI$. We have the following short exact
sequence:
\[
0 \to \tgHI{n} \tlHI{m}(F) \to \tlHI{m}(F) \to 
  \tlHI{n} \tlHI{m}(F) \to 0.
\]
By Lemma \ref{lem_TFHI_properties}, $\tlHI{m} \tlHI{n}(F) 
= \tlHI{n}(F)$, and therefore $\tgHI{n} \tlHI{m}(F)$ is the kernel 
of the surjection $\tlHI{m}(F) \to \tlHI{n}(F)$. But the sequence
\[
\sgHI{n} \tlHI{m}(F) \to \tlHI{m}(F) \to \tlHI{n}(F) \to 0
\]
is exact. Therefore, the induced map from $\sgHI{n} \tlHI{m}(F)$
to $\tgHI{n} \tlHI{m}(F)$ is a surjection as well. Furthermore,
since the commutative diagram
\[
\begin{tikzcd}
{} & \sgHI{n} \tlHI{m}(F) \arrow{r} \arrow[twoheadrightarrow]{d} &
\tlHI{m}(F) \arrow{r} \arrow[equal]{d} &
\tlHI{n}(F) \arrow{r} \arrow[equal]{d} &
0 \\
0 \arrow{r} &
\tgHI{n} \tlHI{m}(F) \arrow{r} &
\tlHI{m}(F) \arrow{r} &
\tlHI{n}(F) \arrow{r} &
0
\end{tikzcd}
\] 
is functorial in $F$, the surjection is natural. This establishes the
first claim of the proposition, since $\tgHI{n} \tlHI{m}$ is naturally
isomorphic to $\tlHI{m} \tgHI{n}$ (Proposition
\ref{prop_THI_properties} (3)) and $\sgHI{n} \tlHI{m}$ is naturally
isomorphic to $\tlHI{m} \sgHI{n}$ (Proposition
\ref{prop_sg_tl_commute}). The second claim follows by setting $n = m
- 1$.
\end{proof}

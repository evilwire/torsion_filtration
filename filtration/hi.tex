\newpage
\section{Filtrations on $\HI$}\label{sect_filtration_hi}

The purpose of this section is to construct three filtrations. 
The focus will be on two of the filtrations featured in
main result (Theorem \ref{thm_main_result}): there is a sequence 
of coradicals on the category $\HI$ which induces a descending 
strong filtration and an ascending cofiltration (defined in 
\ref{def_strong_filtration} below) on the category $\HI$. The key 
ingredient in the constructions of the filtrations are the tensor 
monoidal structure on $\HI$ and the partial internal hom. Both
of these structures are induced by the tensor and partial internal 
hom operators on $\DMeff$. 

To simplify the definition and the proofs in this section and the
next, we invoke Theorem \ref{thm_ALocal_eq_DMeff} and identify the 
category $\DMeff$ with the full triangulated subcategory $\ALocal$ 
of $\A^1$-local complexes from \ref{def_ALocal}. We identify 
objects $M$ in $\DMeff$ with some bounded above complex $F^*$ of 
Nisnevich sheaves with transfers such that $H^nF^*$ is a homotopy 
invariant presheaf with transfers for every $n$. In particular, 
regarding a sheaf with transfers as a chain complex concentrated 
in degree 0, we consider $\HI$ as an additive subcategory of 
$\DMeff$.

Recall the following notions from \cite[1.3]{BBD}:

\begin{defn}\label{def_t_struct}
A \DEF{$t$-category} is a triangulated category $\DCat$ together
with a pair of full subcategories $(\DCat^{\geq 0}, 
\DCat^{\leq 0})$ which satisfies the following:
\begin{enumerate}
\item For all $X$ in $\DCat^{\geq 0}$, and $Y$ in $\DCat^{\geq 1}$, 
$\hom_{\DCat}(X, Y) = 0$.

\item $\DCat^{\leq 0} \subset \DCat^{\leq 1}$ and
$\DCat^{\geq 1} \subset \DCat^{\geq 0}$

\item For all $X$ in $\DCat$, there exists a distinguished 
triangle
\[
A \to X \to B \to A[1]
\]
such that $A \in \DCat^{\leq 0}$ and $B \in \DCat^{\geq 1}$.
\end{enumerate}
\noindent Here we write $\DCat^{\geq n}$ and $\DCat^{\leq n}$ for 
$\DCat^{\geq 0}[n]$ and $\DCat^{\leq 0}[n]$ respectively. We call
the pair $(\DCat^{\geq 0}, \DCat^{\leq 0})$ a \DEF{$t$-structure} 
on $\DCat$.

The \DEF{heart} of a $t$-category is the full subcategory
$\Cat{C} \defeq \DCat^{\geq 0} \cap \DCat^{\leq 0}$.
\end{defn}

If $\DCat$ is a $t$-category, then the inclusion of $\DCat^{\leq 
n}$ into $\DCat$ admits a right adjoint $\gltrunc{n}: \DCat \to 
\DCat^{\leq n}$, and the inclusion of $\DCat^{\geq n}$ admits left 
adjoint $\ggtrunc{n}: \DCat \to \DCat^{\geq n}$. Furthermore, for 
all $X$ in $\DCat$, there exists a unique map $d$ in 
$\hom_{\DCat}(\ggtrunc{1}X, \gltrunc{0}X[1])$ such that
\[
\gltrunc{0} X \to X \to \ggtrunc{1} X \stackrel{d}{\to} \gltrunc{0}X[1]
\]
is distinguished (see \cite[1.3.3]{BBD}). For integers $m$ and $n$ 
such that $m < n$, it is easy to see that $\gltrunc{m} \gltrunc{n} =
\gltrunc{n}\gltrunc{m} = \gltrunc{m}$, and $\ggtrunc{m} 
\ggtrunc{n} = \ggtrunc{n} \ggtrunc{m} = \ggtrunc{n}$. Furthermore, 
$\gltrunc{m} \ggtrunc{n} = \ggtrunc{n} \gltrunc{m} = 0$, and
$\gltrunc{n} \ggtrunc{m} = \ggtrunc{m} \gltrunc{n}$ 
(see \cite[1.3.5]{BBD}). When $n = m = 0$, the composition
$\gltrunc{0} \ggtrunc{0}$ defines an additive functor 
$\HH^n : \DCat \to \Cat{C}$.

We have the following important characterization of the heart
$\Cat{C}$ of the $t$-category $\DCat$.

\begin{thm}[\cite{BBD} 1.3.6]\label{thm_heart_is_abel_cat}
Let $\DCat$ be a $t$-category, and let $(\DCat^{\geq 0}, 
\DCat^{\leq 0})$ be its associated $t$ structure. Then the heart 
$\Cat{C}$ is an admissible abelian category. 
\end{thm}

\begin{ex}\label{ex_DA_t_struct}
Let $\Cat{A}$ be an abelian category, and $\DCat[\Cat{A}]$ be its
derived category. There is a natural $t$-structure on 
$\DCat[\Cat{A}]$. The pair $(\DCat[\Cat{A}]^{\geq 0}, 
\DCat[\Cat{A}]^{\leq 0})$ are a pair of full subcategories whose
objects are those with trivial cohomology in the negative and 
positive degrees respectively. In this case, the reflection 
functors $\ggtrunc{n}$ and $\gltrunc{n}$ are given by the good
truncations. 

The heart of this $t$-structure is precisely $\Cat{A}$, regarded
as a complex concentrated in degree $0$. (See 1.3.2 and the remark
following the statement of 1.3.6 in \cite{BBD}.)
\end{ex}

\begin{ex}
If $\DCat'$ is a full subcategory of a $t$-category $\DCat$, then
$\DCat'$ is also a $t$-category with the $t$-structure given by
$({\DCat'}^{\geq 0}, {\DCat'}^{\leq 0})$, where ${\DCat'}^{\geq 0} 
\defeq \DCat^{\geq 0} \cap \DCat$ and ${\DCat'}^{\leq 0} \defeq 
\DCat^{\leq 0} \cap \DCat'$. 
\end{ex}

\begin{defn}
Let $\phi: \DCat_1 \to \DCat_2$ be a triangulated functor between
$t$-categories. We say that $\phi$ is \DEF{right $t$-exact} if 
$\phi(\DCat_1^{\leq 0}) \subseteq \DCat_2^{\leq 0}$ and $\phi$ is 
\DEF{left $t$-exact} if $\phi(\DCat_1^{\geq 0}) \subseteq 
\DCat_2^{\geq 0}$. We say that $\phi$ is \DEF{$t$-exact} 
if it is both right and left $t$-exact.
\end{defn}

The concept of (left or right) $t$-exactness is a generalization 
exactness in abelian category. We have the following result
regarding $t$-exact functors and the induced functor on the
hearts.

\begin{prop}[\cite{BBD} 1.3.17]\label{prop_t_exact_implies_exact}
Let $\DCat$ and $\DCat'$ be two $t$-categories with hearts 
$\Cat{A}$ and $\Cat{A}'$ respectively. Furthermore, let $F: 
\DCat \to \DCat'$ be a left (resp. right) $t$-exact triangulated 
functor. Then $F$ induces a left (resp. right) exact 
functor from $\Cat{A}$ to $\Cat{A}'$.
\end{prop}

If a $t$-category $\DCat$ is equipped with an additive symmetric 
monoidal structure that is right $t$-exact in both factors, then 
so does its heart $\Cat{C}$. The symmetric monoidal structure on
the heart is defined as follows. Suppose $- \tensor -$ is the 
tensor operator on $\DCat$. For $C, C'$ in $\Cat{C}$, we define $C 
\tensor^{\Cat{C}} C'$ by $\HH^0(C \tensor^{\Cat{C}} C')$. 
Since $\tensor$ is right $t$-exact in both factors, for all $M$ 
and $N$ in $\DCat$,
\[
\HH^0(M \tensor N) = \HH^0(\HH^0(M) \tensor \HH^0(N))
\]
(\cite[5.10]{DegModHom}) and $\tensor^{\Cat{C}}$ is well-defined.
It is now straightforward to verify that $(\Cat{C}, 
\tensor^{\Cat{C}})$ satisfies all the axioms of a symmetric 
monoidal category. 

In addition, if $\DCat$ has a partial internal hom in the sense
that there exists an a bifunctor $\ihom$ defined in the first 
factor for a subcollection $\DCat'$ of objects in $\DCat$ such
that
\[
\hom_{\DCat}(X \tensor Z, Y) \cong 
   \hom_{\DCat}(X, \ihom(Z, Y))
\]
for all $Y$ in $\DCat'$, then the same is true for the heart 
$\Cat{C}$ of $\DCat$. Here, we define the internal hom by
\[
\ihom_{\Cat{C}}(C, C') \defeq \HH^0(\ihom(C, C')).
\]
It is easy to see that $\ihom_{\Cat{C}}$ defines a partial inverse
for all $C'$ in $\Cat{C}$ and $C$ in the essential image of 
$\DCat'$ under $\HH^0$.

Applying the above discussion to the category $\DMeff$, we see 
that there exists a $t$-structure on $\DMeff$, with the abelian 
category $\HI$ as its heart. We write $\gltrunc{0}$, $\ggtrunc{0}$
and $\HH^0$ for the reflection functors from $\DMeff$ to the 
respective subcategories. Moreover, the triangulated monoidal 
structure on $\DMeff$ induces a symmetric monoidal bifunctor on 
$\HI$, which we write as $\tHI$. This bifunctor is uniquely 
characterized by
\[
\hS{X} \tHI \hS{Y} = \hS{X \times Y},
\]
where $\hS{X} = \HH^0(M(X))$ for $X$ in $\Sm$ (see 
\cite{DegModHom}).

Moreover, there is a partial internal hom, defined by
\[
\ihomHI(F, G) = \HH^0(\ihomDMf(F, G))
\]
for all $G$ in $\HI$, and all $F$ in $\HH^0\DMeffgm$. In 
particular, since $\Ox = \Z(1)$, the internal hom on $\HI$ 
defines an endofunctor given by $\ihom(\Ox, -)$. To emphasize the
relationship with its counterpart in $\DMeff$, let
\[
\LHI{F} \defeq F \tHI \Ox \hspace{10pt} \textrm{and}\hspace{10pt} 
   \RHI{F} \defeq \ihomHI(\Ox, F).
\]
We write $\LHI[n]{F}$ for $\LHI{(\LHI[n - 1]{F})}$ and 
$\RHI[n]{F}$ for $\RHI{(\RHI[n + 1]{F})}$.

\begin{rmk}\label{rmk_contraction}
The astute reader may have noticed that $\RHI{F}$ is already used
to denote the contraction of the sheaf $F$ in $\HI$. Recall from 
\ref{def_contract} that $\RHI{F}$ is the sheaf sends $X$ in $\Sm$ to 
\[
\cok( F(X \times \A^1) \to F(X \times (\A^1 - 0))).
\]
In fact, there is no ambiguity here, since the contraction of
$F$ is isomorphic to $\ihomHI(\Ox, F)$. This is not difficult to
see: for $F$ in $\HI$,
\begin{align*}
\RHI{F} &= \mathrm{cok}\left(\ihomDMf(M(\A^1 - 0), F) \to \ihomDMf(\Z, F)\right) \\
&= \ihomDMf(\Z(1), F) = \ihomHI(\Ox, F),
\end{align*}
since $\Z(1) \cong \Ox$.
\end{rmk}

Since $\Z(1)[1] \cong \O^*$ in $\DMeff$, for any $F$ in $\HI$, 
$\HH^0(F \tDM \Z(1)[1]) = \LHI{F}$ and $\HH^0\ihomDMf(\Z(1)[1], F) 
= \RHI{F}$. These observations allow us to conclude a number of useful
facts. First, the functor $F \mapsto \LHI{F}$ is left adjoint to 
$F \mapsto \RHI{F}$. We also have the following result:

\begin{prop}\label{prop_unit_iso}
Let $F$ be a homotopy invariant sheaf with transfers. The unit map 
$F \to \RHI{(\LHI{F})}$ is an isomorphism.
\end{prop}
\begin{proof}
For $F$ in $\HI$, by the Cancellation Theorem 
(\ref{thm_dm_cancellation}), we have that 
\[
\ihomDMf(\Z(1)[1], F(1)[1]) \cong \ihomDMf(\Z, F) 
\cong F. 
\]
Now apply $\HH^0$ to this chain of isomorphisms, and note 
that $\HH^0(F) = F$. The proposition now follows.
\end{proof}

For all positive integer $n$, consider the counit map $\cuHI_F^n: 
\LHI[n]{\RHI[n]{F}} \to F$, and write $\tlHI{n}(F)$ for the 
cokernel of $\cuHI_F^n$. First, we make the following observation:

\begin{prop}\label{prop_counit_iso_for_HIn}
If $F = \LHI[n]{G}$ for some $G$ in $\HI$, then $\cuHI^n_F : 
\LHI[n]{\RHI[n]{F}} \to F$ is an isomorphism.
\end{prop}
\begin{proof}
Suppose $F = \LHI[n]{G}$ for some $G$ in $\HI$. Writing $L$ for 
the functor 
$F \mapsto \LHI[n]F$, by counit-unit adjunction, the composition
\[
\LHI[n]{G} \xrightarrow{L\eta_G} \LHI[n]{(\RHI[n]{(\LHI[n]{G})})}
   \xrightarrow{\cuHI_G L} \LHI[n]{G}
\]
is the identity, where $\eta_{G}$ and $\cuHI_{G}$ are the unit and 
the counit maps respectively. However, by Prop. \ref{prop_unit_iso}, 
$\eta_{G}$ is an isomorphism, and so is $L\eta_{G}$. It follows
that $\cuHI_G L$ is an isomorphism. But $\cuHI_G L$ is the counit
map for $LG = \LHI[n]{G} = F$, and the proposition follows.
\end{proof}

We now define the first filtration on $\HI$. Let $\LHI[0]{\HI} = 
\HI$ and let $\LHI[n]{\HI}$ denote the full subcategory of objects 
$F$ where $F = \LHI[n]{F'}$ for some $F'$ in $\HI$. It is clear 
that if $m \geq n$, then $\LHI[m]{\HI} \subseteq \LHI[n]{\HI}$. 
In particular, we have a tower of subcategories
\[
\HI = \LHI[0]{\HI} \supset \LHI[1]{\HI} \supset \LHI[2]{\HI} 
\subset \cdots.
%\cdots \subset \LHI[2]{\HI} \subset \LHI[1]{\HI} \subset \LHI[0]{\HI} 
% = \HI 
\]
To see that this filtration is not trivial (i.e. $\LHI[n]{\HI} =
\LHI[m]{\HI}$ for all natural numbers $n$ and $m$), notice that
for the constant sheaf $\Z$, it is clear that $\RHI{\Z} = 0$. 
Then $\Z$ is an object in $\HI$ but not in $\LHI{\HI}$. Indeed, if 
$\Z \in \LHI{\HI}$ then $\Z = \LHI{F'}$, but then $\RHI{\Z} = F'$ 
by Prop. \ref{prop_unit_iso}, forcing $\Z = 0$. Similarly
$\RHI{\Ox} = \Z$ and therefore, $\Ox \in \LHI{\HI}$ but $\Ox \notin
\LHI[2]{\HI}$. In general, $\LHI[n - 1]{\Ox}$ is an object of
$\LHI[n]{\HI}$ but not $\LHI[n + 1]{\HI}$.

\begin{rmk}
The subcategories $\LHI[n]{\HI}$ are additive, but \emph{not} 
abelian, except for the case $n = 0$. To see this, consider the 
map
\[
n: \Ox \to \Ox
\]
given by sending $u \in \Ox(X)$ to $u^n$ for each $X$ in $\Sm$.
The kernel of this map is sheaf of $n$-th root of unity $\mu_n$.
But $\RHI{(\mu_n)} = 0$. It follows that $\LHI{\HI}$ is not closed
under kernels. Similar arguments shows that $\LHI[n]{\HI}$ is
not closed under kernel for any positive integer $n$.
\end{rmk}

To show that the full subcategories $\LHI[n]{\HI}$ define a 
filtration, we need to show that there are reflection functors
$\sgHI{n}: \HI \to \LHI[n]{\HI}.$ In fact, these functors are 
given by $F \mapsto \LHI[n]{(\RHI[n]{F})}$. This is precisely
the statement of the following proposition.

\begin{prop}\label{prop_HI_upper_slice}
Let $\sgHI{n}$ denote the functor $F \mapsto 
\LHI[n]{(\RHI[n]{F})}$. Then $\sgHI{n}$ is right adjoint to the 
inclusion of $\LHI[n]{\HI}$. In particular, $(\LHI[*]{\HI}, 
\sgHI{*})$ defines a (nontrivial) weak filtration of $\HI$.
\end{prop}
\begin{proof}
Let $f: F \to G$ be a map in $\HI$ such that $F \in \LHI[n]{\HI}$.
By naturality of $\cuHI$, we have the following commutative 
diagram:
\[
\begin{tikzcd}
\LHI[n]{\RHI[n]{F}} \arrow{r}{\cuHI^n f}\arrow{d}{\cuHI^n_F} 
& \LHI[n]{\RHI[n]{G}} \arrow{d}{\cuHI^n_G} \\
F \arrow{r}{f}
& G.
\end{tikzcd}
\]
Since $F \in \LHI[n]{\HI}$, by Prop. \ref{prop_counit_iso_for_HIn} 
the counit map $\cuHI_F$ is an isomorphism.

Define $\chi: \homHI(F, G) \to \hom_{\LHI[n]{\HI}}(F, 
\LHI[n]{\RHI[n]{G}})$ by $f \mapsto (\cuHI^n_F)^{-1} \comp \cuHI^n 
f$. Since $\cuHI_F$ is an isomorphism, $\chi$ is injective. 
Moreover, given a map $g: F \to \LHI[n]{\RHI[n]{G}}$, set $f' = 
\cuHI_G \comp g$. Then $\chi(f') = g$. Hence $\chi$ is an 
isomorphism as desired.
\end{proof}

We have come to the main result of this section. In order to 
formulate the statement, we need to define the following notion

\begin{defn}\label{def_strong_filtration}
Let $\Cat{A}$ be an abelian category. We say that a descending 
filtration $(\nu_*\Cat{A}, \nu_*)$ is a \DEF{strong filtration} if 
for each $A$ in $\Cat{A}$ and $n$ in $\Z$, $\nu_n A$ is a 
subobject of $A$. A descending filtration $(\Cat{A}, \nu_*)$ 
is a \DEF{strong cofiltration} if $\nu_n A$ is a quotient of 
$A$ for each $n$ and each $A$ in $\Cat{A}$.

We can similarly define ascending strong filtration and 
cofiltration on $\Cat{A}$.
\end{defn}

\begin{rmk}
The difference between the notion of a strong filtrations and 
the notion of a weak filtration as defined in 
\ref{def_cat_filtration} is that the reflection functors  are
required to be subobject functors. This is a requirement that
makes sense only for categories on which one can define
subobjects.
\end{rmk}

We will now state the main theorem:

\begin{thm}\label{thm_main_result}
There exists a sequence of coradicals $\tlHI{n}, n = 0, 1, 2 
\cdots$ on $\HI$ such that the associated torsionfree 
subcategories $\tgHI{*}\HI$ form a descending strong filtration
of $\HI$ and the associated torsion subcategories $\tlHI{*}\HI$ 
form a strong cofiltration.
\end{thm}

Theorem \ref{thm_main_result} will be verified by Propositions
\ref{prop_TFHI_properties}, \ref{prop_tlHIn_corad}, and 
\ref{prop_tgHI_reflection} below.

\begin{rmk}\label{rmk_sgHI_is_not_strong_filt}
We note that $(\sgHI{*}\HI, \sgHI{*})$ does not define a strong 
filtration of $\HI$. The problem here is that the counit is not 
injective in general, as demonstrated by the following example.

Let $(\Ox)^n$ be the sheaf of $n$-th power of global units 
associated to the presheaf where sections of a smooth finite type 
$k$-scheme $X$ is the abelian subgroup of $\Ox$ given by 
\[
\{x : x=y^n\textrm{ for some }y \textrm{ in } \Ox(X)\}.
\]
It is clear that $(\Ox)^n \in \HI$. Furthermore, there exists the 
following exact sequence 
\[
0 \to \mu_n \to \Ox \to (\Ox)^n \to 0
\]
where $\mu_n$ is the constant sheaf of $n$-th roots of unity.
In particular, $\RHI{(\mu_n)} = 0$. Therefore, we have
\[
0 \to \RHI{\Ox} \to \RHI{(\Ox)^{n}} \to 0.
\]
In particular, $\LHI{\RHI{(\Ox)^n}} \cong \LHI{\RHI{\Ox}} = \Ox$, 
and the counit $\LHI{\RHI{(\Ox)^n}} \to (\Ox)^n$ is precisely $x 
\mapsto x^n$, which has a nontrivial kernel.

We can understand the problem in another way, which is that the 
categories $\sgHI{*}\HI$ are too small and do not include all
the kernels of counits $\LHI[n]{(\RHI[n]{F})} \to F$. This can be 
fixed by enlarging the filtration at each level, and to do so, we 
turn to torsion theory.
\end{rmk}

We first define the strong cofiltration, and show that the 
reflection functors are coradicals. Let $n$ be a natural number, 
and let $\tlHI{n}\HI$ be the full subcategories of objects $F$ in 
$\HI$ such that $\RHI[n]{F} = 0$. Here, we define $\RHI[n]{F} 
\defeq F$ for $n = 0$. It is easy to see that we also have the 
following ascending tower of subcategories
\[
0 = \tlHI{0}\HI \subset \tlHI{1}\HI \subset \tlHI{2}\HI \subset 
   \cdots \subset \HI
\]
It is also clear that $\tlHI{n}\HI \neq \tlHI{n + 1}\HI$ since 
$\RHI{\Ox} = \Z$ and by Prop. \ref{prop_unit_iso}, $\LHI[n]{\Ox} 
\in \tlHI{n + 1}\HI$ but $\LHI[n]{\Ox} \notin \tlHI{n}\HI$.

To see that the subcategories $\tlHI{n} \HI$ form a cofiltration
we need to describe the reflection functors $\tlHI{n} : \HI \to 
\tlHI{n}\HI$. Let $\tlHI{n}(F)$ denote the cokernel of the counit 
$\cuHI^n_F: \LHI[n]{\RHI[n]{F}} \to F$.

\begin{prop}\label{prop_HI_lower_slice}
The association $F \mapsto \tlHI{n}(F)$ is a functor, and is left 
adjoint to the inclusion $\tlHI{n}\HI \to \HI$.
\end{prop}
\begin{proof}
\pfitem{Functoriality} : on objects, $\tlHI{n}$ sends $F$ to the 
cokernel $\tlHI{n}(F)$ of the counit $\LHI[n]{\RHI[n]{F}} \to F$.

Let $f: F \to G$ be a morphism in $\HI$.
By naturality of $\cuHI^n$, we have the following commutative
diagram:
\[
\begin{tikzcd}
\LHI[n]{\RHI[n]{F}} \arrow{r}{\cuHI_F^n} \arrow{d}{\LHI[n]{\RHI[n]{f}}}
& F \arrow{r} \arrow{d}{f}
& \tlHI{n}(F) \arrow[dotted]{d}{g} \arrow{r}
& 0 \\
\LHI[n]{\RHI[n]{G}} \arrow{r}{\cuHI_G^n}
& G \arrow{r}
& \tlHI{n}(G) \arrow{r}
& 0
\end{tikzcd}
\]
with the dotted arrow $g$ given by the universal property of 
cokernels. Set $\tlHI{n}(f) = g$. It is clear from definition that
$\tlHI{n}$ is a functor.

\pfitem{Essential image is $\tlHI{n}\HI$} : Let $F$ be an object 
of $\HI$. We need to verify that $\RHI[n]{(\tlHI{n}(F))} = 0$.

By definition, we have
\[
\LHI[n]{\RHI[n]{F}} \to F \to \tlHI{n}(F) \to 0
\]
Since the functor $\RHI{?}$ is exact (see Prop.
\ref{prop_contract_is_exact} and Remark \ref{rmk_contraction}), we 
then have the following exact sequence
\[
\RHI[n]{(\LHI[n]{\RHI[n]{F}})} \to \RHI[n]{F} \to
\RHI[n]{(\tlHI{n}(F))} \to 0.
\]
By Prop. \ref{prop_unit_iso}, $\RHI[n]{(\LHI[n]{\RHI[n]{F}})}
\to \RHI[n]{F}$ is an isomorphism. It follows that $\tlHI{n}(F) = 
0$ as desired.

\pfitem{$\tlHI{n}$ is left adjoint to inclusion} : the proof
of this claim will rely on the following lemma:

\begin{lem}\label{lem_tlHI_id}
The functor $\tlHI{n}$, restricted to $\tlHI{n}\HI$ is the 
identity. Consequently, the functor $\tlHI{n}$ is idempotent.
(See Def. \ref{def_coradical} (2)).
\end{lem}
\begin{proof}[Proof of Lemma]
For $F$ in $\tlHI{n}\HI$, we have the following exact sequence:
\[
\LHI[n]{\RHI[n]{F}} \to F \to \tlHI{n}(F) \to 0
\]
However, since $F \in \tlHI{n}\HI$, $\RHI[n]{(\tlHI{n}(F))} = 0$, 
and therefore counit map is $0$. It follows that $\tlHI{n}(F) = F$ 
as desired.

The second statement follows from the fact that $\tlHI{n}(F) \in 
\tlHI{n}\HI$.
\end{proof}

Continuing with the proof of Prop. \ref{prop_HI_lower_slice}, let
$F$ be a homotopy invariant sheaf with transfers, and let $G$ be 
an object in $\tlHI{n}\HI$. For all $f: F \to G$ we have the 
following commutative diagram:
\[
\begin{tikzcd}
F \arrow{r}{\pi_F} \arrow{d}{f}
& \tlHI{n}(F) \arrow{d}{\tlHI{n}(f)} \\
G \arrow{r}{\pi_G}
& \tlHI{n}(G) 
\end{tikzcd}
\]
where $\pi_F$ and $\pi_G$ are surjections. By Lemma 
\ref{lem_tlHI_id}, since $G \in \tlHI{n}\HI$, the map $G 
\stackrel{\pi'}{\to} \tlHI{n}(G)$ is the identity. Define
\[
\chi : \homHI(F, G) \to \hom_{\tlHI{n}\HI}(\tlHI{n}(F), G)
\]
by $f \mapsto \tlHI{n}(f)$. If $\tlHI{n}(f) = 0$, then $f = 0$.
Therefore $\chi$ is injective. For $g: \tlHI{n}(F) \to G$, then
set $f' = \pi \comp g$. It is easy to see that $\chi(f') = g$.
Thus, $\chi$ is a bijection, as desired.
\end{proof}

As demonstrated, $\tlHI{n}$ is an idempotent quotient functor
for each natural number $n$. In fact, we have the following 
result:

\begin{prop}\label{prop_tlHIn_corad}
For each natural number $n$, $\tlHI{n}$ is a coradical.
\end{prop}
\begin{proof}
By Lemma \ref{lem_tlHI_id}, $\tlHI{n}$ is idempotent. By Prop. 
\ref{prop_HI_lower_slice}, $\tlHI{n}$ is a left adjoint, and
is therefore right exact. All that remains to show is that for
each $F$ in $\HI$,
\[
\tlHI{n}(\ker (F \to \tlHI{n}(F))) = 0.
\]
Fix a positive integer $n$, and let $K$ denote the kernel of the 
surjection $F \to \tlHI{n}(F) \to 0$. We have the following short 
exact sequence 
\[
0 \to K \to F \to \tlHI{n}(F) \to 0
\]
Since $\tlHI{n}(F) \in \tlHI{n}\HI$, by definition 
$\RHI[n]{(\tlHI{n}(F))} = 0$, and therefore, we have the following
commutative diagram
\[
\begin{tikzcd}
{} &\LHI[n]{(\RHI[n]{K})} \arrow{d}{\cuHI_F} \arrow{r}
   &\LHI[n]{(\RHI[n]{F})} \arrow{d}{\cuHI_F} \arrow{r}
   &0 \arrow{d} \arrow{r}
   &0 \\
0 \arrow{r} &
  K \arrow{r}&
  F \arrow{r}&
  \tlHI{n}(F) \arrow{r}&
  0
\end{tikzcd}
\]
By the Snake Lemma, and using the fact that $\cok \cuHI_F = 
\tlHI{n}(F)$, we have the exact sequence
\[
0 \to \tlHI{n}(K) \to \tlHI{n}(F) \to \tlHI{n}(F) \to 0.
\]
And the map $\tlHI{n}(F) \to \tlHI{n}$ is the identity. It follows 
that $\tlHI{n}(K) = 0$ as desired.
\end{proof}

The following is a straightforward consequence of the preceding 
proposition and Thm. \ref{thm_precorad_eq_tt}, and Cor. 
\ref{cor_tt_ref_and_coref}:

\begin{cor}\label{cor_tlHI_prop}
For $n$, there exists a torsion pair 
\[
(\tgHI{n}\HI, \tlHI{n}\HI)
\] 
where $\tgHI{n}\HI$ is the full subcategory of objects $F$ in 
$\HI$ for which $\tlHI{n}(F) = 0$, and $\tlHI{n}\HI$ is the full 
subcategory of $F$ in $\HI$ such that $\tlHI{n}(F) = F$. 
Furthermore, $\tlHI{n} : \HI \to \tlHI{n}\HI$ is a reflection 
functor for the inclusion of $\tlHI{n} \HI$ into $\HI$.
\end{cor}

\begin{defn}
We call the strong filtration and cofiltration defined by
the torsion theories $(\tgHI{n}\HI, \tlHI{n}\HI)$ for $n = 
0,1,2,\dots$ the \DEF{torsion filtration of $\HI$.}

In general, if $\Cat{A}$ is an abelian category, we say that
$\Cat{A}$ has a torsion filtration if there exists a sequence
of idempotent pre-(co)radicals $\corad{*}$ such that the induced
torsion theories fit together to form a descending strong 
filtration
\[
\Cat{C} \supseteq \cdots \supseteq \corad{0}\Cat{C} \supseteq
   \corad{1}\Cat{C} \supseteq \cdots \supseteq \corad{n}\Cat{C} 
   \supseteq \cdots
\]
and an ascending strong cofiltration
\[
0 \subseteq \cdots \subseteq \corad{0}\Cat{C} \subseteq
   \corad{1}\Cat{C} \subseteq \cdots \subseteq \corad{n}\Cat{C} 
   \subseteq \cdots.
\]
\end{defn}

\begin{rmk}
We have used the notation $\tlHI{n}\HI$ to denote both the full
subcategory of objects $F$ for which $\RHI[n]{F} = 0$ and also
the torsionfree subcategory associated to the coradical $\tlHI{n}$
to highlight the fact that these two full subcategories of $\HI$
are the same. This is shown in (1) of the following proposition.
\end{rmk}

\begin{prop}\label{prop_TFHI_properties}
For natural numbers $n$ and $m$ such that $m > n$

\begin{enumerate}
\item for $F$ in the torsion category $\TFHI{n}$, $\RHI[n] F = 0$. 
Conversely, if $\RHI[n] F = 0$, then $F \in \TFHI{n}$.
\tinyskip

\item $\TFHI{n}$ is a full subcategory of $\TFHI{m}$.
\tinyskip

\item $\tlHI{n}\tlHI{m} = \tlHI{m}\tlHI{n} = \tlHI{n}$.
\tinyskip
\end{enumerate}
\end{prop}
\begin{proof}
\pfitem{(1)} : Since $\RHI[n]{(\tlHI{n}(F))} = 0$ for all $F$ in 
$\HI$ (see Prop. \ref{prop_HI_lower_slice}), for all $F$ in
$\TFHI{n}$, $\RHI[n]{F} = \RHI[n]{(\tlHI{n}(F))} = 0$.

Conversely, if $\RHI[n]{F} = 0$, then $\tlHI{n}(F) = F$ by Lemma
\ref{lem_tlHI_id}. That is, $F \in \TFHI{n}$. In particular, the
torsion-free subcategory $\TFHI{n}$ is precisely the full 
subcategory of $F$ in $\HI$ for which $\RHI[n]{F} = 0$.

\pfitem{(2)} : Suppose we have positive integers $m$ and $n$ such
that $m > n$, and $F$ in $\TFHI{n}$, then $\RHI[m]{F} = 0$. By 
part (1), $F \in \TFHI{m}$.

\pfitem{(3)} : Suppose, as in (2), that $m, n$ are two positive
integers such that $m > n$, and $F$ is in $\HI$.

By part (2) and the fact that $\tlHI{m}$ is the identity on 
$\TFHI{m}$ (Lemma \ref{lem_tlHI_id}), it is clear that 
$\tlHI{m}\tlHI{n} = \tlHI{n}$. It remains to show that 
$\tlHI{n}\tlHI{m} = \tlHI{n}$.

We have the following commutative diagram:
\[
\begin{tikzcd}
\sgHI{n}\sgHI{m}(F) \arrow{r}\arrow{d}{\cuHI_{\sgHI{m}(F)}} &
\sgHI{n}(F) \arrow{r}\arrow{d}{\cuHI_F} &
\sgHI{n}\tlHI{m}(F) \arrow{r}\arrow{d}{\cuHI_{\tlHI{m}(F)}} &
0 \\
\sgHI{m}(F) \arrow{r} &
F \arrow{r} &
\tlHI{m}(F) \arrow{r}&
0,
\end{tikzcd}
\]
where the vertical arrows are the counits. Furthermore, by the 
same arguments as in the Snake Lemma, we have the ``snake tail'' 
exact sequence:
\[
\cok \cuHI_{\sgHI{m}(F)} \to \tlHI{n}(F) \to \tlHI{n}\tlHI{m}(F) 
   \to 0.
\]
However, since $\sgHI{m}F \in \LHI[m]{\HI}$, by Prop. 
\ref{prop_unit_iso} $\cuHI_{\sgHI{m}(F)}$ is an isomorphism. 
Therefore, $\tlHI{n}(F) \cong \tlHI{n}\tlHI{m}(F)$. Equality 
follows from the fact that both are quotients by the image of 
$\sgHI{n}(F)$ in $F$.
\end{proof}

The following are a few properties of the functor $\tgHI{*}$:

\begin{prop}\label{prop_tgHI_reflection}
\begin{enumerate}
\item $\tgHI{n}$ is an idempotent pre-radical.
\tinyskip

\item the essential image of $\tgHI{n}$ is $\THI{n}$, the 
restriction to which $\tgHI{n}$ is the identity.
\tinyskip

\item $\tgHI{n}$ is right adjoint to the inclusion $\THI{n} \into 
\HI$.
\tinyskip
\end{enumerate}
\end{prop}
\begin{proof}
Statement (1) follows from Prop. \ref{prop_rad_eq_corad} and
the fact that $\tlHI{n}$ is a coradical (and thus an idempotent 
pre-coradical). To see that $\tgHI{n}$ acts as the identity on 
$\THI{n}$, note that for any $F$ in $\THI{n}$, $\tlHI{n}(F) = 0$. 

Finally, for $F$ in $\HI$, we have the short exact sequence
\[
0 \to (\tgHI{n})^2(F) \to \tgHI{n}(F) \to \tlHI{n} \tgHI{n}(F) 
\to 0.
\]
Since $\tgHI{n}$ is a radical, $\tlHI{n} \tgHI{n}(F) = 0$. 
Therefore, $\tgHI{n}(F) \in \THI{n}$ as desired.

Finally, (3) follows from Cor. \ref{cor_tt_ref_and_coref}.
\end{proof}

\begin{prop}\label{prop_THI_properties}
For all natural numbers $m$ and $n$ such that $m > n$,

\begin{enumerate}
\item $\THI{n}$ is the full subcategory of objects $F$ such that 
$\sgHI{n}(F) \to F$ is surjective.
\tinyskip

\item $\THI{m} \subset \THI{n}$.
\tinyskip

\item $\sgHI{n} \HI$ is a proper full subcategory $\THI{n}$.
\tinyskip

\item $\tgHI{n}\tgHI{m} = \tgHI{m}\tgHI{n} = \tgHI{m}$.
\tinyskip

\item $\tlHI{n}\tgHI{m} = \tgHI{m}\tlHI{n}$, and $\tgHI{n}\tlHI{m} 
= \tlHI{m}\tgHI{n} = 0$ 
\tinyskip
\end{enumerate}
\end{prop}
\begin{proof}
\pfitem{(1)} : For all $F$ in $\HI$ and $n \geq 0$, we have the 
following exact sequence
\[
\sgHI{n}(F) \to F \to \tlHI{n}(F) \to 0.
\]
It is clear that $\tlHI{n}(F) = 0$ if and only if $\sgHI{n}(F) 
\to F$ is a surjection.

\pfitem{(2)} : Let $F$ be an object in $\THI{m}$. Then 
$\tlHI{m}(F) = 0$, and by Prop. \ref{prop_TFHI_properties} (3)
\[
0 = \tlHI{n}\tlHI{m}(F) = \tlHI{n}(F).
\]
Thus, $F \in \THI{m}$.

\pfitem{(3)} : Let $F$ be an object in $\sgHI{n} \HI$. Then $F = 
\LHI[n]{F'}$ for some $F'$ in $\HI$. By Prop. 
\ref{prop_counit_iso_for_HIn}, the counit map is an isomorphism. 
By part (1), $F \in \THI{n}$.

\pfitem{(4)} : By part (2) of the preceding proposition and part 
(2) of this proposition, it is clear that $\tgHI{n} \tgHI{m} = 
\tgHI{m}$. For the other part, it is precisely the dual of the 
argument in Prop. \ref{prop_TFHI_properties} part (3).

\pfitem{(5)} : Let $F$ be a homotopy invariant sheaf with 
transfers. Since $\tgHI{m}(F) \in \THI{n}$, $\tlHI{n} \tgHI{m}(F) 
= 0$ by definition. Furthermore, $\tgHI{m}\tlHI{n}(F) = 0$ since 
it is the kernel of the $\tlHI{m} \tlHI{n}(F) \to \tlHI{n}(F)$ 
which is the identity map by part (3) of the preceding 
proposition.

To show that $\tlHI{n} \tgHI{m}$ is naturally isomorphic to
$\tgHI{m} \tlHI{n}$, let us first consider the following 
diagram:
\[
\begin{tikzcd}
{} &
\sgHI{m} \tgHI{n}(F) \arrow{r} \arrow{d} &
\sgHI{m}(F) \arrow{r} \arrow{d} &
\sgHI{m} \tlHI{n}(F) \arrow{r} \arrow{d} &
0 \\
0 \arrow{r} &
\tgHI{n}(F) \arrow{r} &
F \arrow{r} &
\tlHI{n}(F) \arrow{r}&
0
\end{tikzcd}
\]
where vertical maps are the counits. Notice that the top row is 
exact on the right because $\sgHI{m}$ is right exact. This follows 
from the fact that $\sgHI{m}$ is the composition of the functors 
$F \mapsto \LHI[m]{F}$, which is right exact, and $F \mapsto 
\RHI[m]{F}$, which is exact (Prop. \ref{prop_contract_is_exact}).

Since $m > n$, $\RHI[m]{(\tlHI{n}(F))} = 0$ (see 
\ref{prop_HI_lower_slice}). Therefore, $\sgHI{m}\tlHI{n}(F) = 0$. 
By the Snake Lemma, we have the following exact sequence
\[
0 \to \tlHI{m}\tgHI{n}(F) \to \tlHI{m}(F) \to \tlHI{m} \tlHI{n}(F) 
   \to 0.
\]
Notice that $\tlHI{n} \tlHI{m}(F) = \tlHI{m}(F)$, and the map from
$\tlHI{m}(F) \to \tlHI{n}(F)$ is precisely the unit map associated
to the functor $\tlHI{n}$. It follows that
\[
\tlHI{m} \tgHI{n}(F) \cong \tgHI{n} \tlHI{m}(F).
\]
Naturality follows from the construction of the isomorphism.
\end{proof}

We conclude this section by relating the filtrations on $\HI$
to the slice filtration on $\DMeff$. Recall that the slice 
filtration structure on $\DMeff$ is associated with the (weak) 
filtration $(\sgDM{*}\DMeff, \sgDM{*})$ and the (weak) 
cofiltration $(\sgDM{*}\DMeff, \slDM{*})$ (see Sect. 
\ref{sect_slice_filt_dm}).

First, observe that for every positive integer $n$ and every $F$ 
in $\HI$ (regarded as an object of $\DMeff$), there exists a slice 
triangle:
\[
\sliceTriangle{n}{F}
\]
Applying the cohomological functor $\HH^0$, we obtain the
following long exact sequence
\[
\cdots \stackrel{\delta_{-1}}{\to} \HH^0 \sgDM{n}(F) \to 
   \HH^0 F \to \HH^0 \slDM{n}(F)
   \stackrel{\delta_0}{\to} \HH^1 \sgDM{n}(F) \to \cdots
\]
where $\HH^i F \defeq \HH^0F[i]$. In particular, we
have the following exact sequence
\begin{equation}\label{eq_slice_DM_exact_seq}
\HH^0 \slDM{n}(F) \to \HH^0 F \to \HH^0 
\sgDM{n}(F) \stackrel{\delta_0}{\to} \HH^1 \sgDM{n}(F).
\end{equation}
Notice that $\HH^0F = F$ and $\sgDM{n}(F) = 
\ihomDMf(\Z(1)[1], F)(1)[1]$. By definition, $\HH^0 \sgDM{n}(F) 
= \LHI[n]{(\RHI[n]{F})}$, and therefore it is clear that
that the functor $\HH^0 \sgDM{n}$ restricted to $\HI$ is 
$\sgHI{n}$. In particular, $\HI(n)$ is equal to the essential
image of $\sgDM{n}\DMeff$ under $\HH^0$, and the reflection 
functors from $\HI$ to $\HI(n)$ is compatible with $\HH^0$.

A comparable statement can be made about the filtration 
$(\tlHI{*}\HI, \tlHI{*})$, but the arguments are more involved.
Notice that, since $\HH^0\sgDM{n}F = \LHI[n]{\RHI[n]{F}}$ we get 
the following exact sequence from \eqref{eq_slice_DM_exact_seq}
\[
\LHI[n]{(\RHI[n]{F})} \to F \to \HH^0{\slDM{n}(F)} 
   \stackrel{\delta_0}{\to} \HH^1 \sgDM{n}(F).
\]
where the map $\LHI[n]{(\RHI[n]{F}} \to F$ is the counit. If we 
show that $\HH^1 \sgDM{n}(F) = 0$, then it is clear that
$\HH^0 \slDM{n}(F) \cong \tlHI{n}(F)$. We will prove this 
statement in \ref{lem_H1_sgDM_vanishes}, which will depend
on the following:

\begin{lem}[\cite{DegGenMot} 3.4.5]\label{lem_rhomDM_and_contract}
For $F$ in $\HI$,
\[
\ihomDMf(\Z(1)[1], F) \cong \RHI{F}
\]
as objects in $\DMeff$.
\end{lem}
\begin{proof}
Since $\HH^0 \ihomDMf(\Z(1)[1], F) \cong \RHI{F}$ by definition, 
it suffices to show that $\ihomDMf(\Z(1)[1], F)$ represents a 
chain complex concentrated in degree 0. In particular, we need
to show that the Nisnevich sheaf
\[
\HH^i\ihomDMf(\Z(1)[1], F)
\]
vanishes for all Hensel local schemes $S$.

Let $S$ be a Hensel local scheme. Note that, since $F$ is 
$\A^1$-local,
\begin{align*}
\HH^i\ihomDMf(\Z(1)[1], F)(S) &= H^i\rhomDMf(\CZtr(S) 
   \tDM \Z(1)[1], F) \\
   &\cong H^i\homDSh(\Ztr(S \times 
   \Gm), F).
\end{align*}
Furthermore, there exists a split exact triangle
\[
\Ztr(S \times \A^1) \to \Ztr(S \times (\A^1 - 0)) \to 
   \Ztr(S \times \Gm) \to \Ztr(S \times \A^1)[1],
\]
in $\DShCor$. Applying $\homDSh(-, F)$ to the triangle above, we 
have a long exact sequence in cohomology:
\begin{align*}
\cdots & \rightarrow H^i\homDSh(\Ztr(S \times \A^1), F) 
   \rightarrow H^i\homDSh(\Ztr(S \times (\A^1 - 0)), F) \\
 & \rightarrow H^i\homDSh(\Ztr(S \times \Gm), F) \rightarrow 
   H^i\homDSh(\Ztr(S \times \A^1), F) \rightarrow \cdots
\end{align*}
Note that $H^i\homDSh(\Ztr(X), F) = H_{\Nis}^i(X; F)$. Since
$H^i_{\Nis}(X; F) = 0$ for $i < 0$, and for all $i > 0$, 
\[
H_{\Nis}^i(S \times \A^1; F) = H_{\Nis}^i(S; F) = 0
\] 
and 
\[
H_{\Nis}^i(S \times (\A^1 - 0); F) = 0
\] 
(see \cite[24.5]{MVW}), it follows that 
\[
H^i\homDSh(\Ztr(S \times \Gm), F) = 0 \quad\textrm{for all $i \neq -1, 0$.}
\] 
Thus, we are reduced to showing that $H^{-1}\homDSh(\Ztr(S \times 
\Gm), F) = 0$. However, the map $F(S \times \A^1) = F(S) \to F(S 
\times (\A^1 - 0))$ is a split injection, and the lemma follows.
\end{proof}

\begin{lem}\label{lem_H_com_ihom_DM}
For $M$ in $\DMeff$, 
\[
\HH^i\ihomDMf(\Z(1)[1], M) = \ihomDMf(\Z(1)[1], \HH^i(M)) = 
   \RHI{(\HH^i(M))}
\]
\end{lem}
\begin{proof}
First, the cohomological functor $\ihomDMf(\Z(1)[1], -)$ is left 
$t$-exact (being the right adjoint of $-\tDM \Z(1)[1]$. By the 
preceding lemma and Prop \ref{prop_contract_is_exact},
$\ihomDMf(\Z(1)[1], -)$ is also exact on the heart of $\DMeff$,
whence it is $t$-exact.

That is, $\ihomDMf(\Z(1)[1], -)$ commutes with $\HH^0$,
and the lemma is established.
\end{proof}

\begin{lem}\label{lem_H1_sgDM_vanishes}
For each $F$ in $\HI$, $\HH^1 \sgDM{n}(F) = 0$.
\end{lem}
\begin{proof}
Applying Lemma \ref{lem_rhomDM_and_contract}, we see that the
complex
\[
\ihomDMf(\Z(n)[n], F) \tDM \Z(n)[n] \cong \RHI{F} \tDM \Z(n)[n]
\] 
is concentrated entirely in negative degrees. It follows that 
$\HH^i$ vanishes for $i > 0$, and in particular, for $i = 
1$.
\end{proof}

This shows that $\tlHI{n}\HI$ is contained in the essential image 
of $\sgDM{n}\DMeff$ under $\HH^0$. To show the converse, we 
establish that $\RHI[n]{(\HH^0 M)} = 0$ for $M$ in 
$\slDM{n}\DMeff$. This follows from the definition of 
$\slDM{n}\DMeff$. Indeed, if $M \in \slDM{n}\DMeff$, 
then 
\[
\ihomDMf(\Z(n)[n], M) = 0.
\] 
Applying Lemma \ref{lem_H_com_ihom_DM}, it follows that
\[
0 = \HH^0\ihomDMf(\Z(n)[n], M) = \RHI[n]{(\HH^0(M))}
\]
and thus $\HH^0 M \in \tlHI{n}\HI$ by definition. We
have just proved:

\begin{prop}\label{prop_H_commute_with_filt}
For each positive integer $n$, the following diagram of functors 
commute:
\[
\begin{tikzcd}
\sgDM{n}\DMeff \arrow{d}{\HH^0} &
\DMeff \arrow{l}{\sgDM{n}} \arrow{r}{\slDM{n}} \arrow{d}{\HH^0} &
\slDM{n}\DMeff \arrow{d}{\HH^0} \\
\sgHI{n}\HI &
\HI \arrow{l}{\sgHI{n}} \arrow{r}{\tlHI{n}} &
\tlHI{n}\HI 
\end{tikzcd}
\]
and all vertical arrows are essentially surjective.
\end{prop}

As in the case of $\DMeff$, we can also define the structure 
invariants associated to the filtration and cofiltration. In this
case, for every natural number $n$, there exists a functorial 
exact sequence
\[
\sgHI{n} \to \sgHI{n - 1} \to \tlHI{n}\sgHI{n - 1} \to 0.
\]
\begin{defn}
We define \DEF{the structure constants on $\HI$} to be the 
functors $\slice{n} \defeq \tlHI{n + 1}\sgHI{n}$. In this case we 
call $\slice{n}$ the $n$-th structure constant.
\end{defn}

Recall the definition of the structure constants of $(\DMeff, 
\sgDM{*})$ to be the triangulated endofunctor $\sliceDM{*}$ that
fits into the following exact triangle
\[
\slDM{n} \to \slDM{n - 1} \to \sliceDM{n - 1} \to \slDM{n}[1].
\]
A direct consequence of Prop. \ref{prop_H_commute_with_filt} is 
that the structure constants of $(\LHI[*]{\HI}, \sgHI{*})$ agree 
with the structure constants of $(\sgDM{*}\DMeff, \sgDM{*})$:

\begin{cor}\label{cor_H_commute_with_slice}
Let the structure constants $\slice{*}$ for $(\LHI[*]{\HI}, 
\sgHI{*})$ be defined as above. Then
\[
\HH^0 \sliceDM{n} = \slice{n}.
\]
\end{cor}

Before we proceed, let us record for subsequent results the 
following proposition.

\begin{prop}\label{prop_sg_tl_commute}
For natural numbers $m$ and $n$, $\sgHI{n} \tlHI{m}$ is naturally
isomorphic to $\tlHI{m} \sgHI{n}$, and are both 0 if $m \leq n$.
\end{prop}
\begin{proof}
Let $F$ be an object in $\HI$. Notice that by Prop. 
\ref{prop_unit_iso}, $\LHI[n]{\RHI[n]{(\LHI[m]{\RHI[m]{F}})}} 
\cong \LHI[m]{\RHI[m]{(\LHI[n]{\RHI[n]{F}})}}$. Furthermore, the 
isomorphism fits into a commutative square
\begin{equation}\label{eq_prop_sg_tl_com_sq}
\begin{tikzcd}
\sgHI{m}\sgHI{n}(F) \arrow{r}{f} \arrow{d}{\cong} & 
\sgHI{n}(F) \arrow[equal]{d} \\
\sgHI{n}\sgHI{m}(F) \arrow{r}{g} &
\sgHI{n}(F),
\end{tikzcd}
\end{equation}
where, the map $f$ is counit of $\sgHI{n}(F)$, and the map $g$
is obtained from applying $\sgHI{n}$ to the counit $\sgHI{m}(F) 
\to F$.

The cokernel of $f$ is precisely $\tlHI{m} \sgHI{n}(F)$, and we 
claim that the cokernel of the map on the bottom row map is 
$\sgHI{n} \tlHI{m}(F)$. Indeed, $\sgHI{n}$ is right exact. Applying 
$\sgHI{n}$ to the exact sequence
\[
\sgHI{m}(F) \to F \to \tlHI{m}(F) \to 0,
\]
we have
\[
\sgHI{n} \sgHI{m}(F) \to \sgHI{n}(F) \to \sgHI{n} \tlHI{m}(F) \to 0.
\]
It is clear that $\tlHI{m} \sgHI{n}(F) \cong \sgHI{n} \tlHI{m}(F)$.
Since the square in Display \ref{eq_prop_sg_tl_com_sq} is 
functorial, it follows that the isomorphism identified above is
natural in $F$.

To conclude, suppose $m \leq n$. Then $\RHI[n]{(\tlHI{m}(F))} = 0$
(Prop \ref{prop_HI_lower_slice}). It follows that $\sgHI{n} 
\tlHI{m}(F) = 0$, and so is true of $\tlHI{m} \sgHI{n}(F)$.
\end{proof}

\begin{rmk}
In case the indexing becomes difficult to keep track, one might
wish to consider a ``bread'' analogy. Imagine that a half-infinite 
loaf of bread is laid out on a line marked from 0 to $\infty$ 
(representing an $F$ in $\HI$), and one is allowed to take cuts at 
the marked points, and subsequently pick up all the bread lying 
greater than $n$ or less than $n$. The former action describes the 
functors $\tgHI{n}$ and $\sgHI{n}$, and therefore, the higher the 
$n$, the \emph{less} bread one would take; the latter, functors 
$\tlHI{n}$, wherein the greater the $n$, the \emph{more} bread one 
would takes.

If one finds the analogy useful, one might wish to interpret
Prop. \ref{prop_TFHI_properties} (3) and \ref{prop_THI_properties} (4)
and (5) with this culinary picture in mind.
\end{rmk}

As we did for the filtration $(\HI, \sgHI{*})$, we can define the 
structure invariants for $(\HI, \tgHI{*})$. 
\begin{defn}
For each $F$ in $\HI$ and natural number $n$, write $\tconst{n}$ 
for the functor $\tgHI{n + 1} \tgHI{n}$, which we define to be the 
\DEF{$n$-th structure invariant of $F$ associated to $\tgHI{*}$}. 
\end{defn}

As it turns out, this is \emph{not} the same as the structure 
invariants of $\sgHI{*}$. To see this, consider the example 
introduced in \ref{rmk_sgHI_is_not_strong_filt}. For $(\Ox)^n$,
it is easy to see that 
\[
\tconst{k}(\Ox)^n = \begin{cases}
(\Ox)^n &\textrm{if }k = 1\\
0     &\textrm{otherwise},
\end{cases}
\]
while
\[
\slice{k}(\Ox)^n = \slice{k}(\Ox) = \begin{cases}
\Ox &\textrm{if }k = 1\\
0   &\textrm{otherwise}.
\end{cases}
\]
Nonetheless, the two structure constants are related by the 
following:

\begin{prop}\label{prop_struct_consts}
Let $m$ and $n$ be natural numbers such that $m > n$. There exists 
a natural surjection from $\tlHI{m} \sgHI{n}$ to $\tlHI{m} 
\tgHI{n}$. In particular, for each $F$ in $\HI$, there exists a 
surjection $\pi_m: \slice{m} F \to \tconst{m} F$.
\end{prop}
\begin{proof}
Let $F$ be a homotopy invariant sheaf with transfers, and notice that
\[
0 \to \tgHI{n} \tlHI{m}(F) \to \tlHI{m}(F) \to 
  \tlHI{n} \tlHI{m}(F) \to 0.
\]
By Prop. \ref{prop_TFHI_properties} part (3), $\tlHI{m} \tlHI{n}(F) 
= \tlHI{n}(F)$, and notice that $\tgHI{n} \tlHI{m}(F)$ is the kernel 
of the surjection $\tlHI{m}(F) \to \tlHI{n}(F)$.

But the sequence
\[
\sgHI{n} \tlHI{m}(F) \to \tlHI{m}(F) \to \tlHI{n}(F) \to 0
\]
is exact. Therefore, the induced map from $\sgHI{n} \tlHI{m}(F)$
to $\tgHI{n} \tlHI{m}(F)$ is a surjection as well. Furthermore,
since the commutative diagram
\[
\begin{tikzcd}
{} & \sgHI{n} \tlHI{m}(F) \arrow{r} \arrow[twoheadrightarrow]{d} &
\tlHI{m}(F) \arrow{r} \arrow[equal]{d} &
\tlHI{n}(F) \arrow{r} \arrow[equal]{d} &
0 \\
0 \arrow{r} &
\tgHI{n} \tlHI{m}(F) \arrow{r} &
\tlHI{m}(F) \arrow{r} &
\tlHI{n}(F) \arrow{r} &
0
\end{tikzcd}
\] 
is functorial in $F$, the surjection is natural.

To prove the first claim of the proposition, notice that 
$\tgHI{n} \tlHI{m}$ is naturally isomorphic to $\tlHI{m} \tgHI{n}$ 
(Prop. \ref{prop_THI_properties}) and $\sgHI{n} \tlHI{m}$ is 
naturally isomorphic to $\tlHI{m} \sgHI{n}$ (Prop 
\ref{prop_sg_tl_commute}).

The second claim follows from setting $n = m - 1$.
\end{proof}

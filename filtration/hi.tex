\newpage
\section{Filtration on $\HI$}\label{sect_filtration_hi}

Recall that a weak filtration of a category $\Cat{C}$ is a tower 
of subcategories
\[
\cdots \subset F_n\Cat{C} \subset F_{n - 1}\Cat{C} \subset \cdots 
   \subset \Cat{C}
\]
together with reflection functors $\nu_n : \Cat{C} \to F_n\Cat{C}_n$
(cf. \ref{def_cat_filtration}.) If we have a suitable notion of
subobjects in $\Cat{C}$, e.g. if $\Cat{C}$ is abelian, we can 
define a stronger notion (as the name suggests).

For the remainder, let $\Cat{C}$ be an abelian category. We have:

\begin{defn}
We say that a filtration $(\Cat{C}, \nu_*)$ is a \DEF{strong 
filtration} if for each $A \in \Cat{C}$ and $n \in \Z$, $nu_n A$ is 
a subobject of $A$. Similarly a cofiltration $(\Cat{C}, \gamma_*)$ 
is a \DEF{strong cofiltration} if $\gamma_n A$ is a quotient of 
$A$ for each $n$.
\end{defn}

The purpose of this section and the next is to construct a 
strong filtration on the category $\CycMod$. To do this, we first
construct a weak filtration structure on $\HI$ using a pair
of adjoint functors obtained from the closed symmetric monoidal 
structure of $\HI$. We adopt the same convention as in Sections
\ref{sect_heart_struct} and \ref{sect_slice_filt_dm}: we identify 
$\DMeff$ with the full subcategory of $\DSh$ with homotopy 
invariant cohomologies, and identify $\HI$ with the heart of 
$\DMeff$ under the $t$-structure induced by that of $\DSh$. Once
again, let $\sheaf{H}^0$ denote the reflection functor $\DMeff
\to \HI$.

Recall that the sheaf $\Ox$ of units given by 
\[
\Ox(X) = \{\textrm{invertible elements of }\O(X)\}
\]
is a homotopy invariant sheaf with transfers. For $F \in \HI$, 
let
\[
\LHI{F} \defeq F \tHI \Ox \hspace{10pt} \textrm{and}\hspace{10pt} 
   \RHI{F} \defeq \ihomHI(\Ox, F).  
\]
We write $\LHI[n]{F}$ for $\LHI{(\LHI[n - 1]{F})}$ and 
$\RHI[n]{F}$ for $\RHI{(\RHI[n + 1]{F})}$.

\begin{rmk}\label{rmk_contract_rhi_eq}
In literature, the notation $\RHI{F}$ is often used to represent
the contraction of $F$, which is the sheaf that sends $X \in 
\Cor$ to 
\[
\cok( F(X \times \A^1) \to F(X \times (\A^1 - 0))).
\]
In fact, there is no ambiguity here, since the contraction of
$F$ is isomorphic to $\ihomHI(\Ox, F)$ by Prop. 
\ref{prop_contraction}.
\end{rmk}

The following proposition highlights the connection between the
pair of adjoint functors in the construction of the slice 
filtration on $\DMeff$ (cf.  Section \ref{sect_slice_filt_dm}).

\begin{prop}\label{prop_LRDM_eq_LRHI}
Fix any $F \in \HI$, which we also consider as an object of 
$\DMeff$. Then we have $\sheaf{H}^0(F \tDM \Z(1)[1]) \simeq 
\LHI{F}$ and $\sheaf{H}^0\ihomDMf(\Z(1)[1], F) \simeq \RHI{F}$.
\end{prop}
\begin{proof}
By Prop. \ref{prop_Z1_eq_Ox}, $\Z(1)[1] \simeq \Ox$ in $\DSh$
and hence in $\DMeff$ as well. The isomorphisms now follow
from the definitions of $\tHI$ and $\ihomHI$.
\end{proof}

As constructed, the functor $F \mapsto \LHI{F}$ is left adjoint to 
$F \mapsto \RHI{F}$. In this case,

\begin{prop}\label{prop_unit_iso}
Let $F \in \HI$. The unit map $F \to \RHI{(\LHI{F})}$ is an
isomorphism.
\end{prop}
\begin{proof}
Given $F \in \HI$, consider $F$ as an object in $\DMeff$. By
Cancellation Theorem (Thm. \ref{thm_dm_cancellation}), we have
that $\ihomDMf(\Z(1)[1], F(1)[1]) \simeq \ihomDMf(\Z, F) 
\simeq F$. Now apply $\sheaf{H}^0$ to this chain of isomorphisms,
and note that $\sheaf{H}^0(F) = F$. The proposition follows from 
Prop. \ref{prop_LRDM_eq_LRHI}.
\end{proof}

Fix $n > 0$, and consider the counit map $\cuHI_F^n: 
\LHI[n]{\RHI[n]{F}} \to F$, and write $\tlHI{n} F$ for the 
cokernel of $\cuHI_F$. First, we make the following observation:

\begin{prop}\label{prop_counit_iso_for_HIn}
Let $\cuHI^n$ denote the counit, described above. If $F \in 
\LHI[n]{\HI}$, then $\cuHI^n : \LHI[n]{\RHI[n]{F}} \to F$ is
an isomorphism.
\end{prop}
\begin{proof}
Suppose $F \in \LHI[n]{\HI}$. That is, $F = LHI[n]{F'}$ for some 
$F' \in \HI$. Writing $L$ for the functor $F \mapsto \LHI[n]F$, 
by counit-unit adjunction, the composition
\[
\LHI[n]{F'} \xrightarrow{L\eta} \LHI[n]{(\RHI[n]{(\LHI[n]{F'})})}
   \xrightarrow{\cuHI L} \LHI[n]{F'}
\]
is the identity, where $\eta_{F'}$ is the unit, and $\cuHI_{F'}$ 
is the counit maps. However, by Prop. \ref{prop_unit_iso}, 
$\eta_{F'}$ is an isomorphism, and so is $L\eta_{F'}$. It follows
that $\cuHI L$ is an isomorphism. But $\cuHI L$ is the counit
map for $LF' = \LHI[n]{F'} = F$, and the proposition follows.
\end{proof}

To see this, let $\LHI[0]{\HI} = \HI$ and let $\LHI[n]{\HI}$ 
denote the full subcategory of $F \in \HI$ where $F = \LHI[n]{F'}$ 
for some $F' \in \HI$. It is clear that if $m \geq n$, then 
$\LHI[m]{\HI} \subseteq \LHI[n]{\HI}$. In particular, we have a 
tower of subcategories
\[
\HI = \LHI[0]{\HI} \supset \LHI[1]{\HI} \supset \LHI[2]{\HI} 
\subset \cdots.
%\cdots \subset \LHI[2]{\HI} \subset \LHI[1]{\HI} \subset \LHI[0]{\HI} 
% = \HI 
\]
For the constant sheaf $\Z$, it is clear that $\RHI{\Z} = 0$. 
Then $\Z \in \HI$ but not in $\LHI{\HI}$. Indeed, if $\Z \in 
\LHI{\HI}$ then $\Z = \LHI{F'}$, but then $\RHI{\Z} = F'$ by Prop. 
\ref{prop_unit_iso}, forcing $\Z = 0$. Inductively, we see 
that $\LHI[n]{\HI} \neq \LHI[n + 1]{\HI}$ for all $n \in \N$.

\begin{prop}
Let $\sgHI{n}$ denote the functor $F \mapsto 
\LHI[n]{(\RHI[n]{F})}$. Then $\sgHI{n}$ is right adjoint to the 
inclusion of $\LHI[n]{\HI}$. In particular, $(\HI, \sgHI{n})_{n 
\in \N}$ defines a (nontrivial) weak filtration of $\HI$.
\end{prop}
\begin{proof}
Fix $F \in \LHI[n]{HI}$ and $G \in \HI$, and fix a map $f : 
F \to G$. By naturality of $\cuHI$, we have the following 
commutative diagram:
\[
\begin{tikzcd}
\LHI[n]{\RHI[n]{F}} \arrow{r}{\cuHI^n f}\arrow{d}{\cuHI^n_F} 
& \LHI[n]{\RHI[n]{G}} \arrow{d}{\cuHI^n_G} \\
F \arrow{r}{f}
& G.
\end{tikzcd}
\]
Since $F \in \LHI[n]{\HI}$, by Prop. \ref{prop_counit_iso_for_HIn} 
the counit map $\cuHI_F$ is an isomorphism.

Define $\chi: \homHI(F, G) \to \hom_{\LHI[n]{\HI}}(F, 
\LHI[n]{\RHI[n]{G}})$ by $f \mapsto (\cuHI^n_F)^{-1} \comp \cuHI^n 
f$. Since $\cuHI_F$ is an isomorphism, $\chi$ is injective. 
Moreover, given a map $g: F \to \LHI[n]{\RHI[n]{G}}$, set $f' = 
\cuHI_G \comp g$. Then $\chi(f') = g$. Hence $\chi$ is an 
isomorphism as desired.
\end{proof}

For $n > 0$, let $\tlHI{n}\HI$ be the full subcategories of 
objects $F \in \HI$ such that $\RHI[n]{F} = 0$ (and for $n = 0$, 
we define $\RHI[0]{F} \defeq F$). It is clear that we also have 
the following ascending tower of subcategories
\[
0 = \tlHI{0}\HI \subset \tlHI{1}\HI \subset \tlHI{2}\HI \subset 
   \cdots \subset \HI
\]
It is also clear that $\tlHI{n} \neq \tlHI{n + 1}$ since 
$\RHI{\Ox} = \Z$ and by Prop. \ref{prop_unit_iso} $\LHI[n]{\Ox} 
\in \tlHI{n + 1}\HI$ but not in $\tlHI{n}\HI$.

In fact, we have the following result:

\begin{prop}\label{prop_HI_lower_slice}
Fix a positive integer $n$ and let $\tlHI{n}F$ be cokernel of the 
counit $\cuHI^n_F: \LHI[n]{\RHI[n]{F}} \to F$. Then the 
association $F \mapsto \tlHI{n}F$ is a functor, and is left 
adjoint to the inclusion $\tlHI{n}\HI \to \HI$.

Thus, $(\HI, \tlHI{*})$ defines a strong cofiltration
of $\HI$.
\end{prop}
\begin{proof}
\pfitem{Functoriality} : on objects, $\tlHI{n}$ sends $F$ to the 
cokernel $\tlHI{n}F$ of the counit $\LHI[n]{\RHI[n]{F}} \to F$.

Let $f: F \to G$ be a morphism in $\HI$.
By naturality of $\cuHI^n_F$, we have the following commutative
diagram:
\[
\begin{tikzcd}
\LHI[n]{\RHI[n]{F}} \arrow{r}{\cuHI_F^n} \arrow{d}{\LHI[n]{\RHI[n]{f}}}
& F \arrow{r} \arrow{d}{f}
& \tlHI{n} F \arrow[dotted]{d}{g} \arrow{r}
& 0 \\
\LHI[n]{\RHI[n]{G}} \arrow{r}{\cuHI_F^n}
& G \arrow{r}
& \tlHI{n}G \arrow{r}
& 0
\end{tikzcd}
\]
with the dotted arrow $g$ given by the universal property of 
cokernels. Set $\tlHI{n}f = g$. It is clear from definition that
$\tlHI{n}$ is a functor.

\pfitem{Essential image is $\tlHI{n}\HI$} : Fix $F \in \HI$. To
show that $\tlHI{n}F \in \tlHI{n}\HI$, we verify that 
$\RHI[n]{(\tlHI{n}F)}$ is 0.

By definition, we have
\[
\LHI[n]{\RHI[n]{F}} \to F \to \tlHI{n}F \to 0
\]
Since the functor $\RHI{?}$ is exact (see Prop. 
\ref{prop_contract_is_exact} and Prop. \ref{prop_contraction}), 
we then have the following exact sequence
\[
\RHI[n]{(\LHI[n]{\RHI[n]{F}})} \to \RHI[n]{F} \to
\RHI[n]{(\tlHI{n}F)} \to 0.
\]
By Prop. \ref{prop_unit_iso}, $\RHI[n]{(\LHI[n]{\RHI[n]{F}})}
\to \RHI[n]{F}$ is an isomorphism. It follows that $\tlHI{n}F = 
0$ as desired.

\pfitem{$\tlHI{n}$ is left adjoint to inclusion} : the proof
of this claim will rely on the following lemma:

\begin{lem}\label{lem_tlHI_id}
The functor $\tlHI{n}$, restricted to $\tlHI{n}\HI$ is the 
identity. Consequently, the functor $\tlHI{n}$ is idempotent. 
That is, $(\tlHI{n})^2 = \tlHI{n}$ (see Def. 
\ref{def_coradical}(2)).
\end{lem}
\begin{proof}[Proof of Lemma]
For $F \in \tlHI{n}\HI$, we have
\[
\LHI[n]{\RHI[n]{F}} \to F \to \tlHI{n}F \to 0
\]
However, since $F \in \tlHI{n}\HI$, $\RHI[n]{(\tlHI{n}F)} = 0$, and 
therefore counit map is $0$. It follows that $\tlHI{n}F = F$ as 
desired.

The second statement follows from the fact that $\tlHI{n}F \in 
\tlHI{n}\HI$.
\end{proof}

Continuing with the proof of Prop. \ref{prop_HI_lower_slice}, let
$F \in \HI$ and $G \in \tlHI{n}\HI$. For all $f: F \to G$ we have 
the following commutative diagram:
\[
\begin{tikzcd}
F \arrow{r}{\pi} \arrow{d}{f}
& \tlHI{n}F \arrow{r} \arrow{d}{\tlHI{n}f}
& 0 \\
G \arrow{r}{\pi'}
& \tlHI{n}G \arrow{r}
& 0
\end{tikzcd}
\]
By Lemma \ref{lem_tlHI_id}, since $G \in \tlHI{n}\HI$, the map
$G \stackrel{\pi'}{\to} \tlHI{n}G$ is the identity. Define
\[
\chi : \homHI(F, G) \to \hom_{\tlHI{n}\HI}(\tlHI{n}F, G)
\]
by $f \mapsto \tlHI{n}f$. If $\tlHI{n}f = 0$, then $f = 0$.
Therefore $\chi$ is injective. For $g: \tlHI{n}F \to G$, then
set $f' = \pi \comp g$. It is easy to see that $\chi(f') = g$.
Thus, $\chi$ is a bijection, as desired.
\end{proof}

We now relate the weak filtration $(\HI, \sgHI{*})$ and the strong 
cofiltration $(\HI, \tlHI{*})$ to the slice filtration on 
$\DMeff$. Recall that the slice filtration structure on $\DMeff$ 
is associated with the (weak) filtration $(\DMeff, \sgDM{*})$ and 
the (weak) cofiltration $(\DMeff, \slDM{*})$ (see Sect. 
\ref{sect_slice_filt_dm}).

It is clear from Prop. \ref{prop_LRDM_eq_LRHI} that the essential 
image of $\sgDM{n}\DMeff$ under $\sheaf{H}^0$ is $\sgHI{n}\HI$. 
However, showing that the essential image $\slDM{n}\DMeff$ under 
$\sheaf{H}^0$ is $\tlHI{n}\HI$ requires more proof. 
First, observe that by Prop. \ref{prop_DM_slice_triangle} for 
every $n > 0$ and every $F \in \HI$ (regarded as an object of 
$\DMeff$), there exists a slice triangle:
\[
\sliceTriangle{n}{F}
\]
Applying the cohomologicaly functor $\sheaf{H}^0$, we obtain the
following long exact sequence
\[
\cdots \stackrel{\delta_{-1}}{\to} \sheaf{H}^0 \sgDM{n} F \to 
   \sheaf{H}^0 F \to \sheaf{H}^0 \slDM{n} F
   \stackrel{\delta_0}{\to} \sheaf{H}^1 \sgDM{n} F \to \cdots
\]
where $\sheaf{H}^i F \defeq \sheaf{H}^0F[i]$. In particular, we
have the following exact sequence
\begin{equation}\label{eq_slice_DM_exact_seq}
\sheaf{H}^0 \slDM{n} F \to \sheaf{H}^0 F \to \sheaf{H}^0 
\sgDM{n} F \stackrel{\delta_0}{\to} \sheaf{H}^1 \sgDM{n} F.
\end{equation}
Notice that $\sheaf{H}^0F = F$ and $\sgDM{n} F = \ihomDMf(\Z(1)[1], 
F)(1)[1]$. So by Prop. \ref{prop_LRDM_eq_LRHI}, $\sheaf{H}^0 
\sgDM{n}F = \LHI[n]{(\RHI[n]{F})}$. Therefore, the exact sequence
in Eq. \ref{eq_slice_DM_exact_seq} reduces to the following
\[
\LHI[n]{(\RHI[n]{F})} \to F \to \sheaf{H}^0{\slDM{n} F} 
   \stackrel{\delta_0}{\to} \sheaf{H}^1 \sgDM{n} F.
\]
where the map $\LHI[n]{(\RHI[n]{F}} \to F$ is the counit. If we 
show that $\sheaf{H}^1 \sgDM{n} F = 0$, then it is clear that
$\sheaf{H}^0 \slDM{n} F \simeq \tlHI{n} F$. The claim that 
$\sheaf{H}^1 \sgDM{n}F = 0$ is precisely the content of Lemma 
\ref{lem_H1_sgDM_vanishes}, which we prove shortly. First, we have
the following.

\begin{lem}[D\'eglise]\label{lem_rhomDM_and_contract}
For $F \in \HI$,
\[
\ihomDMf(\Z(1)[1], F) \simeq \RHI{F}
\]
as objects in $\DMeff$.
\end{lem}
\begin{proof}
Since $\sheaf{H}^0 \ihomDMf(\Z(1)[1], F) \simeq \RHI{F}$ by 
\ref{prop_LRDM_eq_LRHI}, it suffices to show that 
$\ihomDMf(\Z(1)[1], F)$, now regarded as a chain complex of
sheafs with transfers, is quasi-isomorphic to a chain complex 
concentrated in degree 0, the Nisnevich sheaf
\[
\sheaf{H}^i\ihomDMf(\Z(1)[1], F)
\]
vanishes for all Hensel local schemes.

Fix $S$ a Hensel local scheme. Note that, since $F$ is $\A^1$-local,
\begin{align*}
\sheaf{H}^i\ihomDMf(\Z(1)[1], F)(S) &= H^i\rhomDMf(\CZtr(S) 
   \tDM \Z(1)[1], F) \\
   &\simeq H^i\homDSh(\Ztr(S \times 
   \Gm), F).
\end{align*}
Furthermore, there exists a split exact triangle
\[
\Ztr(S \times \A^1) \to \Ztr(S \times (\A^1 - 0)) \to 
   \Ztr(S \times \Gm) \to \Ztr(S \times \A^1)[1],
\]
in $\DSh$ and upon applying $\homDSh(-, F)$, we have a long exact 
sequence in cohomology:
\begin{align*}
\cdots & \rightarrow H^i\homDSh(\Ztr(S \times \A^1), F) 
   \rightarrow H^i\homDSh(\Ztr(S \times (\A^1 - 0)), F) \\
 & \rightarrow H^i\homDSh(\Ztr(S \times \Gm), F) \rightarrow 
   H^i\homDSh(\Ztr(S \times \A^1), F) \rightarrow \cdots
\end{align*}
Note that $H^i\homDSh(\Ztr(X), F) = H_{\Nis}^i(X; F)$. Since
$H^i_{\Nis}(X; F) = 0$ for $i < 0$, and for all $i > 0$, 
\[
H_{\Nis}^i(S \times \A^1; F) = H_{\Nis}^i(S; F) = 0
\] 
and 
\[
H_{\Nis}^i(S \times (\A^1 - 0); F) = 0
\] 
(\cite{MVW} Cor. 24.5), it follows that 
\[
H^i\homDSh(\Ztr(S \times \Gm), F) = 0 \quad\textrm{for all $i \neq -1, 0$.}
\] 
Thus, we are reduced to showing that $H^{-1}\homDSh(\Ztr(S \times 
\Gm), F) = 0$. However, the map $F(S \times \A^1) = F(S) \to F(S 
\times (\A^1 - 0))$ is a split injection, and the lemma follows.
\end{proof}

\begin{lem}[D\'eglise]\label{lem_H_com_ihom_DM}
For $M \in \DMeff$, 
\[
\sheaf{H}^i\ihomDMf(\Z(1)[1], M) = \ihomDMf(\Z(1)[1], \sheaf{H}^i(M)) = 
   \RHI{\sheaf{H}^i(M)}
\]
\end{lem}
\begin{proof}
First, the cohomological functor $\ihomDMf(\Z(1)[1], -)$ is left 
$t$-exact (being the right adjoint of $-\tDM \Z(1)[1]$. By the 
preceding lemma and Prop \ref{prop_contract_is_exact},
$\ihomDMf(\Z(1)[1], -)$ is also exact on the heart of $\DMeff$,
whence it is $t$-exact.

That is, $\ihomDMf(\Z(1)[1], -)$ commutes with $\sheaf{H}^0$,
and the lemma is established.
\end{proof}

\begin{lem}\label{lem_H1_sgDM_vanishes}
For $F \in \HI$, $\sheaf{H}^1 \sgDM{n} F = 0$.
\end{lem}
\begin{proof}
Applying Lemma \ref{lem_rhomDM_and_contract}, we see that the
complex
\[
\ihomDMf(\Z(n)[n], F) \tDM \Z(n)[n] \simeq \RHI{F} \tDM \Z(n)[n]
\] 
is concentrated entirely in negative degrees (see Def. \ref{def_z_n}).
It follows that $\sheaf{H}^i$ vanishes for $i > 0$, and in particular,
for $i = 1$.
\end{proof}

This shows that $\tlHI{n}\HI$ is contained in the essential image 
of $\sgDM{n}\DMeff$ under $\sheaf{H}^0$. To show the converse, we 
establish that $\RHI[n]{(\sheaf{H}^0 M)} = 0$ for $M \in 
\slDM{n}\DMeff$. This follows from Prop. 
\ref{prop_lower_slice_char}. Indeed, if $M \in \slDM{n}\DMeff$, 
then 
\[
\ihomDMf(\Z(n)[n], M) = 0.
\] 
Applying Lemma \ref{lem_H_com_ihom_DM}, it follows that
\[
0 = \sheaf{H}^0\ihomDMf(\Z(n)[n], M) = \RHI[n]{(\sheaf{H}^0(M))}
\]
and thus $\sheaf{H}^0 M \in \tlHI{n}\HI$ by definition. We
have just proved:

\begin{prop}\label{prop_H_commute_with_filt}
Let $(\HI, \sgHI{*})$ and $(\HI, \tlHI{*})$ be the respective 
filtration and cofiltration defined above, and let $(\DMeff, 
\sgDM{*})$ and $(\DMeff, \slDM{*})$ be the filtration and
cofiltration defined in Sect. \ref{sect_slice_filt_dm}. Then
for each $n > 0$, the following diagram of functors commute:
\[
\begin{tikzcd}
\sgDM{n}\DMeff \arrow{d}{\sheaf{H}^0} &
\DMeff \arrow{l}{\sgDM{n}} \arrow{r}{\slDM{n}} \arrow{d}{\sheaf{H}^0} &
\slDM{n}\DMeff \arrow{d}{\sheaf{H}^0} \\
\sgHI{n}\HI &
\HI \arrow{l}{\sgHI{n}} \arrow{r}{\tlHI{n}} &
\tlHI{n}\HI 
\end{tikzcd}
\]
and all vertical arrows are essentially surjective.
\end{prop}

As in the case of $\DMeff$, we can also define the structure 
invariants associated to the filtration and cofiltration. In this
case, for every $n$, there exists a functorial exact sequence
\[
\sgHI{n} \to \sgHI{n - 1} \to \tlHI{n}\sgHI{n - 1} \to 0.
\]
\begin{defn}
We define \DEF{the structure constants on $\HI$} or simply 
to be the functors $\slice{n} \defeq \tlHI{n + 1}\sgHI{n}$. In
this case we call $\slice{n}$ the $n$-th structure constant.
\end{defn}

Recall the definition of the structure constants of $(\DMeff, 
\sgDM{*})$ to be the triangulated endofunctor $\sliceDM{*}$ that
fits into the following exact triangle
\[
\slDM{n} \to \slDM{n - 1} \to \sliceDM{n - 1} \to \slDM{n}[1].
\]
A direct consequence of Prop. \ref{prop_H_commute_with_filt} is 
that the structure constants of $(\HI, \sgHI{*})$ agree with the 
structure constants of $(\DMeff, \sgDM{*})$:

\begin{cor}\label{cor_H_commute_with_slice}
Let the structure constants $\slice{*}$ for $(\HI, \sgHI{*})$ be 
defined as above. Then
\[
\sheaf{H}^0 \sliceDM{n} = \slice{n}.
\]
\end{cor}

Before we proceed, let us record for subsequent results the 
following proposition.

\begin{prop}\label{prop_sg_tl_commute}
For each $m, n \geq 0$, $\sgHI{n} \tlHI{m}$ is naturally
isomorphic to $\tlHI{m} \sgHI{n}$, and are both 0 if $m \leq n$.
\end{prop}
\begin{proof}
For $F \in \HI$, notice that by Prop. \ref{prop_unit_iso}, 
$\LHI[n]{\RHI[n]{(\LHI[m]{\RHI[m]{F}})}} \simeq
\LHI[m]{\RHI[m]{(\LHI[n]{\RHI[n]{F}})}}$. Furthermore, the 
isomorphism fits into a commutative square
\begin{equation}\label{eq_prop_sg_tl_com_sq}
\begin{tikzcd}
\sgHI{m}\sgHI{n} F \arrow{r}{f} \arrow{d}{\simeq} & 
\sgHI{n} F \arrow[equal]{d} \\
\sgHI{n}\sgHI{m} F \arrow{r}{g} &
\sgHI{n} F,
\end{tikzcd}
\end{equation}
where, the map $f$ is counit of $\sgHI{n} F$, and the map $g$
is obtained from applying $\sgHI{n}$ to the counit $\sgHI{m} F 
\to F$.

The cokernel of $f$ is precisely $\tlHI{m} \sgHI{n} F$, and we 
claim that the cokernel of the map on the bottom row map is 
$\sgHI{n} \tlHI{m} F$. Indeed, $\sgHI{n}$ is right exact. Applying 
$\sgHI{n}$ to the exact sequence
\[
\sgHI{m} F \to F \to \tlHI{m} F \to 0,
\]
we have
\[
\sgHI{n} \sgHI{m} F \to \sgHI{n} F \to \sgHI{n} \tlHI{m} F \to 0.
\]
It is clear that $\tlHI{m} \sgHI{n} F \simeq \sgHI{n} \tlHI{m} F$.
Since the square in Display \ref{eq_prop_sg_tl_com_sq} is 
functorial, it follows that the isomorphism identified above is
natural in $F$.

To conclude, suppose $m \leq n$. Then $\RHI[n]{(\tlHI{m} F)} = 0$
(Prop \ref{prop_HI_lower_slice}). It follows that $\sgHI{n} 
\tlHI{m} F = 0$, and so is true of $\tlHI{m} \sgHI{n} F$.
\end{proof}

% and we can talk more about the relationship between the slices

In spite of the results of Prop. \ref{prop_H_commute_with_filt} 
and Cor. \ref{cor_H_commute_with_slice}, $(\HI, \sgHI{*})$ does 
not define a strong filtration of $\HI$. The problem here is 
that the counit is not injective in general, as demonstrated by 
the following example: 

\begin{ex}\label{ex_Oxn}
Let $(\Ox)^n$ be the sheaf of $n$-th power of global units 
associated to the presheaf where sections of $X \in \Cor$ is the 
abelian subgroup of $\Ox$ given by 
\[
\{x \in \Ox(X) : x=y^n\textrm{ for some }y \in \Ox(X)\}.
\]
It is clear that $(\Ox)^n \in \HI$. Furthermore, there exists the 
following exact sequence 
\[
0 \to \mu_n \to \Ox \to (\Ox)^n \to 0
\]
where $\mu_n$ is the constant sheaf of $n$-th roots of unity.
In particular, $\RHI{(\mu_n)} = 0$. Therefore, we have
\[
0 \to \RHI{\Ox} \to \RHI{(\Ox)^{n}} \to 0.
\]
In particular, $\LHI{\RHI{(\Ox)^n}} \simeq \LHI{\RHI{\Ox}} = \Ox$, 
and the counit $\LHI{\RHI{(\Ox)^n}} \to (\Ox)^n$ is precisely $x 
\mapsto x^n$, which has a nontrivial kernel.
\end{ex}
\vskip 10pt
We can understand the problem in another way, which is that the 
categories $\sgHI{*}\HI$ are too small and do not include all
the kernels of counits $\LHI[n]{(\RHI[n]{F})} \to F$. This can be 
fixed by enlarging the filtration at each level, and to do so, we 
turn to torsion theory.

Recall from Thm. \ref{thm_precorad_eq_tt} that there is a 
one-to-one correspondence between idempotent pre-coradicals and 
cohereditary torsion theories on a well-powered abelian category 
(See Sect.  \ref{sect_torsion_theory}).

The goal is to show that we have a sequence of coradicals, and
that the associated torsion theory $(\TCat{n}, \TFCat{n})$ has
the properties that for each $n > 0$, and that the torsion 
subcategories fits into descending tower of inclusion of full 
subcategories:
\[
\HI = \TCat{0} \supset \TCat{1} \supset \cdots \supset \TCat{n} 
   \supset \TCat{n + 1} \supset \cdots
%\cdots \into \TCat{i} \into \TCat{i-1} \into \TCat{i-2} \into 
%   \cdots \into \TCat{0} = \HI,
\]
and the torsionfree subcategories fits into an ascending tower
of inclusion of full subcategories
\[
0 = \TFCat{0} \subset \TFCat{1} \subset \cdots \subset \TFCat{n}
   \subset \TCat{n + 1} \subset \cdots \subset \HI
\]
In this case, the coreflection and reflection will be given by
the radical and coradical respectively. In fact, we have a
natural set of candidates for the coradicals:

\begin{prop}
The functors $\tlHI{n}$ are coradicals for each $n > 0$.
\end{prop}
\begin{proof}
By Lemma \ref{lem_tlHI_id}, $\tlHI{n}$ is idempotent. By Prop. 
\ref{prop_HI_lower_slice}, $\tlHI{n}$ is a left adjoint, and
is therefore right exact. All that remains to show is that for
each $F \in \HI$,
\[
\tlHI{n}(\ker (F \to \tlHI{n} F)) = 0.
\]
Fix an $F \in \HI$ and $n > 0$, and let $K$ denote the kernel of 
the surjection $F \to \tlHI{n}F \to 0$. We have the following
short exact sequence 
\[
0 \to K \to F \to \tlHI{n} F \to 0
\]
Since $\tlHI{n} F \in \tlHI{n}\HI$, by definition 
$\RHI[n]{(\tlHI{n} F)} = 0$, and therefore, we have the following
commutative diagram
\[
\begin{tikzcd}
{} &\LHI[n]{(\RHI[n]{K})} \arrow{d}{\cuHI_F} \arrow{r}
   &\LHI[n]{(\RHI[n]{F})} \arrow{d}{\cuHI_F} \arrow{r}
   &0 \arrow{d} \arrow{r}
   &0 \\
0 \arrow{r} &
  K \arrow{r}&
  F \arrow{r}&
  \tlHI{n}F \arrow{r}&
  0
\end{tikzcd}
\]
By the Snake Lemma, and using the fact that $\cok \cuHI_F = 
\tlHI{n} F$ for $F \in \HI$, we have the exact sequence
\[
0 \to \tlHI{n}K \to \tlHI{n}F \to \tlHI{n} F \to 0.
\]
And the map $\tlHI{n}F \to \tlHI{n}$ is the identity. It follows 
that $\tlHI{n} K = 0$ as desired.
\end{proof}

The following is a straightforward consequence of the preceding 
proposition and Thm. \ref{thm_precorad_eq_tt}, and Cor. 
\ref{cor_tt_ref_and_coref}:

\begin{cor}
Let the functor $\tlHI{n}$ be given as above.

For each $n \geq 0$, there exists a torsion pair
\[
(\tgHI{n}\HI, \tlHI{n}\HI)
\]
where $\tgHI{n}\HI$ is the full subcategory of objects $F \in \HI$
for which $\tlHI{n} F = 0$, and $\tlHI{n}\HI$ is the full 
subcategory of $F \in \HI$ such that $\tlHI{n} F = F$. Furthermore, 
$\tlHI{n} : \HI \to \tlHI{n}\HI$ is a reflection functor for the
inclusion of $\tlHI{n} \HI$ into $\HI$.
\end{cor}

\begin{rmk}
Once again, the astute reader may take issue with our using 
$\tlHI{n} \HI$ to represent the torsion-free subcategory of $\HI$ 
associated to $\tlHI{n}$, since we have also used $\tlHI{n} \HI$ 
to represent the full subcategory of $\HI$ consisting of those 
objects $F$ for which $\RHI[n]{F} = 0$.

In fact, there is no ambiguities here. We argue that these two 
subcategories are the same. This is precisely the content of the
first part of the next proposition.
\end{rmk}

\begin{prop}\label{prop_TFHI_properties}
For each $n$, let $(\THI{n}, \TFHI{n})$ and $\tlHI{n}$ be given
as above. Then we the following are true:

\begin{enumerate}
\item for all $n \geq 0$, and each $F$ in the torsion category 
$\TFHI{n}$, $\RHI[n] F = 0$. Conversely, if $\RHI[n] F = 0$, then
$F \in \TFHI{n}$.
\tinyskip

\item for $m > n$, $\TFHI{n}$ is a full subcategory of $\TFHI{m}$.
\tinyskip

\item for $m > n$, $\tlHI{n}\tlHI{m} = \tlHI{m}\tlHI{n} = \tlHI{n}$.
\tinyskip
\end{enumerate}
\end{prop}
\begin{proof}
\pfitem{(1)} : Since $\RHI[n]{(\tlHI{n} F)} = 0$ for all $F \in 
\HI$ (see Prop. \ref{prop_HI_lower_slice}), then for all $F \in 
\TFHI{n}$, $\RHI[n]{F} = \RHI[n]{(\tlHI{n} F)} = 0$.

Conversely, if $\RHI[n]{F} = 0$, then $\tlHI{n} F = F$ by Lemma
\ref{lem_tlHI_id}. That is, $F \in \TFHI{n}$. In particular, the
torsion-free subcategory $\TFHI{n}$ is precisely the full 
subcategory of $F \in \HI$ for which $\RHI[n]{F} = 0$.

\pfitem{(2)} : Suppose $m > n > 0$, and $F \in \TFHI{n}$, then
$\RHI[m]{F} = 0$. By part (1), $F \in \TFHI{m}$.

\pfitem{(3)} : Suppose, as in (2), that $m > n > 0$.

By part (2) and the fact that $\tlHI{m}$ is the identity on 
$\TFHI{m}$ (Lemma \ref{lem_tlHI_id}), it is clear that 
$\tlHI{m}\tlHI{n} = \tlHI{n}$. It remains to show that 
$\tlHI{n}\tlHI{m} = \tlHI{n}$.

Fix $F \in \HI$. We have the following commutative diagram:
\[
\begin{tikzcd}
\sgHI{n}(\sgHI{m} F) \arrow{r}\arrow{d}{\cuHI_{\sgHI{m} F}} &
\sgHI{n} F \arrow{r}\arrow{d}{\cuHI_F} &
\sgHI{n}(\tlHI{m} F) \arrow{r}\arrow{d}{\cuHI_{\tlHI{m}F}} &
0 \\
\sgHI{m} F \arrow{r} &
F \arrow{r} &
\tlHI{m} F \arrow{r}&
0,
\end{tikzcd}
\]
where $\sgHI{n}$ denotes the functor $F \mapsto 
\LHI[n]{\RHI[n]{F}}$, and likewise for $\sgHI{m}$. Notice that the 
vertical arrows are the counits, which we have labelled 
$\cuHI_{\sgHI{m}F}$, $\cuHI_F$, and $\cuHI_{\tlHI{m}F}$. 
Furthermore, by the same arguments as in the Snake Lemma, we have 
the ``snake tail'' exact sequence:
\[
\cok \cuHI_{\sgHI{m}F} \to \tlHI{n}F \to \tlHI{n}\tlHI{m}F \to 0.
\]
However, by Prop. \ref{prop_unit_iso} $\cuHI_{\sgHI{m} F}$ is an 
isomorphism, whence $\tlHI{n}F \simeq \tlHI{n}\tlHI{m} F$. 
Equality follows from the fact that both are quotients by the 
image of $\sgHI{n} F$ in $F$.
\end{proof}

It is clear that $\TFHI{*}$ define the same strong cofiltration as
defined previously. 

We end this section with a more complete description of the 
torsion subcategories, $\THI{*}$, its associated coreflections, and
the structure invariants. The following are a few properties that 
are straight-forward to verify:

\begin{prop} 
Let $\tgHI{n}$ denote the functor that sends $F$ to the kernel of
the surjection $F \to \tlHI{n}F$. Then the following are true:
\begin{enumerate}
\item $\tgHI{n}$ is an idempotent pre-radical.
\tinyskip

\item the essential image of $\tgHI{n}$ is $\THI{n}$, the 
restriction to which $\tgHI{n}$ is the identity.
\tinyskip

\item $\tgHI{n}$ is right adjoint to the inclusion $\THI{n} \into 
\HI$.
\tinyskip
\end{enumerate}
\end{prop}
\begin{proof}
Statement (1) follows from Prop. \ref{prop_rad_eq_corad} and
the fact that $\tlHI{n}$ is a coradical (and thus an idempotent 
pre-coradical). To see that $\tgHI{n}$
acts as the identity on $\THI{n}$, note that for any $F \in \THI{n}$
$\tlHI{n} F = 0$. 

Finally, for $F \in \HI$, we have the short exact sequence
\[
0 (\tgHI{n})^2 F \to \tgHI{n}F \to \tlHI{n} \tgHI{n}F \to 0.
\]
Since $\tgHI{n}$ is a radical, $\tlHI{n} \tgHI{n}F = 0$. Therefore,
$\tgHI{n} \in \THI{n}$ as desired.

Finally, (3) follows from Cor. \ref{cor_tt_ref_and_coref}.
\end{proof}

\begin{prop}\label{prop_THI_properties}
Let $\THI{n}$, $n = 0, 1,\dots$ be defined as above. Furthermore,

\begin{enumerate}
\item for all $n \geq 0$, $\THI{n}$ is the full subcategory of
objects $F \in \HI$ such that $\sgHI{n}F \to F$ is surjective.
\tinyskip

\item for $m > n > 0$, $\THI{m} \subset \THI{n}$.
\tinyskip

\item for $n > 0$, $\sgHI{n} \HI$ is a proper full subcategory 
$\THI{n}$.
\tinyskip

\item for $m \geq n \geq 0$, $\tgHI{n}\tgHI{m} = \tgHI{m}\tgHI{n} =
\tgHI{m}$.
\tinyskip

\item for $m, n > 0$, $\tlHI{n}\tgHI{m}$ is naturally isomorphic 
to $\tgHI{m}\tlHI{n}$, and is the (constant) $0$ functor if $m > n$.
\tinyskip

\end{enumerate}
\end{prop}

\begin{proof}
\pfitem{(1)} : For all $F \in \HI$ and $n \geq 0$, we have the 
following exact sequence
\[
\sgHI{n} F \to F \to \tlHI{n} F \to 0.
\]
It is clear that $\tlHI{n} = 0$ if and only if $\sgHI{n} F \to F$
is a surjection.

\pfitem{(2)} : Let $m > n > 0$, and fix $F \in \THI{m}$. Then 
$\tlHI{m} F = 0$, and by Prop. \ref{prop_TFHI_properties} (3)
\[
0 = \tlHI{n}\tlHI{m} F = \tlHI{n} F.
\]
Thus $F \in \THI{m}$.

\pfitem{(3)} : Let $F \in \sgHI{n} \HI$, then $F = \LHI[n]{F'}$ 
for some $F' \in \HI$. By Prop. \ref{prop_counit_iso_for_HIn},
the counit map is an isomorphism. By part (1), $F \in \THI{n}$.

\pfitem{(4)} : By part (2) of the preceding proposition and (2), 
it is clear that $\tgHI{n} \tgHI{m} = \tgHI{m}$. For the other
part, it is precisely the dual of the argument in Prop. 
\ref{prop_TFHI_properties} part (3).

\pfitem{(5)} : Let $F \in \HI$. In the case where $m \geq n$,
the claim follows since $\tgHI{m} F \in \THI{n}$, and by 
definition $\tlHI{n} \tgHI{m} F = 0$ for all $n \leq m$. 
Furthermore, $\tgHI{m} F = 0$ since it is the kernel of the 
$\tlHI{m} \tlHI{n} F \to \tlHI{n} F$ which is the identity map
by part (3) of the preceding proposition.

Now, consider the case $n > m$. We have the following 
commutative diagram:
\[
\begin{tikzcd}
{} &
\sgHI{n} \tgHI{m} F \arrow{r} \arrow{d}{\cuHI_{\tgHI{m}F}} &
\sgHI{n} F \arrow{r} \arrow{d}{\cuHI_{F}} &
\sgHI{n} \tlHI{m} F \arrow{r} \arrow{d}{\cuHI_{\tlHI{m}F}} &
0 \\
0 \arrow{r} &
\tgHI{m} F \arrow{r} &
F \arrow{r} &
\tlHI{m} F \arrow{r}&
0
\end{tikzcd}
\]
where vertical maps are the counits, labeled with their respective
targets. Notice that the top row is exact on the right because 
$\sgHI{n}$ is right exact. This follows from the fact that the 
functor $F \mapsto \LHI[n]{F}$ is a left adjoint, and $F \mapsto 
\RHI[n]{F}$ is exact (Prop. \ref{prop_contract_is_exact}), and 
$\sgHI{n}$ is their composition.

Since $n > m$, since $\RHI[n](\tlHI{m} F) = 0$ (Prop. 
\ref{prop_HI_lower_slice}) $\sgHI{n}\tlHI{m} F = 0$. By the Snake
Lemma, we have the following exact sequence
\[
0 \to \tlHI{n}\tgHI{m} F \to \tlHI{n} F \to \tlHI{n} \tlHI{m} F 
   \to 0.
\]
Notice that $\tlHI{n} \tlHI{m} F = \tlHI{m} F$, and the map from
$\tlHI{n} F \to \tlHI{m} F$ is precisely the unit map associated
to the functor $\tlHI{m}$. It follows that
\[
\tlHI{n} \tgHI{m} F \simeq \tgHI{m} \tlHI{n} F.
\]
Naturality follows from the construction of the isomorphism.
\end{proof}

\begin{rmk}
In case the indexing becomes difficult to keep track, one might
wish to consider a ``bread'' analogy. Imagine that a half-infinite 
loaf of bread is laid out on a line marked from 0 to $\infty$ 
(representing an $F \in \HI$), and one is allowed to take cuts at 
the marked points, and subsequently pick up all the bread lying 
greater than $n$ or less than $n$. The former action describes the 
functors $\tgHI{n}$ and $\sgHI{n}$, and therefore, the higher the 
$n$, the \emph{less} bread one would take; the latter, functors 
$\tlHI{n}$, wherein the greater the $n$, the \emph{more} bread one 
would takes.

If one finds the analogy useful, one might wish to interpret
Prop. \ref{prop_TFHI_properties} (3) and \ref{prop_THI_properties} (4)
and (5) with this culinary picture in mind.
\end{rmk}

As we did for the filtration $(\HI, \sgHI{*})$, we can define the 
structure invariants for $(\HI, \tgHI{*})$. 
\begin{defn}
For each $F \in \HI$ and $n \geq 0$, write $\tconst{n}$ for the functor 
$\tgHI{n + 1} \tgHI{n}$, which we hereafter refer to as the 
\DEF{$n$-th structure invariant of $F$ associated to $\tgHI{*}$}. 
\end{defn}

As it turns out, this is \emph{not} the same as the structure 
invariants of $\sgHI{*}$, though they are related in the following:

\begin{prop}\label{prop_struct_consts}
Let $m > n > 0$. Then there exists a natural surjection from 
$\tlHI{m} \sgHI{n}$ to $\tlHI{m} \tgHI{n}$. In particular, for 
each $F \in \HI$, there exists a surjection $\pi_m: \slice{m} F \to 
\tconst{m} F$.
\end{prop}
\begin{proof}
Let $F \in \HI$, and notice that
\[
0 \to \tgHI{n} \tlHI{m} F \to \tlHI{m} F \to 
  \tlHI{n} \tlHI{m} F \to 0.
\]
By Prop. \ref{prop_TFHI_properties} part (3), $\tlHI{m} \tlHI{n} F 
= \tlHI{n} F$, and notice that $\tgHI{n} \tlHI{m} F$ is the kernel 
of the surjection $\tlHI{m} F \to \tlHI{n} F$.

But the sequence
\[
\sgHI{n} \tlHI{m} F \to \tlHI{m} F \to \tlHI{n} F \to 0
\]
is exact. Therefore, the induced map from $\sgHI{n} \tlHI{m} F$
to $\tgHI{n} \tlHI{m} F$ is a surjection as well. Furthermore,
since the commutative diagram
\[
\begin{tikzcd}
{} & \sgHI{n} \tlHI{m} F \arrow{r} \arrow[twoheadrightarrow]{d} &
\tlHI{m} F \arrow{r} \arrow[equal]{d} &
\tlHI{n} F \arrow{r} \arrow[equal]{d} &
0 \\
0 \arrow{r} &
\tgHI{n} \tlHI{m} F \arrow{r} &
\tlHI{m} F \arrow{r} &
\tlHI{n} F \arrow{r} &
0
\end{tikzcd}
\] 
is functorial in $F$, the surjection is natural.

To prove the first claim of the proposition, notice that 
$\tgHI{n} \tlHI{m}$ is naturally isomorphic to $\tlHI{m} \tgHI{n}$ 
(Prop. \ref{prop_THI_properties}) and $\sgHI{n} \tlHI{m}$ is 
naturally isomorphic to $\tlHI{m} \sgHI{n}$ (Prop 
\ref{prop_sg_tl_commute}).

The second claim follows from setting $n = m - 1$.
\end{proof}

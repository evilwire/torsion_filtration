\newpage
\section{Filtration on $\HI$}\label{sect_filtration_hi}

Recall that a weak filtration of a category $\Cat{C}$ is a tower 
of subcategories
\[
\cdots \subset F_n\Cat{C} \subset F_{n - 1}\Cat{C} \subset \cdots 
   \subset \Cat{C}
\]
together with reflection functors $\nu_n : \Cat{C} \to F_n\Cat{C}_n$
(cf. \ref{def_cat_filtration}.) If we have a suitable notion of
subobjects in $\Cat{C}$, e.g. if $\Cat{C}$ is abelian, we can 
define a stronger notion (as the name suggests).

For the remainder, let $\Cat{C}$ be an abelian category. We have:

\begin{defn}
We say that a filtration $(\Cat{C}, \nu_*)$ is a \DEF{strong 
filtration} if for each $A \in \Cat{C}$ and $n \in \Z$, $f_n A$ is 
a subobject of $A$. Similarly a cofiltration $(\Cat{C}, \gamma_*)$ 
is a \DEF{strong cofiltration} if $c_n A$ is a quotient of $A$ for 
each $n$.
\end{defn}

The purpose of this section and the next is to construct a 
strong filtration on the category $\CycMod$. To do this, we first
construct a weak filtration structure on $\HI$ using a pair
of adjoint functors obtained from the closed symmetric monoidal 
structure of $\HI$. We adopt the same convention as in Sections
\ref{sect_heart_struct} and \ref{sect_slice_filt_dm}: we identify 
$\DMeff$ with the full subcategory of $\DSh$ with homotopy 
invariant cohomologies, and identify $\HI$ with the heart of 
$\DMeff$ under the $t$-structure induced by that of $\DSh$. Once
again, let $\sheaf{H}^0$ denote the reflection functor $\DMeff
\to \HI$.

Recall that the sheaf $\Ox$ of units given by 
\[
\Ox(X) = \{\textrm{invertible elements of }\O(X)\}
\]
is a homotopy invariant sheaf with transfers. For $F \in \HI$, 
let
\[
\LHI{F} \defeq F \tHI \Ox \hspace{10pt} \textrm{and}\hspace{10pt} 
   \RHI{F} \defeq \ihomHI(\Ox, F).  
\]
We write $\LHI[n]{F}$ for $\LHI{(\LHI[n - 1]{F})}$ and 
$\RHI[n]{F}$ for $\RHI{(\RHI[n + 1]{F})}$.

\begin{rmk}\label{rmk_contract_rhi_eq}
In literature, the notation $\RHI{F}$ is often used to represent
the contraction of $F$, which is the sheaf that sends $X \in 
\Cor$ to 
\[
\cok( F(X \times \A^1) \to F(X \times (\A^1 - 0))).
\]
In fact, there is no ambiguity here, since the contraction of
$F$ is isomorphic to $\ihomHI(\Ox, F)$ by Prop. 
\ref{prop_contraction}.
\end{rmk}

The following proposition highlights the connection between the
pair of adjoint functors in the construction of the slice 
filtration on $\DMeff$ (cf.  Section \ref{sect_slice_filt_dm}).

\begin{prop}\label{prop_LRDM_eq_LRHI}
Fix any $F \in \HI$, which we also consider as an object of 
$\DMeff$. Then we have $\sheaf{H}^0(F \tDM \Z(1)[1]) \simeq 
\LHI{F}$ and $\sheaf{H}^0\ihomDMf(\Z(1)[1], F) \simeq \RHI{F}$.
\end{prop}
\begin{proof}
By Prop. \ref{prop_Z1_eq_Ox}, $\Z(1)[1] \simeq \Ox$ in $\DSh$
and hence in $\DMeff$ as well. The isomorphisms now follow
from the definitions of $\tHI$ and $\ihomHI$.
\end{proof}

As constructed, the functor $F \mapsto \LHI{F}$ is left adjoint to 
$F \mapsto \RHI{F}$. In this case,

\begin{prop}\label{prop_unit_iso}
Let $F \in \HI$. The unit map $F \to \RHI{(\LHI{F})}$ is an
isomorphism.
\end{prop}
\begin{proof}
Given $F \in \HI$, consider $F$ as an object in $\DMeff$. By
Cancellation Theorem (Thm. \ref{thm_dm_cancellation}), we have
that $\ihomDMf(\Z(1)[1], F(1)[1]) \simeq \ihomDMf(\Z, F) 
\simeq F$. Now apply $\sheaf{H}^0$ to this chain of isomorphisms,
and note that $\sheaf{H}^0(F) = F$. The proposition follows from 
Prop. \ref{prop_LRDM_eq_LRHI}.
\end{proof}

Fix $n > 0$, and consider the counit map $\cuHI_F^n: 
\LHI[n]{\RHI[n]{F}} \to F$, and write $\tlHI{n} F$ for the 
cokernel of $\cuHI_F$. First, we make the following observation:

\begin{prop}\label{prop_counit_iso_for_HIn}
Let $\cuHI^n$ denote the counit, described above. If $F \in 
\LHI[n]{\HI}$, then $\cuHI^n : \LHI[n]{\RHI[n]{F}} \to F$ is
an isomorphism.
\end{prop}
\begin{proof}
Suppose $F \in \LHI[n]{\HI}$. That is, $F = LHI[n]{F'}$ for some 
$F' \in \HI$. Writing $L$ for the functor $F \mapsto \LHI[n]F$, 
by counit-unit adjunction, the composition
\[
\LHI[n]{F'} \xrightarrow{L\eta} \LHI[n]{(\RHI[n]{(\LHI[n]{F'})})}
   \xrightarrow{\cuHI L} \LHI[n]{F'}
\]
is the identity, where $\eta_{F'}$ is the unit, and $\cuHI_{F'}$ 
is the counit maps. However, by Prop. \ref{prop_unit_iso}, 
$\eta_{F'}$ is an isomorphism, and so is $L\eta_{F'}$. It follows
that $\cuHI L$ is an isomorphism. But $\cuHI L$ is the counit
map for $LF' = \LHI[n]{F'} = F$, and the proposition follows.
\end{proof}

To see this, let $\LHI[0]{\HI} = \HI$ and let $\LHI[n]{\HI}$ 
denote the full subcategory of $F \in \HI$ where $F = \LHI[n]{F'}$ 
for some $F' \in \HI$. It is clear that if $m \geq n$, then 
$\LHI[m]{\HI} \subseteq \LHI[n]{\HI}$. In particular, we have a 
tower of subcategories
\[
\HI = \LHI[0]{\HI} \supset \LHI[1]{\HI} \supset \LHI[2]{\HI} 
\subset \cdots.
%\cdots \subset \LHI[2]{\HI} \subset \LHI[1]{\HI} \subset \LHI[0]{\HI} 
% = \HI 
\]
For the constant sheaf $\Z$, it is clear that $\RHI{\Z} = 0$. 
Then $\Z \in \HI$ but not in $\LHI{\HI}$. Indeed, if $\Z \in 
\LHI{\HI}$ then $\Z = \LHI{F'}$, but then $\RHI{\Z} = F'$ by Prop. 
\ref{prop_unit_iso}, forcing $\Z = 0$. Inductively, we see 
that $\LHI[n]{\HI} \neq \LHI[n + 1]{\HI}$ for all $n \in \N$.

\begin{prop}
Let $\sgHI{n}$ denote the functor $F \mapsto 
\LHI[n]{(\RHI[n]{F})}$. Then $\sgHI{n}$ is right adjoint to the 
inclusion of $\LHI[n]{\HI}$. In particular, $(\HI, \sgHI{n})_{n 
\in \N}$ defines a (nontrivial) weak filtration of $\HI$.
\end{prop}
\begin{proof}
Fix $F \in \LHI[n]{HI}$ and $G \in \HI$, and fix a map $f : 
F \to G$. By naturality of $\cuHI$, we have the following 
commutative diagram:
\[
\begin{tikzcd}
\LHI[n]{\RHI[n]{F}} \arrow{r}{\cuHI^n f}\arrow{d}{\cuHI^n_F} 
& \LHI[n]{\RHI[n]{G}} \arrow{d}{\cuHI^n_G} \\
F \arrow{r}{f}
& G.
\end{tikzcd}
\]
Since $F \in \LHI[n]{\HI}$, by Prop. \ref{prop_counit_iso_for_HIn} 
the counit map $\cuHI_F$ is an isomorphism.

Define $\chi: \homHI(F, G) \to \hom_{\LHI[n]{\HI}}(F, 
\LHI[n]{\RHI[n]{G}})$ by $f \mapsto (\cuHI^n_F)^{-1} \comp \cuHI^n 
f$. Since $\cuHI_F$ is an isomorphism, $\chi$ is injective. 
Moreover, given a map $g: F \to \LHI[n]{\RHI[n]{G}}$, set $f' = 
\cuHI_G \comp g$. Then $\chi(f') = g$. Hence $\chi$ is an 
isomorphism as desired.
\end{proof}

For $n > 0$, let $\tlHI{n}\HI$ be the full subcategories of 
objects $F \in \HI$ such that $\RHI[n]{F} = 0$ (where $\RHI[0]{F} 
\defeq F$). It is clear that we also have the following ascending 
tower of subcategories
\[
0 = \tlHI{0}\HI \subset \tlHI{1}\HI \subset \tlHI{2}\HI \subset 
   \cdots \subset \HI
\]
It is also clear that $\tlHI{n} \neq \tlHI{n + 1}$ since 
$\RHI{\Ox} = \Z$ and by Prop. \ref{prop_unit_iso} $\LHI[n]{\Ox} 
\in \tlHI{n + 1}\HI$ but not in $\tlHI{n}\HI$.

In fact, we have the following result:

\begin{prop}\label{prop_HI_lower_slice}
Fix a positive integer $n$ and let $\tlHI{n}F$ be cokernel of the 
counit $\cuHI^n_F: \LHI[n]{\RHI[n]{F}} \to F$. Then the 
association $F \mapsto \tlHI{n}F$ is a functor, and is left 
adjoint to the inclusion $\tlHI{n}\HI \to \HI$.

Thus, $(\HI, \tlHI{*})$ defines a strong cofiltration
of $\HI$.
\end{prop}
\begin{proof}
\pfitem{Functoriality} : on objects, $\tlHI{n}$ sends $F$ to the 
cokernel $\tlHI{n}F$ of the counit $\LHI[n]{\RHI[n]{F}} \to F$.

Let $f: F \to G$ be a morphism in $\HI$.
By naturality of $\cuHI^n_F$, we have the following commutative
diagram:
\[
\begin{tikzcd}
\LHI[n]{\RHI[n]{F}} \arrow{r}{\cuHI_F^n} \arrow{d}{\LHI[n]{\RHI[n]{f}}}
& F \arrow{r} \arrow{d}{f}
& \tlHI{n} F \arrow[dotted]{d}{g} \arrow{r}
& 0 \\
\LHI[n]{\RHI[n]{G}} \arrow{r}{\cuHI_F^n}
& G \arrow{r}
& \tlHI{n}G \arrow{r}
& 0
\end{tikzcd}
\]
with the dotted arrow $g$ given by the universal property of 
cokernels. Set $\tlHI{n}f = g$. It is clear from definition that
$\tlHI{n}$ is a functor.

\pfitem{Essential image is $\tlHI{n}\HI$} : Fix $F \in \HI$. To
show that $\tlHI{n}F \in \tlHI{n}\HI$, we need only to verify
that $\RHI[n]{(\tlHI{n}F)}$ is 0.

By definition, we have
\[
\LHI[n]{\RHI[n]{F}} \to F \to \tlHI{n}F \to 0
\]
Since the functor $\RHI{?}$ is exact (see Prop. 
\ref{prop_contract_is_exact} and Prop. \ref{prop_contraction}), 
we then have the following exact sequence
\[
\RHI[n]{(\LHI[n]{\RHI[n]{F}})} \to \RHI[n]{F} \to
\RHI[n]{(\tlHI{n}F)} \to 0.
\]
By Prop. \ref{prop_unit_iso}, $\RHI[n]{(\LHI[n]{\RHI[n]{F}})}
\to \RHI[n]{F}$ is an isomorphism. It follows that $\tlHI{n}F = 
0$ as desired.

\pfitem{$\tlHI{n}$ is left adjoint to inclusion} : the proof
of this will rely on the following lemma:

\begin{lem}\label{lem_tlHI_id}
The functor $\tlHI{n}$, restricted to $\tlHI{n}\HI$ is the 
identity. Consequently, the functor $\tlHI{n}$ is idempotent. 
That is, $(\tlHI{n})^2 = \tlHI{n}$ (see Def. \ref{def_idempotence})
\end{lem}
\begin{proof}[Proof of Lemma]
For $F \in \tlHI{n}\HI$, we have
\[
\LHI[n]{\RHI[n]{F}} \to F \to \tlHI{n}F \to 0
\]
However, since $F \in \tlHI{n}\HI$, $\RHI[n]{\tlHI{n}F} = 0$, and 
therefore counit map is $0$. It follows that $\tlHI{n}F = F$ as 
desired.

The second statement follows from the fact that $\tlHI{n}F \in 
\tlHI{n}\HI$.
\end{proof}

Continuing with the proof of Prop. \ref{prop_HI_lower_slice}, let
$F \in \HI$ and $G \in \tlHI{n}\HI$. For all $f: F \to G$ we have 
the following commutative diagram:
\[
\begin{tikzcd}
F \arrow{r}{\pi} \arrow{d}{f}
& \tlHI{n}F \arrow{r} \arrow{d}{\tlHI{n}f}
& 0 \\
G \arrow{r}{\pi'}
& \tlHI{n}G \arrow{r}
& 0
\end{tikzcd}
\]
By Lemma \ref{lem_tlHI_id}, since $G \in \tlHI{n}\HI$, the map
$G \stackrel{\pi'}{\to} \tlHI{n}G$ is the identity. Define
\[
\chi : \homHI(F, G) \to \hom_{\tlHI{n}\HI}(\tlHI{n}F, G)
\]
by $f \mapsto \tlHI{n}f$. If $\tlHI{n}f = 0$, then $f = 0$.
Therefore $\chi$ is injective. For $g: \tlHI{n}F \to G$, then
set $f' = \pi \comp g$. It is easy to see that $\chi(f') = g$.
Thus, $\chi$ is a bijection, as desired.
\end{proof}

We now relate the weak filtration $(\HI, \sgHI{*})$ and the strong 
cofiltration $(\HI, \tlHI{*})$ to the slice filtration on 
$\DMeff$. Recall that the slice filtration structure on $\DMeff$ 
is associated with the (weak) filtration $(\DMeff, \sgDM{*})$ and 
the (weak) cofiltration $(\DMeff, \slDM{*})$ (see Sect. 
\ref{sect_slice_filt_dm}).

It is clear from Prop. \ref{prop_LRDM_eq_LRHI} that the essential 
image of $\sgDM{n}\DMeff$ under $\sheaf{H}^0$ is $\sgHI{n}\HI$. 
However, showing that the essential image $\slDM{n}\DMeff$ under 
$\sheaf{H}^0$ is $\tlHI{n}\HI$ requires more proof. 
First, observe that by Prop. \ref{prop_DM_slice_triangle} for 
every $n > 0$ and every $F \in \HI$ (regarded as an object of 
$\DMeff$), there exists a slice triangle:
\[
\sliceTriangle{n}{F}
\]
Applying the cohomologicaly functor $\sheaf{H}^0$, we obtain the
following long exact sequence
\[
\cdots \stackrel{\delta_{-1}}{\to} \sheaf{H}^0 \sgDM{n} F \to 
   \sheaf{H}^0 F \to \sheaf{H}^0 \slDM{n} F
   \stackrel{\delta_0}{\to} \sheaf{H}^1 \sgDM{n} F \to \cdots
\]
where $\sheaf{H}^i F \defeq \sheaf{H}^0F[i]$. In particular, we
have the following exact sequence
\begin{equation}\label{eq_slice_DM_exact_seq}
\sheaf{H}^0 \slDM{n} F \to \sheaf{H}^0 F \to \sheaf{H}^0 
\sgDM{n} F \stackrel{\delta_0}{\to} \sheaf{H}^1 \sgDM{n} F.
\end{equation}
Notice that $\sheaf{H}^0F = F$ and $\sgDM{n} F = \ihomDMf(\Z(1)[1], 
F)(1)[1]$. So by Prop. \ref{prop_LRDM_eq_LRHI}, $\sheaf{H}^0 
\sgDM{n}F = \LHI[n]{(\RHI[n]{F})}$. Therefore, the exact sequence
in Eq. \ref{eq_slice_DM_exact_seq} reduces to the following
\[
\LHI[n]{(\RHI[n]{F})} \to F \to \sheaf{H}^0{\slDM{n} F} 
   \stackrel{\delta_0}{\to} \sheaf{H}^1 \sgDM{n} F.
\]
where the map $\LHI[n]{(\RHI[n]{F}} \to F$ is the counit. If we 
show that $\sheaf{H}^1 \sgDM{n} F = 0$, then it is clear that
$\sheaf{H}^0 \slDM{n} F = \tlHI{n} F$. The claim that $\sheaf{H}^1 
\sgDM{n}F = 0$ is precisely the content of Lemma 
\ref{lem_H1_sgDM_vanishes}, which we prove shortly. First, we have
the following.

\begin{lem}[D\'eglise]\label{lem_contract_exact}
For $F \in \HI$,
\[
\ihomDMf(\Z(1)[1], F) \simeq \RHI{F}
\]
as objects in $\DMeff$.
\end{lem}
\begin{proof}
Since $\sheaf{H}^0 \ihomDMf(\Z(1)[1], F) \simeq \RHI{F}$ by 
\ref{prop_LRDM_eq_LRHI}, it suffices to show that 
$\ihomDMf(\Z(1)[1], F)$, now regarded as a chain complex of
sheafs with transfers, is quasi-isomorphic to a chain complex 
concentrated in degree 0, the Nisnevich sheaf
\[
\sheaf{H}^i\ihomDMf(\Z(1)[1], F)
\]
vanishes for all Hensel local schemes.

Fix $S$ a Hensel local scheme. Note that, since $F$ is $\A^1$-local,
\begin{align*}
\sheaf{H}^i\ihomDMf(\Z(1)[1], F)(S) &= H^i\rhomDMf(\CZtr(S) 
   \tDM \Z(1)[1], F) \\
   &\simeq H^i\homDSh(\Ztr(S \times 
   \Gm), F).
\end{align*}
Furthermore, there exists a split exact triangle
\[
\Ztr(S \times \A^1) \to \Ztr(S \times (\A^1 - 0)) \to 
   \Ztr(S \times \Gm) \to \Ztr(S \times \A^1)[1],
\]
in $\DSh$ and upon applying $\homDSh(-, F)$, we have a long exact 
sequence in cohomology:
\begin{align*}
\cdots & \rightarrow H^i\homDSh(\Ztr(S \times \A^1), F) 
   \rightarrow H^i\homDSh(\Ztr(S \times (\A^1 - 0)), F) \\
 & \rightarrow H^i\homDSh(\Ztr(S \times \Gm), F) \rightarrow 
   H^i\homDSh(\Ztr(S \times \A^1), F) \rightarrow \cdots
\end{align*}
Note that $H^i\homDSh(\Ztr(X), F) = H_{\Nis}^i(X; F)$. Since
$H^i_{\Nis}(X; F) = 0$ for $i < 0$, and for all $i > 0$, 
\[
H_{\Nis}^i(S \times \A^1; F) = H_{\Nis}^i(S; F) = 0
\] 
and 
\[
H_{\Nis}^i(S \times (\A^1 - 0); F) = 0
\] 
(\cite{MVW} Cor. 24.5), it follows that 
\[
H^i\homDSh(\Ztr(S \times \Gm), F) = 0 \quad\textrm{for all $i \neq -1, 0$.}
\] 
Thus, we are reduced to showing that $H^{-1}\homDSh(\Ztr(S \times 
\Gm), F) = 0$. This follows from the fact that the map $F(S \times 
\A^1) = F(S) \to F(S \times (\A^1 - 0))$ is a split injection, and 
the lemma follows.
\end{proof}

\begin{lem}[D\'eglise]\label{lem_H_com_ihom_DM}
For $M \in \DMeff$, 
\[
\sheaf{H}^i\ihomDMf(\Z(1)[1], M) = \ihomDMf(\Z(1)[1], \sheaf{H}^i(M)).
\]
\end{lem}
\begin{proof}

\end{proof}

\begin{rmk}
Lemma \ref{lem_contract_exact} appears in \cite{DegGenMot} as Lem. 
3.4.5. However, we give a slightly different proof here. Lemma 
\ref{lem_H_com_ihom_DM} follows from Prop. 3.4.4 of \emph{Loc. Cit.}
and 
\end{rmk}

\begin{lem}\label{lem_H1_sgDM_vanishes}
For $F \in \HI$, $\sheaf{H}^1 \sgDM{n} F = 0$.
\end{lem}
\begin{proof}
First, note that
\end{proof}

This shows that $\tlHI{n}\HI$ is contained in the essential image 
of $\sgDM{n}\DMeff$ under $\sheaf{H}^0$. To show the converse, we 
establish that $\RHI[n](\sheaf{H}^0 M) = 0$ for $M \in 
\slDM{n}\DMeff$. It is a 

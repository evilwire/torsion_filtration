\newpage
\chapter{Filtrations on $\HI$}\label{sect_filtration_hi}

The purpose of this chapter is to construct three filtrations of 
$\HI$. The main result of this chapter is that
there is a sequence of coradicals (see Definition 
\ref{def_coradical}) on the category $\HI$ which induces a 
descending strong filtration and an ascending cofiltration (see 
Definition \ref{def_strong_filtration} below) of $\HI$ by the 
associated subcategories (see Theorem \ref{thm_main_result}). The 
key ingredient in the constructions of the filtrations is the 
tensor monoidal structure on $\HI$ and the partial internal hom. 
These structures are induced by the tensor and partial 
internal hom operators on $\DMeff$ introduced in Section 
\ref{sect_TMS_DMeff}. All uncredited results in this section are new.

\section{Tensor and partial internal hom structure on $\HI$}

To simplify the definition and the proofs in this chapter and the
next, we invoke Theorem \ref{thm_ALocal_eq_DMeff} and identify the 
category $\DMeff$ with the full triangulated subcategory $\ALocal$ 
of $\A^1$-local complexes from Definition \ref{def_ALocal}. We identify 
objects $M$ in $\DMeff$ with bounded above complexes $F^*$ of 
Nisnevich sheaves with transfers such that $H^nF^*$ is a homotopy 
invariant presheaf with transfers for every $n$. In particular, 
regarding a sheaf with transfers as a cochain complex concentrated 
in degree 0, we consider $\HI$ as an additive subcategory of 
$\DMeff$.

Recall the following notions from \cite[1.3]{BBD}:

\begin{defn}\label{def_t_struct}
A \DEF{$t$-category} is a triangulated category $\DCat$ together
with a pair of full subcategories $(\DCat^{\geq 0}, 
\DCat^{\leq 0})$, called the \DEF{positive objects} and 
\DEF{negative object} of $\DCat$ respective, which satisfies the 
following properties:
\begin{enumerate}
\item For all $X$ in $\DCat^{\leq 0}$, and $Y$ in $\DCat^{\geq 1}$, 
$\hom_{\DCat}(X, Y) = 0$.

\item $\DCat^{\leq 0} \subset \DCat^{\leq 1}$ and
$\DCat^{\geq 1} \subset \DCat^{\geq 0}$

\item For all $X$ in $\DCat$, there exists a distinguished 
triangle
\[
A \to X \to B \to A[1]
\]
such that $A$ is in $\DCat^{\leq 0}$ and $B$ is in 
$\DCat^{\geq 1}$.
\end{enumerate}
\noindent Here we write $\DCat^{\geq n}$ and $\DCat^{\leq n}$ for 
$\DCat^{\geq 0}[n]$ and $\DCat^{\leq 0}[n]$ respectively. We call
the pair $(\DCat^{\geq 0}, \DCat^{\leq 0})$ a \DEF{$t$-structure} 
on $\DCat$.

The \DEF{heart} of a $t$-category is the full subcategory
$\Cat{C} \defeq \DCat^{\geq 0} \cap \DCat^{\leq 0}$.
\end{defn}

If $\DCat$ is a $t$-category, then the inclusion of $\DCat^{\leq 
n}$ in $\DCat$ admits a right adjoint $\gltrunc{n}: \DCat \to 
\DCat^{\leq n}$, and the inclusion of $\DCat^{\geq n}$ in $\DCat$ 
admits left adjoint $\ggtrunc{n}: \DCat \to \DCat^{\geq n}$. 
Furthermore, for all $X$ in $\DCat$, there exists a unique map $d$ 
in $\hom_{\DCat}(\ggtrunc{1}X, \gltrunc{0}X[1])$ such that
\[
\gltrunc{0} X \to X \to \ggtrunc{1} X \stackrel{d}{\to} \gltrunc{0}X[1]
\]
is distinguished (see \cite[1.3.3]{BBD}). For integers $m$ and $n$ 
such that $m < n$, $\gltrunc{m} \gltrunc{n} = 
\gltrunc{n}\gltrunc{m} = \gltrunc{m}$, and $\ggtrunc{m} 
\ggtrunc{n} = \ggtrunc{n} \ggtrunc{m} = \ggtrunc{n}$. Furthermore, 
$\gltrunc{m} \ggtrunc{n} = \ggtrunc{n} \gltrunc{m} = 0$, and
$\gltrunc{n} \ggtrunc{m} = \ggtrunc{m} \gltrunc{n}$ 
(see \cite[1.3.5]{BBD}). When $n = m = 0$, the composition
$\gltrunc{0} \ggtrunc{0}$ defines an additive functor 
$\HH^0 : \DCat \to \Cat{C}$.

Recall from \cite[1.2.5]{BBD} that an abelian subcategory 
$\Cat{C}$ of $\DCat$ is \DEF{admissible} if for all $C$ and $D$ in 
$\Cat{C}$ and $i < 0$, $\hom_{\DCat}(C, D[i]) = 0$, and all exact
sequences in $\Cat{C}$ come from distinguished triangles in 
$\DCat$.

\begin{thm}[\cite{BBD} 1.3.6]\label{thm_heart_is_abel_cat}
Let $\DCat$ be a $t$-category, and let $(\DCat^{\geq 0}, 
\DCat^{\leq 0})$ be its associated $t$ structure. Then the heart 
$\Cat{C}$ is an admissible abelian category, stable under
taking extensions. 
\end{thm}

\begin{ex}[\cite{BBD} 1.3.2]\label{ex_DA_t_struct}
Let $\Cat{A}$ be an abelian category, and $\DCat[D]\Cat{A}$ be its
derived category. There is a natural $t$-structure on 
$\DCat[D]\Cat{A}$. The pair $(\DCat[D]\Cat{A}^{\geq 0}, 
\DCat[D]\Cat{A}^{\leq 0})$ is a pair of full subcategories whose
objects are those with trivial cohomology in the negative and 
positive degrees respectively. In this case, the functors 
$\ggtrunc{n}$ and $\gltrunc{n}$ are given by good truncations. 

The heart of this $t$-structure is precisely $\Cat{A}$, where an
object of $\Cat{A}$ is regarded as a complex concentrated in degree
$0$. (See the example following the statement of 1.3.6 in \cite{BBD}.)
\end{ex}

\begin{ex}[\cite{BBD} 1.3.16]
If $\DCat'$ is a full triangulated subcategory of a $t$-category 
$\DCat$, then $\DCat'$ is also a $t$-category with the 
$t$-structure given by $({\DCat'}^{\geq 0}, {\DCat'}^{\leq 0})$, 
where ${\DCat'}^{\geq 0} \defeq \DCat^{\geq 0} \cap \DCat$ and 
${\DCat'}^{\leq 0} \defeq \DCat^{\leq 0} \cap \DCat'$. 
\end{ex}

\begin{defn}
Let $\phi: \DCat \to \DCat'$ be a triangulated functor between
$t$-categories. We say that $\phi$ is \DEF{right $t$-exact} if 
$\phi(\DCat^{\leq 0}) \subseteq {\DCat'}^{\leq 0}$, and  
\DEF{left $t$-exact} if $\phi(\DCat^{\geq 0}) \subseteq 
{\DCat'}^{\geq 0}$. We say that $\phi$ is \DEF{$t$-exact} 
if it is both right and left $t$-exact.
\end{defn}

The concept of (left or right) $t$-exactness is a generalization 
of exactness in abelian category. We have the following result
regarding $t$-exact functors and the induced functor on the
hearts.

\begin{prop}[\cite{BBD} 1.3.17]\label{prop_t_exact_implies_exact}
Let $\DCat$ and $\DCat'$ be $t$-categories with hearts 
$\Cat{A}$ and $\Cat{A}'$ respectively. Furthermore, let $F: 
\DCat \to \DCat'$ be a left (resp., right) $t$-exact triangulated 
functor. Then $\HH^0 F$ is a left (resp., right) exact 
functor from $\Cat{A}$ to $\Cat{A}'$.
\end{prop}

If a $t$-category $\DCat$ is equipped with an additive symmetric 
monoidal structure that is right $t$-exact in both factors, then 
so is its heart $\Cat{C}$. The symmetric monoidal structure on
the heart is defined as follows. Suppose $- \tensor -$ is the 
tensor operator on $\DCat$. For $C, C'$ in $\Cat{C}$, we define $C 
\tensor^{\Cat{C}} C'$ by $\HH^0(C \tensor C')$. 
Since $\tensor$ is right $t$-exact in both factors, for all $M$ 
and $N$ in $\DCat^{\leq 0}$,
\[
\HH^0(M \tensor N) = \HH^0(\HH^0(M) \tensor \HH^0(N))
\]
(\cite[5.10]{DegModHom}) and $\tensor^{\Cat{C}}$ is well-defined.
It is now straightforward to verify that $(\Cat{C}, 
\tensor^{\Cat{C}})$ satisfies all the axioms of a symmetric 
monoidal category. 

In addition, if $\DCat$ has a partial internal hom structure
$(\ihom, \DCat^{\compact})$ as defined in Definition 
\ref{def_tensor_triang_cat}, then $\Cat{C}$ is also equipped
with a partial internal hom. For $C, C'$ in $\Cat{C}$, let us
set
\[
\ihom_{\Cat{C}}(C, C') \defeq \HH^0(\ihom(C, C')).
\]
\begin{prop}\label{prop_ihomC_is_partial_ihom}
Let $C$ be an object in $\DCat^{\compact} \cap \Cat{C}$ such that 
$\ihom(C, -)$ is right $t$-exact. Then $\ihom_{\Cat{C}}(C, -)$ is 
right adjoint to $C \tensor^{\Cat{C}} -$ as endofunctors on 
$\Cat{C}$.
\end{prop}
\begin{proof}
Notice that for $M$ in $\DCat^{\leq 0}$ and $M'$ in 
$\DCat^{\geq 0}$,
\begin{equation}\label{eq_t_struct_hom_eq_1}
\hom_{\DCat}(\HH^0(M), M') \cong \hom_{\DCat}(M, M')
\end{equation}
and
\begin{equation}\label{eq_t_struct_hom_eq_2}
\hom_{\DCat}(M, \HH^0(M')) \cong \hom_{\DCat}(M, M').
\end{equation}
Fix any $C_1, C_2$ in $\Cat{C}$. Since $\tensor$ is right 
$t$-exact in both factors, $C_1 \tensor C$ is in $\DCat^{\leq 0}$, 
and using the isomorphism in \eqref{eq_t_struct_hom_eq_1}, we 
obtain the following isomorphism:
\[
\hom_{\Cat{C}}(C_1 \tensor^{\Cat{C}} C, C_2) = 
\hom_{\Cat{C}}(\HH^0(C_1 \tensor C), C_2) 
\cong \hom_{\DCat}(C_1 \tensor C, C_3).
\]
Since $C$ is in $\DCat^{\compact}$, the functor $- \tensor C$ is 
left adjoint to $\ihom(C, -)$. By assumption, $\ihom(C, -)$ is 
right $t$-exact, and since $C_2$ is in the heart, $\ihom(C, C_2)$ 
is an object in $\DCat^{\geq 0}$. Therefore, using the isomorphism 
in \eqref{eq_t_struct_hom_eq_2}, we obtain the following chain of
isomorphisms:
\begin{align*}
\hom_{\Cat{C}}(C_1 \tensor^{\Cat{C}} C, C_2)
&\cong \hom_{\DCat}(C_1 \tensor C, C_2) \\
&\cong \hom_{\DCat}(C_1, \ihom(C, C_2)) \\
&\cong \hom_{\DCat}(C_1, \HH^0(\ihom(C, C_2))) \\
&= \hom_{\Cat{C}}(C_1, \ihom_{\Cat{C}}(C, C_2)).
\end{align*}
Since each of the above isomorphism is natural in $C_1$ and $C_2$, 
the proposition now follows.
\end{proof}

Proposition \ref{prop_ihomC_is_partial_ihom} shows that the 
bifunctor $\ihom_{\Cat{C}}$ defines a partial internal hom on the
category $\Cat{C}$ where the collection of semi-representable 
objects of $\ihom_{\Cat{C}}$ contains at least those objects $C$ 
in $\Cat{C} \cap \DCat^{\compact}$ such that $\ihom(C, -)$ is left 
$t$-exact.

The above discussion applies to the category $\DMeff$ since
that there exists a $t$-structure on $\DMeff$, with the abelian 
category $\HI$ as its heart (see Theorem 
\ref{thm_ALocal_eq_DMeff}). 

\begin{defn}\label{def_t_struct_DMeff}
We write $\gltrunc{0}\DMeff$ for the negative objects of $\DMeff$, and 
$\ggtrunc{0}\DMeff$ for the positive objects of $\DMeff$. We will 
also let $\gltrunc{0}$, $\ggtrunc{0}$ and $\HH^0$ denote the 
functors from $\DMeff$ to $\gltrunc{0}\DMeff$, $\ggtrunc{0}\DMeff$ 
and $\HI$, respectively. 
\end{defn}

The triangulated monoidal structure on $\DMeff$ induces a 
symmetric monoidal bifunctor on $\HI$, which we write as $\tHI$. 
This bifunctor is uniquely characterized by
\[
\hS{X} \tHI \hS{Y} = \hS{X \times Y},
\]
where $\hS{X} = \HH^0(M(X))$ for $X$ in $\Sm$ (see 
\cite[1.8]{DegModHom}).

Moreover, there is a partial internal hom, defined by
\[
\ihomHI(F, G) = \HH^0(\ihomDMf(F, G))
\]
for all $G$ in $\HI$, and $F$ in $\HI \cap \DMeffgm$ for which 
$\ihomDMf(F, -)$ is left $t$-exact. Our first goal is to show that 
$\Ox$ is \SemiInvertible with respect to $\ihomHI$ by showing that
$\ihomDMf(\Ox, -)$ is right $t$-exact. This is established by the 
following lemma. Recall from Definition \ref{def_contract} that 
for $F$ in $\HI$, $\RHI{F}$ is the contraction of $F$, which is 
the homotopy invariant sheaf with transfers defined by
\[
X \mapsto \mathrm{cok}(F(X \times \A^1) \to F(X \times 
(\A^1 - 0)) ).
\]
Moreover, $F \mapsto \RHI{F}$ defines an endofunctor on $\HI$. 

\begin{lem}\label{lem_HH_commutes_with_contract}
There exists a natural isomorphism of homotopy invariant sheaves
with transfers
\[
\HH^i \ihomDMf(\Z(n)[n], M) \cong \RHI[n]{(\HH^i M)}.
\]
In particular, $\ihomDMf(\Ox, -)$ is left $t$-exact.
\end{lem}
\begin{proof}
Fix an object $M$ in $\DMeff$, regarded as a cochain complex of 
sheaves with transfers with homotopy invariant cohomology 
presheaves. By \cite[3.4.4]{DegGenMot}, there exists a natural
morphism between homotopy invariant presheaves with transfers:
\[
i : \RHI[n]{(H^i M)} \to H^i \ihomDMf(\Z(n)[n], M)
\] 
such that for all fields $E$ over $\basefield$, the following
is an isomorphism:
\[
\RHI[n]{(H^i M)}(\Spec E) \cong H^i\ihomDMf(\Z(n)[n], M)(\Spec E).
\]
By \cite[11.2]{MVW}, $i$ induces a natural isomorphism 
of the associated homotopy invariant Nisnevich sheaves with 
transfers:
\[
\RHI[n]{(\HH^i M)} \stackrel{\cong}{\to} \HH^i \ihomDMf(\Z(n)[n], 
   M).
\]
This proves the first claim in the Lemma. 

To see that $\ihomDMf(\Ox, -)$ is left $t$-exact, suppose $M$
is a positive object, i.e. $\HH^i M = 0$ for all $i < 0$. Since
$\Ox \cong \Z(1)[1]$, applying the above for $n = 1$, we see
that for all $i < 0$,
\[
\HH^i \ihomDMf(\Ox, M) \cong \HH^i \ihomDMf(\Z(1)[1], M) \cong 
   \RHI{(\HH^i M)} = 0.
\]
Thus, $\ihomDMf(\Ox, M)$ is also a positive object in $\DMeff$,
and the lemma is now established.
\end{proof}

\begin{defn}
To emphasize the relationship with corresponding operations in
$\DMeff$, let us set
\[
\LHIs{F} \defeq F \tHI \Ox \hspace{10pt} \textrm{and}\hspace{10pt} 
   \RHIs{F} \defeq \ihomHI(\Ox, F).
\]
We write $\LHIs[n]{F}$ for $\LHIs{(\LHIs[n - 1]{F})}$ and 
$\RHIs[n]{F}$ for $\RHIs{(\RHIs[n + 1]{F})}$. 
\end{defn}

By Lemma \ref{lem_HH_commutes_with_contract} and preceding 
comments, $F \mapsto \LHIs{F}$ is left adjoint to $F \mapsto 
\RHIs{F}$, and therefore $F \mapsto \LHIs[n]{F}$ is 
left adjoint to $F \mapsto \RHIs[n]{F}$ for all $n > 0$.

\begin{rmk}\label{rmk_contraction}
To simplify notation, we will drop the ``HI'', and simply write 
$\LHI[n]{F}$ and $\RHI[n]{F}$ for $\LHIs[n]{F}$ and $\RHIs[n]{F}$.
Doing so introduces a number of potential sources of ambiguity. 
The first is that $\LHI[n]{F}$ is already used to represent $F 
\tDM \Z(n)$, where $\Z(n)$ is the motivic complex in $\DMeff$ 
introduced in Section \ref{sect_motivic_complex}. In particular, 
$\LHI[n]{\Z}$ may refer to the motivic complexes as well as the 
objects $\Z \tHI (\Ox)^{\tensor n}$. To resolve this ambiguity, we 
adopt the following convention: For the remainder of the thesis, 
unless otherwise specified, for an object $F$ in $\HI$, 
$\LHI[n]{F}$ will denote $\LHIs[n]{F} \defeq F \tHI 
(\Ox)^{\tensor n}$. All mentions of $\Z(n)$ will refer to the 
motivic complex in $\DMeff$.

The second source of potential ambiguity comes from the fact that 
$\RHI{F}$ is already used to represent the contraction of the 
sheaf $F$ in $\HI$. Recall from Definition \ref{def_contract} that 
$\RHI{F}$ is the sheaf that sends $X$ in $\Sm$ to $\cok p^*$, 
where 
\[
p^* : F(X) \to F(X \times (\A^1 - 0))
\]
is the map induced by the projection $X \times (\A^1 - 0) \to X$.
In fact, there is no ambiguity here, since the contraction of $F$ 
is isomorphic to the sheaf $\ihomHI(\Ox, F)$. Indeed, by 
\cite[3.4.5]{DegGenMot}, the contraction of $F$ is isomorphic to
$\ihomDMf(\Z(1)[1], F)$. Recall from \cite[4.1]{MVW} that 
$\Z(1)[1] \cong \Ox$ in $\DMeff$. Hence, we have that
\[
\ihomHI(\Ox, F) \cong \HH^0 \ihomDMf(\Z(1)[1], F) \cong \HH^0 
   \RHI{F} = \RHI{F}.
\]
\end{rmk}

Finally, we make some observations that will be useful in 
subsequent sections.

\begin{prop}\label{prop_LR_commute_with_HH}
For all negative objects $M$,
\[
\HH^0(M \tDM \Z(n)[n]) = \LHI[n]{\HH^0(M)}.
\]
\end{prop}
\begin{proof}
By construction, the tensor operation $\tDM$ is right 
$t$-exact in both factors. Therefore, for negative objects $M$ and 
$N$ of $\DMeff$, we have that
\[
\HH^0M\tHI \HH^0N = \HH^0(\HH^0(M) \tDM \HH^0(N)) = 
   \HH^0(M \tDM N).
\]
Since $\Z(n)[n] = \Z(n - 1)[n - 1] \tDM \Z(1)[1]$, and $\Z(1)[1] 
\cong \Ox$, by induction on $n$, $\Z(n)[n]$ is also a negative 
object and $\HH^0(\Z(n)[n]) \cong (\Ox)^{\tensor n}$. Moreover, for 
a negative object $M$ in $\DMeff$, we obtain the following:
\[
\HH^0(M \tDM \Z(n)[n]) \cong \HH^0(M) \tHI (\Ox)^{\tensor n} = 
   \LHI[n]{\HH^0(M)}. \qedhere
\]
\end{proof}

\begin{prop}\label{prop_unit_iso}
Let $F$ be a homotopy invariant sheaf with transfers. The unit map 
$F \to \RHI[n]{\LHI[n]{F}}$ is an isomorphism.
\end{prop}
\begin{proof}
For $F$ in $\HI$, by the Cancellation Theorem 
\ref{thm_dm_cancellation}, we have that 
\[
\ihomDMf(\Z(n)[n], F(n)[n]) \cong \ihomDMf(\Z, F) 
\cong F. 
\]
Now apply $\HH^0$ to this chain of isomorphisms. Using Lemma
\ref{lem_HH_commutes_with_contract} and the fact that $\HH^0(F) = 
F$, we obtain the desired isomorphism.
\end{proof}

\begin{prop}\label{prop_counit_iso_for_HIn}
If $F = \LHI[n]{G}$ for some $G$ in $\HI$, then $\cuHI^n_F : 
\LHI[n]{\RHI[n]{F}} \to F$ is an isomorphism.
\end{prop}
\begin{proof}
Suppose $F = \LHI[n]{G}$ for some $G$ in $\HI$. Writing $L$ for 
the functor 
$F \mapsto \LHI[n]F$, by counit-unit adjunction, the composition
\[
\LHI[n]{G} \xrightarrow{L\eta_G} \LHI[n]{(\RHI[n]{\LHI[n]{G}})}
   \xrightarrow{\cuHI_G L} \LHI[n]{G}
\]
is the identity, where $\eta_{G}$ and $\cuHI_{G}$ are the unit and 
the counit maps respectively. By Proposition \ref{prop_unit_iso}, 
$\eta_{G}$ is an isomorphism, and so is $L\eta_{G}$. It follows
that $\cuHI_G L$ is an isomorphism as well. Since $\cuHI_G L$ is the 
counit map for $LG = \LHI[n]{G} = F$, the proposition follows.
\end{proof}

\section{Torsion filtration on $\HI$}
\label{sect_torsion_filt_on_HI}

We now define the first filtration on $\HI$. Let $\LHI[0]{\HI} = 
\HI$ and let $\LHI[n]{\HI}$ denote the full subcategory of objects 
$F$ where $F \cong \LHI[n]{F'}$ for some $F'$ in $\HI$. It is clear 
that if $m \geq n$, then $\LHI[m]{\HI} \subseteq \LHI[n]{\HI}$. 
In particular, we have a tower of subcategories
\[
\HI = \LHI[0]{\HI} \supset \LHI[1]{\HI} \supset \LHI[2]{\HI} 
\subset \cdots.
%\cdots \subset \LHI[2]{\HI} \subset \LHI[1]{\HI} \subset \LHI[0]{\HI} 
% = \HI 
\]
To see that this filtration is not trivial (i.e., $\LHI[n]{\HI} \neq
\LHI[m]{\HI}$ for all natural numbers $n$ and $m$), notice that
for the constant sheaf $\Z$, it is clear that $\RHI{\Z} = 0$. 
Then $\Z$ is an object in $\HI$ but not in $\LHI{\HI}$. Indeed, if 
$\Z \in \LHI{\HI}$ then $\Z \cong \LHI{F'}$, but then $\RHI{\Z} = F'$ 
by Proposition \ref{prop_unit_iso}, forcing $\Z = 0$. Similarly, 
since $\RHI{\Ox} = \Z$, $\Ox \in \LHI{\HI}$ but $\Ox \notin
\LHI[2]{\HI}$. In general, $\LHI[n - 1]{\Ox}$ is an object of
$\LHI[n]{\HI}$ but not $\LHI[n + 1]{\HI}$.

\begin{rmk}
The subcategories $\LHI[n]{\HI}$ are additive, but \emph{not} 
abelian, except for the case $n = 0$. To see this, consider the 
map
\[
n: \Ox \to \Ox
\]
given by sending $u \in \Ox(X)$ to $u^n$ for each $X$ in $\Sm$.  The
kernel of this map is the sheaf of $n$-th roots of unity $\mu_n$.  But
$\RHI{(\mu_n)} = 0$. If $\mu_n$ were in $\LHI[n]{\HI}$, then by
Proposition \ref{prop_counit_iso_for_HIn}, we would have $\mu_n \cong
\LHI[1]{\RHI[1]{(\mu_n)}} = 0$, which is a contradiction. It follows
that $\LHI{\HI}$ is not closed under kernels. Similar arguments show
that $\LHI[n]{\HI}$ is not closed under kernel for any positive
integer $n$.
\end{rmk}

Recall from Definition \ref{def_cat_filtration} that a descending weak
filtration $(\Cat{A}_*, \phi_*)$ is a tower of subcategories
$\Cat{A}_i$ together with coreflection functors $\phi_i:
\Cat{A} \to \Cat{A}_i$ such that $\phi_i$ restricted to $\Cat{A}_i$ is
naturally isomorphic to the identity.  To show that the full
subcategories $\LHI[n]{\HI}$ define a descending weak filtration, we
need to show that there exist coreflection functors $\sgHI{n}: \HI \to
\LHI[n]{\HI}.$

\begin{defn}
Let $\sgHI{n}$ denote the functor $F \mapsto \LHI[n]{(\RHI[n]{F})}$.
Since $F \mapsto \LHI[n]{F}$ is right exact, and $\RHI[n]{F}$ is
exact (Proposition \ref{prop_contract_is_exact}), $\sgHI{n}$ is right
exact. However, $\sgHI{n}$ is \emph{not} always left exact (see Example
\ref{rmk_sgHI_is_not_strong_filt} below).
\end{defn}

\begin{prop}\label{prop_HI_upper_slice}
The functor $\sgHI{n}$ is right adjoint to the 
inclusion of $\LHI[n]{\HI}$. In particular, $(\LHI[*]{\HI}, 
\sgHI{*})$ defines a (nontrivial) descending weak filtration of 
$\HI$.
\end{prop}
\begin{proof}
Let $f: F \to G$ be a map in $\HI$, with $F$ in $\LHI[n]{\HI}$,
and let $\cuHI^n$ denote the counit $\sgHI{n} \to \id$.
By naturality of $\cuHI^n$, we have the following commutative 
diagram:
\[
\begin{tikzcd}
\LHI[n]{\RHI[n]{F}} \arrow{r}{\cuHI^n f}\arrow{d}{\cuHI^n_F} 
& \LHI[n]{\RHI[n]{G}} \arrow{d}{\cuHI^n_G} \\
F \arrow{r}{f}
& G.
\end{tikzcd}
\]
Since $F \in \LHI[n]{\HI}$, by Proposition 
\ref{prop_counit_iso_for_HIn} the counit map $\cuHI^n_F$ is an 
isomorphism.

Define the map $\chi: \homHI(F, G) \to \hom_{\LHI[n]{\HI}}(F, 
\LHI[n]{\RHI[n]{G}})$ by $f \mapsto 
\cuHI^n f \comp (\cuHI^n_F)^{-1}$. Since $\cuHI^n_G \comp \chi(f) 
= f$, $\chi$ is injective.  Moreover, given a map $g: F 
\to \LHI[n]{\RHI[n]{G}}$, set $f' = \cuHI_G^n \comp g$. Then 
$\chi(f') = g$. Hence $\chi$ is an isomorphism as desired.
From the way $\chi$ is defined, it is clear that $\chi$ is 
functorial in both $F$ and $G$, and therefore $\sgHI{n}$ is
right adjoint to the inclusion of $\LHI[n]{\HI}$ into $\HI$.

To show that $(\LHI[*]{\HI}, \sgHI{*})$ define a weak descending
filtration, the only criterion left to check is that $\sgHI{n}$
restricted to $\LHI[n]{\HI}$ is naturally isomorphic to the
identity. By Proposition \ref{prop_counit_iso_for_HIn}, the
counit map $\cuHI^n: \sgHI{n} F \to F$ is an isomorphism for
all $F$ in $\LHI[n]{\HI}$, and the proposition follows.
\end{proof}

\begin{ex}\label{rmk_sgHI_is_not_strong_filt}
While $(\HI(*), \sgHI{*})$ forms a weak filtration of $\HI$, for
a given sheaf $F$ in $\HI$, the objects $\sgHI{n} F$ are not 
in general subobjects of $F$, because the counit map $\sgHI{n} F
\to F$ is not always injective. Here is an example.

Let $\Oxn$ be the sheaf of $n$-th power of global units 
associated to the presheaf where sections of a smooth finite type 
$k$-scheme $X$ is the abelian subgroup of $\Ox$ given by 
\[
\Oxn(X) = \{x : x=y^n\textrm{ for some }y \textrm{ in } \Ox(X)\}.
\]
It is clear that $\Oxn \in \HI$. Furthermore, there exists the 
following exact sequence 
\[
0 \to \mu_n \to \Ox \to \Oxn \to 0
\]
where $\mu_n$ is the constant sheaf of $n$-th roots of unity.
In particular, $\RHI{(\mu_n)} = 0$. By Proposition 
\ref{prop_contract_is_exact}, the functor $F \mapsto \RHI{F}$ is
exact. Therefore, the map $\RHI{\Ox} \to \RHI{(\Oxn)}$ is an
isomorphism, and
\[
\LHI{\RHI{(\Oxn)}} \cong \LHI{\RHI{\Ox}} = \Ox,
\]
and the counit $\LHI{\RHI{(\Oxn)}} \to \Oxn$ is given precisely by
$x \mapsto x^n$, which has a nontrivial kernel.

We can understand the problem in another way, which is that the 
categories $\HI(*)$ are too small and do not include all
the kernels of counits $\LHI[n]{(\RHI[n]{F})} \to F$. This can be 
fixed by enlarging the filtration at each level, and to do so, we 
turn to torsion theory.
\end{ex}

Motivated by Example \ref{rmk_sgHI_is_not_strong_filt}, we 
introduce the following more stringent criteria on weak
filtrations.

\begin{defn}\label{def_strong_filtration}
  Let $\Cat{A}$ be an abelian category. We say that a $\Z$-indexed
  descending weak filtration $(\Cat{A}_*, \phi_*)$ is a \DEF{strong
    filtration} if for each $A$ in $\Cat{A}$ and $n$ in $\Z$, $\phi_n
  A \to A$ is a monomorphism of $A$. An ascending weak filtration
  $(\Cat{A}_*, \phi_*)$ is a \DEF{strong cofiltration} if $A \to
  \phi_n A$ is a quotient of $A$ for each $n$ and each $A$ in
  $\Cat{A}$.

  Similarly, we can define ascending strong filtration and descending
  strong cofiltration on $\Cat{A}$.
\end{defn}

\begin{ex}
Here is an example of a strong ascending filtration and a strong
descending filtration on the category $\QCoh$ of quasi-coherent
sheaves on $\P^n$. Let $i_k$ denote the closed immersion of $\P^k$ 
into $\P^n$ as a subscheme identified by the vanishing of the last 
$k$ homogeneous coordinates, and let $j_k$ denote the open 
immersion of $U_k \defeq \P^n - \P^k$ into $\P^n$. 

Let $\QCoh^k$ be the full subcategory of quasi-coherent sheaves
supported in $U_k$ and let $\QCoh_k$ denote the full subcategory of
sheaves on $\P^n$ supported in $\P^k$. Since
\[
U_0 \supseteq U_1 \supseteq U_2 \supseteq \cdots \supseteq U_n
\]
and
\[
\P^0 \subseteq \P^1 \subseteq \P^2 \subseteq \cdots \subseteq \P^n
\]
we have the following towers of subcategories:
\[
\QCoh^0 \supseteq \QCoh^1 \supseteq \QCoh^2 \supseteq \cdots
   \supseteq \QCoh^n
\]
and 
\[
\QCoh_0 \subseteq \QCoh_1 \subseteq \QCoh_2 \subseteq \cdots
   \subseteq \QCoh_n.
\]
We will show that the towers of subcategories define a strong 
filtration and strong cofiltration on $\QCoh$. For each positive 
integer $k$ less than $n$ and each $F$ in $\QCoh$, we have the 
following exact sequence of quasi-coherent sheaves on $\P^n$:
\begin{equation}\label{eq_qc_sheaf_ses}
0 \to (j_k)_!(F|_U) \to F \to (i_k)_*(F|_Z) \to 0
\end{equation}
where $(j_k)_!(F|_U)$ is the sheaf associated with the presheaf 
given by
\[
V \mapsto \begin{cases}
F(V) &\textrm{if }V \subseteq U_k\\
0    &\textrm{otherwise}.
\end{cases}
\]
In this case $(j_k)_!(F)$ is in $\QCoh^k$ and $(i_k)_*(F)$ is in 
$\QCoh_k$ (see \cite[Ex. 1.19]{Hart}). In fact, $F \mapsto 
(j_k)_!(F|_{U_k})$ and $F \mapsto (i_k)_*(F|_{\P^k})$ define
functors from $\QCoh$ to $\QCoh^k$ and $\QCoh_k$
respectively. In this case $(j_k)_!$ is right adjoint to 
inclusion, and $(i_k)_*$ is left adjoint to inclusion.

In general, let $Z_1 \subseteq Z_2 \subseteq \cdots Z_n$ be
a sequence of subschemes of some scheme $X$, and let $U_k =
X - Z_k$. Let $\QCoh(X)$ be the abelian category of quasi-coherent 
sheaves on $X$. Then there exists a strong descending filtration 
\begin{equation}\label{eq_qcoh_desc_filt}
\QCoh^0(X) \supseteq \QCoh^1(X) \supseteq \QCoh^2(X) \supseteq 
   \cdots \supseteq \QCoh^n(X)
\end{equation}
where $\QCoh^k(X)$ is the full subcategory of quasi-coherent sheaves
supported on $U_k$, and a strong ascending cofiltration on 
$\QCoh(X)$
\begin{equation}\label{eq_qcoh_asc_filt}
\QCoh_0(X) \subseteq \QCoh_1(X) \subseteq \QCoh_2(X) \subseteq \cdots
\subseteq \QCoh_n(X)
\end{equation}
where $\QCoh_k(X)$ is the full subcategory of quasi-coherent 
sheaves on $X$ supported on $Z_k$. As above, the coreflection 
functors $\phi^k$ from $\QCoh$ to $\QCoh^k$ are given by $F 
\mapsto (j_k)_!(F|U_k)$ where $j_k$ is the open immersion $j_k: 
U_k \to X$; the reflection functors $\phi_k$ from $\QCoh$ to 
$\QCoh_k$ are given by $F \mapsto (i_k)_*(F|_{Z_k})$, where $i_k: 
Z_k \to X$ is the evident closed immersion. Since, for each 
$k$, we have an exact sequence of quasi-coherent sheaves as in
\eqref{eq_qc_sheaf_ses}, $\phi^k(F)$ is a subobject of $F$ and
$\phi_k(F)$ is a quotient of $F$ for each $F$ in $\QCoh(X)$.
The claim that \eqref{eq_qcoh_desc_filt} defines a strong 
filtration and \eqref{eq_qcoh_asc_filt} defines a strong 
cofiltration now follows.
\end{ex}

We will now state the main theorem. Recall from Theorem 
\ref{thm_corad_equiv_htt} that if $\phi$ is a coradical, the 
associated torsion theory is a pair of full subcategories
$(\Cat{T}, \Cat{F})$ where the torsion subcategory $\Cat{T}$ consists
of the objects $T$ such that $\phi(T) = 0$, and the torsionfree
subcategory $\Cat{F}$ consists of the objects $F$ such that the map
$F \to \phi(F)$ is an isomorphism.

\begin{thm}\label{thm_main_result}
There exists a sequence of coradicals $\tlHI{n}, n = 0, 1, 2 
\cdots$ on $\HI$ such that the associated torsionfree 
subcategories $\THI{*}$ form a descending strong filtration
of $\HI$ and the associated torsion subcategories $\TFHI{*}$ 
form a strong cofiltration.
\end{thm}

Theorem \ref{thm_main_result} will be verified by Propositions
\ref{prop_tsubcat_eq_tlHI} and \ref{prop_THI_form_strong_filt} 
below. We first define the strong cofiltration, and show that the 
reflection functors are coradicals. 

\begin{defn}\label{def_TFHI}
If $n$ is a natural number, let $\TFHI{n}$ be the full 
subcategory of objects $F$ in $\HI$ such that $\RHI[n]{F} = 0$. 
By Proposition \ref{prop_contract_is_exact}, $F \mapsto 
\RHI[n]{F}$ is exact. Therefore, $\TFHI{n}$ is an abelian 
subcategory closed under extensions.
\end{defn}

By convention, define $\RHI[0]{F}$ to be $F$. Since 
$\RHI[n - 1]{F} = \RHI{(\RHI[n]{F})}$, we obtain the following 
ascending tower of subcategories
\[
0 = \TLHI{0}{\HI} \subset \TLHI{1}{\HI} \subset \TLHI{2}{\HI} 
   \subset \cdots \subset \HI
\]
Since $\RHI{\Ox} = \Z$ and $\RHI{\Z} = 0$, $\Ox$ is in 
$\TLHI{2}{\HI}$ but not in $\TLHI{1}{\HI}$. By Proposition 
\ref{prop_unit_iso} and by induction, $\LHI[n]{\Ox}$ is in 
$\TLHI{n + 1}{\HI}$ but not in $\TLHI{n}{\HI}$.

We now describe the reflection functors $\tlHI{n} : \HI \to 
\TLHI{n}{\HI}$. 

\begin{defn}
Let $n$ be a positive integer, and let $\tlHI{n}(F)$ denote the 
cokernel of the counit $\cuHI^n_F: \LHI[n]{\RHI[n]{F}} \to F$. 
Since $\cuHI^n_F$ is natural in $F$, $\tlHI{n}$ is a functor.
\end{defn}

We will show that $\tlHI{n}$ is the desired reflection functor
from $\HI$ to $\TFHI{n}$. This is established in Proposition
\ref{prop_HI_lower_slice}. We proceed by first considering the
following lemmas:

\begin{lem}\label{lem_tlHI_in_TLHI}
The image of $\HI$ under $\tlHI{n}$ is contained in 
$\TLHI{n}{\HI}$.
\end{lem}

\begin{proof}
Let $F$ be an object of $\HI$. We need to verify that 
$\RHI[n]{\tlHI{n}(F)} = 0$. By definition, we have an exact
sequence
\[
\LHI[n]{\RHI[n]{F}} \to F \to \tlHI{n}(F) \to 0.
\]
Since the functor $F \mapsto \RHI[n]{F}$ is exact (see Proposition
\ref{prop_contract_is_exact} and Remark \ref{rmk_contraction}), we 
then have the following exact sequence
\[
\RHI[n]{\LHI[n]{\RHI[n]{F}}} \to \RHI[n]{F} \to
\RHI[n]{\tlHI{n}(F)} \to 0.
\]
By Proposition \ref{prop_unit_iso}, $\RHI[n]{(\LHI[n]{\RHI[n]{F}})}
\to \RHI[n]{F}$ is an isomorphism. Hence,
$\RHI[n]{\tlHI{n}(F)} = 0$ as desired.
\end{proof}

\begin{lem}\label{lem_tlHI_id}
The functor $\tlHI{n}$, restricted to $\TFHI{n}$ is the 
identity. Consequently, the functor $\tlHI{n}$ is idempotent
(see Definition \ref{def_coradical} (2)), and the image of $\HI$ 
under $\tlHI{n}$ is $\TFHI{n}$.
\end{lem}
\begin{proof}
For each $F$ in $\TFHI{n}$, we have the following exact sequence:
\[
\LHI[n]{\RHI[n]{F}} \to F \to \tlHI{n}(F) \to 0
\]
Since $F \in \TFHI{n}$, $\RHI[n]{F} = 0$, and therefore the counit
map is $0$. It follows that $\tlHI{n}(F) = F$ as desired. The 
first statement follows from the fact that $\tlHI{n}(F)$ is in
$\TLHI{n}{\HI}$, which is established in Lemma \ref{lem_tlHI_in_TLHI}.
\end{proof}

\begin{prop}\label{prop_HI_lower_slice}
For each $n$, the functor $\tlHI{n}$ is left adjoint to the 
inclusion of $\TLHI{n}{\HI}$ into $\HI$.
\end{prop}
\begin{proof}
Let $F$ be a homotopy invariant sheaf with transfers, and let $G$ 
be an object in $\TLHI{n}{\HI}$. For all $f: F \to G$ we have the 
following commutative diagram:
\[
\begin{tikzcd}
F \arrow{r}{\pi_F} \arrow{d}{f}
& \tlHI{n}(F) \arrow{d}{\tlHI{n}(f)} \\
G \arrow{r}{\pi_G}
& \tlHI{n}(G) 
\end{tikzcd}
\]
where $\pi_F$ and $\pi_G$ are surjections. By Lemma \ref{lem_tlHI_id},
the map $G \stackrel{\pi_G}{\to} \tlHI{n}(G)$ is an
isomorphism. Define
\[
\chi : \homHI(F, G) \to \hom_{\TLHI{n}{\HI}}(\tlHI{n}(F), G)
\]
by $f \mapsto \pi_G^{-1} \circ \tlHI{n}(f)$. Since $\chi(f) \comp 
\pi_F = f$, $\chi$ is injective. For $g: \tlHI{n}(F) \to G$, set 
$f' = \pi \comp g$. Since $\chi(f') = g$, $\chi$ is a bijection, 
as desired.

From the way $\chi$ is defined, it is clear that $\chi$ is 
functorial in both $F$ and $G$. The proposition now follows.
\end{proof}

This shows that $\tlHI{n}$ is an idempotent quotient functor
for each natural number $n$. In fact, we have the following 
result:

\begin{prop}\label{prop_tlHIn_corad}
For each natural number $n$, $\tlHI{n}$ is a coradical.
\end{prop}
\begin{proof}
By Lemma \ref{lem_tlHI_id}, $\tlHI{n}$ is idempotent. 
By Proposition \ref{prop_HI_lower_slice}, $\tlHI{n}$ is a left 
adjoint, and since $\TFHI{n}$ is an abelian category (see 
Definition \ref{def_TFHI}), $\tlHI{n}$ is therefore right exact. 
All that remains to show is that for each $F$ in $\HI$,
\[
\tlHI{n}(\ker (F \to \tlHI{n}(F))) = 0.
\]
Fix a positive integer $n$, and let $K$ denote the kernel of the 
surjection $F \to \tlHI{n}(F)$. Since $\tlHI{n}(F)$ is in 
$\TLHI{n}{\HI}$, by definition $\RHI[n]{\tlHI{n}(F)} = 0$. 
Therefore, we have the following commutative diagram with exact
rows:
\[
\begin{tikzcd}
{} &\sgHI{n} K \arrow{d}{\cuHI_F} \arrow{r}
   &\sgHI{n} F \arrow{d}{\cuHI_F} \arrow{r}
   &0 \arrow{d} \arrow{r}
   &0 \\
0 \arrow{r} &
  K \arrow{r}&
  F \arrow{r}&
  \tlHI{n}(F) \arrow{r}&
  0.
\end{tikzcd}
\]
By the Snake Lemma, and using the fact that $\cok \cuHI_F = 
\tlHI{n}(F)$, we have the exact sequence
\[
0 \to \tlHI{n}(K) \to \tlHI{n}(F) \stackrel{q}{\to} \tlHI{n}(F) 
   \to 0.
\]
And the map $q$ is the identity. It follows that $\tlHI{n}(K) = 0$ 
as desired.
\end{proof}

Since $\tlHI{n}$ is a coradical, by Theorem \ref{thm_precorad_eq_tt},
there exists a torsion theory $(\Cat{T}_n, \Cat{F}_n)$ associated with
each $\tlHI{n}$. We now give another description of the torsionfree
subcategories.

\begin{prop}\label{prop_tsubcat_eq_tlHI}
For each positive integer $n$, the full subcategory $\TFHI{n}$ and
the torsionfree subcategory $\Cat{F}_n$ are the same. Hence, the
torsionfree subcategories form an ascending strong cofiltration of
$\HI$.
\end{prop}
\begin{proof}
Recall from Lemma \ref{lem_tlHI_in_TLHI} that 
$\RHI[n]{\tlHI{n}(F)} = 0$ for all $F$ in $\HI$. Hence, if $F$ is 
in $\Cat{F}_n$, $\RHI[n]{F} = \RHI[n]{\tlHI{n}(F)} = 0$.

Conversely, if $\RHI[n]{F} = 0$, then $\tlHI{n}(F) = F$ by Lemma
\ref{lem_tlHI_id}, and by definition $F$ is an object of 
$\TFHI{n}$. Hence, the torsionfree subcategory $\Cat{F}_n$ is 
precisely the full subcategory $\TFHI{n}$ of the sheaves $F$ in 
$\HI$ for which $\RHI[n]{F} = 0$. This proves the first claim
in the proposition. Since $\TFHI{n}$ form an ascending strong 
cofiltration, the second claim now follows.
\end{proof}

We still have to show that the torsion subcategories $\Cat{T}_n$
form a strong descending filtration. Let us first introduce a more
appropriate notation for the torsion subcategory.  

\begin{defn}\label{def_upper_slice_functor}
Let $\THI{n}$ denote the torsion subcategory $\Cat{T}_n$, and
$\tgHI{n}$ denote the kernel of the natural surjection $\id 
\to \tlHI{n}$. By Proposition \ref{prop_rad_eq_corad} and 
Corollary \ref{cor_tt_ref_and_coref}, $\tgHI{n}$ is an idempotent 
pre-radical, and is right adjoint to the inclusion of $\THI{n}$ in
$\HI$.
\end{defn}

We will now show that $(\THI{*}, \tgHI{*})$ defines a descending 
strong filtration on $\HI$.

\begin{lem}\label{lem_tgHI_reflection}
The essential image of $\tgHI{n}$ is $\THI{n}$, and the 
restriction of $\tgHI{n}$ to $\THI{n}$ is the identity.
\end{lem}
\begin{proof}
Recall from the definition of $\tgHI{n}$ that for each $F$
in $\HI$, there exists a short exact sequence:
\[
0 \to \tgHI{n} F \to F \to \tlHI{n} F \to 0.
\]
Furthermore, recall from Theorem \ref{thm_precorad_eq_tt} that
the for all $F$ in $\THI{n}$, $\tlHI{n} F = 0$. The lemma now
follows.
\end{proof}

\begin{lem}\label{lem_TFHI_properties}
For natural numbers $n$ and $m$ such that $m > n$, $\tlHI{m} 
\tlHI{n} = \tlHI{n}$ and there exist a
natural isomorphism $\tlHI{n}\tlHI{m} \cong \tlHI{n}$. 
\end{lem}
\begin{proof}
Suppose $F$ is in $\HI$. Since $\TFHI{n}$ is a
full subcategory of $\TFHI{m}$, and $\tlHI{m}$ is the identity on 
$\TFHI{m}$ (Lemma \ref{lem_tlHI_id}), we have $\tlHI{m}\tlHI{n} = 
\tlHI{n}$. It remains for us to show that $\tlHI{n}\tlHI{m} \cong
\tlHI{n}$.

We have the following commutative diagram:
\begin{equation}\label{eq_sg_nat_diag}
\begin{tikzcd}
\sgHI{n}\sgHI{m}(F) \arrow{r}\arrow{d}{\cuHI_{\sgHI{m}(F)}} &
\sgHI{n}(F) \arrow{r}\arrow{d}{\cuHI_F} &
\sgHI{n}\tlHI{m}(F) \arrow{r}\arrow{d}{\cuHI_{\tlHI{m}(F)}} &
0 \\
\sgHI{m}(F) \arrow{r} &
F \arrow{r} &
\tlHI{m}(F) \arrow{r}&
0,
\end{tikzcd}
\end{equation}
where the vertical arrows are the counits. Furthermore, by the 
same arguments as in the Snake Lemma, we have the ``snake tail'' 
exact sequence:
\[
\cok \cuHI_{\sgHI{m}(F)} \to \tlHI{n}(F) \to \tlHI{n}\tlHI{m}(F) 
   \to 0.
\]
However, since $\sgHI{m}F \in \LHI[m]{\HI}$, by Proposition
\ref{prop_unit_iso} $\cuHI_{\sgHI{m}(F)}$ is an isomorphism.
Therefore, the natural map $\tlHI{n}(F) \stackrel{\cong}{\to} 
\tlHI{n}\tlHI{m}(F)$ is an isomorphism.
\end{proof}

\begin{prop}\label{prop_THI_form_strong_filt}
The collection $(\THI{*}, \tgHI{*})$ form a descending strong 
filtration of $\HI$, i.e. we have the following descending
tower of subcategories
\[
\HI = \THI{0} \supseteq \THI{1} \supseteq \cdots \supseteq \THI{n} \supseteq \THI{n + 1}
\supseteq \cdots
\]
and coreflection functors $\tgHI{n} : \HI \to \THI{n}$ such
that $\tgHI{n}$ restricted to $\THI{n}$ is the identity, and
$\tgHI{n}(F)$ is a subobject of $F$ for all $n$.
\end{prop}
\begin{proof}
The only claim left to show is that $\THI{m} \subseteq \THI{n}$
for $n \leq m$.

Let $F$ be an object in $\THI{m}$. Then 
$\tlHI{m}(F) = 0$, and by Lemma \ref{lem_TFHI_properties}
\[
0 = \tlHI{n}\tlHI{m}(F) = \tlHI{n}(F).
\]
Thus, $F$ is in $\THI{n}$.
\end{proof}

We introduce the following notion to describe the strong filtration
and cofiltration on $\HI$ and its relationship to the coradicals
$\tlHI{*}$.

\begin{defn}
We call the strong filtration and cofiltration defined by
the torsion theories $(\TGHI{n}{\HI}, \TLHI{n}{\HI})$ for $n = 
0,1,2,\dots$ the \DEF{torsion filtration of $\HI$.}

In general, if $\Cat{A}$ is an abelian category, we say that
$\Cat{A}$ has a torsion filtration if there exists a sequence
of idempotent pre-(co)radicals $\corad{*}$ such that the induced
torsion theories $(\Prerad{n}{\Cat{A}}, \Corad{n}{\Cat{A}})$ (for 
$n$ in $\Z$) fit together to form a descending strong filtration
\[
\Cat{A} \supseteq \cdots \supseteq \Prerad{0}{\Cat{A}} \supseteq
   \Prerad{1}{\Cat{A}} \supseteq \cdots \supseteq \Prerad{n}{\Cat{A}}
   \supseteq \cdots
\]
and an ascending strong cofiltration
\[
0 \subseteq \cdots \subseteq \Corad{0}{\Cat{A}} \subseteq
   \Corad{1}{\Cat{A}} \subseteq \cdots \subseteq \Corad{n}{\Cat{A}}
   \subseteq \cdots.
\]
\end{defn}

We conclude this section by presenting some additional properties of
the torsion subcategories and the functor $\tgHI{n}$. Recall from
Proposition \ref{prop_HI_upper_slice} that $\sgHI{n}F = 
\LHI[n]{\RHI[n]{F}}$.

\begin{prop}\label{prop_THI_properties}
For all natural numbers $m$ and $n$ such that $m > n$,

\begin{enumerate}
\item $\THI{n}$ is the full subcategory of objects $F$ for which
the counit map $\sgHI{n}(F) \to F$ is onto.
\tinyskip

\item $\HI(n)$ is a proper full subcategory of $\THI{n}$.
\tinyskip

\item there exists a natural isomorphism between 
$\tlHI{n}\tgHI{m}$ and $\tgHI{m}\tlHI{n}$. Furthermore,
$\tgHI{n}\tlHI{m} = \tlHI{m}\tgHI{n} = 0$ 
\tinyskip

\item there exists natural isomorphisms:
$\tgHI{n}\tgHI{m} \cong \tgHI{m}\tgHI{n} \cong \tgHI{m}$.
\tinyskip
\end{enumerate}
\end{prop}
\begin{proof}
\pfitem{(1)} : For all $F$ in $\HI$ and $n \geq 0$, we have the 
following exact sequence
\[
\sgHI{n}(F) \to F \to \tlHI{n}(F) \to 0.
\]
Therefore, $\tlHI{n}(F) = 0$ if and only if $\sgHI{n}(F) 
\to F$ is a surjection.

\pfitem{(2)} : Let $F$ be an object in $\HI(n)$. Then $F \cong
\LHI[n]{F'}$ for some $F'$ in $\HI$. By Proposition
\ref{prop_counit_iso_for_HIn}, the counit map $\sgHI{n}(F) \to F$ is
an isomorphism.  By part (1), $F \in \THI{n}$.

\pfitem{(3)} : Let $F$ be a homotopy invariant sheaf with 
transfers. Since $\tgHI{m}(F) \in \THI{n}$, $\tlHI{n} \tgHI{m}(F) 
= 0$ by definition. Furthermore, $\tgHI{m}\tlHI{n}(F) = 0$ since 
it is the kernel of $\tlHI{m} \tlHI{n}(F) \to \tlHI{n}(F)$ 
which is an isomorphism by Lemma \ref{lem_tgHI_reflection}.

To show that $\tlHI{n} \tgHI{m}$ is naturally isomorphic to
$\tgHI{m} \tlHI{n}$, let us first consider the following 
diagram:
\begin{equation}\label{eq_sg_tgHI_nat_diag}
\begin{tikzcd}
{} &
\sgHI{m} \tgHI{n}(F) \arrow{r} \arrow{d} &
\sgHI{m}(F) \arrow{r} \arrow{d} &
\sgHI{m} \tlHI{n}(F) \arrow{r} \arrow{d} &
0 \\
0 \arrow{r} &
\tgHI{n}(F) \arrow{r} &
F \arrow{r} &
\tlHI{n}(F) \arrow{r}&
0
\end{tikzcd}
\end{equation}
where vertical maps are the counits. Notice that the top row is 
exact on the right because $\sgHI{m}$ is right exact.                   %% THESIS_FLAG
Since $m > n$, by Lemma \ref{lem_tlHI_in_TLHI},
$\tlHI{n}(F)$ is in $\TFHI{n}$, which is a subcategory of $\TFHI{m}$
by Proposition \ref{prop_tsubcat_eq_tlHI}. Since $G$ is in $\TFHI{m}$
if and only if $\RHI[m]{G} = 0$, it follows that 
$\RHI[m]{(\tlHI{n}(F))} = 0$. Hence, $\sgHI{m}\tlHI{n}(F) = 0$. 

Applying the Snake Lemma to \eqref{eq_sg_tgHI_nat_diag}, we obtain the 
following exact sequence:
\begin{equation}\label{eq_tlHI_exact_sequence}
0 \to \tlHI{m}\tgHI{n}(F) \to \tlHI{m}(F) \to \tlHI{m} \tlHI{n}(F) 
   \to 0.
\end{equation}
Now $\tlHI{n} \tlHI{m}(F) \cong \tlHI{n}(F)$, and the 
composition $\tlHI{m}(F) \to \tlHI{n}\tlHI{m} (F) \to \tlHI{n}(F)$ 
is precisely the natural surjection associated to $\tlHI{n}(F)$. It 
follows that
\[
\tlHI{m} \tgHI{n}(F) \cong \tgHI{n} \tlHI{m}(F).
\]
Since \eqref{eq_sg_tgHI_nat_diag} is natural in $F$, the isomorphism
is natural in $F$ as well.

\pfitem{(4)} : By Proposition \ref{prop_THI_form_strong_filt},
$\THI{m} \subseteq \THI{n}$. Since $\tgHI{n}$ restricted to
$\THI{n}$ is the identity by Lemma \ref{lem_tgHI_reflection}, 
$\tgHI{n} \tgHI{m} = \tgHI{m}$. 

To show that $\tgHI{m} \tgHI{n} \cong \tgHI{m}$, notice that
for a given $F$ in $\HI$ and positive integer $n$, there exists a 
commutative diagram
\begin{equation}\label{eq_tgHI_tlHI_diag}
\begin{tikzcd}
0 \arrow{r} &
\tgHI{n}(F) \arrow{r} \arrow{d}{\eta_{\tgHI{n}(F)}} &
F \arrow{r} \arrow{d}{\eta_F} &
\tlHI{n}(F) \arrow{r} \arrow{d}{\cong} &
0 \\
0 \arrow{r} &
\tlHI{m} \tgHI{n}(F) \arrow{r} &
\tlHI{m}(F) \arrow{r} &
\tlHI{m}\tlHI{n}(F) \arrow{r} &
0,
\end{tikzcd}
\end{equation}
where $\eta$ is the natural surjection $\id \to \tlHI{m}$, and
the bottom row is precisely the short exact sequence 
\eqref{eq_tlHI_exact_sequence}. By Lemma 
\ref{lem_TFHI_properties}, the map $\tlHI{n}(F) \to \tlHI{n} 
\tlHI{m}(F)$ is an isomorphism. Therefore, by the Snake Lemma, we 
have $\tgHI{m} \tgHI{n}(F) \cong \tgHI{m}(F)$. Since 
\eqref{eq_tgHI_tlHI_diag} is natural in $F$, it follows that the 
isomorphism $\tgHI{m}\tgHI{n} \to \tgHI{m}$ is natural as well.
\end{proof}

\section{Slice Filtration on $\DMeff$ and Torsion Filtration on $\HI$}
\label{sect_relation_sfilt_tfilt}

In this section, we want to relate the filtrations on $\HI$
that we have developed with the slice filtration on $\DMeff$. 
Recall that the slice filtration structure on $\DMeff$ is 
associated with the weak filtration $(\GFiltDM[*]{\DMeff}, 
\sgDM{*})$ and the weak cofiltration $(\LFiltDM[*]{\DMeff}, 
\slDM{*})$ (see Section \ref{sect_slice_filt_dm}). The main result 
that we will verify is Proposition \ref{prop_H_commute_with_filt}.
Recall from Definition \ref{def_t_struct_DMeff} that 
$\tau_{\leq 0}\DMeff$ is the full subcategory of negative objects in
$\DMeff$, i.e. the objects $M$ in $\DMeff$ such that $\HH^nM = 0$
for all $n > 0$.

\begin{prop}\label{prop_H_commute_with_filt}
For each positive integer $n$, the following diagram of functors 
commute, with surjective vertical arrows:
\[
\begin{tikzcd}
\GFiltDM[n]{\DMeff} \arrow{d}{\HH^0} &
\tau_{\leq 0}\DMeff \arrow{l}{\sgDM{n}} \arrow{r}{\slDM{n}} \arrow{d}{\HH^0} &
\LFiltDM[n]{\DMeff} \arrow{d}{\HH^0} \\
\HI(n) &
\HI \arrow{l}{\sgHI{n}}\arrow{r}{\tlHI{n}} &
\TFHI{n}.
\end{tikzcd}
\]
\end{prop}

The rest of the section will be devoted to the proof of
Proposition \ref{prop_H_commute_with_filt}. 
First, observe that for every positive integer $n$ and every $M$ 
in $\tau_{\leq 0}\DMeff$, there exists a slice triangle:
\[
\sliceTriangle{n}{M}
\]
Applying the cohomological functor $\HH^0$, we obtain the
following long exact sequence
\begin{equation}\label{eq_slice_DM_exact_seq}
\cdots \stackrel{\delta_{-1}}{\to} \HH^0 \sgDM{n}(M) \to 
   \HH^0 M \to \HH^0 \slDM{n}(M)
   \stackrel{\delta_0}{\to} \HH^1 \sgDM{n}(M) \to \cdots
\end{equation}
where $\HH^i M \defeq \HH^0M[i]$. 

Since $\ihomDMf(\Z(n)[n], -)$ is $t$-exact as shown in Lemma 
\ref{lem_HH_commutes_with_contract}, the cochain complex $\ihomDMf(\Z(n)[n], M)$
is also in $\tau_{\leq 0}\DMeff$. By Proposition \ref{prop_LR_commute_with_HH},
\begin{align*}
\HH^0 \sgDM{n}M &= \HH^0 (\ihomDMf(\Z(n)[n], M) \tDM \Z(n)[n]) \\
&\cong \LHI[n]{\HH^0(\ihomDMf(\Z(n)[n], M))} \\
&= \LHI[n]{\RHI[n]{\HH^0(M)}} \\
&= \sgHI{n}\HH^0M.
\end{align*}
This shows that the left square of Proposition 
\ref{prop_H_commute_with_filt} commutes, that $\HI(n)$ is equal to 
the image of $\tau_{\leq 0}\GFiltDM[n]{\DMeff}$ under $\HH^0$, and the 
coreflection functors from $\tau_{\leq 0}\DMeff$ to $\HI(n)$ is compatible with
$\HH^0$.

To prove the commutativity of the right square,
notice that, for $M$ in $\tau_{\leq 0}\DMeff$, 
$\HH^0\sgDM{n}M = \LHI[n]{\RHI[n]{(\HH^{0}M)}}$. Hence, we get 
the following exact sequence from \eqref{eq_slice_DM_exact_seq}
\[
\LHI[n]{(\RHI[n]{(\HH^0M)})} \to \HH^0M \to \HH^0{\slDM{n}(M)} 
   \stackrel{\delta_0}{\to} \HH^1 \sgDM{n}(M).
\]
where the map $\LHI[n]{(\RHI[n]{(\HH^0M)}} \to \HH^0M$ is the counit. If we show
that $\HH^1 \sgDM{n}(M) = 0$, then it is clear that $\HH^0 \slDM{n}(M)
\cong \tlHI{n}(\HH^0M)$. This shows that the right square of Proposition
\ref{prop_H_commute_with_filt} commutes, completing the proof of
Proposition \ref{prop_H_commute_with_filt}. The vanishing of 
$\HH^1\sgDM{n}$ is established by the following lemma:

\begin{lem}\label{lem_Hsg_vanishes}
For all positive integers $n$ and all $M$ in $\tau_{\leq 0}\DMeff$, $\HH^1 
\sgDM{n}(M) = 0$.
\end{lem}
\begin{proof}
Since $\sgDM{n} M = \Z(n)[n] \tDM \ihomDMf(\Z(n)[n], M)$, we have 
already shown in the preceding discussion that $\sgDM{n} M$
is a negative object. Therefore, $\HH^1 \sgDM{n} M = 0$, as desired.
\end{proof}

\section{Fundamental Invariants of the Torsion Filtration}

As in the case of $\DMeff$, we can also define the structure 
invariants associated to the filtration and cofiltration. In this
case, for every natural number $n$, there exists a functorial 
exact sequence
\[
\sgHI{n} \to \sgHI{n - 1} \to \tlHI{n}\sgHI{n - 1} \to 0.
\]
\begin{defn}\label{defn_sliceHI}
We define \DEF{$n$-th slice functor on $\HI$} to be the functor
$\slice{n} \defeq \tlHI{n + 1}\sgHI{n}$.
\end{defn}

Recall from Definition \ref{def_slice_functors_DMeff} that the 
$n$-th slice functor of $(\GFiltDM{\DMeff}, 
\sgDM{*})$ is the triangulated endofunctor $\sliceDM{*}$ that
fits into the following exact triangle
\[
\slDM{n + 1} \to \slDM{n} \to \sliceDM{n} \to \slDM{n + 1}[1].
\]
A consequence of Proposition \ref{prop_H_commute_with_filt} is 
that the slice functors $\slice{n}$ on $\HI$ agree 
with the slice functors $\sliceDM{n}$ on $\DMeff$ in the following
sense:

\begin{cor}\label{cor_H_commute_with_slice}
For all natural numbers $n$, the slice functors satisfy
\[
\HH^0 \sliceDM{n} = \slice{n}.
\]
\end{cor}
\begin{proof}
Applying $\HH^0$ to the functorial triangle from 
\eqref{eq_slice_triangle_3}, we obtain the following functorial 
exact sequence in $\HI$:
\[
\HH^0 \sgDM{n + 1} \to \HH^0 \sgDM{n} \to \HH^0 \sliceDM{n} 
   \to \HH^1 \sgDM{n + 1}.
\]
By Proposition \ref{prop_LR_commute_with_HH}, $\HH^0 \sgDM{n} =
\sgHI{n}$ and $\HH^0 \sgDM{n + 1} = \sgHI{n + 1}$, and by Lemma 
\ref{lem_Hsg_vanishes}, $\HH^1\sgDM{n + 1} = 0$. It follows that
$\HH^0 \sliceDM{n} = \slice{n}$ as desired.
\end{proof}

Let us first consider the following proposition:

\begin{prop}\label{prop_sg_tl_commute}
For natural numbers $m$ and $n$, $\sgHI{n} \tlHI{m}$ is naturally
isomorphic to $\tlHI{m} \sgHI{n}$, and are both 0 if $m \leq n$.
\end{prop}
\begin{proof}
Let $F$ be an object in $\HI$, and write $L$ for the functor
$F \mapsto \LHI{F}$ and $R$ for the functor $F \mapsto \RHI{F}$.
Since $\sgHI{n} = L^nR^n$, by Lemma \ref{lem_LR_commute_LR}, 
we have the commutative square
\begin{equation}\label{eq_prop_sg_tl_com_sq}
\begin{tikzcd}
\sgHI{m}\sgHI{n}(F) \arrow{r}{f} \arrow{d}{\cong} & 
\sgHI{n}(F) \arrow[equal]{d} \\
\sgHI{n}\sgHI{m}(F) \arrow{r}{g} &
\sgHI{n}(F),
\end{tikzcd}
\end{equation}
where $f$ is the counit of $\sgHI{m}\sgHI{n}(F) \to \sgHI{n}(F)$ and 
$g$ is obtained by applying $\sgHI{n}$ to the counit $\sgHI{m}(F) 
\to F$. The cokernel of $f$ is precisely $\tlHI{m} \sgHI{n}(F)$.
Since $\sgHI{n}$ is right exact, and the
following sequence is exact
\[
\sgHI{m}(F) \to F \to \tlHI{m}(F) \to 0,
\]
the following sequence is also exact.
\[
\sgHI{n} \sgHI{m}(F) \to \sgHI{n}(F) \to \sgHI{n} \tlHI{m}(F) 
   \to 0.
\]
It follows that the cokernel of $g$ is $\sgHI{n} \tlHI{m}(F)$.  By the
Five Lemma \ref{prop_sg_tl_commute}, $\tlHI{m} \sgHI{n}(F) \cong
\sgHI{n} \tlHI{m}(F)$. Since the square in Lemma
\ref{lem_tlHI_in_TLHI} is functorial, it follows that the isomorphism
identified above is natural in $F$.

Finally, suppose $m \leq n$. Then by Proposition 
\ref{prop_HI_lower_slice} $\RHI[n]{\tlHI{m}(F)} = 0$. It follows 
that $\sgHI{n} \tlHI{m}(F) = 0$, and $\tlHI{m} \sgHI{n}(F) = 0$ as 
well.
\end{proof}

\begin{rmk}
  In case the indexing becomes difficult to keep track, one might wish
  to consider a ``bread'' analogy. Imagine that a half-infinite loaf
  of bread is laid out on a line marked from 0 to $\infty$
  (representing an $F$ in $\HI$), and one is allowed to take cuts at
  the marked points and subsequently pick up all the bread lying
  greater than $n$ or less than $n$. For the functors $\tgHI{n}$ and
  $\sgHI{n}$, the higher the $n$, the less bread one would
  \emph{take}. For the functors $\tlHI{n}$, the greater the $n$, the
  less bread one would \emph{leave}.

If one finds the analogy useful, one might wish to interpret
Lemma \ref{lem_TFHI_properties}, and Propositions
\ref{prop_THI_properties} (3) and (4) with this culinary picture 
in mind.
\end{rmk}

As we did for the filtration $(\HI(*), \sgHI{*})$ in Definition
\ref{defn_sliceHI}, we can define the structure invariants for
$(\THI{*}, \tgHI{*})$.
\begin{defn}
For each $F$ in $\HI$ and natural number $n$, write $\tconst{n}$ 
for the functor $\tlHI{n + 1} \tgHI{n}$, which we define to be the 
\DEF{$n$-th fundamental invariant of $F$ associated to $\tgHI{*}$}
or simply the \DEF{$n$-th fundamental invariant}.
\end{defn}

As it turns out, the $n$-th fundamental invariant is \emph{not} 
the same as the $n$-th slice functor on $\HI$. To see this, 
consider the example introduced in Example
\ref{rmk_sgHI_is_not_strong_filt}. For $\Oxn$, following the
discussion in \loccit, we have that
\[
\slice{k}(\Oxn) = \slice{k}(\Ox) = \begin{cases}
\Ox &\textrm{if }k = 1\\
0   &\textrm{otherwise}.
\end{cases}
\]
However, a simple calculation reveals that
\[
\tconst{k}(\Oxn) = \begin{cases}
\Oxn &\textrm{if }k = 1\\
0     &\textrm{otherwise}.
\end{cases}
\]
Nonetheless, the $n$-th slice functor is related to the $n$-th
fundamental invariant via the following proposition:

\begin{prop}\label{prop_struct_consts}
Let $m$ and $n$ be natural numbers such that $m > n$. There exists 
a natural surjection from $\tlHI{m} \sgHI{n}$ to $\tlHI{m} 
\tgHI{n}$. In particular, for each $F$ in $\HI$, there exists a 
natural surjection $\pi_m: \slice{m} F \to \tconst{m} F$.
\end{prop}
\begin{proof}
Let $F$ be an object of $\HI$. We have the following short exact
sequence:
\[
0 \to \tgHI{n} \tlHI{m}(F) \to \tlHI{m}(F) \to 
  \tlHI{n} \tlHI{m}(F) \to 0.
\]
By Lemma \ref{lem_TFHI_properties}, $\tlHI{m} \tlHI{n}(F) 
= \tlHI{n}(F)$, and therefore $\tgHI{n} \tlHI{m}(F)$ is the kernel 
of the surjection $\tlHI{m}(F) \to \tlHI{n}(F)$. But the sequence
\[
\sgHI{n} \tlHI{m}(F) \to \tlHI{m}(F) \to \tlHI{n}(F) \to 0
\]
is exact. Therefore, the induced map from $\sgHI{n} \tlHI{m}(F)$
to $\tgHI{n} \tlHI{m}(F)$ is a surjection as well. Furthermore,
since the commutative diagram
\[
\begin{tikzcd}
{} & \sgHI{n} \tlHI{m}(F) \arrow{r} \arrow[twoheadrightarrow]{d} &
\tlHI{m}(F) \arrow{r} \arrow[equal]{d} &
\tlHI{n}(F) \arrow{r} \arrow[equal]{d} &
0 \\
0 \arrow{r} &
\tgHI{n} \tlHI{m}(F) \arrow{r} &
\tlHI{m}(F) \arrow{r} &
\tlHI{n}(F) \arrow{r} &
0
\end{tikzcd}
\] 
is functorial in $F$, the surjection is natural. This establishes the
first claim of the proposition, since $\tgHI{n} \tlHI{m}$ is naturally
isomorphic to $\tlHI{m} \tgHI{n}$ (Proposition
\ref{prop_THI_properties} (3)) and $\sgHI{n} \tlHI{m}$ is naturally
isomorphic to $\tlHI{m} \sgHI{n}$ (Proposition
\ref{prop_sg_tl_commute}). The second claim follows by setting $n = m
- 1$.
\end{proof}

\section{Weakly Filtered Monoidal Structure on $\HI$}

We end this chapter by discussing the tensorial properties of the 
torsion filtration. Let us first consider the following notion:

\begin{defn}\label{def_graded_tensor}
Let $(\Cat{C}, \tensor, \Unit)$ be a monoidal category. We say 
that $\Cat{C}$ is a \DEF{weakly filtered monoidal category} if there 
exists a weak filtration $(\Cat{C}_*, \phi_*)$ such that for all
integers $m$ and $n$, $\Cat{C}_m \tensor \Cat{C}_n \subseteq
\Cat{C}_{m + n}$.
\end{defn}

\begin{ex}
Here are two examples of weakly filtered monoidal categories
that we have encountered in this thesis. Recall from Definition 
\ref{def_GFiltDM} that $\GFiltDM[k]{\DM}$ is the full subcategory
of the objects $(M, n)$ in $\DM$ such that $n \geq k$. For $(M, n)$
in $\GFiltDM[k]{\DM}$ and $(M', n')$ in $\GFiltDM[l]{\DM}$, $(M, n) \tDM (M', n')$
is equal to $(M \tensor M', n + n')$, which is an object in $\GFiltDM[k + l]{\DM}$.
Therefore, the triangulated tensor product on $\DM$ is weakly filtered
by $(\GFiltDM[*]{\DM}, \sgDM{*})$.

Similarly, $(\HI(*), \sgHI{*})$ defines a graded symmetric 
monoidal category on $\HI$ under $\tHI$. To see this, recall
from the first paragraph in Section \ref{sect_torsion_filt_on_HI}
that $F$ is in $\LHI[n]{\HI}$ if $F \cong \LHI[n]{F'}$. 
Furthermore, since $\LHI[n]{F} = F \tHI (\Ox)^{\tensor n}$, 
$\LHI[n]{F} \tHI \LHI[m]{G} = \LHI[n + m]{(F \tHI G)}.$ Therefore, 
$\LHI[n]{\HI} \tHI \LHI[m]{\HI} \subseteq \LHI[n + m]{\HI}$.
\end{ex}

We now will show that $(\THI{*}, \tgHI{*})$ defines a weakly filtered 
monoidal category on $\HI$. We begin by proving the following
proposition:

\begin{prop}\label{prop_tensor_and_tfilt_HI}
For $F$ in $\THI{n}$ and $G$ in $\THI{m}$, $F \tHI G$
is an object of $\THI{n + m}$.
\end{prop}
\begin{proof}
Since $\THI{n + m}$ is the torsion subcategory associated to the
coradical $\tlHI{n + m}$, to show that $F \tHI G$ is in 
$\THI{n + m}$, it suffices to show that $\tlHI{n + m}(F \tHI G) = 
0$. Since $G$ is in $\THI{n}$, by Proposition 
\ref{prop_THI_properties}(1), the counit $\epsilon: L^mR^m(G) \to 
G$ is surjective. Since $\tDM$ is right $t$-exact in both factors, 
the functor $F \tHI -$ is right exact by Proposition 
\ref{prop_t_exact_implies_exact}, and the following map is surjective:
\begin{equation}\label{eq_tensor_tfilt_HI_1}
F \tHI L^mR^m(G) \xrightarrow{\epsilon_F \tHI G} F \tHI L^mR^m(G).
\end{equation}
Similarly, we see that the following map is
also surjective:
\begin{equation}\label{eq_tensor_tfilt_HI_2}
F \tHI L^mR^m(G) \xrightarrow{L^nR^n(F) \tHI \epsilon_G}
F \tHI G.
\end{equation}
Composing \eqref{eq_tensor_tfilt_HI_1} and 
\eqref{eq_tensor_tfilt_HI_2}, we obtain a surjection
\[
f: L^nR^n(F) \tHI L^mR^m(G) \to F \tHI G.
\]
On the other hand, since $L^nR^n(F) \tHI L^mR^m(G) = 
L^{n + m}(R^n(F) \tHI R^m(G))$, the object $L^nR^n(F) \tHI L^mR^m(G)$
is in $\LHI[n + m]{\HI}$, and by Proposition
\ref{prop_THI_properties}, 
\[
\tlHI{n + m}(L^nR^n(F) \tHI L^mR^m(G)) = 0. 
\]
Since $\tlHI{n + m}$ is a coradical, which is right exact, the map
\[
\tlHI{n + m}(f) : \tlHI{n + m}(L^nR^n(F) \tHI L^mR^m(G)) \to
   \tlHI{n + m}(F \tHI G)
\]
is onto. Therefore, $\tlHI{n + m}(F \tHI G) = 0$ and 
$F \tHI G$ is an object in $\THI{n + m}$, as desired.
\end{proof}

The following is a direct consequence of Proposition 
\ref{prop_tensor_and_tfilt_HI}.

\begin{cor}\label{cor_graded_tensor_HI}
  Let $\tHI$ be the tensor product on $\HI$ defined in Definition
  \ref{def_t_struct_DMeff}. The strong filtration $(\THI{*},
  \tgHI{*})$ makes $(\HI, \tHI)$ into a weakly filtered monoidal
  category.

 There exists a weakly filtered symmetric
  monoidal structure on $\HI$ by $(\THI{*}, \tgHI{*})$.
\end{cor}


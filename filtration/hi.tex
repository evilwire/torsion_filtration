\newpage
\section{Filtration on $\HI$}\label{sect_filtration_hi}

Recall that a weak filtration of a category $\Cat{C}$ is a tower 
of subcategories
\[
\cdots \subset F_n\Cat{C} \subset F_{n - 1}\Cat{C} \subset \cdots 
   \subset \Cat{C}
\]
together with reflection functors $f_n : \Cat{C} \to F_n\Cat{C}_n$
(cf. \ref{def_cat_filtration}.) If we have a suitable notion of
subobjects in $\Cat{C}$, e.g. if $\Cat{C}$ is abelian, we can 
define a stronger notion (as the name suggests).

For the remainder, let $\Cat{C}$ be an abelian category. We have:

\begin{defn}
We say that a filtration $(\Cat{C}, f_*)$ is \emph{strong} if for
each $A \in \Cat{C}$ and $n \in \Z$, $f_n A$ is a subobject of 
$A$. Similarly a cofiltration $(\Cat{C}, c_*)$ is a \emph{strong} 
if $c_n A$ is a quotient of $A$ for each $n$.
\end{defn}

The purpose of this section and the next is to construct a 
strong filtration on the category $\CycMod$. To do this, we first
construct a weak filtration structure on $\HI$ using a pair
of adjoint functors obtained from the closed symmetric monoidal 
structure of $\HI$. We adopt the same convention as in Sections
\ref{sect_heart_struct} and \ref{sect_slice_filt_dm}: we identify 
$\DMeff$ with the full subcategory of $\DSh$ with homotopy 
invariant cohomologies, and identify $\HI$ with the heart of 
$\DMeff$ under the $t$-structure induced by that of $\DSh$. Once
again, let $\sheaf{H}^0$ denote the reflection functor $\DMeff
\to \HI$.

Recall that the sheaf $\Ox$ of units given by 
\[
\Ox(X) = \{\textrm{invertible elements of }\O(X)\}
\]
is a homotopy invariant sheaf with transfers. For $F \in \HI$, 
set
\[
\LHI{F} = F \tHI \Ox \hspace{10pt} \textrm{and}\hspace{10pt} 
   \RHI{F} = \ihomHI(\Ox, F).  
\]
We write $\LHI[n]{F}$ for $\LHI{(\LHI[n - 1]{F})}$ and 
$\RHI[n]{F}$ for $\RHI{(\RHI[n - 1]{F})}$.

\begin{rmk}
In literature, the notation $\RHI{F}$ is often used to represent
the contraction of $F$, which is the sheaf that sends $X \in 
\Cor$ to 
\[
\cok( F(X \times \A^1) \to F(X \times (\A^1 - 0))).
\]
In fact, there is no ambiguity here, since the contraction of
$F$ is isomorphic to $\ihomHI(\Ox, F)$ by Prop. 
\ref{prop_contraction}.
\end{rmk}

The following proposition highlights the connection between the
pair of adjoint functors in the construction of $\DMeff$ (cf.
Section \ref{sect_slice_filt_dm}).

\begin{prop}\label{prop_LRDM_eq_LRHI}
Fix any $F \in \HI$, which we also consider as an object of 
$\DMeff$. Then we have $\sheaf{H}^0(F \tDM \Z(1)[1]) \simeq 
\LHI{F}$ and $\sheaf{H}^0\ihomDMf(\Z(1)[1], F) \simeq \RHI{F}$.
\end{prop}
\begin{proof}
By Prop. \ref{prop_Z1_eq_Ox}, $\Z(1)[1] \simeq \Ox$ in $\DSh$
and hence in $\DMeff$ as well. The isomorphisms now follow
from the definitions of $\tHI$ and $\ihomHI$.
\end{proof}

As constructed, the functor $F \mapsto \LHI{F}$ is left adjoint to 
$F \mapsto \RHI{F}$. In this case,

\begin{prop}\label{prop_unit_eq}
Let $F \in \HI$. The unit map $F \to \RHI{(\LHI{F})}$ is an
isomorphism.
\end{prop}
\begin{proof}
Given $F \in \HI$, consider $F$ as an object in $\DMeff$. By
Cancellation Theorem (Thm. \ref{thm_dm_cancellation}), we have
that $\ihomDMf(\Z(1)[1], F(1)[1]) \simeq \ihomDMf(\Z, F) 
\simeq F$. Now apply $\sheaf{H}^0$ to this chain of isomorphisms,
and note that $\sheaf{H}^0(F) = F$. The proposition follows from 
Prop. \ref{prop_LRDM_eq_LRHI}.
\end{proof}

Now consider the counit map $\cuHI_F: \LHI[n]{\RHI[n]{F}} \to F$, 
and write $\sgHI{n} F$ for the cokernel of $\cuHI_F$. We want to 
show that the functors $F \mapsto \LHI[n]{\RHI[n]{F}}$ 
defines a weak filtration of and $\sgHI{n}$ defines a
strong cofiltration of $\HI$.

To see this, let $\LHI[0]{\HI} = \HI$ and let $\LHI[n]{\HI}$ 
denote the full subcategory of $F \in \HI$ where $F = \LHI[n]{F'}$ 
for some $F' \in \HI$. It is clear that if $m \geq n$, then 
$\LHI[m]{\HI} \subseteq \LHI[n]{\HI}$. In particular, we have a 
tower of subcategories
\[
\HI = \LHI[0]{\HI} \supset \LHI[1]{\HI} \supset \LHI[2]{\HI} 
\subset \cdots.
%\cdots \subset \LHI[2]{\HI} \subset \LHI[1]{\HI} \subset \LHI[0]{\HI} 
% = \HI 
\]
Since for the constant sheaf $\Z$, $\RHI{\Z} = 0$, by Prop.
\ref{prop_unit_eq}, $\LHI[1]{\HI} \neq \HI$. Inductively, we
see that $\LHI[n]{\HI} \neq \LHI[n + 1]{\HI}$.

\begin{prop}
Let $\sgHI{n}$ denote the functor $F \mapsto 
\LHI[n]{(\RHI[n]{F})}$. Then $\sgHI{n}$ is left adjoint to the 
inclusion of $\LHI[n]{\HI}$. In particular, $(\HI, \sgHI{n})$ 
defines a weak filtration of $\HI$.
\end{prop}

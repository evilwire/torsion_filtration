\section{Graded monoidal structures on torsion monoidal categories}

Finally, the descending filtrations, together with
the respective symmetric monoidal structure, give rise to a
graded monoidal structure (see Definition 
\ref{def_graded_tensor}). In order to formulate our final set of 
results, let us first consider the following proposition which 
follows straightforward from the Cancellation property (Definition
\ref{def_torsion_monoidal_category} (2)):

\begin{prop}
There exists a fully faithful $t$-exact functor $i: \DCat \to 
\DCatLoc{S}$, given by $M \mapsto (M, 0)$ that realizes $\DCat$ 
as a full subcategory of $\DCatLoc{S}$. The restriction of $i$ 
to the heart of $\DCat$ induces a corresponding fully faithful 
functor $i : \Cat{C} \to \CatLoc{C}{S}$.
\end{prop}

By definition of $i$, $\DCat$ is identified with the full 
subcategory $\Prerad{0}{\DCatLoc{S}}$ of $\DCatLoc{S}$, and
$\Cat{C}$ is identified with $\Prerad{0}{\CatLoc{C}{S}}$.
The following is an easy of the definition:

\begin{thm}\label{thm_graded_monoid_DCat}
There is a graded monoidal structure on $\DCatLoc{S}$ defined by
the filtration $(\Prerad{*}{\DCatLoc{S}}, \preradD{*})$ and a 
graded monoidal structure on $\DCat$ defined by the filtration
$(\Prerad{*}{\DCat}, \preradD{*})$. The two graded monoidal 
structures respects the inclusion of $\DCat$ into $\DCatLoc{S}$ in 
the sense that, for all natural numbers $n$ and $m$, the following 
functor diagram commutes:
\[
\begin{tikzcd}
\Prerad{n}{\DCat} \tensor \Prerad{m}{\DCat} \arrow{r} \arrow{d}{i} &
\Prerad{n + m}{\DCat} \arrow{d}{i} \\
\Prerad{n}{\DCatLoc{S}} \tensor \Prerad{m}{\DCatLoc{S}} \arrow{r} &
\Prerad{n + m}{\DCatLoc{S}}
\end{tikzcd}
\]
\end{thm}

By applying $\HH^0$ to the diagram in Theorem 
\ref{thm_graded_monoid_DCat}, we obtain the following:

\begin{cor}
There is a graded monoidal structure on $\CatLoc{C}{S}$ defined
by the filtration $(\SGFilt{*}{\CatLoc{C}{S}}, \sgHI{*})$ and a graded 
monoidal structure on $\Cat{C}$ defined by $(\SGFilt{*}{\Cat{C}}, 
\sgHI{*})$. That is, for natural numbers $n$ and $m$, the 
following functor diagram commutes:
\[
\begin{tikzcd}
\SGFilt{n}{\Cat{C}} \tensor \SGFilt{m}{\Cat{C}} \arrow{r} \arrow{d}{i} &
\SGFilt{n + m}{\Cat{C}} \arrow{d}{i} \\
\SGFilt{n}{\CatLoc{C}{S}} \tensor \SGFilt{m}{\CatLoc{C}{S}} \arrow{r} &
\SGFilt{n + m}{\CatLoc{C}{S}}
\end{tikzcd}
\]
\end{cor}

Finally, the arguments for Proposition \ref{prop_graded_mon_struct_HM} go 
through to show that we also have a graded monoidal structure 
associated with the torsion filtration:

\begin{thm}
There is a graded monoidal structure on $\CatLoc{C}{S}$ defined
by the strong filtration $(\Prerad{*}{\CatLoc{C}{S}}, \prerad{*})$ 
and a graded monoidal structure on $\Cat{C}$ defined by 
$(\Prerad{*}{\Cat{C}}, \prerad{*})$. As in the case for $\DCat$ and 
$\DCatLoc{S}$, the two graded monoidal structures are compatible 
via the inclusion of $\Cat{C}$ into $\CatLoc{C}{S}$. That is, for 
natural numbers $n$ and $m$, the following functor diagram 
commutes:
\[
\begin{tikzcd}
\Prerad{n}{\Cat{C}} \tensor \Prerad{m}{\Cat{C}} \arrow{r} \arrow{d}{i} &
\Prerad{n + m}{\Cat{C}} \arrow{d}{i} \\
\Prerad{n}{\CatLoc{C}{S}} \tensor \Prerad{m}{\CatLoc{C}{S}} \arrow{r} &
\Prerad{n + m}{\CatLoc{C}{S}}.
\end{tikzcd}
\]
\end{thm}

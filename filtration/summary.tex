\newpage
\chapter{Torsion Filtrations on Torsion Monoidal Categories}
\label{sect_filtration_general}

In this chapter, we generalize the key results proven in the last
three chapters by axiomatizing the necessary components to define
torsion filtrations on $t$-categories $\DCat$ with a triangulated 
tensor structure. Let us begin by defining the following notion:

\begin{defn}\label{def_torsion_monoidal_category}
Let $(\DCat, \tensor, \Unit)$ be a tensor monoidal category with 
a $t$-structure, and let $\Cat{C}$ be its heart. We say that 
$\DCat$ is a \DEF{torsion monoidal category} if $\DCat$ is 
equipped with 
\begin{enumerate}
\item (\itemhead{Partial Internal Hom}) a partial internal hom
structure $(\ihom, \DCat^{\compact})$ (see Definition
\ref{def_tensor_triang_cat}).

\item (\itemhead{Tate Object}) an object $S$ in both 
   $\DCat^{\compact}$ and the heart of $\DCat$ called the 
   \DEF{Tate object}. In particular, $\ihom(S, -)$ is right
   adjoint to $S \tensor -$.
\end{enumerate}

\noindent such that the following conditions hold:

\begin{enumerate}
\item $\Unit$ is an object of $\Cat{C}$,

\item $\tensor$ is right $t$-exact in both factors,

\item (\itemhead{Cancellation}) $\ihom(S, S \tensor M) = M$,

\item $\ihom(S, -)$ is $t$-exact.
\end{enumerate}
\end{defn}

If $\DCat$ is a torsion monoidal category, we will write $\HH^0$
for the cohomological functor from $\DCat$ to its heart. We also 
write $L: \DCat \to \DCat$ for the functor sending an object $M$ 
in $\DCat$ to $M \tensor S$, and $R$ for the functor sending $M$
to $\ihom(S, M)$, where $S$ is the Tate object. By assumption,
$(L, R)$ is an adjoint pair. Let $L^n$ and $R^n$ denote the $n$-th 
iterations of $L$ and $R$ respectively. Since $L$ is left adjoint 
to $R$, $L^n$ is left adjoint to $R^n$. Furthermore, by the
Cancellation axiom (Definition 
\ref{def_torsion_monoidal_category}(3)), $R^nL^n$ is naturally 
isomorphic to the identity.

Since $\tensor$ is right $t$-exact, it induces a symmetric 
monoidal and a partial internal hom structure on the heart 
$\Cat{C}$ of $\DCat$, which we represent by $\tensor^{\Cat{C}}$ 
and $\ihom_{\Cat{C}}$. The tensor and internal hom bifunctors
are given by
\[
C \tensor^{\Cat{C}} C' \defeq \HH^0(C \tensor C') 
   \;\;\;\textrm{ and }\;\;\;
\ihom_{\Cat{C}}(C, C') \defeq \HH^0(\ihom(C, C')).
\]

Since $\ihom(S, -)$ is assumed to be $t$-exact, Proposition 
\ref{prop_ihomC_is_partial_ihom} states that $S$ is a \SemiInvertible
object of $\Cat{C}$, i.e. $\ihom_{\Cat{C}}(S, -)$ is right
adjoint to $- \tensor^{\Cat{C}} S$. We let $L_H$ and $R_H$ denote 
the endofunctor on $\Cat{C}$ given by $F \mapsto F 
\tensor^{\Cat{C}} S$ and $F \mapsto \ihom_{\Cat{C}}(S, F)$ 
respectively. By convention, let $L_H^0$ and $R_H^0$ be the 
identity functor on $\Cat{C}$, and let $L_H^n$ and $R_H^n$ denote 
the $n$-th iteration of $L_H$ and $R_H$ respectively. Since $L_H$ 
is left adjoint to $R_H$, $L_H^n$ is left adjoint to $R_H^n$ for
every integer $n > 0$.

Here are some additional results about the functors $L_H$ and 
$R_H$ that we will refer to throughout the remainder of this 
chapter. The following proposition generalizes results from
Lemma \ref{lem_HH_commutes_with_contract}, Proposition 
\ref{prop_LR_commute_with_HH}, and Proposition 
\ref{prop_unit_iso}.

\begin{prop}\label{prop_LH_RH_properties}
For all integers $n > 0$,
\begin{enumerate}
\item $R_H^n$ is an exact functor,

\item there exists a natural isomorphism between $R^n$
and $R_H^n$ as endofunctors on $\HI$,

\item there exists a natural isomorphism between $\HH^0L^n$
and $L_H^n$ as endofunctors of $\HI$,

\item there exists a natural isomorphism from $\id$ to 
$R_H^nL_H^n$ as endofunctors on $\HI$.
\end{enumerate}
\end{prop}
\begin{proof}
Since $R$ is $t$-exact, $R_H = \HH^0 R$ is exact as an endofunctor 
on $\Cat{C}$ by Proposition \ref{prop_t_exact_implies_exact}. 
Since composition of exact functors is exact, $R_H^n$ is exact. 
This proves part (1).

To verify (2), we proceed by induction on $n$. The case $n = 1$
follows by definition. Now suppose $\HH^0R^{n - 1}$ is naturally
isomorphic to $R_H^{n - 1}$. Since $R$ is $t$-exact, by 
\cite[1.3.17(ii)]{BBD}, there exists a natural isomorphism between 
$\HH^0R$ and $\HH^0R\HH^0$. Therefore, we obtain the following chain
of natural isomorphisms: $R_H^n \cong \HH^0R\HH^0R^{n - 1} \cong
\HH^0R^n$. Since $R$ is $t$-exact, so is $R^n$. Furthermore, by
definition of $t$-exactness, for all $C$ in $\Cat{C}$, $R^n(C)$ is 
an object of $\Cat{C}$. It follows that $\HH^0 R^n = R^n$.

For (3), since $S$ is in $\Cat{C}$ and $\tensor$ is right 
$t$-exact in both factors, $\HH^0L$ is naturally isomorphic to 
$\HH^0L\HH^0$ as functors on $\Cat{C}$ by \cite[1.3.7(ii)]{BBD}. 
Using similar inductive arguments as in (2), we obtain a natural 
isomorphism between $\HH^0L^n$ and $L_H^n$. 

Since the unit $\id \to R^nL^n$ is a natural
isomorphism in $\DCat$, $\HH^0 \to \HH^0R^nL^n$ is also 
a natural isomorphism. Notice that $\HH^0$ is the identity
functor on $\Cat{C}$ and by part (2) and (3) $\HH^0R^nL^n$ is
naturally isomorphic to $R_H^nL_H^n$. The composition
\[
\id \to \HH^0R^nL^n \to R_H^nL_H^n
\] 
gives us the desired natural isomorphism, which proves part (4).
\end{proof}

\section{Slice filtration of torsion monoidal categories}
\label{sect_summary_slice_filt_on_DCat}

We begin by constructing the slice filtration on $\DCat$. This will
generalize the results in Section
\ref{sect_slice_filtration_DMeff}.

\begin{defn}\label{def_slice_filt_general}
Let $\Corad{n}{\DCat}$ be the full subcategory of $\DCat$ consisting of the objects
$M$ in $\DCat$ for which $R^n M = 0$. Let $\Prerad{n}{\DCat}$ be 
the subcategory of objects $M$ such that $M = L^n(M')$ for some 
$M'$ in $\DCat$. Let $\preradD{n}$ be the functor  $L^nR^n$.
\end{defn}

Notice that the arguments in the proof of \cite[1.1]{HuKa} rely 
only on the Cancellation axiom of $\DMeff$, which is fulfilled by 
Definition \ref{def_torsion_monoidal_category}(3) of $\DCat$. 
Therefore, the proof of \emph{loc. cit.} generalizes to show that 
$\preradD{n}$ is right adjoint to the inclusion of 
$\Prerad{n}{\DCat}$ into $\DCat$. 

Let $M$ be an object of $\DCat$, and let $\eta^n : \preradD{n} M 
\to M$ be the counit. Complete $\eta^n$ to a triangle:
\[
\preradD{n} M \to M \to M' \to \preradD{n} M'[1].
\]
Copying the proof of \cite[1.3]{HuKa}, we see that $M'$ is 
uniquely determined up to unique isomorphism, and $M \mapsto
M'$ defines a triangulated endofunctor $\coradD{n}$ that is
left adjoint to the inclusion of $\Corad{n}{\DCat}$ in $\DCat$.
Finally, the discussion in the paragraphs preceding Proposition 
\ref{prop_slice_DMeff} can be adapted to this more general setting
to show that $\preradD{n}$ restricted to $\Prerad{n}{\DCat}$ is 
naturally isomorphic to the identity, and $\coradD{n}$ restricted 
to $\Corad{n}{\DCat}$ is also naturally isomorphic to the 
identity. We have just verified the following theorem, which is a 
generalization of Proposition \ref{prop_slice_DMeff}:

\begin{thm}\label{thm_summary_triang_cat}
If $\DCat$ is a torsion monoidal category, then there exists an
$\N$-indexed ascending weak filtration $(\Corad{*}{\DCat},
\coradD{*})$ given by
\[
0 = \Corad{0}{\DCat} \subseteq \cdots \subseteq \Corad{n}{\DCat}
   \subseteq \Corad{n + 1}{\DCat} \subseteq \cdots
\]
and a descending weak filtration $(\Prerad{*}{\DCat}, \preradD{*})$
given by
\[
\DCat = \Prerad{0}{\DCat} \supseteq \cdots \supseteq \Prerad{n}{\DCat}
\supseteq \Prerad{n + 1}{\DCat} \supseteq \cdots
\]
\end{thm}

It is possible that the weak filtrations are degenerate. However, as 
the following result shows, the filtration being degenerate is 
related to the invertibility of $S$. Recall that $S$ is 
invertible if there exists an object $T$ in $\DCat$ such that $T 
\tensor S = \Unit$.

\begin{prop}\label{prop_filt_trivial_cond}
The following are equivalent:
\begin{enumerate}
\item the filtration $(\Corad{*}{\DCat}, \coradD{*})$ is trivial,
  i.e., each $\Corad{n}{\DCat}$ is zero.

\item the filtration $(\Prerad{*}{\DCat}, \preradD{*})$ is 
degenerate with $\Prerad{n}{\DCat} = \DCat$ for all $n$.

\item the Tate object is invertible in $\DCat$.
\end{enumerate}
\end{prop}
\begin{proof}
We first show that (1) is equivalent to (2). To see that (1) implies
(2), suppose $\Corad{n}{\DCat} = 0$ for all $n$. We need to show
that every $M$ in $\DCat$ is isomorphic to $L^n M'$ for some $M'$
in $\DCat$. However, for every $M$ in $\DCat$, the following is
a distinguished triangle:
\[
\preradD{n}M \to M \to \coradD{n}M \to \preradD{n} M[1]
\]
where $\coradD{n}M$ is in $\Corad{n}{\DCat}$. But the assumption that
$\Corad{n}{\DCat} = 0$ implies that $\coradD{n}M = 0$. Therefore, $L^n R^n M
\cong M$. It follows that $M$ is in $\Prerad{n}{\DCat}$, and
$\Prerad{n}{\DCat} = \DCat$ as desired. Conversely, if
$\Prerad{n}{\DCat} = \DCat$ then for every $M$ in $\DCat$, $M \cong
L^n M'$ for some $M'$. Suppose $M$ is in $\Corad{n}{\DCat}$, then by
definition $0 = R^n M = R^nL^n M' \cong M'$.  Therefore, $M = L^n 0 =
0$, and $\Corad{n}{\DCat} = 0$.

Now we show that (2) is equivalent to (3). Indeed, if $S$ is 
invertible with inverse $T$, then $M = S^{\tensor n} \tensor 
T^{\tensor n} \tensor M = L^n(T^n \tensor M)$ which is an object 
of $\Prerad{n}{\DCat}$. Conversely, if $\Prerad{n}{\DCat} = 
\DCat$, then, in particular, $\Prerad{1}{\DCat} = \DCat$. This
implies that the unit object $\Unit$ is an object of $\Prerad{1}{\DCat}$.
In other words, $\Unit = T \tensor S$ for some $T$, which shows
that $S$ is invertible.
\end{proof}

\section{Torsion filtration on the heart}

Let us now focus on the heart $\Cat{C}$ of $\DCat$. In this 
section, we will generalize the results developed in Section
\ref{sect_torsion_filt_on_HI} for $\HI$. Recall from Proposition 
\ref{prop_LH_RH_properties} and preceding paragraphs that the 
endofunctors $L_H^n = \HH^0 L^n$ and $R_H^n = \HH^0 R^n$ are 
adjoint. For $F$ in $\Cat{C}$, let $\corad{n} F$ denote the 
cokernel of the counit map $L_H^n R_H^n F \to F$. Since the counit
is natural in $F$, $F \mapsto \corad{n}F$ defines an endofunctor 
of $\Cat{C}$. Let $\Corad{n}{\Cat{C}}$ be the full subcategory of 
all objects $C$ in $\Cat{C}$ with $\corad{n}(C) = C$, and let 
$\Prerad{n}{\Cat{C}}$ be the full subcategory of all objects $C$ 
in $\Cat{C}$ with $\corad{n}(C) = 0$. The arguments for Theorem 
\ref{thm_main_result} go through to give us the following result. 

\begin{thm}\label{thm_sum_heart}
The functors $\corad{n}, n = 1, 2\dots,$ define a sequence of
coradicals, whose associated torsion theories $(\Cat{T}_n,
\Cat{F}_n) = (\Prerad{n}{\Cat{C}}, \Corad{n}{\Cat{C}})$ fit together
to define a strong ascending cofiltration of $\Cat{C}$:
\[
0 = \Corad{0}{\Cat{C}} \subseteq \cdots \subseteq \Corad{n}{\Cat{C}}
   \subseteq \Corad{n + 1}{\Cat{C}} \subseteq \cdots
\]
and a strong descending filtration of $\Cat{C}$:
\[
\Cat{C} = \Prerad{0}{\Cat{C}} \supseteq \cdots \supseteq \Prerad{n}{\Cat{C}}
\supseteq \Prerad{n + 1}{\Cat{C}} \supseteq \cdots.
\]
\end{thm}

Following Definition \ref{def_upper_slice_functor}, we define 
$\prerad{n}$ to be the kernel of the natural surjection $\id \to 
\corad{n}$. By Proposition \ref{prop_rad_eq_corad} and Corollary 
\ref{cor_tt_ref_and_coref}, $\prerad{n}$ is an idempotent 
pre-radical, and is right adjoint to the inclusion of 
$\Prerad{n}{\Cat{C}}$ in $\Cat{C}$. Furthermore, an object $F$
is in $\Prerad{n}{\Cat{C}}$ if and only if $\prerad{n}{F} = F$.

As in the case for $\HI$, we can define $\SGFilt{n}{\Cat{C}}$ to be
the full subcategory of objects $F$ such that $F \cong L_H^nF'$
for some $F'$ in $\Cat{C}$. As defined, $\SGFilt{n}{\Cat{C}}
\subseteq \SGFilt{m}{\Cat{C}}$ if $n < m$. The arguments of 
Proposition \ref{prop_HI_upper_slice} go through to give us the 
following proposition:

\begin{prop}
The tower of full subcategories
\[
\Cat{C} = \SGFilt{0}{\Cat{C}} \supseteq \cdots \supseteq \SGFilt{n - 1}{\Cat{C}}
\supseteq \SGFilt{n}{\Cat{C}} \supseteq \cdots
\]
defines a weak filtration on $\Cat{C}$.
\end{prop}

Drawing on the analogy with $\HI$, the coreflection functor from 
$\Cat{C}$ to $\SGFilt{n}{\Cat{C}}$ is given by $F \mapsto 
L_H^n R_H^n F$. We also have the following relationship between 
$\SGFilt{n}{\Cat{C}}$ and $\Prerad{n}{\Cat{C}}$:

\begin{cor}\label{cor_sg_sub_prerad}
For $F$ in $\SGFilt{n}{\Cat{C}}$, $\hom_{\Cat{C}}(F, G) = 0$ for
all $G$ in $\Corad{n}{\Cat{C}}$. In particular, $\SGFilt{n}{\Cat{C}}$
is a full subcategory of $\Prerad{n}{\Cat{C}}$.
\end{cor}
\begin{proof}
If $F$ is an object of $\SGFilt{n}{\Cat{C}}$, then $F = L_H^n F'$ for
some $F'$ in $\Cat{C}$. Since $R_H^n G = 0$ for all $G$ in 
$\Corad{n}{\Cat{C}}$,
\[
\hom_{\Cat{C}}(F, G) = \hom_{\Cat{C}}(L_H^n F', G) = \hom_{\Cat{C}}(F', R_H^n G) = 0
\]
for all $G$ in $\Corad{n}{\Cat{C}}$. The first statement of the 
corollary is now proven. The second statement follows from the 
definition of $\Prerad{n}{\Cat{C}}$ as the torsion subcategory of 
$\Corad{n}{\Cat{C}}$.
\end{proof}

The filtrations on $\Cat{C}$ may also be trivial. Proposition
\ref{prop_torsion_filt_degen_cond} and Corollary 
\ref{cor_invert_implies_C_invert} below show that, as in the case 
for $\DCat$, the degeneracy of the filtrations are related to the 
the invertibility of $S$. Let us first consider the following 
lemma. Recall from Definition \ref{def_slice_filt_general} that 
for a given $n$, $\Prerad{n}{\DCat}$ is the full subcategory of 
$\DCat$ whose objects are the objects $M$ in $\DCat$ such that $M 
\cong L^nM'$ for some $M'$ in $\DCat$, and $\Corad{n}{\DCat}$ is 
the full subcategory of $\DCat$ whose objects are the objects $M$ 
in $\DCat$ such that $R^nM = 0$.

\begin{lem}\label{lem_trivial_and_trivial}
If $\Corad{n}{\DCat} = 0$ then $\Corad{n}{\Cat{C}} = 0$.
\end{lem}
\begin{proof}
Recall from Proposition \ref{prop_LH_RH_properties}(2) that $R_H^n$
is naturally isomorphic to $R^n$ on $\Cat{C}$. Therefore, if 
$R_H^n C = 0$, then $R^n C = 0$. Hence, $\Corad{n}{\Cat{C}} 
\subset \Prerad{n}{\DCat}$.
\end{proof}

If $S$ is invertible in $\DCat$, then by Proposition 
\ref{prop_filt_trivial_cond}, $\Corad{n}{\DCat} = 0$ for all $n$, 
and by the preceding lemma, $\Corad{n}{\Cat{C}} = 0$ for all $n$. 
As we will see in Proposition \ref{prop_torsion_filt_degen_cond}, 
$\Corad{n}{\Cat{C}} = 0$ for all $n$ implies $\Prerad{n}{\Cat{C}} = 
\Cat{C}$ for all $n$. This shows that if $S$ is invertible in $\DCat$,
then the strong filtration and cofiltration are degenerate. 
However, the converse does not necessarily hold. Rather, the 
converse is related to a weaker condition. 

\begin{defn}
We say that $S$ is \DEF{$\Cat{C}$-invertible} if there exists some 
$T$ in $\Cat{C}$ such that $T \tensor^{\Cat{C}} S = \Unit$. 
\end{defn}

\begin{prop}\label{prop_torsion_filt_degen_cond}
The following are equivalent:
\begin{enumerate}
\item $\SGFilt{*}{\Cat{C}}$ is degenerate with $\SGFilt{n}{\Cat{C}} = 
\Cat{C}$ for all $n$,

\item $\Corad{*}{\Cat{C}}$ is trivial,
for all $n$,

\item $\Prerad{*}{\Cat{C}}$ is degenerate with $\Prerad{n}{\Cat{C}} = 
\Cat{C}$ for all $n$,

\item $S$ is $\Cat{C}$-invertible.
\end{enumerate}
\end{prop}
\begin{proof}
We first show that (1), (2), and (4) are equivalent. The proof 
that (1) and (4) are equivalent is the same as the proof that (2) 
and (3) of Proposition \ref{prop_filt_trivial_cond} are 
equivalent. To see that (2) implies (1), let $F$ be an object in 
$\Cat{C}$, and let $K$ be the kernel of the counit $L_H^n R_H^n F 
\to F$. We have the following exact sequence:
\[
0 \to K \to L_H^n R_H^n F \to F \to \corad{n} F \to 0.
\]
By Proposition \ref{prop_LH_RH_properties}(1), $R_H^n$ is exact. 
Applying $R_H^n$, we obtain the following exact sequence:
\[
0 \to R_H^n K \to R_H^n L_H^n R_H^n F \to R_H^n F \to R_H^n \corad{n} F \to 0.
\]
But since $R_H^n L_H^n \cong \id$, $R_H^n L_H^n R_H^n F \to R_H^n F$ is 
an isomorphism. Therefore, $R_H^n K = R_H^n \corad{n} F = 0$. That is,
$K$ and $\corad{n} F$ are in $\Corad{n}{\Cat{C}}$. It follows that
$K = \corad{n} F = 0$, and therefore $F = L_H^n R_H^n F$. It follows that
$\SGFilt{n}{\Cat{C}} = \Cat{C}$.

To show that (4) implies (2), suppose $F$ is in 
$\Corad{n}{\Cat{C}}$. Then by (4), $F \cong S^{\tensor n} 
\tensor^{\Cat{C}} T^{\tensor n}$ for some $T$ in $\Cat{C}$. Therefore,
$L_H^n R_H^n F \cong F$. However, this means that $\corad{n} F = 0$.
By Theorem \ref{thm_precorad_eq_tt}, $\corad{n} F = F$. Therefore,
$F = 0$ and $\Corad{n}{\Cat{C}} = 0$.

To show that (3) is equivalent to the rest, we first show that (2) 
implies (3). Suppose $\Corad{n}{\Cat{C}}$ is trivial. Since 
for all $F$, $\corad{n} F$ is an object of $\Corad{n}{\Cat{C}}$, it 
follows that $\corad{n}F = 0$. Therefore, $\prerad{n}F = F$. Thus, 
$\Prerad{n}{\Cat{C}} = \Cat{C}$, as desired.

Finally, to show that (3) implies (2), suppose 
$\Prerad{n}{\Cat{C}} = \Cat{C}$. By Proposition \ref{prop_tt_suff_cond}, 
$\Prerad{n}{\Cat{C}} \cap \Corad{n}{\Cat{C}} = 0$. Hence,
$\Corad{n}{\Cat{C}} = 0$, as desired.
\end{proof}

The following corollary is a direct direct consequence of Lemma 
\ref{lem_trivial_and_trivial} and Proposition 
\ref{prop_torsion_filt_degen_cond}.

\begin{cor}\label{cor_invert_implies_C_invert}
If $S$ is invertible in $\DCat$, then $S$ is $\Cat{C}$-invertible.
\end{cor}

\section{Slice filtration on the localization of $\DCat$ by $S$}

Next, for a torsion monoidal category $\DCat$, we can form the
localization $\DCatLoc{S}$ of $\DCat$ by $S$ (see \cite[8A]{MVW}). 
The objects of $\DCatLoc{S}$ are pairs $(M, n)$, where $M$ is 
in $\DCat$, and $n$ is some integer, and $(M, n + 1) \cong (LM, 
n)$ for all $M$ and $n$. Morphisms between $(M, n)$ and $(M', n')$
are elements of the direct limit $\varinjlim_{k} \hom(M(n + k),
M'(n' + k))$. The relationship between $\DCat$ and $\DCatLoc{S}$
is analogous to the relationship between $\DMeff$ and $\DM$. In
particular, by the Cancellation axiom (Definition 
\ref{def_torsion_monoidal_category}(3)) the localization functor 
$\Sigma^\infty: \DCat \to \DCatLoc{S}$ which sends an object $M$ 
in $\DCat$ to the object $(M, 0)$ is fully faithful. Therefore, 
we can identify $\DCat$ as a full subcategory of $\DCatLoc{S}$. 

There is also a tensor product on $\DCatLoc{S}$, given by
\[
(M, n) \tensor (M', n') = (M \tensor M', n + n')
\]
(see \cite[8A]{MVW}). In the case that the cyclic permutation of 
$(S, 0)^{\tensor 3}$ is the identity in $\DCatLoc{S}$, by 
\cite[8A.10, 8A.11]{MVW} the tensor product is a
triangulated symmetric tensor on $\DCatLoc{S}$. In this case, 
$\DCatLoc{S}$ is also a torsion monoidal category. However, since 
$S$ is invertible in $\DCatLoc{S}$, defining the weak filtrations as we have done 
in Section \ref{sect_summary_slice_filt_on_DCat} will result in 
trivial weak filtrations, as we have shown in Proposition 
\ref{prop_filt_trivial_cond}. Nonetheless, we can still construct
weak filtrations on $\DCatLoc{S}$ as follows.

\begin{defn}\label{def_GFiltDLoc_general}
Let $\Prerad{n}{\DCatLoc{S}}$ be the full subcategory of 
$\DCatLoc{S}$ with objects $(M, k)$ such that $(M, k) \cong
(M', n)$ for some $M'$ in $\DCat$. Since $(M, n + 1) \cong 
(LM, n)$, we have the following descending tower of full 
subcategories:
\[
\DCatLoc{S} \supseteq \cdots \supseteq \Prerad{0}{\DCatLoc{S}}
   \supseteq \cdots \supseteq \Prerad{n}{\DCatLoc{S}} \supseteq 
   \Prerad{n + 1}{\DCatLoc{S}} \supseteq \cdots.
\]
\end{defn}

To show that the nested sequence of subcategories is a descending 
weak filtration, we need to show that for each integer $n$, there 
exists a coreflection $\preradD{n}: \DCatLoc{S} \to 
\Prerad{n}{\DCatLoc{S}}$. Copying the definition of the functor 
$\preradD{n}$ on $\DM$ as given in Definition \ref{def_sgDM_DM}, 
we define $\preradD{k}$ by setting
\[
\preradD{k}(M, n) \defeq \begin{cases}
(\preradD{k - n}M, n) &\textrm{if }k > n\\
(M, n) &\textrm{otherwise}.
\end{cases}
\]
Copying the proof of Proposition \ref{prop_sDM_reflection}, we
see that $\preradD{k}$ is right adjoint to the inclusion of
$\Prerad{k}{\DCatLoc{S}}$ into $\DCatLoc{S}$. Furthermore,
the restriction of $\preradD{k}$ to $\Prerad{k}{\DCatLoc{S}}$
is the identity. This shows that $(\Prerad{*}{\DCatLoc{S}},
\preradD{*})$ is a descending weak filtration of $\DCatLoc{S}$.

\begin{rmk}
Since $(M, 0) \cong (M', n)$ if and only if $M \cong L^nM'$, the 
image of $\Prerad{n}{\DCat}$ under $\Sigma^\infty$ is precisely
$\Prerad{n}{\DCatLoc{S}}$. In particular, we can identify $\DCat$
with the full subcategory $\Prerad{0}{\DCatLoc{S}}$, and the 
preceding discussion shows that $\preradD{0}$ is a right
adjoint to $\Sigma^\infty$.
\end{rmk}

Next, let $\Corad{n}{\DCatLoc{S}}$ be the full subcategory of
objects $(M, k)$ where $\preradD{n}(M, k) = 0$. Since 
$\preradD{k}(M, n) = 0$ implies that $\preradD{k + 1}(M, n) = 0$,
we obtain the following ascending tower of full subcategories of
$\DCatLoc{S}$:
\[
\cdots \subseteq \Corad{0}{\DCatLoc{S}} \subseteq \cdots \subseteq 
   \Corad{n}{\DCatLoc{S}} \subseteq \Corad{n + 1}{\DCatLoc{S}}
   \subseteq \cdots \DCatLoc{S}.
\]
We want to show that this tower of full subcategories defines
a weak filtration of $\DCatLoc{S}$ by showing that, for each $n$,
there exists a reflection $\coradD{n}: \DCatLoc{S} \to 
\Corad{n}{\DCatLoc{S}}$. Copying the definition of the functor
$\coradD{k}$ on $\DM$ as given in Definition \ref{def_slDM_DM},
we define $\coradD{k}$ by setting:
\[
\coradD{k}(M, n) = \begin{cases}
(\coradD{n - k}M, n) &\textrm{if }n > k\\
0                    &\textrm{otherwise}.
\end{cases}
\]
The arguments for \cite[1.3(i)]{HuKa} go through
in this general setting to show that for each $k$, $\coradD{k}$ is 
a triangulated functor that is right adjoint to the inclusion of 
$\Corad{k}{\DCatLoc{S}}$ into $\DCatLoc{S}$. Moreover, the 
restriction of $\coradD{k}$ to $\Corad{k}{\DCatLoc{S}}$ is 
naturally isomorphic to the identity (\CF Proposition 
\ref{prop_slDM_functor}). We have just proved the following
theorem, which generalizes the results in Section 
\ref{sect_slice_filt_on_DM}. 

\begin{thm}\label{thm_summary_dloc}
The category of $\DCatLoc{S}$ is equipped with a descending 
weak filtration given by $(\Prerad{*}{\DCatLoc{S}}, \prerad{*})$
and an ascending weak filtration given by $(\Corad{*}{\DCatLoc{S}},
\corad{*})$.
\end{thm}

The following proposition, which is a consequence of Proposition 
\ref{prop_filt_trivial_cond}, relate the degeneracy of the weak 
filtrations that we defined above with the invertibility of the
Tate object $S$.

\begin{prop}
The following are equivalent:
\begin{enumerate}
\item the categories $\Corad{n}{\DCatLoc{S}}$ are trivial,

\item the categories $\Prerad{n}{\DCatLoc{S}}$ is degenerate with
$\Prerad{n}{\DCatLoc{S}} = \DCatLoc{S}$ for all $n$,

\item $S$ is invertible in $\DCat$.
\end{enumerate}
\end{prop}
\begin{proof}
If $S$ is invertible, then $\DCatLoc{S}$ is equivalent to $\DCat$.
The fact (3) implies (1) and (2) follows directly from Proposition 
\ref{prop_filt_trivial_cond}.

To show that (1) implies (2), suppose $\Corad{n}{\DCatLoc{S}} = 0$
for all $n$. For any object $(M, k)$ of $\DCatLoc{S}$ and any integer
$n$, we have the following distinguished triangle
\[
\preradD{n}(M, k) \to (M, k) \to \coradD{n}(M, k) \to \preradD{n}(M, k)[1].
\]
Notice that $\coradD{n}(M, k)$ is an object of $\Corad{n}{\DCatLoc{S}}$
and $\preradD{n}(M, k)$ is an object of $\Prerad{n}{\DCatLoc{S}}$. By
the assumption that $\Corad{n}{\DCatLoc{S}} = 0$, we see that
$(M, k) \cong \preradD{n}(M, k)$, and therefore $(M, k)$ is in
$\Prerad{n}{\DCatLoc{S}}$. Therefore, $\Prerad{n}{\DCatLoc{S}} = 
\DCatLoc{S}$ for all $n$.

To show that (2) implies (3), suppose $\Prerad{n}{\DCatLoc{S}} = 
\DCatLoc{S}$ for all $n$. In particular, $\Prerad{1}{\DCatLoc{S}}
= \DCatLoc{S}$. This implies that $(\Unit, 0)$ is an object of
$\Prerad{1}{\DCatLoc{S}}$. By definition of $\Prerad{1}{\DCatLoc{S}}$, 
$(\Unit, 0) \cong (T, 1)$ for some $T$. Therefore, $\Unit \cong
LT = S \tensor T$, and $S$ is invertible in $\DCat$.
\end{proof}

\section{Torsion Filtration on the Stable Localization of 
$\Cat{C}$ by $S$}

In this section, we generalize the results of Section 
\ref{sect_torsion_filt_HM}. Copying the construction of $\HM$, we
define the stable localization of $\Cat{C}$ as follows:

\begin{defn}\label{def_C_loc_S}
  Let $\CatLoc{C}{S}$ denote the category whose objects are the
  $\Z$-graded objects $C_*$ in $\Cat{C}$ together with a map $s_n:
  L_HC_n \to C_{n + 1}$ for each integer $n$ such that the
  the corresponding adjunction map $w_n: C_{n + 1} \to R_HC_n$ is an
  isomorphism. Following Section \ref{sect_torsion_filt_HM}, we call
  $s_n$ the \DEF{$n$-th suspension map}, and $w_n$ the \DEF{$n$-th
    delooping map}. We will represent an object of $\CatLoc{C}{S}$ by
  $(C_*, w_*)$ or simply $C_*$ if the collection of delooping maps are
  clear.
\end{defn}

By the Cancellation axiom, $C \mapsto L_HC$ is fully faithful,
and therefore, the arguments in \cite[1.8]{DegModHom} go through
for $\CatLoc{C}{S}$ to show that there is a fully faithful 
functor $\sigma^\infty$ from $\Cat{C}$ to $\CatLoc{C}{S}$ given by
$C \mapsto (C_*, w_*)$, where the $n$-th graded component of $C_*$
is given by
\[
C_n \defeq \begin{cases}
L_H^nC &\textrm{if }n > 0\\
C      &\textrm{if }n = 0\\
R_H^{|n|} C &\textrm{otherwise},
\end{cases}
\]
and the $n$-th delooping $w_n: C_n \to R_H(C_{n + 1})$ is the unit
map for $n \geq 0$ and the identity for $n < 0$. As in the case for
$\HM$, $\sigma^\infty$ has a right adjoint $\omega^\infty: \Cat{C}
\to \CatLoc{C}{S}$, given by $(C_*, w_*) \mapsto C_0$. Thus, we
can view $\Cat{C}$ as a full coreflective subcategory of 
$\CatLoc{C}{S}$ whose objects are the $(C_*, w_*)$ such that 
$C_n = L_H^nC_0$ for all $n > 0$. 

\begin{defn}\label{def_corad_loc_gen}
Copying the definition of $\corad{n}$ in Definition 
\ref{def_corad_HM}, for an object $(C_*, w_*)$ of $\CatLoc{C}{S}$,
we define $\corad{n}(C_*)$ to be the object in 
$\CatLoc{C}{S}$ where
\[
(\corad{n}(C_*))_k \defeq
\begin{cases}
\corad{n + k}(C_k) &\textrm{if }n + k > 0\\
0  &\textrm{otherwise}.
\end{cases}
\]
For ease of notation, we will write $\corad{n}_k(C_*)$ for the 
$k$-th graded component of $\corad{n}(C_*)$.
\end{defn}

As the arguments of Theorem \ref{thm_tlHM_corad} are 
entirely formal, replacing $\HM$ by $\CatLoc{C}{S}$ and $\tlHM{k}$ by
$\corad{k}$, we obtain the following proposition
which is needed in the construction of the strong filtration and
cofiltration on $\CatLoc{C}{S}$. 

\begin{prop}\label{prop_corad_loc_general}
For each integer $n$, $\corad{n}$ is a coradical of the category
$\CatLoc{C}{S}$.
\end{prop}

We can now define the full subcategories in the strong filtration
and cofiltration of $\CatLoc{C}{S}$. Recall from Theorem
\ref{thm_precorad_eq_tt} that if $\phi$ is a coradical, then the
torsion subcategory of $\phi$ is the full subcategory $\Cat{T}$
consisting of the objects $T$ such that $\phi(T) = 0$, and the
torsionfree subcategory of $\phi$ is the full subcategory $\Cat{F}$
whose objects are the objects $F$ such that the natural map $F 
\to \phi(F)$ is an isomorphism.

\begin{defn}\label{def_torsion_filt_general}
Let $\Corad{n}{\CatLoc{C}{S}}$ be the torsionfree subcategory of
$\corad{n}$, i.e., $C_*$ is an object of $\Corad{n}{\CatLoc{C}{S}}$
if and only if $\corad{n}(C_*) = C_*$. Let 
$\Prerad{n}{\CatLoc{C}{S}}$ be the torsion subcategory of $\corad{n}$.
The objects of $\Prerad{n}{\CatLoc{C}{S}}$ are the $C_*$ in 
$\CatLoc{C}{S}$ such that $\corad{n}(C_*) = 0$.
\end{defn}

Copying the proof for Corollary \ref{cor_tor_filt_on_HM} we obtain 
the following theorem.

\begin{thm}
\label{thm_sum_heart_loc}
The sequence of functors $\corad{n}, n = \dots, -1, 0, 1, \dots$ on
$\CatLoc{C}{S}$ is a $\Z$-indexed sequence of coradicals whose 
associated torsion theories 
\[
(\Cat{T}_n, \Cat{F}_n) = (\Prerad{n}{\CatLoc{C}{S}}, \Corad{n}{\CatLoc{C}{S}})
\]
define an ascending strong cofiltration
\[
\cdots \subseteq \Corad{0}{\CatLoc{C}{S}} \subseteq \cdots \subseteq 
   \Corad{n}{\CatLoc{C}{S}} \subseteq \Corad{n + 1}{\CatLoc{C}{S}}
   \subseteq \cdots
\]
and a strong descending filtration
\[
\CatLoc{C}{S} \supseteq \cdots \supseteq \Prerad{0}{\CatLoc{C}{S}}
   \supseteq \cdots \supseteq \Prerad{n}{\CatLoc{C}{S}} \supseteq 
   \Prerad{n + 1}{\CatLoc{C}{S}} \supseteq \cdots.
\]
on $\CatLoc{C}{S}$.
\end{thm}

\comment{
Notice that we can extend the weak filtration $(\SGFilt{*}{\Cat{C}},
\sgHI{*})$ to a weak filtration on $\CatLoc{C}{S}$. Let 
$\SGFilt{n}{\Cat{C}}$ be the full subcategory of objects $(C_*, w_*)$ 
such that $k \geq n$. As defined, we have a tower of subcategories
\[
\CatLoc{C}{S} \supseteq \SGFilt{-1}{\CatLoc{C}{S}} \supseteq 
   \SGFilt{0}{\CatLoc{C}{S}} \subseteq \cdots \supseteq 
   \SGFilt{n - 1}{\CatLoc{C}{S}} 
   \supseteq \SGFilt{n}{\CatLoc{C}{S}} \supseteq \cdots.
\]
Furthermore, $\SGFilt{n}{\CatLoc{C}{S}}$ is the essential image 
under $\HH^0$ of $\Prerad{n}{\DCatLoc{S}}$. The coreflection functor
$\preradD{n}: \DCatLoc{S} \to \Prerad{n}{\DCatLoc{S}}$ induces a
coreflection functor $\sgHI{n}$ from $\CatLoc{C}{S}$ to 
$\Prerad{n}{\Cat{C}{S}}$ given by
\[
(F, k) \mapsto \HH^0\preradD{n}(F, k) = (\prerad{n - k}F, k).
\]
It follows that $(\SGFilt{*}{\CatLoc{C}{S}}, \sgHI{n})$ defines
a weak filtration of $\CatLoc{C}{S}$. 
}

The following proposition shows that $S$ is $\Cat{C}$-invertible if
and only if each of the weak filtrations above are degenerate:

\begin{prop}
The following are equivalent:

\begin{enumerate}
\comment{\item $\SGFilt{*}{\CatLoc{C}{S}}$ is degenerate with 
$\SGFilt{n}{\CatLoc{C}{S}} = \Cat{C}$ for all $n$,}

\item $\Corad{*}{\CatLoc{C}{S}}$ is trivial,
for all $n$,

\item $\Prerad{*}{\CatLoc{C}{S}}$ is degenerate with 
$\Prerad{n}{\CatLoc{C}{S}} = \Cat{C}$ for all $n$,

\item $S$ is $\Cat{C}$-invertible.
\end{enumerate}
\end{prop}
\begin{proof}
To see that (1) implies (2), suppose $\Corad{n}{\CatLoc{C}{S}} = 
0$ for all integers $n$. Then $\corad{n}(C_*) = 0$ for all $C_*$
in $\CatLoc{C}{S}$, and therefore $\corad{n} = 0$ for all $n$.
Since $C_*$ is in $\Prerad{n}{\CatLoc{C}{S}}$ if and only if
$\corad{n}(C_*) = 0$ (see Definition 
\ref{def_torsion_filt_general}), it follows that $\Prerad{n}{\CatLoc{C}{S}}
= \CatLoc{C}{S}$ for all $n$.

Now, assume $\Prerad{n}{\CatLoc{C}{S}} = \CatLoc{C}{S}$ for
all $n$. By Proposition \ref{prop_tt_suff_cond}, $\Corad{n}{\CatLoc{C}{S}}
\cap \Prerad{n}{\CatLoc{C}{S}} = 0$. Hence, $\Corad{n}{\CatLoc{C}{S}} = 0$
for all $n$. This shows that (2) implies (1).

To show that (1) implies (3), suppose $\Corad{n}{\CatLoc{C}{S}} = 0$
for all $n$. As we have shown in the proof of (1) implies (2), 
$\corad{n} = 0$ as an endofunctor on $\CatLoc{C}{S}$ for all $n$, 
which further implies that $\corad{n} = 0$ as an endofunctor on
$\Cat{C}$ for all $n > 0$. This implies that $\Corad{n}{\Cat{C}} = 0$
for all $n$, and by Proposition \ref{cor_sg_sub_prerad}, $S$ is
invertible in $\Cat{C}$.

To see that (3) implies (1), suppose $S$ is invertible in
$\Cat{C}$. Then by Proposition \ref{cor_sg_sub_prerad},
$\Corad{n}{\Cat{C}} = 0$ for all $n$. Therefore, $\corad{n}(C) = 0$
for all $C$ in $\Cat{C}$ and $n > 0$. Hence, $\corad{n}(C_*) = 0$
for all $C_*$ in $\CatLoc{C}{S}$ and integer $n$. It follows that
the torsionfree categories are trivial, i.e. 
$\Corad{n}{\CatLoc{C}{S}} = 0$ for all integer $n$.
\end{proof}

%\section{Graded monoidal structures on torsion monoidal categories}

Finally, the descending filtrations, together with
the respective symmetric monoidal structure, give rise to a
graded monoidal structure (see Definition 
\ref{def_graded_tensor}). In order to formulate our final set of 
results, let us first consider the following proposition which 
follows straightforward from the Cancellation property (Definition
\ref{def_torsion_monoidal_category} (2)):

\begin{prop}
There exists a fully faithful $t$-exact functor $i: \DCat \to 
\DCatLoc{S}$, given by $M \mapsto (M, 0)$ that realizes $\DCat$ 
as a full subcategory of $\DCatLoc{S}$. The restriction of $i$ 
to the heart of $\DCat$ induces a corresponding fully faithful 
functor $i : \Cat{C} \to \CatLoc{C}{S}$.
\end{prop}

By definition of $i$, $\DCat$ is identified with the full 
subcategory $\Prerad{0}{\DCatLoc{S}}$ of $\DCatLoc{S}$, and
$\Cat{C}$ is identified with $\Prerad{0}{\CatLoc{C}{S}}$.
The following is an easy of the definition:

\begin{thm}\label{thm_graded_monoid_DCat}
There is a graded monoidal structure on $\DCatLoc{S}$ defined by
the filtration $(\Prerad{*}{\DCatLoc{S}}, \preradD{*})$ and a 
graded monoidal structure on $\DCat$ defined by the filtration
$(\Prerad{*}{\DCat}, \preradD{*})$. The two graded monoidal 
structures respects the inclusion of $\DCat$ into $\DCatLoc{S}$ in 
the sense that, for all natural numbers $n$ and $m$, the following 
functor diagram commutes:
\[
\begin{tikzcd}
\Prerad{n}{\DCat} \tensor \Prerad{m}{\DCat} \arrow{r} \arrow{d}{i} &
\Prerad{n + m}{\DCat} \arrow{d}{i} \\
\Prerad{n}{\DCatLoc{S}} \tensor \Prerad{m}{\DCatLoc{S}} \arrow{r} &
\Prerad{n + m}{\DCatLoc{S}}
\end{tikzcd}
\]
\end{thm}

By applying $\HH^0$ to the diagram in Theorem 
\ref{thm_graded_monoid_DCat}, we obtain the following:

\begin{cor}
There is a graded monoidal structure on $\CatLoc{C}{S}$ defined
by the filtration $(\SGFilt{*}{\CatLoc{C}{S}}, \sgHI{*})$ and a graded 
monoidal structure on $\Cat{C}$ defined by $(\SGFilt{*}{\Cat{C}}, 
\sgHI{*})$. That is, for natural numbers $n$ and $m$, the 
following functor diagram commutes:
\[
\begin{tikzcd}
\SGFilt{n}{\Cat{C}} \tensor \SGFilt{m}{\Cat{C}} \arrow{r} \arrow{d}{i} &
\SGFilt{n + m}{\Cat{C}} \arrow{d}{i} \\
\SGFilt{n}{\CatLoc{C}{S}} \tensor \SGFilt{m}{\CatLoc{C}{S}} \arrow{r} &
\SGFilt{n + m}{\CatLoc{C}{S}}
\end{tikzcd}
\]
\end{cor}

Finally, the arguments for Proposition \ref{prop_graded_mon_struct_HM} go 
through to show that we also have a graded monoidal structure 
associated with the torsion filtration:

\begin{thm}
There is a graded monoidal structure on $\CatLoc{C}{S}$ defined
by the strong filtration $(\Prerad{*}{\CatLoc{C}{S}}, \prerad{*})$ 
and a graded monoidal structure on $\Cat{C}$ defined by 
$(\Prerad{*}{\Cat{C}}, \prerad{*})$. As in the case for $\DCat$ and 
$\DCatLoc{S}$, the two graded monoidal structures are compatible 
via the inclusion of $\Cat{C}$ into $\CatLoc{C}{S}$. That is, for 
natural numbers $n$ and $m$, the following functor diagram 
commutes:
\[
\begin{tikzcd}
\Prerad{n}{\Cat{C}} \tensor \Prerad{m}{\Cat{C}} \arrow{r} \arrow{d}{i} &
\Prerad{n + m}{\Cat{C}} \arrow{d}{i} \\
\Prerad{n}{\CatLoc{C}{S}} \tensor \Prerad{m}{\CatLoc{C}{S}} \arrow{r} &
\Prerad{n + m}{\CatLoc{C}{S}}.
\end{tikzcd}
\]
\end{thm}


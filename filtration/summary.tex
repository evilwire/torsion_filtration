\newpage
\chapter{Torsion Filtrations on Torsion Monoidal Categories}
\label{sect_filtration_general}

In this section, we generalize the ideas developed by the last
two sections by axiomatizing the necessary components to define
torsion filtrations on $t$-categories $\DCat$ with a symmetric
monoidal structure. Let us begin by defining the following notion:

\begin{defn}\label{def_torsion_monoidal_category}
Let $\DCat$ be a tensor monoidal category with a $t$-structure 
given by $(\DCat^{\geq 0}, \DCat^{\leq 0})$. Let 
$(\tensor, \Unit)$ represent its symmetric monoidal structure. We 
say that $\DCat$ is a \DEF{torsion monoidal category} if $\DCat$ 
is equipped with 
\begin{enumerate}
\item (\itemhead{Partial Internal Hom}) a partial internal hom
structure $(\ihom, \DCat^{\compact})$ (see Definition
\ref{def_tensor_triang_cat}).

\item (\itemhead{Tate Object}) an compact object $S$ in the hear
   of $\DCat$ called the \DEF{Tate object}
\end{enumerate}

\noindent subject to the following conditions:

\begin{enumerate}
\item $\Unit$ is an object of $\Cat{C}$,

\item $\tensor$ is right $t$-exact in both factors,

\item (\itemhead{Cancellation}) $\ihom(S, S \tensor X) = X$,

\item $\ihom(S, -)$ is $t$-exact.
\end{enumerate}
\end{defn}

If $\DCat$ is a torsion monoidal category, we will write $\HH^0$
for the cohomological functor from $\DCat$ to its heart. We also 
write $L: \DCat \to \DCat$ for the functor sending an object $M$ 
in $\DCat$ to $M \tensor S$, and $R$ for the functor sending $M$
to $\ihom(S, M)$, where $S$ is the Tate object. 

Since $\tensor$ is right $t$-exact, it induces a symmetric 
monoidal and a partial internal hom structure on the heart 
$\Cat{C}$ of $\DCat$, which we represent by $\tensor^{\Cat{C}}$ 
and $\ihom_{\Cat{C}}$. The tensor and internal hom bifunctors
are given by
\[
C \tensor^{\Cat{C}} C' \defeq \HH^0(C \tensor C') 
   \;\;\;\textrm{ and }\;\;\;
\ihom_{\Cat{C}}(C, C') \defeq \HH^0(\ihom(C, C')).
\]
We let $L_H$ and $R_H$ denote the endofunctor on 
$\Cat{C}$ given by $F \mapsto F \tensor^{\Cat{C}} S$ and $F \mapsto
\ihom_{\Cat{C}}(S, F)$ respectively.

Let $L^n$ and $R^n$ denote the $n$-th iterations of $L$ and $R$ 
respectively; we define $L_H^n$ and $R_H^n$ similarly. Notice that 
the Cancellation property (Definition 
\ref{def_torsion_monoidal_category}(2)) implies that $R^nL^n$ is 
naturally isomorphic to the identity, and the same is true for
$R_H^nL_H^n$. We now introduce the following theorems that 
generalize and summarize the theory we have developed in this 
thesis.

\section{Slice filtration of torsion monoidal categories}

We begin by introducing the slice filtrations on $\DCat$. Let 
$\Corad{n}{\DCat}$ be the full subcategory of $\DCat$ by those 
$M$ in $\DCat$ for which $R^n M = 0$. Let $\Prerad{n}{\DCat}$ be 
the subcategory of objects $M$ such that $M = L^n(M')$ for some 
$M'$ in $\DCat$.

\begin{thm}\label{thm_summary_triang_cat}
If $\DCat$ is a torsion monoidal category, then there exists
an $\N$-indexed ascending filtration
\[
0 = \Corad{0}{\DCat} \subseteq \cdots \subseteq \Corad{n}{\DCat}
   \subseteq \Corad{n + 1}{\DCat} \subseteq \cdots
\]
and a descending filtration
\[
\DCat = \Prerad{0}{\DCat} \supseteq \cdots \supseteq \Prerad{n}{\DCat}
\supseteq \Prerad{n + 1}{\DCat} \supseteq \cdots
\]

The reflection $\coradD{n} : \DCat \to \Corad{n}{\DCat}$ is defined 
by sending $M$ in $\DCat$ to $M'$ in the following 
triangle\footnote{Implicit in the statement is that $M'$ is 
defined up to unique isomorphism.}
\[
L^nR^n M \to M \to M' \to L^nR^n M[1],
\]
and the reflection $\preradD{n} : \DCat \to \Prerad{n}{\DCat}$ is
given by $M \mapsto L^nR^n M$.
\end{thm}
\begin{proof}
The theorem relies heavily on the following claims. First, 
$\prerad{n}$ is right adjoint to the inclusion of 
$\Prerad{n}{\DCat}$ to $\DCat$, and the object $M'$ in the triangle
\[
\prerad{n}M \to M \to M' \to \prerad{n}M[1]
\]
is uniquely defined.

These statements rely only on the the Cancellation property of the 
torsion monoidal structure. The arguments in \cite[1.1, 1.4i]{HuKa}
can be generalized to verify these statements. Having established
the uniqueness of $M'$, it is straightforward to verify that 
$\corad{n}$ is a functor and defines a left adjoint to the 
inclusion of $\Corad{n}{\DCat}$ into $\DCat$. (\emph{cf}. 
\cite[1.4]{HuKa})
\end{proof}

It is possible that the filtration is trivial. However, as the
following result shows, the filtration being trivial is equivalent
to the invertibility of $S$ (Recall that $S$ is invertible if 
there exists an object $T$ in $\DCat$ such that $T \tensor S = 
\Unit$):

\begin{prop}\label{prop_filt_trivial_cond}
The following are equivalent:
\begin{enumerate}
\item the filtration $(\Corad{*}{\DCat}, \coradD{*})$ is trivial. 
That is $\Corad{n}{\DCat} = 0$ for all $n$.

\item the filtration $(\Prerad{*}{\DCat}, \preradD{*})$ is trivial.
That is $\Prerad{n}{\DCat} = \DCat$ for all $n$.

\item the Tate object is invertible.
\end{enumerate}
\end{prop}
\begin{proof}
It is easy to see that (2) is equivalent to (3). Indeed, if such
a $T$ exists, then $X = X \tensor T^{\tensor n} \tensor 
S^{\tensor n}$ is an object of $\Prerad{n}{\DCat}$. Conversely, if
$\Prerad{n}{\DCat} = \DCat$, then $\Unit = T \tensor S$ for some
$T$ in $\DCat$.

It remains to show that $\Corad{n}{\DCat} = 0$ if and only if
$\Prerad{n}{\DCat} = \DCat$. First assume that $\Prerad{n}{\DCat} = 
\DCat$. Since $M \in \Corad{n}{\DCat}$ if and only if $R^n M = 0$,
we see that
\begin{align*}
\hom_{\DCat}(X, X) &= \hom_{\DCat}(X \tensor \Unit, X) \\
&= \hom_{\DCat}(L^n(X \tensor T^{\tensor n}), X) \\
&= \hom_{\DCat}(X \tensor T^{\tensor n}, R^n X) \\
&= 0.
\end{align*}
It follows that $X = 0$.

Conversely, suppose $\Corad{n}{\DCat} = 0$. Then $\coradD{n}X = 0$
for all $n$. It follows that $X \cong L^n R^n X \in 
\Prerad{n}{\DCat}$. It follows that $\DCat = \Prerad{n}{\DCat}$.
\end{proof}

Let us now focus on the heart $\Cat{C}$ of $\DCat$. By definition,
the endofunctors $L$ and $R$ are adjoint, and for $F$ in 
$\Cat{C}$, let $\corad{n} F$ denote the cokernel of the counit map 
$L^n R^n F \to F$. The proof of Proposition \ref{prop_HI_lower_slice} can be
modified to show that $\corad{n}$ is a functor. In fact, the 
arguments for Propositions \ref{prop_HI_upper_slice}, \ref{thm_main_result} and 
Corollary \ref{cor_tlHI_prop} go through to show:

\begin{thm}\label{thm_sum_heart}
There exists a descending weak filtration $(\SGFilt{*}{\Cat{C}}, 
\sgHI{*})$ where the objects of the full subcategories 
$\SGFilt{n}{\Cat{C}}$ are those $F$ in $\Cat{C}$ for which $F = L^nF'$ 
for some $F'$ in $\Cat{C}$. In this case, the reflection functor 
$\sgHI{n}: \Cat{C} \to \SGFilt{n}{\Cat{C}}$ is defined by $F \mapsto 
L^nR^n F$.

Moreover, there exists a torsion filtration on $\Cat{C}$. That is, 
there exists a $\N$-indexed sequence of coradicals $\corad{n}$ 
whose corresponding torsion theories define an ascending strong 
cofiltration
\[
0 = \Corad{0}{\Cat{C}} \subseteq \cdots \subseteq \Corad{n}{\Cat{C}}
   \subseteq \Corad{n + 1}{\Cat{C}} \subseteq \cdots
\]
and a strong descending filtration
\[
\Cat{C} = \Prerad{0}{\Cat{C}} \supseteq \cdots \supseteq \Prerad{n}{\Cat{C}}
\supseteq \Prerad{n + 1}{\Cat{C}} \supseteq \cdots.
\]

Furthermore, the strong filtration defines a graded monoidal 
structure on $\Cat{C}$.
\end{thm}
\noproof

As in the case for $\HI$, we have the following relationship 
between $\SGFilt{n}{\Cat{C}}$ and $\Prerad{n}{\Cat{C}}$:

\begin{cor}\label{cor_sg_sub_prerad}
For $F$ in $\SGFilt{n}{\Cat{C}}$, $\hom_{\Cat{C}}(F, G) = 0$ for
all $G$ in $\Prerad{n}{\Cat{C}}$. In particular, $\SGFilt{n}{\Cat{C}}$
is a full subcategory of $\Prerad{n}{\Cat{C}}$.
\end{cor}
\begin{proof}
If $F$ is an object of $\SGFilt{n}{\Cat{C}}$, then $F = L^n F'$ for
some $F'$ in $\Cat{C}$. Since $R^n G = 0$ for all $G$ in 
$\Corad{n}{\Cat{C}}$,
\[
\hom_{\Cat{C}}(F, G) = \hom_{\Cat{C}}(F', R^n G) = 0.
\]
The corollary now follows.
\end{proof}

As in the case for $\DCat$, the filtrations on $\Cat{C}$ may be
trivial. However, this is not the case if $S$ is not invertible.

\begin{cor}
The following are equivalent:
\begin{enumerate}
\item $\SGFilt{*}{\Cat{C}}$ is trivial, i.e. $\SGFilt{n}{\Cat{C}} = 
\Cat{C}$ for all $n$,

\item $\Corad{*}{\Cat{C}}$ is trivial, i.e. $\Corad{n}{\Cat{C}} = 0$ 
for all $n$,

\item $\Prerad{*}{\Cat{C}}$ is trivial, i.e. $\Prerad{n}{\Cat{C}} = 
\Cat{C}$ for all $n$,

\item $S$ is invertible.
\end{enumerate}
\end{cor}
\begin{proof}
The equivalence of (1) and (2) with (4) follows from Proposition
\ref{prop_filt_trivial_cond}. To show that (3) is equivalent to
the rest, first suppose $\Corad{n}{\Cat{C}} = 0$. It follows that 
for all $F$, $\corad{n}F = 0$. Therefore, $\prerad{n}F = F$. It 
follows that $\Prerad{n}{\Cat{C}} = \Cat{C}$.

By Corollary \ref{cor_sg_sub_prerad}, $\SGFilt{n}{\Cat{C}} = \Cat{C}$ implies
that $\Prerad{n}{\Cat{C}} = \Cat{C}$.
\end{proof}

Finally, for a torsion monoidal category $\DCat$, we can form the
localization $\DCatLoc{S}$ of $\DCat$ by $S$. Here the objects of
$\DCatLoc{S}$ are pairs $(M, n)$, where $M$ is in $\DCat$, and $n$
is some integer, and $(M, n + 1) = (LM, n)$ for all $M$ and $n$.
The category $\DCatLoc{S}$ is also a triangulated category with a
$t$-structure. It also inherits the symmetric monoidal structure
from $\DCat$. In fact, $\DCatLoc{S}$ is also a torsion monoidal 
category. However, since $S$ is invertible, we must define the
filtration by some other means.

Nonetheless, we still have weak filtrations on $\DCatLoc{S}$:

\begin{thm}\label{thm_summary_dloc}
There exists an ascending weak filtration
\[
\cdots \subseteq \Corad{0}{\DCatLoc{S}} \subseteq \cdots \subseteq 
   \Corad{n}{\DCatLoc{S}} \subseteq \Corad{n + 1}{\DCatLoc{S}}
   \subseteq \cdots
\]
and a descending weak filtration
\[
\cdots \supseteq \Prerad{0}{\DCatLoc{S}} \supseteq \cdots \supseteq 
   \Prerad{n}{\DCatLoc{S}} \supseteq \Prerad{n + 1}{\DCatLoc{S}}
   \supseteq \cdots
\]
\end{thm}
\begin{proof}
Let $\Prerad{k}{\DCatLoc{S}}$ be the full subcategory of 
$\DCatLoc{S}$ with objects $(M, n)$ such that $n \geq k$. It is
easy to see that
\[
\DCatLoc{S} \supseteq \cdots \supseteq \Prerad{0}{\DCatLoc{S}}
   \supseteq \cdots \supseteq \Prerad{n}{\DCatLoc{S}} \supseteq 
   \Prerad{n + 1}{\DCatLoc{S}} \supseteq \cdots.
\]
We now need to show that for each integer $n$, there exists a 
reflection $\preradD{n}: \DCatLoc{S} \to \Prerad{n}{\DCatLoc{S}}$.
As in Proposition \ref{prop_sDM_reflection}, this is given by the
functors $\preradD{k}$ defined by
\[
\preradD{k}(M, n) = \begin{cases}
(\preradD{k - n}M, n) &\textrm{if }k > n\\
(M, n) &\textrm{otherwise}.
\end{cases}
\]

Now let $\Corad{k}{\DCatLoc{S}}$ be the full subcategory of
objects $(M, n)$ where $\preradD{k}(M, n) = 0$. It is easy to
see that
\[
\cdots \subseteq \Corad{0}{\DCatLoc{S}} \subseteq \cdots \subseteq 
   \Corad{n}{\DCatLoc{S}} \subseteq \Corad{n + 1}{\DCatLoc{S}}
   \subseteq \cdots \DCatLoc{S}.
\]
In this case, the arguments for Proposition \ref{prop_slDM_functor} go through 
to show that the reflection functors $\coradD{k} : \DCatLoc{S}
\to \Corad{k}{\DCatLoc{S}}$ are given by
\[
\coradD{k}(M, n) = \begin{cases}
(\coradD{n - k}M, n) &\textrm{if }n > k\\
0                    &\textrm{otherwise}.
\end{cases}
\]
\end{proof}

Notice that $\DCatLoc{S}$ is a $t$-category, and we write the
heart of its $t$-structure as $\CatLoc{C}{S}$. Objects of 
$\CatLoc{C}{S}$ can be represented as a pair $(F, n)$ where
$F$ is in $\Cat{C}$ and $n$ is an integer such that $(LF, n) 
\cong (F, n + 1)$. The category $\CatLoc{C}{S}$ is also equivalent
to the category of $\Z$-graded objects $(F_*, \deloop_*)$ of 
$\Cat{C}$ such that $RF_n \cong F_{n - 1}$.

The arguments in Theorem \ref{thm_tlHM_corad} go through to show that:

\begin{thm}
\label{thm_sum_heart_loc}
There exists a descending weak filtraion $(\SGFilt{*}{\CatLoc{C}{S}}, 
\sgHI{*})$ where the objects of the full subcategories 
$\SGFilt{k}{\CatLoc{C}{S}}$ are those $(F, n)$ in $\CatLoc{C}{S}$ for 
which $n \geq k$. In this case, the reflection functor 
$\sgHI{k}: \CatLoc{C}{S} \to \SGFilt{k}{\CatLoc{C}{S}}$ is defined by 
\[
\sgHI{k}(F,n) = \begin{cases}
(\sgHI{k - n}F, n) &\textrm{if }k > n\\
(F, n)             &\textrm{otherwise}
\end{cases}
\]

Moreover, there exists a torsion filtration on $\CatLoc{C}{S}$. 
That is, there exists a $\Z$-indexed sequence of coradicals $\corad{n}$ 
whose corresponding torsion theories define an ascending strong 
cofiltration
\[
\cdots \subseteq \Corad{0}{\CatLoc{C}{S}} \subseteq \cdots \subseteq 
   \Corad{n}{\CatLoc{C}{S}} \subseteq \Corad{n + 1}{\CatLoc{C}{S}}
   \subseteq \cdots
\]
and a strong descending filtration
\[
\CatLoc{C}{S} \supseteq \cdots \supseteq \Prerad{0}{\CatLoc{C}{S}}
   \supseteq \cdots \supseteq \Prerad{n}{\CatLoc{C}{S}} \supseteq 
   \Prerad{n + 1}{\CatLoc{C}{S}} \supseteq \cdots.
\]

Furthermore, the strong filtration defines a graded monoidal 
structure on $\Cat{C}$.
\end{thm}
\noproof

Finally, in all cases, the descending filtrations, together with
the respective symmetric monoidal structure, give rise to a
graded monoidal structure. (See Definition \ref{def_graded_tensor}.) In order
to formulate our final set of results, let us first consider the
following proposition which follows straightforward from 
Cancellation:

\begin{prop}
There exists a fully faithful $t$-exact functor $i: \DCat \to 
\DCatLoc{S}$, given by $M \mapsto (M, 0)$ that realizes $\DCat$ 
as a full subcategory of $\DCatLoc{S}$. The restriction of $i$ 
to the heart of $\DCat$ induces a corresponding fully faithful 
functor $i : \Cat{C} \to \CatLoc{C}{S}$.
\end{prop}

By definition of $i$, $\DCat$ is identified with the full 
subcategory $\Prerad{0}{\DCatLoc{S}}$ of $\DCatLoc{S}$, and
$\Cat{C}$ is identified with $\Prerad{0}{\CatLoc{C}{S}}$.
The following is an easy of the definition:

\begin{thm}\label{thm_graded_monoid_DCat}
There is a graded monoidal structure on $\DCatLoc{S}$ defined by
the filtration $(\Prerad{*}{\DCatLoc{S}}, \preradD{*})$ and a 
graded monoidal structure on $\DCat$ defined by the filtration
$(\Prerad{*}{\DCat}, \preradD{*})$. The two graded monoidal 
structures respects the inclusion of $\DCat$ into $\DCatLoc{S}$ in 
the sense that, for all natural numbers $n$ and $m$, the following 
functor diagram commutes:
\[
\begin{tikzcd}
\Prerad{n}{\DCat} \tensor \Prerad{m}{\DCat} \arrow{r} \arrow{d}{i} &
\Prerad{n + m}{\DCat} \arrow{d}{i} \\
\Prerad{n}{\DCatLoc{S}} \tensor \Prerad{m}{\DCatLoc{S}} \arrow{r} &
\Prerad{n + m}{\DCatLoc{S}}
\end{tikzcd}
\]
\end{thm}

By applying $\HH^0$ to the diagram in Theorem \ref{thm_graded_monoid_DCat},
we obtain the following:

\begin{cor}
There is a graded monoidal structure on $\CatLoc{C}{S}$ defined
by the filtration $(\SGFilt{*}{\CatLoc{C}{S}}, \sgHI{*})$ and a graded 
monoidal structure on $\Cat{C}$ defined by $(\SGFilt{*}{\Cat{C}}, 
\sgHI{*})$. That is, for natural numbers $n$ and $m$, the 
following functor diagram commutes:
\[
\begin{tikzcd}
\SGFilt{n}{\Cat{C}} \tensor \SGFilt{m}{\Cat{C}} \arrow{r} \arrow{d}{i} &
\SGFilt{n + m}{\Cat{C}} \arrow{d}{i} \\
\SGFilt{n}{\CatLoc{C}{S}} \tensor \SGFilt{m}{\CatLoc{C}{S}} \arrow{r} &
\SGFilt{n + m}{\CatLoc{C}{S}}
\end{tikzcd}
\]
\end{cor}

Finally, the arguments for Proposition \ref{prop_graded_mon_struct_HM} go 
through to show that we also have a graded monoidal structure 
associated with the torsion filtration:

\begin{thm}
There is a graded monoidal structure on $\CatLoc{C}{S}$ defined
by the strong filtration $(\Prerad{*}{\CatLoc{C}{S}}, \prerad{*})$ 
and a graded monoidal structure on $\Cat{C}$ defined by 
$(\Prerad{*}{\Cat{C}}, \prerad{*})$. As in the case for $\DCat$ and 
$\DCatLoc{S}$, the two graded monoidal structures are compatible 
via the inclusion of $\Cat{C}$ into $\CatLoc{C}{S}$. That is, for 
natural numbers $n$ and $m$, the following functor diagram 
commutes:
\[
\begin{tikzcd}
\Prerad{n}{\Cat{C}} \tensor \Prerad{m}{\Cat{C}} \arrow{r} \arrow{d}{i} &
\Prerad{n + m}{\Cat{C}} \arrow{d}{i} \\
\Prerad{n}{\CatLoc{C}{S}} \tensor \Prerad{m}{\CatLoc{C}{S}} \arrow{r} &
\Prerad{n + m}{\CatLoc{C}{S}}.
\end{tikzcd}
\]
\end{thm}

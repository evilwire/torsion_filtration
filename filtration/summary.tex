\newpage
\chapter{Torsion Filtrations on Torsion Monoidal Categories}
\label{sect_filtration_general}

In this section, we generalize the ideas developed by the last
two sections by axiomatizing the necessary components to define
torsion filtrations on $t$-categories $\DCat$ with a symmetric
monoidal structure. Let us begin by defining the following notion:

\begin{defn}\label{def_torsion_monoidal_category}
Let $(\DCat, \tensor, \Unit)$ be a tensor monoidal category with 
a $t$-structure, and let $\Cat{C}$ be the heart. We say that 
$\DCat$ is a \DEF{torsion monoidal category} if $\DCat$ is 
equipped with 
\begin{enumerate}
\item (\itemhead{Partial Internal Hom}) a partial internal hom
structure $(\ihom, \DCat^{\compact})$ (see Definition
\ref{def_tensor_triang_cat}).

\item (\itemhead{Tate Object}) an object $S$ in both 
   $\DCat^{\compact}$ and the heart of $\DCat$ called the 
   \DEF{Tate object}. In particular, $\ihom(S, -)$ is right
   adjoint to $S \tensor -$.
\end{enumerate}

\noindent such that the following conditions hold:

\begin{enumerate}
\item $\Unit$ is an object of $\Cat{C}$,

\item $\tensor$ is right $t$-exact in both factors,

\item (\itemhead{Cancellation}) $\ihom(S, S \tensor M) = M$,

\item $\ihom(S, -)$ is $t$-exact.
\end{enumerate}
\end{defn}

If $\DCat$ is a torsion monoidal category, we will write $\HH^0$
for the cohomological functor from $\DCat$ to its heart. We also 
write $L: \DCat \to \DCat$ for the functor sending an object $M$ 
in $\DCat$ to $M \tensor S$, and $R$ for the functor sending $M$
to $\ihom(S, M)$, where $S$ is the Tate object. By assumption,
$(L, R)$ is an adjoint pair.

Since $\tensor$ is right $t$-exact, it induces a symmetric 
monoidal and a partial internal hom structure on the heart 
$\Cat{C}$ of $\DCat$, which we represent by $\tensor^{\Cat{C}}$ 
and $\ihom_{\Cat{C}}$. The tensor and internal hom bifunctors
are given by
\[
C \tensor^{\Cat{C}} C' \defeq \HH^0(C \tensor C') 
   \;\;\;\textrm{ and }\;\;\;
\ihom_{\Cat{C}}(C, C') \defeq \HH^0(\ihom(C, C')).
\]
We let $L_H$ and $R_H$ denote the endofunctor on 
$\Cat{C}$ given by $F \mapsto F \tensor^{\Cat{C}} S$ and $F \mapsto
\ihom_{\Cat{C}}(S, F)$ respectively.

Let $L^n$ and $R^n$ denote the $n$-th iterations of $L$ and $R$ 
respectively; we define $L_H^n$ and $R_H^n$ similarly. By induction,
$L^n$ is left adjoint to $R^n$. Furthermore, by Cancellation 
(Definition \ref{def_torsion_monoidal_category}(2)) implies that 
$R^nL^n$ is naturally isomorphic to the identity, and the same is 
true for $R_H^nL_H^n$. We now introduce the following theorems 
that generalize and summarize the theory we have developed in the
earlier chapters of this thesis.

\section{Slice filtration of torsion monoidal categories}
\label{sect_summary_slice_filt_on_DCat}

We begin by introducing the slice filtrations on $\DCat$. 

\begin{defn}
Let $\Corad{n}{\DCat}$ be the full subcategory of $\DCat$ by those 
$M$ in $\DCat$ for which $R^n M = 0$. Let $\Prerad{n}{\DCat}$ be 
the subcategory of objects $M$ such that $M = L^n(M')$ for some 
$M'$ in $\DCat$. Let $\preradD{n}$ be the functor  $L^nR^n$.
\end{defn}

The arguments in \cite[1.1]{HuKa} generalize to show that 
$\preradD{n}$ is right adjoint to the inclusion of 
$\Prerad{n}{\DCat}$ into $\DCat$. The crucial piece that
allows the arguments to go through is precisely the Cancellation
property (Definition \ref{def_torsion_monoidal_category} (3)).

Let $M$ be an object of $\DCat$, and let $\eta^n : \preradD{n} M 
\to M$ be the counit. Complete $\eta^n$ to a triangle:
\[
\preradD{n} M \to M \to M' \to \preradD{n} M'[1].
\]
Since 
Copying the proof of Proposition \ref{prop_slice_DMeff}, we 

\begin{thm}\label{thm_summary_triang_cat}
  If $\DCat$ is a torsion monoidal category, then there exists an
  $\N$-indexed ascending weak filtration $(\Corad{*}{\DCat},
  \coradD{*})$ given by
\[
0 = \Corad{0}{\DCat} \subseteq \cdots \subseteq \Corad{n}{\DCat}
   \subseteq \Corad{n + 1}{\DCat} \subseteq \cdots
\]
and a descending weak filtration $(\Prerad{*}{\DCat}, \preradD{*})$
given by
\[
\DCat = \Prerad{0}{\DCat} \supseteq \cdots \supseteq \Prerad{n}{\DCat}
\supseteq \Prerad{n + 1}{\DCat} \supseteq \cdots
\]
\end{thm}

It is possible that the weak filtrations are degenerate. However, as 
the following result shows, the filtration being degenerate is 
related to the invertibility of $S$ (Recall that $S$ is 
invertible if there exists an object $T$ in $\DCat$ such that $T 
\tensor S = \Unit$):

\begin{prop}\label{prop_filt_trivial_cond}
The following are equivalent:
\begin{enumerate}
\item the filtration $(\Corad{*}{\DCat}, \coradD{*})$ is trivial,
  i.e., if each $(\Corad{n}{\DCat}$ is zero.

\item the filtration $(\Prerad{*}{\DCat}, \preradD{*})$ is 
degenerate with $\Prerad{n}{\DCat} = \DCat$ for all $n$.

\item the Tate object is invertible.
\end{enumerate}
\end{prop}
\begin{proof}
We first show that (1) is equivalent to (2). To see that (1) implies
(2), suppose $\Corad{n}{\DCat} = 0$ for all $n$. We need to show
that every $M$ in $\DCat$ is isomorphic to $L^n M'$ for some $M'$
in $\DCat$. However, for every $M$ in $\DCat$, the following is
a distinguished triangle:
\[
\preradD{*}M \to M \to \coradD{*}M \to \preradD{*} M[1]
\]
where $M'$ is in $\Corad{n}{\DCat}$. But the assumption that
$\Corad{n}{\DCat} = 0$ implies that $M' = 0$. Therefore, $L^n R^n M
\cong M$. It follows that $M$ is in $\Prerad{n}{\DCat}$, and
$\Prerad{n}{\DCat} = \DCat$ as desired. Conversely, if
$\Prerad{n}{\DCat} = \DCat$ then for every $M$ in $\DCat$, $M \cong
L^n M'$ for some $M'$. Suppose $M$ is in $\Corad{n}{\DCat}$, then by
definition $0 = R^n M = R^nL^n M' \cong M'$.  Therefore, $M = L^n 0 =
0$, and $\Corad{n}{\DCat} = 0$.

Now we show that (2) is equivalent to (3). Indeed, if $S$ is 
invertible with inverse $T$, then $M = S^{\tensor n} \tensor 
T^{\tensor n} \tensor M = L^n(T^n \tensor M)$ which is an object 
of $\Prerad{n}{\DCat}$. Conversely, if $\Prerad{n}{\DCat} = 
\DCat$, then, in particular, $\Prerad{1}{\DCat} = \DCat$. This
implies that the unit object $\Unit$ is an object of $\Prerad{1}{\DCat}$.
In other words, $\Unit = T \tensor S$ for some $T$, and $S$ is
invertible.
\end{proof}

\section{Torsion filtration on the heart}

Let us now focus on the heart $\Cat{C}$ of $\DCat$. Recall from
an earlier discussion that the endofunctors $L_H^n$ and $R_H^n$ 
are adjoint. For $F$ in $\Cat{C}$, let $\corad{n} F$ denote the 
cokernel of the counit map $L^n R^n F \to F$. The proof of 
Proposition \ref{prop_HI_lower_slice} can be generalized in this 
setting to show that $\corad{n}$ is a functor. In fact, the 
arguments for Theorem \ref{thm_main_result} go through to give us
the following result:

\begin{thm}\label{thm_sum_heart}
The functors $\corad{n}, n = 1, 2\dots,$ define a sequence of
coradicals, whose associated torsion theories $(\Cat{T}_n,
\Cat{F}_i) = (\Prerad{n}{\Cat{C}}, \Corad{n}{\Cat{C}})$ fit together
to define a strong ascending cofiltration of $\Cat{C}$:
\[
0 = \Corad{0}{\Cat{C}} \subseteq \cdots \subseteq \Corad{n}{\Cat{C}}
   \subseteq \Corad{n + 1}{\Cat{C}} \subseteq \cdots
\]
and a strong descending filtration of $\Cat{C}$:
\[
\Cat{C} = \Prerad{0}{\Cat{C}} \supseteq \cdots \supseteq \Prerad{n}{\Cat{C}}
\supseteq \Prerad{n + 1}{\Cat{C}} \supseteq \cdots.
\]
\end{thm}

As in the case for $\HI$, we can define $\SGFilt{n}{\Cat{C}}$ to be
the full subcategory of objects $F$ such that $F \cong L_H^nF'$
for some $F'$ in $\Cat{C}$. As defined, $\SGFilt{n}{\Cat{C}}
\subseteq \SGFilt{m}{\Cat{C}}$ if $n < m$. The arguments of 
Proposition \ref{prop_HI_upper_slice} go through to give us the 
following proposition:

\begin{prop}
The tower of full subcategories
\[
\Cat{C} = \SGFilt{0}{\Cat{C}} \supseteq \cdots \supseteq \SGFilt{n - 1}{\Cat{C}}
\supseteq \SGFilt{n}{\Cat{C}} \supseteq \cdots
\]
defines a weak filtration on $\Cat{C}$.
\end{prop}

As in the case for $\HI$, the coreflection functors from $\Cat{C}$ 
to $\SGFilt{n}{\Cat{C}}$ is given by $F \mapsto L_H^n R_H^n F$. We 
also have the following relationship between $\SGFilt{n}{\Cat{C}}$ 
and $\Prerad{n}{\Cat{C}}$:

\begin{cor}\label{cor_sg_sub_prerad}
For $F$ in $\SGFilt{n}{\Cat{C}}$, $\hom_{\Cat{C}}(F, G) = 0$ for
all $G$ in $\Prerad{n}{\Cat{C}}$. In particular, $\SGFilt{n}{\Cat{C}}$
is a full subcategory of $\Prerad{n}{\Cat{C}}$.
\end{cor}
\begin{proof}
If $F$ is an object of $\SGFilt{n}{\Cat{C}}$, then $F = L_H^n F'$ for
some $F'$ in $\Cat{C}$. Since $R_H^n G = 0$ for all $G$ in 
$\Corad{n}{\Cat{C}}$,
\[
\hom_{\Cat{C}}(F, G) = \hom_{\Cat{C}}(L_H^n F', G) = \hom_{\Cat{C}}(F', R_H^n G) = 0
\]
for all $G$ in $\Corad{n}{\Cat{C}}$.

The first statement of the corollary is now proven. The second 
statement follows from the definition of $\Prerad{n}{\Cat{C}}$ as 
the torsion subcategory of $\Corad{n}{\Cat{C}}$.
\end{proof}

The filtrations on $\Cat{C}$ may be trivial, as well. The following
proposition shows that, as in the case for $\DCat$, the degeneracy
of the filtrations are related to the the invertibility of $S$.
Since $\HH^0(\Corad{n}{\DCat}) = \Corad{n}{\Cat{C}}$ and
$\HH^0(\Prerad{n}{\DCat}) = \SGFilt{n}{\Cat{C}}$ (\emph{cf}. 
\ref{prop_H_commute_with_filt}), if $S$ is invertible, then certainly
$\Corad{n}{\Cat{C}} = 0$ and $\Prerad{n}{\DCat} = \DCat$ for all $n$.
However, the converse does not necessarily hold. Rather, the converse
is related to a weaker condition: we say that $S$ is 
\DEF{$\tensor^{\Cat{C}}$-invertible} if there exists some $T$ such
that $T \tensor^{\Cat{C}} S = \Unit$. Since $\Unit$ is an object of
$\Cat{C}$ (Definition \ref{def_torsion_monoidal_category} (1)), if
$S$ is invertible, then $S$ is $\tensor^{\Cat{C}}$-invertible.

\begin{prop}\label{prop_torsion_filt_degen_cond}
The following are equivalent:
\begin{enumerate}
\item $\SGFilt{*}{\Cat{C}}$ is degenerate with $\SGFilt{n}{\Cat{C}} = 
\Cat{C}$ for all $n$,

\item $\Corad{*}{\Cat{C}}$ is trivial,
for all $n$,

\item $\Prerad{*}{\Cat{C}}$ is degenerate with $\Prerad{n}{\Cat{C}} = 
\Cat{C}$ for all $n$,

\item $S$ is $\tensor^{\Cat{C}}$-invertible.
\end{enumerate}
\end{prop}
\begin{proof}
We first show that (1), (2), and (4) are equivalent. The proof that
(1) and (4) are equivalent is the same as in the case for Proposition
\ref{prop_filt_trivial_cond}. Next, we show that (2) implies (1).
 Fix $F$ in $\Cat{C}$, and let $K$ be the kernel of
the counit $L_H^n R_H^n F \to F$. Then we have the following exact
sequence:
\[
0 \to K \to L_H^n R_H^n F \to F \to \corad{n} F \to 0.
\]
By Definition \ref{def_torsion_monoidal_category} (4) and Proposition
\ref{prop_t_exact_implies_exact}, $R_H$ is exact, and therefore, $R_H^n$
is exact as well. Applying $R_H^n$, we obtain the following exact sequence:
\[
0 \to R_H^n K \to R_H^n L_H^n R_H^n F \to R_H^n F \to R_H^n \corad{n} F \to 0.
\]
But since $R_H^n L_H^n \cong \id$, $R_H^n L_H^n R_H^n F \to R_H^n F$ is 
an isomorphism. Therefore, $R_H^n K = R_H^n \corad{n} F = 0$. That is,
$K$ and $\corad{n} K$ are in $\Corad{n}{\Cat{C}}$. It follows that
$K = \corad{n} K = 0$, and therefore $F = L_H^n R_H^n F$. It follows that
$\SGFilt{n}{\Cat{C}} = \Cat{C}$.

Next, we show that (4) implies (2). To see this, suppose $F$ is in 
$\Corad{n}{\Cat{C}}$. Then by (4), $F \cong S^{\tensor n} 
\tensor^{\Cat{C}} T^{\tensor n}$ for some $T$ in $\Cat{C}$. Therefore,
$L_H^n R_H^n F \cong F$. However, this means that $\corad{n} F = 0$.
By Theorem \ref{thm_precorad_eq_tt}, $\corad{n} F = F$. Therefore,
$F = 0$ and $\Corad{n}{\Cat{C}} = 0$.

To show that (3) is equivalent to the rest, we first show that (2) 
implies (3). Suppose $\Corad{n}{\Cat{C}}$ is trivial. Since 
$\corad{n} F$ is an object of $\Corad{n}{\Cat{C}}$ for all $F$, it 
follows that $\corad{n}F = 0$. Therefore, $\prerad{n}F = F$. Thus, 
$\Prerad{n}{\Cat{C}} = \Cat{C}$, as desired.

Finally, we show that (3) implies (2). Suppose now that 
$\Prerad{n}{\Cat{C}} = \Cat{C}$. That is, for all $F$ in 
$\Cat{C}$, $\corad{n}(F) = 0$. Now suppose that $F$ is in 
$\Corad{n}{\Cat{C}}$. Since $\Corad{n}{\Cat{C}}$ is the 
torsionfree subcategory associated to the coradical $\corad{n}$, 
by Theorem \ref{thm_precorad_eq_tt}, $0 = \corad{n}(F) = F$. 
Therefore, $\Corad{n}{\Cat{C}} = 0$.
\end{proof}

\section{Slice filtration on the localization of $\DCat$ by $S$}

Next, for a torsion monoidal category $\DCat$, we can form the
localization $\DCatLoc{S}$ of $\DCat$ by $S$ (see \cite[8A]{MVW}). 
Here the objects of $\DCatLoc{S}$ are pairs $(M, n)$, where $M$ is 
in $\DCat$, and $n$ is some integer, and $(M, n + 1) \cong (LM, 
n)$ for all $M$ and $n$. As in the case for $\DM$, $\DCatLoc{S}$
has the structure of a triangulated category, and $\DCat$ can be
realized as a full subcategory of $\DCatLoc{S}$ whose objects
are the objects $(M, n)$ such that $n = 0$. To be notationally
consistent, let $\Sigma^\infty$ denote the inclusion of $\DCat$
into $\DCatLoc{S}$.

Furthermore, we can extend the definition of $\HH^0$ on $\DCat$ to 
$\DCatLoc{S}$. Define $\CatLoc{C}{S}$ to be the full subcategory
of $\DCatLoc{S}$ whose objects are the pairs $(F, n)$ where $F$.
The restriction of $\Sigma^\infty$ to $\Cat{C}$ defines an 
inclusion into $\CatLoc{C}{S}$, which we represent by 
$\sigma^\infty$. The arguments of \cite[5.6]{DegModHom} can be 
adapted to show that the triangulated category $\DCatLoc{S}$ has 
a $t$-structure, whose heart is $\CatLoc{C}{S}$. Furthermore, 
there exists a commutative diagram of categories:
\[
\begin{tikzcd}[column sep=75pt, row sep=large]
\DCat \arrow[bend right]{r}{\Sigma^{\infty}} \arrow{d}{\HH^0} &
\DCatLoc{S} \arrow[bend right]{l}{\Omega^\infty} \arrow{d}{\HH^0} \\
\Cat{C} \arrow[bend right]{r}{\sigma^\infty}  &
\CatLoc{C}{S} \arrow[bend right]{l}{\omega^\infty}
\end{tikzcd}
\]

There is also a tensor product on $\DCatLoc{S}$, given by
\[
(M, n) \tensor (M', n') = (M \tensor M', n + n')
\]
(see \cite[8A]{MVW}). In the case that the cyclic permutation of 
$(S, 0)^{\tensor 3}$ is the identity in $\DCatLoc{S}$, by 
\cite[8A.10, 8A.11]{DegModHom} the tensor product is also 
triangulated symmetric tensor on $\DCatLoc{S}$. In this case, 
$\DCatLoc{S}$ is also a torsion monoidal category. However, since 
$S$ is invertible, defining the weak filtrations as we have done 
in Section \ref{sect_summary_slice_filt_on_DCat} will result in 
trivial weak filtrations, as we have shown in Proposition 
\ref{prop_filt_trivial_cond}. Nonetheless, there still exist 
weak filtrations on $\DCatLoc{S}$.

\begin{thm}\label{thm_summary_dloc}
There exists an ascending weak filtration
\[
\cdots \subseteq \Corad{0}{\DCatLoc{S}} \subseteq \cdots \subseteq 
   \Corad{n}{\DCatLoc{S}} \subseteq \Corad{n + 1}{\DCatLoc{S}}
   \subseteq \cdots
\]
and a descending weak filtration
\[
\cdots \supseteq \Prerad{0}{\DCatLoc{S}} \supseteq \cdots \supseteq 
   \Prerad{n}{\DCatLoc{S}} \supseteq \Prerad{n + 1}{\DCatLoc{S}}
   \supseteq \cdots
\]
\end{thm}
\begin{proof}
Let $\Prerad{k}{\DCatLoc{S}}$ be the full subcategory of 
$\DCatLoc{S}$ with objects $(M, n)$ such that $(M, n) \cong
(M', k)$ for some $M'$ in $\DCat$. Since $(M, k + 1) \cong 
(LM, k)$, we have the following descending tower of full 
subcategories:
\[
\DCatLoc{S} \supseteq \cdots \supseteq \Prerad{0}{\DCatLoc{S}}
   \supseteq \cdots \supseteq \Prerad{n}{\DCatLoc{S}} \supseteq 
   \Prerad{n + 1}{\DCatLoc{S}} \supseteq \cdots.
\]
We now need to show that for each integer $n$, there exists a 
coreflection $\preradD{n}: \DCatLoc{S} \to \Prerad{n}{\DCatLoc{S}}$.
As in Proposition \ref{prop_sDM_reflection}, this is given by the
functors $\preradD{k}$ defined by
\[
\preradD{k}(M, n) = \begin{cases}
(\preradD{k - n}M, n) &\textrm{if }k > n\\
(M, n) &\textrm{otherwise}.
\end{cases}
\]

Now let $\Corad{k}{\DCatLoc{S}}$ be the full subcategory of
objects $(M, n)$ where $\preradD{k}(M, n) = 0$. It is easy to
see that
\[
\cdots \subseteq \Corad{0}{\DCatLoc{S}} \subseteq \cdots \subseteq 
   \Corad{n}{\DCatLoc{S}} \subseteq \Corad{n + 1}{\DCatLoc{S}}
   \subseteq \cdots \DCatLoc{S}.
\]
In this case, the arguments for Proposition 
\ref{prop_slDM_functor} go through to show that the 
functors $\coradD{k} : \DCatLoc{S}
\to \Corad{k}{\DCatLoc{S}}$ given by
\[
\coradD{k}(M, n) = \begin{cases}
(\coradD{n - k}M, n) &\textrm{if }n > k\\
0                    &\textrm{otherwise}
\end{cases}
\]
define are right adjoint to the inclusion of 
$\Corad{k}{\DCatLoc{S}}$ into $\DCatLoc{S}$.
\end{proof}

The following proposition, which is a consequence of Proposition 
\ref{prop_filt_trivial_cond}, relate the degeneracy of the weak 
filtrations that we defined above with the invertibility of the
Tate object $S$.

\begin{prop}
The following are equivalent:
\begin{enumerate}
\item the categories $\Corad{n}{\DCatLoc{S}}$ are trivial,

\item the categories $\Prerad{n}{\DCatLoc{S}}$ is degenerate with
$\Prerad{n}{\DCatLoc{S}} = \DCatLoc{S}$ for all $n$,

\item $S$ is invertible.
\end{enumerate}
\end{prop}
\begin{proof}
If $S$ is invertible, then $\DCatLoc{S}$ is equivalent to $\DCat$.
Therefore, the proposition follows directly from Proposition
\ref{prop_filt_trivial_cond}.
\end{proof}

\section{Torsion Filtration on the Localization of $\Cat{C}$ by $S$}

Let $(F, k)$ be an object of $\CatLoc{C}{S}$, and we define
$\corad{n}(F, k)$ to be $(\corad{n - k}F, k)$. Here, we 
adopt the convention that $\corad{n - k} = 0$ if $n - k \leq 0$.
The arguments of Lemma \ref{lem_tlHM_is_functor} go through to
show that $\corad{n}$ is an endofunctor on $\Cat{C}{S}$ for each
integer $n$. In fact, Theorem \ref{thm_tlHM_corad} generalizes to
the following:

\begin{thm}
\label{thm_sum_heart_loc}
The sequence of functors $\corad{n}, n = \dots, -1, 0, 1, \dots$ on
$\CatLoc{C}{S}$ is a $\Z$-indexed sequence of coradicals whose 
associated torsion theories 
\[
(\Cat{T}_n, \Cat{F}_n) = (\Prerad{n}{\CatLoc{C}{S}}, \Corad{n}{\CatLoc{C}{S}})
\]
define an ascending strong cofiltration
\[
\cdots \subseteq \Corad{0}{\CatLoc{C}{S}} \subseteq \cdots \subseteq 
   \Corad{n}{\CatLoc{C}{S}} \subseteq \Corad{n + 1}{\CatLoc{C}{S}}
   \subseteq \cdots
\]
and a strong descending filtration
\[
\CatLoc{C}{S} \supseteq \cdots \supseteq \Prerad{0}{\CatLoc{C}{S}}
   \supseteq \cdots \supseteq \Prerad{n}{\CatLoc{C}{S}} \supseteq 
   \Prerad{n + 1}{\CatLoc{C}{S}} \supseteq \cdots.
\]
on $\CatLoc{C}{S}$.
\end{thm}

Notice that we can extend the weak filtration $(\SGFilt{*}{\Cat{C}},
\sgHI{*})$ to a weak filtration on $\CatLoc{C}{S}$. Let 
$\SGFilt{n}{\Cat{C}}$ be the full subcategory of objects $(F, k)$ 
such that $k \geq n$. As defined, we have a tower of subcategories
\[
\CatLoc{C}{S} \supseteq \SGFilt{-1}{\CatLoc{C}{S}} \supseteq 
   \SGFilt{0}{\CatLoc{C}{S}} \subseteq \cdots \supseteq 
   \SGFilt{n - 1}{\CatLoc{C}{S}} 
   \supseteq \SGFilt{n}{\CatLoc{C}{S}} \supseteq \cdots.
\]
Furthermore, $\SGFilt{n}{\CatLoc{C}{S}}$ is the essential image 
under $\HH^0$ of $\Prerad{n}{\DCatLoc{S}}$. The coreflection functor
$\preradD{n}: \DCatLoc{S} \to \Prerad{n}{\DCatLoc{S}}$ induces a
coreflection functor $\sgHI{n}$ from $\CatLoc{C}{S}$ to 
$\Prerad{n}{\Cat{C}{S}}$ given by
\[
(F, k) \mapsto \HH^0\preradD{n}(F, k) = (\prerad{n - k}F, k).
\]
It follows that $(\SGFilt{*}{\CatLoc{C}{S}}, \sgHI{n})$ defines
a weak filtration of $\CatLoc{C}{S}$. The following proposition,
which is a direct consequence of Proposition
\ref{prop_torsion_filt_degen_cond}, shows that $S$ is 
$\tensor^{\Cat{C}}$-invertibility if and only if each of the
above weak filtrations are degenerate:

\begin{prop}
The following are equivalent:

\begin{enumerate}
\item $\SGFilt{*}{\CatLoc{C}{S}}$ is degenerate with 
$\SGFilt{n}{\CatLoc{C}{S}} = \Cat{C}$ for all $n$,

\item $\Corad{*}{\CatLoc{C}{S}}$ is trivial,
for all $n$,

\item $\Prerad{*}{\CatLoc{C}{S}}$ is degenerate with 
$\Prerad{n}{\CatLoc{C}{S}} = \Cat{C}$ for all $n$,

\item $S$ is $\tensor^{\Cat{C}}$-invertible.
\end{enumerate}
\end{prop}
\begin{proof}
Notice that $S$ is $\tensor^{\Cat{C}}$-invertible if and
only if $\CatLoc{C}{S}$ is equivalent to $\CatLoc{C}{S}$.
Therefore, the equivalence between (1), (2), (3), and (4)
follows directly from Proposition 
\ref{prop_torsion_filt_degen_cond}.
\end{proof}

\section{Graded monoidal structures on torsion monoidal categories}

Finally, the descending filtrations, together with
the respective symmetric monoidal structure, give rise to a
graded monoidal structure (see Definition 
\ref{def_graded_tensor}). In order to formulate our final set of 
results, let us first consider the following proposition which 
follows straightforward from the Cancellation property (Definition
\ref{def_torsion_monoidal_category} (2)):

\begin{prop}
There exists a fully faithful $t$-exact functor $i: \DCat \to 
\DCatLoc{S}$, given by $M \mapsto (M, 0)$ that realizes $\DCat$ 
as a full subcategory of $\DCatLoc{S}$. The restriction of $i$ 
to the heart of $\DCat$ induces a corresponding fully faithful 
functor $i : \Cat{C} \to \CatLoc{C}{S}$.
\end{prop}

By definition of $i$, $\DCat$ is identified with the full 
subcategory $\Prerad{0}{\DCatLoc{S}}$ of $\DCatLoc{S}$, and
$\Cat{C}$ is identified with $\Prerad{0}{\CatLoc{C}{S}}$.
The following is an easy of the definition:

\begin{thm}\label{thm_graded_monoid_DCat}
There is a graded monoidal structure on $\DCatLoc{S}$ defined by
the filtration $(\Prerad{*}{\DCatLoc{S}}, \preradD{*})$ and a 
graded monoidal structure on $\DCat$ defined by the filtration
$(\Prerad{*}{\DCat}, \preradD{*})$. The two graded monoidal 
structures respects the inclusion of $\DCat$ into $\DCatLoc{S}$ in 
the sense that, for all natural numbers $n$ and $m$, the following 
functor diagram commutes:
\[
\begin{tikzcd}
\Prerad{n}{\DCat} \tensor \Prerad{m}{\DCat} \arrow{r} \arrow{d}{i} &
\Prerad{n + m}{\DCat} \arrow{d}{i} \\
\Prerad{n}{\DCatLoc{S}} \tensor \Prerad{m}{\DCatLoc{S}} \arrow{r} &
\Prerad{n + m}{\DCatLoc{S}}
\end{tikzcd}
\]
\end{thm}

By applying $\HH^0$ to the diagram in Theorem 
\ref{thm_graded_monoid_DCat}, we obtain the following:

\begin{cor}
There is a graded monoidal structure on $\CatLoc{C}{S}$ defined
by the filtration $(\SGFilt{*}{\CatLoc{C}{S}}, \sgHI{*})$ and a graded 
monoidal structure on $\Cat{C}$ defined by $(\SGFilt{*}{\Cat{C}}, 
\sgHI{*})$. That is, for natural numbers $n$ and $m$, the 
following functor diagram commutes:
\[
\begin{tikzcd}
\SGFilt{n}{\Cat{C}} \tensor \SGFilt{m}{\Cat{C}} \arrow{r} \arrow{d}{i} &
\SGFilt{n + m}{\Cat{C}} \arrow{d}{i} \\
\SGFilt{n}{\CatLoc{C}{S}} \tensor \SGFilt{m}{\CatLoc{C}{S}} \arrow{r} &
\SGFilt{n + m}{\CatLoc{C}{S}}
\end{tikzcd}
\]
\end{cor}

Finally, the arguments for Proposition \ref{prop_graded_mon_struct_HM} go 
through to show that we also have a graded monoidal structure 
associated with the torsion filtration:

\begin{thm}
There is a graded monoidal structure on $\CatLoc{C}{S}$ defined
by the strong filtration $(\Prerad{*}{\CatLoc{C}{S}}, \prerad{*})$ 
and a graded monoidal structure on $\Cat{C}$ defined by 
$(\Prerad{*}{\Cat{C}}, \prerad{*})$. As in the case for $\DCat$ and 
$\DCatLoc{S}$, the two graded monoidal structures are compatible 
via the inclusion of $\Cat{C}$ into $\CatLoc{C}{S}$. That is, for 
natural numbers $n$ and $m$, the following functor diagram 
commutes:
\[
\begin{tikzcd}
\Prerad{n}{\Cat{C}} \tensor \Prerad{m}{\Cat{C}} \arrow{r} \arrow{d}{i} &
\Prerad{n + m}{\Cat{C}} \arrow{d}{i} \\
\Prerad{n}{\CatLoc{C}{S}} \tensor \Prerad{m}{\CatLoc{C}{S}} \arrow{r} &
\Prerad{n + m}{\CatLoc{C}{S}}.
\end{tikzcd}
\]
\end{thm}

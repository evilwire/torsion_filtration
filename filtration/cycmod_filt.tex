\section{Torsion filtration on cycle modules}

We conclude this chapter by showing that there is a torsion 
filtration structure on the category of cycle modules (defined 
below). Recall from \cite{MilK} that for a field $F$, the Milnor 
$K$-theory of $F$ is the graded commutative ring given by
\[
\milK_*(F) \defeq T^*(F^*)/I
\]
where $T^*(F^*)$ denotes the tensor algebra of the multiplicative 
group $F^*$, and $I$ denotes the ideal generated by $a \tensor 
(1 - a)$ for all $a$ in $F^*$. We define $\milK_n(F)$ to be $0$
for $n < 0$ and let $\milK_n(F)$ be the $n$-th graded piece of 
$\milK_*(F)$. We call $\milK_n(F)$ the \DEF{$n$-th Milnor 
$K$-theory of $F$}.

\begin{defn}[\cite{Rost96} 1.1]\label{def_pre_cycmod}
Let $X$ be a finite-type $\basefield$-scheme, and let 
$\fields(\basefield)$ be the category of function fields $E$ of 
$\Sm$, i.e., $E$ is the function field of some $\basefield$-scheme 
$X$ in $\Sm$, and any morphism $E \to E'$ in $\fields(\basefield)$
is a field homomorphism such that the restriction to $\basefield$ 
is the identity. A \DEF{cycle premodule} $M$ is a functor which 
assigns to every field $E$ in $\fields(\basefield)$ a $\Z$-graded abelian 
group $\displaystyle M(E) = \{M_i\}_{i \in \Z}$, together with
the following data:

\begin{enumerate}[label=\bfseries D\arabic*.]
\item[\textbf{D1.}] For each field extension $\phi: E' \to E$, 
there is a degree 0 map $\phi_*: M(E') \to M(E)$ called the 
\emph{restriction map associated to $\phi$}

\item[\textbf{D2.}] For each finite extension $\phi: E' \to E$, 
there is a degree 0 map $\phi^*: M(E) \to M(E')$ called the 
\emph{corestriction map associated to $\phi$}

\item[\textbf{D3.}] For each $E$ in $\fields(\basefield)$, the group $M(E)$ 
is equipped with the structure of a left $\milK_*(E)$-module, where
$\milK_*(E)$ is the Milnor $K$-ring of $E$.

\item[\textbf{D4.}] For a given valuation $v$ of $E$ in 
$\fields(\basefield)$, there exists a map of degree -1 $\res{v}: M(E) \to 
M(\resf{v})$ called the \emph{residue} map, where $\resf{v}$ is 
the residue field of $v$.
\end{enumerate}

The data given in D1 - D4 satisfy the following criteria.
For a given valuation $v$ of $E$ in $\fields(\basefield)$,
fix $p$ to be a prime of $v$. The $\milK_*(E)$-module structure
in D3 and the residue map in D4 give
rise to a degree preserving map $\Special{v}{p} : M(E) \to
M(\resf{v})$ defined by $\Special{v}{p}(\rho) =
\res{v}(\{p\} \cdot \rho)$, where $\{p\}$ the element in 
$\milK_1(E)$ represented by $E$. Following \cite[1.1]{Rost96},
we call $\Special{v}{p}$ the \DEF{specialization} map.

\begin{enumerate}[label=\bfseries R1\alph*., leftmargin=3em]
\item For each field extension $\phi: E' \to E$ and 
field extension $\psi: E \to E''$, $(\psi \comp \phi)_* = \psi_* 
\comp \phi_*$

\item For each finite extension $\phi: E' \to E$ and 
finite extension $\psi: E \to E''$, $(\psi \comp \phi)^* = \phi^* 
\comp \psi^*$

\item For finite extension $\phi: E' \to E$ and any 
field extension $\psi: E' \to E''$ with $\phi$ finite, define $R = E 
\otimes_{E'} E''$, and let $\ideal{p}$ be any prime ideal of $R$. (As 
$R$ is Artin, let $l_p$ be the length of the local ring 
$R_{(\ideal{p})}$), and $\phi_{\ideal{p}}: E'' \to R/{\ideal{p}}$ 
and $\psi_{\ideal{p}}: E \to R/{\ideal{p}}$ be natural maps.
\[
\psi_*\comp \phi^* = \sum_{\ideal{p}} l_p \cdot 
(\phi_{\ideal{p}})^* \comp (\psi_{\ideal{p}})_*.
\]
\end{enumerate}
\begin{enumerate}[label=\bfseries R2\alph*., leftmargin=3em]
\item[\textbf{R2.}] For any extension $\phi: E' \to E$, $x \in 
\milK_*(E')$, $y \in \milK_*(E)$, $\rho \in M(E')$, and $\mu \in 
M(E)$, then:

\item $\phi_*(x \cdot \rho) = \phi_*(x) \cdot 
\phi_*(\rho)$.

\item if $\phi$ is finite, $\phi^*(\phi_*(x) \cdot 
\mu) = x \cdot \phi^*(\mu)$.

\item if $\phi$ is finite, $\phi^*(y \cdot 
\phi_*(\rho)) = \phi^*(y) \cdot \rho$.
\end{enumerate}
\begin{enumerate}[label=\bfseries R3\alph*., leftmargin=3em]
\item[\textbf{R3.}] For any field extension $\phi: E' \to E$, $v$ 
a valuation on $E$ and $w$ and a valuation on $E'$:

\item Suppose $w$ is a nontrivial restriction of 
$v$ with ramification index $e$. Let $\phib: \resf{w} \to 
\resf{v}$ be the induced map. Then:
\[
\res{v} \comp \phi_* = e \cdot \phib_* \comp \res{w}.
\]

\item Let $\phi$ be a finite extension, suppose
$w$ is an extension of $v$ to $E$. Let $\phi_v: \resf{w} \to 
\resf{v}$ be the induced map on the residue fields. Then
\[
\res{v} \comp \phi^* \sum_{v} \comp \res{v}.
\]

\item Suppose $v$ restricts to a trivial valuation 
on $E'$. Then
\[
\res{v} \comp \phi_* = 0
\]

\item Suppose $v$ restricts to a trivial valuation 
on $E'$. Let $\phib: F \to \resf{v}$ be the induced map on 
the residue fields. Let $p$ a prime of $v$. Then
\[
\Special{v}{p} \comp \phi_* = \phib_*
\]

\item Let $u$ be an element of $E$ such that 
$v(u) = 0$. Given $\rho$ in $M(F)$, one has
\[
\res{v}(\{u\} \cdot \rho) = -\{\overline{u}\} \cdot \res{v}(\rho).
\]
\end{enumerate}
\end{defn}

For $X$ a $k$-scheme, let $\subsch{1}{X}$ denote the collection of 
codimension 1 subschemes. Let $\xi_X$ be the generic point of an
irreducible $X$ with $K_X = \O_X,\xi_X$. If $X$ is normal, then 
for $x$ in $\subsch{1}{X}$, the local ring $\O_{x,X}$ is a valuation 
ring of $K_X$ with residue field $\resf{x}$. Write $M(x)$ for 
$M(\resf{x})$, and $\res{x}: M(\xi_X) \to M(x)$ for the 
restriction map.

Furthermore, for $x, y \in X$, let $Z$ be the closed subscheme 
determined by $x$, and $\overline{Z}$ be the normalization $Z$.
Define
\[
\ptres{x}{y}: M(x) \to M(y)
\]
by
\[
\ptres{x}{y} = 
\begin{cases}
0 & y \notin \subsch{1}{Z} \\
\sum_{z|y} \phi_{\resf{z},\resf{x}}^* \comp \res{z} & \textrm{otherwise}.
\end{cases}
\]
Here, following \cite{Rost96}, $z|y$ denotes the relation that $z$ 
lies over $y$. In particular, if $y \in \subsch{1}{Z}$, the sum 
is taken over all $z$ lying over $y \in \subsch{1}{Z}$. In this
case, $\phi_{\resf{z},\resf{y}}^*$ is the corestriction map 
associated to the finite field extension $\resf{y} \to \resf{z}$.

\begin{defn}[\cite{Rost96} 2.1]
A cycle module $M$ on $\fields(\basefield)$ is a cycle premodule that
satisfies the following conditions:

\begin{enumerate}[leftmargin=3em]
\item[\textbf{(FD)}] \itemhead{Finite support of divisors.} 
$X$ be a normal scheme and $\rho \in M(\xi_X)$. Then $\res{x}: 
M(\xi_X) \to M(X)$ is 0 for all but finitely many $x \in 
\subsch{1}{X}$.

\item[\textbf{(C)}] \itemhead{Closedness.} If $X$ is an integral
local scheme of dimension 2 with closed point $x_0$, then the map 
from $M(\xi_X)$ to $M(x_0)$ given by
\[
\sum_{x \in \subsch{1}{X}} \ptres{x_0}{x} \comp \ptres{x}{\xi}
\]
is 0.
\end{enumerate}
\end{defn}

D\'eglise showed in \cite{DegModHom} that a homotopy module $(F_*, 
\deloop_*)$ gives rise to a unique cycle module $\assocCM{F_*}$,
and that this association defines an equivalence between the 
category of homotopy modules and cycle modules (see 
\cite[3.7]{DegModHom}). Via this categorical equivalence, we 
obtain the following corollary: 

\begin{cor}\label{cor_tor_filt_on_CycMod}
There exists a $\Z$-indexed torsion filtration on $\CycMod$. That
is, there exists a $\Z$-indexed sequence of coradicals, which by abuse
of notation, we also represent by $\tlHM{i}$ such that the 
associated torsion subcategories $\TFCycMod{i}$ form an ascending 
strong cofiltration of $\CycMod$:
\[
\cdots \subseteq \TFCycMod{-1} \subseteq \TFCycMod{0} \subseteq \cdots 
   \subseteq \TFCycMod{i} \subseteq \cdots 
   \subseteq \CycMod
\]
and the associated torsionfree subcategories $\TCycMod{i}$
form a descending strong filtration of $\CycMod$:
\[
\cdots \subseteq \TCycMod{i} \subseteq 
   \cdots \subseteq \TCycMod{0} \subseteq \TCycMod{-1} \subseteq \cdots
   \cdots \subseteq \CycMod.
\]
\end{cor}

\begin{ex}\label{ex_milK}
Milnor $K$-theory $\milK_*$, defined in the paragraph preceding 
Definition \ref{def_pre_cycmod}, is an example of a cycle module 
(see \cite[1.4, 2.5]{Rost96}). By \cite[3.7]{DegModHom}, the 
homotopy module corresponding to $\milK_*$ is $\spectHI(\Z)$. As 
we have shown in Example \ref{ex_TFHI_eq_TFHM}, $\spectHI(\Z)$ is 
an object of $\THM{1} \cap \TFHM{0}$. Hence, $\milK_*
\in \TCycMod{1} \cap \TFCycMod{0}$.
\end{ex}

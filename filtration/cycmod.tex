\newpage
\section{Filtration on $\CycMod$}\label{sect_filtration_cycmod}

In this section, we extend the filtration structure defined on 
$\HI$ to the category of homotopy modules $\HM$. Via the 
categorical equivalence between $\CycMod$ and $\HM$ (Thm. 
\ref{thm_cycmod_eq_hm}), we obtain a filtration on the abelian 
category $\CycMod$.

Throughout the section, we adopt the same notation convention as 
in Section \ref{sect_cycmod_and_hm} and write $(F_*, \deloop_*)$ 
for a homotopy module where the $\deloop_n: F_{n} \to 
\RHI{(F_{n + 1})}$ denotes the $n$-th delooping map (see Def. 
\ref{def_hm}).

The following lemma will be key to extending the functors 
$\tlHI{n}$, $\tgHI{n}$ and $\sgHI{n}$ on $\HM$:

\begin{prop}\label{prop_tl_L_R}
For $F \in \HI$, 
\[
\LHI{(\tlHI{n} F)} \simeq \tlHI{n + 1}(\LHI{F})
\]
and
\[
\RHI{(\tlHI{n} F)} \simeq \tlHI{n - 1}(\RHI{F}).
\] 
Both isomorphisms are natural in $F$.
\end{prop}
\begin{proof}
For ease of notation, let $L$ be the functor defined by $F \mapsto
\LHI{F}$, and $R$ be the functor given by $F \mapsto \RHI{F}$, and
$(L, R)$ form an adjoint pair, and $RL$ is naturally ismorphic to
the identity (Prop. \ref{prop_unit_iso}).

The key is to prove that the diagram
\begin{equation}\label{eq_Ltl_com_diag}
\begin{tikzcd}
\L^{n + 1}R^{n + 1}LF \arrow{r}{\eta^{n + 1}_{LF}} \arrow{d} &
LF \arrow[equals]{d} \\
L(L^{n}R^nF) \arrow{r}{L\eta^{n}_F} &
LF.
\end{tikzcd}
\end{equation}
is commutative. Here, $\eta^n$ denotes the counit $L^nR^n \to \id$,
and the vertical map $L^{n + 1}R^{n + 1}LF \to L(L^nR^n F)$ is 
given by the map $L^{n + 1}R^n \epsilon^{-1}$, where $\epsilon$ is
the unit $\id \to RL$ and an isomorphism.

To establish that the above diagram is commutative, we proceed by 
induction on $n$. The case $n = 0$ follows by the counit-unit 
adjunction. That is, the composition
\[
LF \stackrel{L\epsilon}{\to} LRLF \stackrel{\eta L}{\to} LF.
\]
is the identity. Therefore, $\eta L = L\epsilon^{-1}$, and the
following diagram commutes:
\[
\begin{tikzcd}
LRL F \arrow{r}{\eta L} \arrow{d}{L \epsilon^{-1}} &
LF \arrow[equals]{d} \\
LF \arrow[equals]{r} & LF.
\end{tikzcd}
\]

Now, suppose we have
\begin{equation}\label{eq_induct_hyp_tl_diag}
\begin{tikzcd}
\L^{n}R^{n}LF \arrow{r}{\eta^{n}_{LF}} \arrow{d} &
LF \arrow[equals]{d} \\
L(L^{n - 1}R^{n - 1}F) \arrow{r}{L\eta^{n - 1}_F} &
LF.
\end{tikzcd}
\end{equation}
There exists a natural transformation $\eta': L^n R^n \to 
L^{n - 1} R^{n - 1}$. Applying this to $\epsilon^{-1}: RL F \to 
F$, we have the following commutative diagram
\[
\begin{tikzcd}
L^n R^n RL F \arrow{r}{\eta'_{RLF}} 
   \arrow{d}{L^nR^n \epsilon^{-1}} &
L^{n - 1}R^{n - 1} RLF 
   \arrow{d}{L^{n - 1}R^{n - 1}\epsilon^{-1}} \\
L^nR^n F \arrow{r}{\eta'_F} &
L^{n - 1}R^{n - 1} F,
\end{tikzcd}
\]
and applying $L$, we have
\[
\begin{tikzcd}
L^{n + 1} R^{n + 1} L F \arrow{r}{L\eta'_{RLF}} 
   \arrow{d}{L^{n + 1}R^n \epsilon^{-1}} &
L^{n}R^n LF \arrow{d}{L^nR^{n - 1}\epsilon^{-1}} \\
L^{n + 1}R^n F \arrow{r}{L\eta'_F} &
L^{n}R^{n - 1} F,
\end{tikzcd}
\]
which fits together with Diagram \ref{eq_induct_hyp_tl_diag} to 
give the following commutative diagram:
\[
\begin{tikzcd}
L^{n + 1} R^{n + 1} L F \arrow{r}{L\eta'_{RLF}} 
   \arrow{d}{L^{n + 1}R^n \epsilon^{-1}}
   \arrow[bend left]{rr}{\eta^{n + 1}_{LF}} &
L^{n}R^n LF \arrow{d}{L^nR^{n - 1}\epsilon^{-1}}
   \arrow{r}{\eta^{n}_{LF}} &
LF \arrow[equals]{d} \\
L^{n + 1}R^n F \arrow{r}{L\eta'_F} 
   \arrow[bend right]{rr}{L\eta^n_F} &
L^{n}R^{n - 1} F \arrow{r}{L\eta^{n - 1}_F} &
LF.
\end{tikzcd}
\]
Notice that the composition of the top and bottom horizontal 
arrows (indicated by the bent arrows) in the diagram above are 
precisely $\eta^{n + 1}_KF$ and $L \eta^n_F$ respectively, and
the claim is proved.

Returning to Diagram \ref{eq_Ltl_com_diag}, which we now know to 
be commutative, notice that the cokernel of the top row is 
$\tlHI{n + 1}LF$, and since $L$ is right exact, the cokernel of 
the bottom row is $L\tlHI{n} F$. By the Five Lemma, it is clear 
that $\tlHI{n + 1}LF \simeq L\tlHI{n} F$.

For the other isomorphism, apply similar arguments to the
diagram
\[
\begin{tikzcd}
L^{n - 1}R^{n - 1}(RF) \arrow{r}{\eta^n_{RF}} 
   \arrow{d}{\eta_{L^{n - 1}R^{n - 1}F}} &
RF \arrow[equals]{d} \\
R (L^n R^n F) \arrow{r}{R \eta^n} &
RF,
\end{tikzcd}
\]
using the fact that the functor $R$ is exact. The diagram itself
can be shown to be commutative by using similar arguments as in
the case for Diagram \ref{eq_Ltl_com_diag}.
\end{proof}


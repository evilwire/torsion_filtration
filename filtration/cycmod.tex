\newpage
\chapter{Filtration on $\CycMod$}\label{sect_filtration_cycmod}

In this chapter, we will extend the three filtrations on
$\HI$ to the Rost-D\'eglise category of homotopy modules $\HM$ 
(see Definition \ref{def_hm} below). To further simplify notation, 
in this chapter, let $L: \HI \to \HI$ denote the functor $F 
\mapsto \LHI{F}$, and let $R: \HI \to \HI$ denote the functor 
given by $F \mapsto \RHI{F}$. We write $\epsilon^n : \id \to 
R^nL^n$ and $\eta^n: L^nR^n \to \id$ for the unit and counit maps; 
we abbreviate $\eta^1$ as $\eta$, and $\epsilon^1$ as $\epsilon$.
The extension of these filtrations to $\HM$ is new.

\section{Torsion filtration on $\HM$}
\label{sect_torsion_filt_HM}

Recall from \cite[1.17]{DegModHom} the following definition:

\begin{defn}\label{def_hm}
A \DEF{homotopy module} is a $\Z$-graded homotopy 
invariant sheaf with transfers $F_*$ such that for every $n$, 
there exists a map $\susp_n: \LHI{F_n} \to F_{n+1}$ such that the 
corresponding adjunction map $\deloop_n: F_n \to \RHI{(F_{n+1})}$
is an isomorphism. We call $\susp_n$ and $\deloop_n$ 
the \DEF{$n$-th suspension} and the \DEF{$n$-th delooping} 
respectively. A morphism $F_* \to G_*$ between homotopy module
is a sequence of morphisms $F_n \to G_n$ of homotopy invariant
sheaves with transfers that commute with $s_n$ and $w_n$.

Let $\HM$ denote the category of $\DEF{homotopy modules}$.
Objects in $\HM$ will be represented by $(F_*, \deloop_*)$ where 
$F_*$ is the $\Z$-graded homotopy invariant sheaf with transfers, 
and $\deloop_*$ is the sequence of deloopings.
\end{defn}

There is a fully faithful functor $\spectHI: \HI \to \HM$ given
by $F \mapsto (F_*, \deloop_*)$ where
\[
F_n = \begin{cases}
F(n) &\textrm{if }n > 0 \\
F    &\textrm{if }n = 0 \\
\RHI[|n|]F &\textrm{otherwise},
\end{cases}
\]
and the $n$-th delooping $F_n \to \RHI{(F_{n + 1})}$ is the unit map
for $n \geq 0$ and the tautological natural isomorphism for $n < 0$. Furthermore,
$\spectHI$ has a right adjoint $\loopHM: \HM \to \HI$ given by
$(F_*, \deloop_*) \mapsto F_0$ (see \cite[1.18]{DegModHom}).  Since
$\spectHI$ is fully faithful and admits a right adjoint, we can regard
$\HI$ as a full coreflective subcategory of $\HM$. The torsion
filtration on $\HI$, defined in Definitions
\ref{def_upper_slice_functor} and \ref{def_TFHI},
gives rise to two $\N$-indexed weak filtrations of $\HM$. The goal 
is to extend these filtrations to a $\Z$-indexed strong filtration 
and a $\Z$-indexed cofiltration of $\HM$. In particular, we show
that there is a sequence of coradicals $\tlHM{n}$
on $\HM$ such that for nonnegative $n$, the restriction of 
$\tlHM{n}$ to $\HI$ is $\tlHI{n}$. In this case, the associated
torsion theories will extend the torsion 
filtrations on $\HI$ to $\HM$.

The following proposition will be crucial to extending the 
functors $\tlHI{n}$:

\begin{prop}\label{prop_tl_L_R}
For $F$ in $\HI$, there are natural isomorphisms:
\[
L \tlHI{n}(F) \cong \tlHI{n + 1} L(F)
\]
and
\[
R\tlHI{n}(F) \cong \tlHI{n - 1}R(F).
\] 
\end{prop}
\begin{proof}
By Lemma \ref{lem_LRcommute}, the following diagram, natural in $F$, 
is commutative:
\begin{equation}\label{eq_Ltl_com_diag_HI}
\begin{tikzcd}[column sep=80pt]
L^{n + 1}R^{n + 1}L \arrow{r}{\cuHI^{n + 1}L} \arrow{d} &
L \arrow[equals]{d} \\
L(L^{n}R^n) \arrow{r}{L\cuHI^{n}} &
L.
\end{tikzcd}
\end{equation}
Here, $\cuHI^n$ denotes the counit $L^nR^n \to \id$,
and the vertical map $L^{n + 1}R^{n + 1}L(F) \to L(L^nR^n (F))$, which is 
given by the map $L^{n + 1}R^n \eta^{-1}$, where $\eta$ is
the unit $\id \to RL$, is an isomorphism by Proposition 
\ref{prop_unit_iso}.

The cokernel of $\cuHI^{n + 1}L$ is $\tlHI{n + 1}L(F)$. Since $L$ 
is right exact, the cokernel of $L\epsilon^n$ is $L\tlHI{n}(F)$. 
By the Five Lemma, it is clear that $\tlHI{n + 1}L(F) \cong 
L\tlHI{n}(F)$. Since \eqref{eq_Ltl_com_diag_HI} is natural in $F$,
the isomorphism $\tlHI{n + 1}L \to L\tlHI{n}$ is natural as well.
By similar arguments, one can show that $R\tlHI{n}$ is naturally
isomorphic to $\tlHI{n - 1}R$ as well.
\end{proof}

We will now define the coradicals on $\HM$. 

\begin{defn}\label{def_corad_HM}
Let $(F_*, \deloop_*)$ 
be an object of $\HM$, and write $\tlHM{n}(F)$ for the graded 
homotopy invariant sheaf with transfers where
\[
(\tlHM{n}(F))_k \defeq
\begin{cases}
   \tlHI{n + k}(F_k) & \textrm{if } n + k > 0 \\
   0                 & \textrm{otherwise}.
\end{cases}
\]
For ease of notation, we will write $\tlHM[k]{n}(F)$ for the 
$k$-th graded component of $\tlHM{n}(F)$.
\end{defn}

Using the isomorphism $R\tlHI{n}(F_k) \cong \tlHI{n - 1}R(F_k)$ 
established in Proposition \ref{prop_tl_L_R}, let
\[
\deloopTL[k]{n} : \tlHM[k - 1]{n}(F) \to R \tlHM[k]{n}(F) 
\]
denote the composition
\begin{equation}\label{eq_deloop_def}
\tlHI{n + k - 1}(F_{k - 1})
   \xrightarrow{\;\tlHI{n + k - 1}(\deloop_k)\;} 
   \tlHI{n + k - 1} R(F_k) \stackrel{\cong}{\to} 
   R \tlHI{n + k}(F_k),
\end{equation}
where $\deloop_k: F_{k - 1} \to R(F_k)$ is the $k$-th delooping
of $(F_*, \deloop_*)$. Since $\deloop_k$ is an isomorphism 
for all $k$, so is $\deloopTL[k]{n}$. Defining $\suspTL[k]{n}$
to be the adjoint of $\deloopTL[k]{n}$, we immediately have that
$(\tlHM{n}(F), \deloopTL[k]{n})$ is an object of $\HM$. 

\begin{rmk*}
In the discussion above, the map $\suspTL[k]{n}: L\tlHM[k]{n}(F) \to
\tlHM[k + 1]{n}(F)$ is actually given by the composition
\[
L \tlHI{n + k}(F_{k}) \stackrel{\cong}{\to}
\tlHI{n + k + 1}L(F_{k}) \xrightarrow{\tlHI{n + k + 1}(s_k)}
\tlHI{n + k + 1}(F_{k + 1}),
\]
where $s_k: LF_k \to F_{k + 1}$ is $k$-th suspension map. This also
follows from Proposition \ref{prop_tl_L_R}.
\end{rmk*}

\begin{lem}\label{lem_tlHM_is_functor}
For each integer $n$, $\tlHM{n}$ is an endofunctor of $\HM$.
\end{lem}

\begin{proof}
Let $f_*: (F_*,\deloop_*) \to (G_*, \deloop_*')$ be a map between 
homotopy modules, and let $\tlHM{n}(f)$ be a map of graded 
homotopy invariant sheaves with transfers whose
$k$-th graded component is $\tlHM[k]{n}(f) \defeq \tlHI{n + k}(f_k)$. 
If we can show that $\tlHM{n}(f)$ is a map in $\HM$, then it will
be clear that $\tlHM{n}$ preserves identity maps and compositions.

By naturality of $\rho : R\tlHI{n + 1} \to \tlHI{n} R$ and 
$\lambda: L\tlHI{n} \to \tlHI{n + 1}L$ and also by the above 
arguments, the following two squares are commutative
\[
\begin{tikzcd}[column sep=12em, row sep=huge]
R\tlHI{n + k}(F_k) \arrow{r}{R \tlHI{n + k}(f_k)} 
   \arrow{d}{\tlHI{n + k - 1}R(\deloop) \rho} &
R\tlHI{n + k}(G_k) 
   \arrow{d}{\tlHI{n + k - 1}R(\deloop') \rho} \\
\tlHI{n + k - 1}(F_{k - 1}) 
   \arrow{r}{\tlHI{n + k - 1}(f_{k - 1})} &
\tlHI{n + k - 1}(G_{k - 1}) 
\end{tikzcd}
\]
\[
\begin{tikzcd}[column sep=12em, row sep=huge]
L\tlHI{n + k}(F_k) \arrow{r}{L \tlHI{n + k}(f_k)} 
   \arrow{d}{\tlHI{n + k + 1}L(\susp) \rho} &
L\tlHI{n + k}(G_k) 
   \arrow{d}{\tlHI{n + k + 1}L(\susp') \rho} \\
\tlHI{n + k + 1}(F_{k + 1}) 
   \arrow{r}{\tlHI{n + k + 1}(f_{k - 1})} &
\tlHI{n + k + 1}(G_{k + 1}).
\end{tikzcd}
\]

Here, $\tlHM{n}(f)$ is a map from $\tlHM{n}(F)$ to 
$\tlHM{n}(G)$ as homotopy modules. The fact that $\tlHM{n}$ 
respects composition follows from the functoriality of $\tlHI{*}$.
\end{proof}

We now verify the main result of this section:

\begin{thm}\label{thm_tlHM_corad}
For each integer $n$, $\tlHM{n}$ is a coradical of $\HM$.
\end{thm}

\begin{proof}
\pfitem{$\tlHM{n}$ is a quotient functor}: certainly $F_* \to 
\tlHM{n}(F)$ is surjective for each $n$ since it is a surjection 
at each degree. What we need to verify is that the degree-wise 
surjection gives rise to a map of homotopy modules. In particular, 
we need to verify that the following diagram is commutative
\[
\begin{tikzcd}
L(F_k) \arrow{r}{\susp} \arrow{d} &
F_{k + 1} \arrow{d} \\
L \tlHI{n + k}(F_k) \arrow{r}{\susp} &
\tlHI{n + k - 1}(F_{k + 1}).
\end{tikzcd}
\]

To see this, notice that the above diagram is the outer 
square of the diagram:
\[
\begin{tikzcd}
L(F_k) \arrow{r}{\susp} \arrow{d} &
F_{k + 1} \arrow{d} \\
\tlHI{n + k + 1} L(F_k) \arrow{r} \arrow{d}{\lambda^{-1}} &
\tlHI{n + k + 1} F_{k + 1} \arrow[equals]{d} \\
L \tlHI{n + k}(F_k) \arrow{r}{\susp} &
\tlHI{n + k + 1} F_{k + 1}.
\end{tikzcd}
\]
Here, the top square commutes by the naturality of $\id \to 
\tlHI{n + k}$, and the bottom square commutes by the definition of 
the suspension map 
$\susp: L \tlHI{n + k}(F_k) \to \tlHI{n + k + 1}(F_{k + 1})$. 

The fact that $\tlHM{n}$ respects delooping follows from 
the duality of the suspension and delooping as established by the
preceding lemma.

\pfitem{$\tlHM{n}$ is a pre-coradical}: The kernel of $F_* \to 
\tlHM{n}(F)$ is a homotopy module $K_*$ whose $k$-th graded term is 
\[
\mathrm{ker}(F_k \to \tlHI{n + k}(F_k)). 
\]
But $\tlHI{n}$ is a coradical; hence, $\tlHM{n}(K_*) = 
\tlHI{n + k}(K_k) = 0$. That is $\tlHM{n}(K) = 0$, as desired.

\pfitem{$\tlHM{n}$ is a right exact}: since $\tlHI{n + k}$ is 
right exact for each $k$, $\tlHM[k]{n}$ is right exact for each 
associated graded term. It follows that $\tlHM{n}$ is right exact.
\end{proof}

Recall from Theorem \ref{thm_precorad_eq_tt} that if $\phi$ is 
a coradical, then the torsion subcategory of $\phi$ is the full 
subcategory $\Cat{T}$ consisting of the objects $T$ such that 
$\phi(T) = 0$, and the torsionfree subcategory of $\phi$ is the 
full subcategory $\Cat{F}$ whose objects are the objects $F$ such 
that $\phi(F) = 0$. Furthermore, by Corollary 
\ref{cor_tt_ref_and_coref}, the inclusion of $\Cat{F}$ into the
ambient category admits a right adjoint given by the kernel of
the natural surjection $\id \to \phi$. 

\begin{defn}
For each integer $i$, let $\THM{n}$ and $\TFHM{n}$ denote the 
torsion and torsionfree subcategory of $\tlHM{n}$ respectively.
Let $\tgHM{n}$ denote the kernel of the natural surjection
$\id \to \tlHM{n}$. By the preceding remarks, $\tgHM{n}$ is right
adjoint to the inclusion $\THM{n}$ in $\HM$.
\end{defn}

Here is a straightforward consequence of Theorem
\ref{thm_precorad_eq_tt} and Theorem \ref{thm_tlHM_corad}:

\begin{cor}\label{cor_tgHM_prop}
An object $(F_*, \deloop_*)$ is in $\THM{n}$ if and only
if $\tgHM{n}(F) = (F_*, \deloop_*)$.
\end{cor}

We now verify the main result of this section.

\begin{cor}\label{cor_tor_filt_on_HM}
There exists a $\Z$-indexed torsion filtration on $\HM$. That 
is, there exists a $\Z$-indexed sequence of coradicals $\tlHM{i}$
such that the associated torsion subcategories, which are given
by $\Cat{F}_i = \TFHM{i}$
form an ascending strong cofiltration of $\HM$:
\[
\cdots \subseteq \TFHM{-1} \subseteq \TFHM{0} \subseteq \TFHM{1} \subseteq \cdots 
   \subseteq \TFHM{i} \subseteq \TFHM{i + 1} \subseteq \cdots 
   \subseteq \HM
\]
and the associated torsionfree subcategories $\Cat{T}_i = \THM{i}$
form a descending strong filtration of $\HM$:
\[
\cdots \subseteq \THM{i} \subseteq \THM{i - 1} \subseteq 
   \cdots \subseteq \THM{1} \subseteq \THM{0} \subseteq \THM{-1} \subseteq \cdots
   \cdots \subseteq \HM.
\]
\end{cor}
\begin{proof}
Since $(F_*, \deloop_*)$ is in $\HM$ if and only if $\tlHM{n}(F) =
(F_*, \deloop_*)$ and $\tlHM{n}$ is idempotent, the restriction
of $\tlHM{n}$ to $\TFHM{n}$ is therefore the identity. Similarly,
the restriction of $\tgHM{n}$ to $\THM{n}$ is the identity.

What remains to be checked are that $\TFHM{n} \subset \TFHM{n + 1}$
and $\THM{n + 1} \subset \THM{n}$ for each integer $n$. To 
proceed, notice that by Lemma \ref{lem_TFHI_properties}, 
$\tlHI{n + 1}\tlHI{n} = \tlHI{n}$. Therefore, for $(F_*, 
\deloop_*)$ in $\TFHM{n}$, if $\tlHM{n}(F) = (F_*, \deloop_*)$ 
then for every $k$, $\tlHI{n + k + 1}(F_k) = \tlHI{n + k + 1}\tlHI{n + k}(F_k) = 
\tlHI{n + k}(F_k) = F_k$. It follows that $\tlHM[k]{n + 1}(F) = 
F_k$ for all $k$, and $(F_*, \deloop_*)$ is in $\TFHM{n + 1}$.
Hence, $\TFHM{n + 1} \subset \TFHM{n + 1}$ for every $n$. Using 
Proposition \ref{prop_THI_properties}(4) and Corollary 
\ref{cor_tor_filt_on_HM}, we can show that $\THM{n + 1} \subset 
\THM{n}$ for every $n$ by using similar arguments.
\end{proof}

\begin{ex}\label{ex_TFHI_eq_TFHM}
Suppose $F$ is a homotopy invariant sheaf with transfers. The
image of $F$ under $\spectHI$ (see the paragraph after
Definition \ref{def_hm}) is the homotopy module $F_*$ where
\[
F_k \defeq \begin{cases}
\LHI[n]{F} & \textrm{if }k > 0\\
F &\textrm{if }k = 0\\
\RHI[|k|]{F} &\textrm{otherwise}.
\end{cases}
\]
Iterating on the result of Proposition 
\ref{prop_tl_L_R}, we see that for every positive integers $n$ and 
$k$, there are natural isomorphisms $\tlHI{n + k}(\LHI[k]{F}) 
\cong \LHI[k]{(\tlHI{n}F)}$ and $\tlHI{n}(\RHI[k]{F}) \cong 
\RHI[k]{(\tlHI{n + k}F)}$. If $F$ is in $\TFHI{n}$, then $\tlHI{n}F 
= F$ by Proposition \ref{prop_tsubcat_eq_tlHI}, and therefore
$\tlHM{n}(F_*) \cong F_*$. Similarly, if $F$ is in $\THI{n}$,
then $\tgHM{n}(F_*) \cong F_*$. It follows that the image of
$\THI{n}$ under $\spectHI$ is $\THM{n}$ and $\TFHI{n}$ under
$\spectHI$ is $\TFHM{n}$ for each positive integer $n$.
\end{ex}

We conclude this section by showing that there is a torsion 
filtration structure on the category of cycle modules (defined 
below). Recall from \cite{MilK} that for a field $F$, the Milnor 
$K$-theory of $F$ is the graded commutative ring given by
\[
\milK_*(F) \defeq T^*(F^*)/I
\]
where $T^*(F^*)$ denotes the tensor algebra of the multiplicative 
group $F^*$, and $I$ denotes the ideal generated by $a \tensor 
(1 - a)$ for all $a$ in $F^*$. We define $\milK_n(F)$ to be $0$
for $n < 0$ and let $\milK_n(F)$ be the $n$-th graded piece of 
$\milK_*(F)$. We call $\milK_n(F)$ the \DEF{$n$-th Milnor 
$K$-theory of $F$}.

\begin{defn}[\cite{Rost96} 1.1]\label{def_pre_cycmod}
Let $X$ be a finite-type $\basefield$-scheme, and let 
$\fields(\basefield)$ be the category of function fields $E$ of 
$\Sm$, i.e. $E$ is the function field of some $\basefield$-scheme 
$X$ in $\Sm$, and any morphism $E \to E'$ in $\fields(\basefield)$
is a field homomorphism such that the restriction to $\basefield$ 
is the identity. A \DEF{cycle premodule} $M$ is a functor which 
assigns to every field $E$ in $\fields(\basefield)$ a $\Z$-graded abelian 
group $\displaystyle M(E) = \{M_i\}_{i \in \Z}$, together with
the following data:

\begin{enumerate}[label=\bfseries D\arabic*.]
\item[\textbf{D1.}] For each field extension $\phi: E' \to E$, 
there is a degree 0 map $\phi_*: M(E') \to M(E)$ called the 
\emph{restriction map associated to $\phi$}

\item[\textbf{D2.}] For each finite extension $\phi: E' \to E$, 
there is a degree 0 map $\phi^*: M(E) \to M(E')$ called the 
\emph{corestriction map associated to $\phi$}

\item[\textbf{D3.}] For each $E$ in $\fields(\basefield)$, the group $M(E)$ 
is equipped with the structure of a left $\milK_*(E)$-module, where
$\milK_*(E)$ is the Milnor $K$-ring of $E$.

\item[\textbf{D4.}] For a given valuation $v$ of $E$ in 
$\fields(\basefield)$, there exists a map of degree -1 $\res{v}: M(E) \to 
M(\resf{v})$ called the \emph{residue} map, where $\resf{v}$ is 
the residue field of $v$.
\end{enumerate}

The data given in D1 - D4 satisfy the following criteria.
For a given valuation $v$ of $E$ in $\fields(\basefield)$,
fix $p$ be a prime of $v$. The $\milK_*(E)$-module structure
in D3 and the residue map in D4 gives
rise to a degree preserving map $\Special{v}{p} : M(E) \to
M(\resf{v})$ defined by $\Special{v}{p}(\rho) =
\res{v}(\{p\} \cdot \rho)$, where $\{p\}$ the element in 
$\milK_1(E)$ represented by $E$. Following \cite[1.1]{Rost96},
we call $\Special{v}{p}$ the \DEF{specialization} map.

\begin{enumerate}[label=\bfseries R1\alph*., leftmargin=3em]
\item For each field extension $\phi: E' \to E$ and 
field extension $\psi: E \to E''$, $(\psi \comp \phi)_* = \psi_* 
\comp \phi_*$

\item For each finite extension $\phi: E' \to E$ and 
finite extension $\psi: E \to E''$, $(\psi \comp \phi)^* = \phi^* 
\comp \psi^*$

\item For finite extension $\phi: E' \to E$ and any 
field extension $\psi: E' \to E''$ with $\phi$ finite, define $R = E 
\otimes_{E'} E''$, and let $\ideal{p}$ be any prime ideal of $R$. (As 
$R$ is Artin, let $l_p$ be the length of the local ring 
$R_{(\ideal{p})}$), and $\phi_{\ideal{p}}: E'' \to R/{\ideal{p}}$ 
and $\psi_{\ideal{p}}: E \to R/{\ideal{p}}$ be natural maps.
\[
\psi_*\comp \phi^* = \sum_{\ideal{p}} l_p \cdot 
(\phi_{\ideal{p}})^* \comp (\psi_{\ideal{p}})_*.
\]
\end{enumerate}
\begin{enumerate}[label=\bfseries R2\alph*., leftmargin=3em]
\item[\textbf{R2.}] For any extension $\phi: E' \to E$, $x \in 
\milK_*(E')$, $y \in \milK_*(E)$, $\rho \in M(E')$, and $\mu \in 
M(E)$, then:

\item $\phi_*(x \cdot \rho) = \phi_*(x) \cdot 
\phi_*(\rho)$.

\item if $\phi$ is finite, $\phi^*(\phi_*(x) \cdot 
\mu) = x \cdot \phi^*(\mu)$.

\item if $\phi$ is finite, $\phi^*(y \cdot 
\phi_*(\rho)) = \phi^*(y) \cdot \rho$.
\end{enumerate}
\begin{enumerate}[label=\bfseries R3\alph*., leftmargin=3em]
\item[\textbf{R3.}] For any field extension $\phi: E' \to E$, $v$ 
a valuation on $E$ and $w$ and a valuation on $E'$:

\item Suppose $w$ is a nontrivial restriction of 
$v$ with ramification index $e$. Let $\phib: \resf{w} \to 
\resf{v}$ be the induced map. Then:
\[
\res{v} \comp \phi_* = e \cdot \phib_* \comp \res{w}.
\]

\item Let $\phi$ be a finite extension, suppose
$w$ is an extension of $v$ to $E$. Let $\phi_v: \resf{w} \to 
\resf{v}$ be the induced map on the residue fields. Then
\[
\res{v} \comp \phi^* \sum_{v} \comp \res{v}.
\]

\item Suppose $v$ restricts to a trivial valuation 
on $E'$. Then
\[
\res{v} \comp \phi_* = 0
\]

\item Suppose $v$ restricts to a trivial valuation 
on $E'$, and let $\phib: F \to \resf{v}$ be the induced map on 
the residue fields. Let $\ideal{p}$ a prime of $v$. Then
\[
\Special{v}{\ideal{p}} \comp \phi_* = \phib_*
\]

\item Let $u$ be an element of $E$ such that 
$v(u) = 0$. Given $\rho$ in $M(F)$, one has
\[
\res{v}(\{u\} \cdot \rho) = -\{\overline{u}\} \cdot \res{v}(\rho).
\]
\end{enumerate}
\end{defn}

For $X$ a $k$-scheme, let $\subsch{1}{X}$ denote the collection of 
codimension 1 subschemes. Let $\xi_X$ be the generic point of an
irreducible $X$ with $K_X = \O_X,\xi_X$. If $X$ is normal, then 
for $x$ in $\subsch{1}{X}$, the local ring $\O_{x,X}$ is a valuation 
ring of $K_X$ with residue field $\resf{x}$. Write $M(x)$ for 
$M(\resf{x})$, and $\res{x}: M(\xi_X) \to M(x)$ for the 
restriction map.

Furthermore, for $x, y \in X$, let $Z$ be the closed subscheme 
determined by $x$, and $\overline{Z}$ be the normalization $Z$.
Define
\[
\ptres{x}{y}: M(x) \to M(y)
\]
by
\[
\ptres{x}{y} = 
\begin{cases}
0 & y \notin \subsch{1}{Z} \\
\sum_{z|y} \phi_{\resf{z},\resf{x}}^* \comp \res{z} & \textrm{otherwise}.
\end{cases}
\]
Here, following \cite{Rost96}, $z|y$ denotes the relation that $z$ 
lies over $y$. In particular, if $y \in \subsch{1}{Z}$, the sum 
is taken over all $z$ lying over $y \in \subsch{1}{Z}$. In this
case, $\phi_{\resf{z},\resf{y}}^*$ is the corestriction map 
associated to the finite field extension $\resf{y} \to \resf{z}$.

\begin{defn}[\cite{Rost96} 2.1]
A cycle module $M$ on $\fields(\basefield)$ is a cycle premodule that
satisfies the following conditions:

\begin{enumerate}[leftmargin=3em]
\item[\textbf{(FD)}] \itemhead{Finite support of divisors.} 
$X$ be a normal scheme and $\rho \in M(\xi_X)$. Then $\res{x}: 
M(\xi_X) \to M(X)$ is 0 for all but finitely many $x \in 
\subsch{1}{X}$.

\item[\textbf{(C)}] \itemhead{Closedness.} If $X$ is an integral
local scheme of dimension 2 with closed point $x_0$, then the map 
from $M(\xi_X)$ to $M(x_0)$ given by
\[
\sum_{x \in \subsch{1}{X}} \ptres{x_0}{x} \comp \ptres{x}{\xi}
\]
is 0.
\end{enumerate}
\end{defn}

D\'eglise showed in \cite{DegModHom} that a homotopy module $(F_*, 
\deloop_*)$ gives rise to a unique cycle module $\assocCM{F_*}$,
and that this association defines an equivalence between the 
category of homotopy modules and cycle modules (see 
\cite[3.7]{DegModHom}). Via this categorical equivalence, we 
obtain the following corollary: 

\begin{cor}\label{cor_tor_filt_on_CycMod}
There exists a $\Z$-indexed torsion filtration on $\CycMod$. That
is, there exists a $\Z$-indexed sequence of coradicals, which by abuse
of notation, we also represent by $\tlHM{i}$ such that the 
associated torsion subcategories $\TFCycMod{i}$ form an ascending 
strong cofiltration of $\CycMod$:
\[
\cdots \subseteq \TFCycMod{-1} \subseteq \TFCycMod{0} \subseteq \cdots 
   \subseteq \TFCycMod{i} \subseteq \cdots 
   \subseteq \CycMod
\]
and the associated torsionfree subcategories $\TCycMod{i}$
form a descending strong filtration of $\CycMod$:
\[
\cdots \subseteq \TCycMod{i} \subseteq 
   \cdots \subseteq \TCycMod{0} \subseteq \TCycMod{-1} \subseteq \cdots
   \cdots \subseteq \CycMod.
\]
\end{cor}

\begin{ex}\label{ex_milK}
Milnor $K$-theory $\milK_*$, defined in the paragraph preceding 
Definition \ref{def_pre_cycmod}, is an example of a cycle module 
(see \cite[1.4, 2.5]{Rost96}). By \cite[3.7]{DegModHom}, the 
homotopy module corresponding to $\milK_*$ is $\spectHI(\Ox)$. As 
we have shown in Example \ref{ex_TFHI_eq_TFHM}, $\spectHI(\Ox)$ is 
an object of $\THM{2} \cap \TFHM{1}$. Hence, $\milK_*
\in \TCycMod{2} \cap \TFCycMod{1}$.
\end{ex}

%We now discuss the tensor properties of the filtration. Let us
first consider the following notion:

\begin{defn}\label{def_graded_tensor}
Let $(\Cat{C}, \tensor, \Unit)$ be a monoidal category. We say 
that $\Cat{C}$ is a weakly filtered monoidal category if there 
exists a weak filtration $(\Cat{C}_*, \phi_*)$ such that for all
integers $m$ and $n$, $\Cat{C}_m \tensor \Cat{C}_n \subseteq
\Cat{C}_{m + n}$.
\end{defn}

\begin{ex}
Here are two examples of weakly filtered monoidal categories
that we have encountered in this thesis. Recall from Definition 
\ref{def_GFiltDM} that $\GFiltDM[k]{\DM}$ is the full subcategory
of the objects $(M, n)$ in $\DM$ such that $n \geq k$. Using the
description of the tensor product on $\DM$ given in Definition
\ref{def_tensor_DM}, we see that for $(M, n)$ in $\GFiltDM[k]{\DM}$
and $(M', n')$ in $\GFiltDM[l]{\DM}$, $(M, n) \tDM (M', n') =
(M \tensor M', n + n')$ is an object in $\GFiltDM[k + l]{\DM}$.

Similarly, $(\HI(*), \sgHI{*})$ defines a graded symmetric 
monoidal category on $\HI$ under $\tHI$. To see this, recall
from the first paragraph in Section \ref{sect_torsion_filt_on_HI}
that $F$ is in $\LHI[n]{\HI}$ if $F \cong \LHI[n]{F'}$. 
Furthermore, since $\LHI[n]{F} = F \tHI (\Ox)^{\tensor n}$, 
$\LHI[n]{F} \tHI \LHI[m]{G} = \LHI[n + m](F \tHI G).$ Therefore, 
$\LHI[n]{\HI} \tHI \LHI[m]{\HI} \subseteq \LHI[n + m]{\HI}$.
\end{ex}

We first show that $(\THI{*}, \tgHI{*})$ defines a weakly filtered 
monoidal category on $\HI$. We begin by proving the following
proposition:

\begin{prop}\label{prop_tensor_and_tfilt_HI}
For $F$ in $\THI{n}$ and $G$ in $\THI{m}$, $F \tHI G$
is an object of $\THI{n + m}$.
\end{prop}
\begin{proof}
Since $\THI{n + m}$ is the torsion subcategory associated to the
coradical $\tlHI{n + m}$, to show that $F \tHI G$ is in 
$\THI{n + m}$, it suffices to show that $\tlHI{n + m}(F \tHI G) = 
0$. Since $G$ is in $\THI{n}$, by Proposition 
\ref{prop_THI_properties}(1), the counit $\epsilon: L^mR^m(G) \to 
G$ is surjective. By \cite[5.2]{DegModHom}, $F \tDM -$ is right 
$t$-exact on $\DMeff$. Therefore, by Proposition 
\ref{prop_t_exact_implies_exact}, the functor $F \tHI -$ is right 
exact, and the following map is surjective:
\begin{equation}\label{eq_tensor_tfilt_HI_1}
F \tHI L^mR^m(G) \xrightarrow{\epsilon_F \tHI G} F \tHI L^mR^m(G).
\end{equation}
Replacing $F$ by $L^nR^n(F)$, we see that the following map is
also surjective:
\begin{equation}\label{eq_tensor_tfilt_HI_2}
L^nR^n(F) \tHI L^mR^m(G) \xrightarrow{L^nR^n(F) \tHI \epsilon_G}
L^nR^n(F) \tHI G.
\end{equation}
Composing \eqref{eq_tensor_tfilt_HI_1} and 
\eqref{eq_tensor_tfilt_HI_2}, we obtain a surjection
\[
f: L^nR^n(F) \tHI L^mR^m(G) \to F \tHI G.
\]
On the other hand, since $L^nR^n(F) \tHI L^mR^m(G) = 
L^{n + m}(R^n(F) \tHI R^m(G))$, the object $L^nR^n(F) \tHI L^mR^m(G)$
is in $\LHI[n + m]{\HI}$, and by Proposition
\ref{prop_THI_properties}, 
\[
\tlHI{n + m}(L^nR^n(F) \tHI L^mR^m(G)) = 0. 
\]
Since $\tlHI{n + m}$ is a coradical, which is right exact, the map
\[
\tlHI{n + m}(f) : \tlHI{n + m}(L^nR^n(F) \tHI L^mR^m(G)) \to
   \tlHI{n + m}(F \tHI G)
\]
is onto. Therefore, $\tlHI{n + m}(F \tHI G) = 0$ and 
$F \tHI G$ is an object in $\THI{n + m}$, as desired.
\end{proof}

The following is a direct consequence of the proposition above:

\begin{cor}\label{cor_graded_tensor_HI}
There exists a graded symmetric monoidal structure on $\HI$ by
$(\THI{*}, \tgHI{*})$.
\end{cor}

We now show that the weakly filtered symmetric monoidal structure 
on $\HI$ can be extended to a weakly filtered symmetric monoidal 
structure on $\HM$. For the following, by abuse of notation, let 
$F_*$ denote the object $(F_*, \deloop_*)$ in $\HM$. Recall from
\cite[1.16]{DegModHom} the following definition of the symmetric 
tensor product on $\HM$:

\begin{defn}
Let $F_*$ and $G_*$ be two $\Z$-graded homotopy invariant sheaves
with transfers. We define $F_* \tHM G_*$ to be the $\Z$-graded 
homotopy invariant sheaf with transfers given by
\[
(F_* \tHM G_*)_n \defeq \bigoplus_{p + q = n} F_p \tHI G_q,
\]
where $F_p$ and $G_q$ are the $p$-th and $q$-th graded component
of $F_*$ and $G_*$ respectively. This construction defines a
symmetric tensor product $\tHM$ on the category of $\Z$-graded
homotopy invariant sheaves with transfers.

By \cite[1.19]{DegModHom}, $\tHM$ induces a symmetric tensor 
product on $\HM$, which is uniquely determined by
\end{defn}

\begin{prop}\label{prop_graded_mon_struct_HM}
Let $n$ and $m$ be arbitrary integers, and fix $F_*$ in $\THM{n}$ 
and $G_*$ in $\THM{m}$. Then, $F_* \tHI G_*$ is an object of 
$\THM{n + m}$.
\end{prop}
\begin{proof}
By Corollary \ref{cor_tgHM_prop}, it suffices to show that $\tgHM{n + m}(F_*
\tHI G_*) = F_* \tHI G_*$.

Regard the objects of $\HM$ as pairs $(F, k)$, where $F \in \HI$
and $k$ is an arbitrary integer. It is straightforward to 
verify from the definition of $\tgHM{*}$ and Proposition \ref{prop_tl_L_R} 
that
\[
\tgHM{n}((F, k)) = \begin{cases}
(\tgHI{n - k}F, k) &\textrm{if }n > k\\
(F, k)             &\textrm{otherwise}.
\end{cases}
\]
Therefore, we are reduced to showing that for $F$ and $G$ in
$\THI{n}$ and $\THI{n}$ respectively, 
$\tgHI{n + m}(F \tHI G) = F \tHI G$. This follows directly from 
Corollary \ref{cor_graded_tensor_HI}.
\end{proof}

Finally, we obtain the following corollary as a direct consequence
of the proposition above:

\begin{cor}\label{cor_graded_mon_struct_HM}
There exists a weakly filtered symmetric monoidal structure on $\HM$ by
$(\THM{*}, \tgHM{*})$.
\end{cor}



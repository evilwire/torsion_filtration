\newpage
\chapter{Filtration on $\CycMod$}\label{sect_filtration_cycmod}

In this section, just as we have extend the slice filtration from
$\DMeff$ to $\DM$, we will extend the three filtrations on
$\HI$ to the category of homotopy modules $\HM$ (defined below).
To further simplify notation, let $L: \HI \to \HI$ denote the 
functor $F \mapsto \LHI{F}$, and let $R: \HI \to \HI$ denote the 
functor given by $F \mapsto \RHI{F}$. We write $\epsilon^n : \id 
\to R^nL^n$ and $\eta^n: L^nR^n \to \id$ for the unit and counit 
maps; we abbreviate $\eta^1$ as $\eta$, and $\epsilon^1$ as 
$\epsilon$.

Recall from \cite[1.17]{DegModHom} the following definition:

\begin{defn}\label{def_hm}
A \DEF{homotopy module} is a $\Z$-graded homotopy 
invariant sheaf with transfers $F_*$ such that for every $n$, 
there exists a map $\susp_n: \LHI{F_n} \to F_{n+1}$ such that the 
corresponding adjunction map $\deloop_n: F_n \to \RHI{(F_{n+1})}$
is an isomorphism.

We call the morphisms $\susp_n$ and $\deloop_n$ the \DEF{$n$-th 
suspensions} and the \DEF{$n$-th delooping} respectively.
\end{defn}

The relationship between $\HM$ and $\HI$ is analogous to the 
relationship between $\DM$ and $\DMeff$. For example, it is easy 
to see that every object in $\HM$ can be represented by $(F, n)$, 
where $F$ is an object of $\HI$ and $n$ is an integer. 
Furthermore, there exists isomorphisms between $(F, n + 1)$ and
$(\LHI{F}, n)$. 

There a fully faithful functor $\spectHI : \HI \to \HM$, given by 
$F \mapsto (F, 0)$, which realizes $\HI$ as a full subcategory of 
objects in $\HM$ of the form $(F, n)$ where $n \geq 0$. Just as in
the case for $\DM$ and $\DMeff$, $\spectHI$ admits a right adjoint
$\loopHM : \HM \to \HI$ given by 
\[
\loopHM(F, n) = \begin{cases}
\LHI[n]F &\textrm{for }n \geq 0\\
0 &\textrm{otherwise}.
\end{cases}
\]

The relationships between $\HM$ and the categories $\HI$, $\DMeff$ 
and $\DM$ can be best summarized by the following results taken
from \cite{DegModHom}.

\begin{prop}
The category $\DM$ is equipped with a $t$-structure, where the 
heart is categorical equivalent to $\HM$. In particular, $\HM$ is 
an abelian category equipped a symmetric monoidal structure and a 
partial internal hom. Furthermore, $\HM$ fits into the following 
commutative diagram of symmetric monoidal categories:
\begin{equation}\label{eq_hm_hi_com_diag}
\begin{tikzcd}[column sep=huge, row sep=huge]
\DMeff \arrow[hookrightarrow, bend right]{r}{\spectDM} \arrow{d}{\HH^0} &
\DM \arrow{d}{\HH^0} \arrow[bend right]{l}{\loopDM}\\
\HI \arrow[hookrightarrow, bend right]{r}{\spectHI} &
\HM \arrow[bend right]{l}{\loopHM}
\end{tikzcd}
\end{equation}
where $\spectDM$ and $\loopDM$ are the inclusion of $\DMeff$ into
$\DM$ and its right adjoint, and $\spectHI$ and $\loopHM$ are the 
functors defined above.
\end{prop}
\begin{proof}
Let $\DM^{\geq 0}$ (resp. $\DM^{\leq 0}$) to be full 
subcategory of $\DM$ where the objects are of the form $(M, n)$
such that $M \in {\DMeff}^{\geq 0}$ (resp ${\DMeff}^{\leq 0}$).
Then $(\DM^{\geq 0}, \DM^{\leq 0})$ define a $t$-structure on
$\DM$. The heart of this $t$-structure is $\HM$. Therefore $\HM$
is an abelian category by Thm. \ref{thm_heart_is_abel_cat}.

Since the tensor product on $\DM$ is right $t$-exact 
(\cite[5.8]{DegModHom}), $\HM$ is endowed with an additive 
symmetric monoidal structure. Similarly, $\HM$ inherits the
partial internal hom structure from $\DM$.

The commutativity of \eqref{eq_hm_hi_com_diag} follows easily
from the discussion above. (See \cite[5.7]{DegModHom}.)
\end{proof}

By abuse of notation, let $\tHI$ denote the additive tensor 
product on $\HM$ as well. We will discuss the relationship between
the tensor monoidal structure on $\HM$ and the filtrations that 
we are about to introduce.

For the remainder of this section, we will \emph{not} represent 
objects of $\HM$ as a pair $(F, n)$ where $F$ is in $\HI$ and $n$ 
is an integer. Instead, objects in $\HM$ are represented by $(F_*, 
\deloop_*)$ where $F_*$ is the $\Z$-graded homotopy invariant 
sheaf with transfers, and $\deloop_*$ is the sequence of 
deloopings.

Since $\HI$ can be recognized as a subcategory of $\HM$, the
three filtrations defined in the previous sections also define
a $\N$-indexed filtration of $\HM$. The goal is to extend these
filtrations to $\Z$-filtrations of $\HM$. In particular, we show
that there is a $\Z$-indexed sequence of coradicals $\tlHM{n}$
such that for nonnegative $n$, the restriction of $\tlHM{n}$ to
$\HI$ is $\tlHI{n}$. If this is possible, then the associated
torsion theories will define the extension of the torsion 
filtrations on $\HI$.

The following proposition will be crucial to extending the 
functors $\tlHI{n}$:

\begin{prop}\label{prop_tl_L_R}
For $F$ in $\HI$, 
\[
L \tlHI{n}(F) \cong \tlHI{n + 1} L(F)
\]
and
\[
R\tlHI{n}(F) \cong \tlHI{n - 1}R(F).
\] 
Both isomorphisms are natural in $F$.
\end{prop}
\begin{proof}
The key is to prove that the diagram
\begin{equation}\label{eq_Ltl_com_diag}
\begin{tikzcd}
L^{n + 1}R^{n + 1}L(F) \arrow{r}{\eta^{n + 1}} \arrow{d} &
L(F) \arrow[equals]{d} \\
L(L^{n}R^n(F)) \arrow{r}{L\eta^{n}} &
L(F).
\end{tikzcd}
\end{equation}
is commutative. Here, $\eta^n$ denotes the counit $L^nR^n \to \id$,
and the vertical map $L^{n + 1}R^{n + 1}L(F) \to L(L^nR^n (F))$ is 
given by the map $L^{n + 1}R^n \epsilon^{-1}$, where $\epsilon$ is
the unit $\id \to RL$, which is an isomorphism.

To establish that the above diagram is commutative, we proceed by 
induction on $n$. The case $n = 0$ follows by the counit-unit 
adjunction. That is, the composition
\[
L(F) \stackrel{L(\epsilon)}{\to} LRL(F) \stackrel{\eta L}{\to} 
   L(F).
\]
is the identity. Therefore, $\eta L = L(\epsilon^{-1})$, and the
following diagram commutes:
\[
\begin{tikzcd}
LRL(F) \arrow{r}{\eta L} \arrow{d}{L(\epsilon^{-1})} &
L(F) \arrow[equals]{d} \\
L(F) \arrow[equals]{r} & L(F).
\end{tikzcd}
\]

Now, suppose we have
\begin{equation}\label{eq_induct_hyp_tl_diag}
\begin{tikzcd}
L^{n}R^{n}L(F) \arrow{r}{\eta^{n}} \arrow{d} &
L(F) \arrow[equals]{d} \\
L(L^{n - 1}R^{n - 1}(F)) \arrow{r}{L\eta^{n - 1}} &
L(F).
\end{tikzcd}
\end{equation}
There exists a natural transformation $\eta': L^n R^n \to 
L^{n - 1} R^{n - 1}$. Applying this to $\epsilon^{-1}: RL(F) \to 
F$, we have the following commutative diagram
\[
\begin{tikzcd}
L^n R^n RL(F) \arrow{r}{\eta'} 
   \arrow{d}{L^nR^n \epsilon^{-1}} &
L^{n - 1}R^{n - 1} RL(F) 
   \arrow{d}{L^{n - 1}R^{n - 1}\epsilon^{-1}} \\
L^nR^n(F) \arrow{r}{\eta'} &
L^{n - 1}R^{n - 1} (F),
\end{tikzcd}
\]
and applying $L$, we have
\[
\begin{tikzcd}
L^{n + 1} R^{n + 1} L(F) \arrow{r}{L\eta'} 
   \arrow{d}{L^{n + 1}R^n \epsilon^{-1}} &
L^{n}R^n L(F) \arrow{d}{L^nR^{n - 1}\epsilon^{-1}} \\
L^{n + 1}R^n(F) \arrow{r}{L\eta'} &
L^{n}R^{n - 1}(F),
\end{tikzcd}
\]
which fits together with \eqref{eq_induct_hyp_tl_diag} to give the 
following commutative diagram:
\[
\begin{tikzcd}
L^{n + 1} R^{n + 1} L(F) \arrow{r}{L\eta'} 
   \arrow{d}{L^{n + 1}R^n(\epsilon^{-1})}
   \arrow[bend left]{rr}{\eta^{n + 1}} &
L^{n}R^n L(F) \arrow{d}{L^nR^{n - 1}(\epsilon^{-1})}
   \arrow{r}{\eta^{n}} &
L(F) \arrow[equals]{d} \\
L^{n + 1}R^n(F) \arrow{r}{L\eta'} 
   \arrow[bend right]{rr}{L\eta^n} &
L^{n}R^{n - 1}(F) \arrow{r}{L\eta^{n - 1}} &
L(F).
\end{tikzcd}
\]
Notice that $\eta^n L\eta' = \eta^{n + 1}L$ (the composition of
the top horizontal arrows is precisely the top bent arrow), and
$L\eta^{n - 1}L\eta' = L\eta^n$ (the composition of the bottom
horizontal arrows is precisely the bottom bent arrow). The claim
follows.

Returning to \eqref{eq_Ltl_com_diag}, which is now shown to 
be commutative, notice that the cokernel of the top row is 
$\tlHI{n + 1}L(F)$. Since $L$ is right exact, the cokernel of 
the bottom row is $L\tlHI{n}(F)$. By the Five Lemma, it is clear 
that $\tlHI{n + 1}L(F) \cong L\tlHI{n}(F)$.

For the other isomorphism, apply similar arguments to the
diagram
\[
\begin{tikzcd}
L^{n - 1}R^{n - 1}R(F) \arrow{r}{\eta^n} 
   \arrow{d}{\eta} &
R(F) \arrow[equals]{d} \\
R (L^n R^n(F)) \arrow{r}{R \eta^n} &
R(F),
\end{tikzcd}
\]
using the fact that the functor $R$ is exact.
\end{proof}

We can now define the cofiltration endofunctor on $\HM$. Let
$(F_*, \deloop_*)$ be an object of $\HM$, and write $\tlHM{n}(F_*)$ 
for the graded homotopy invariant sheaf with transfers where
\[
(\tlHM{n}(F_*))_k \defeq
\begin{cases}
   \tlHI{n + k}(F_k) & \textrm{if } n + k > 0 \\
   0                 & \textrm{otherwise}.
\end{cases}
\]
For ease of notation, we write $\tlHM{n}(F)$ for the essential 
image of $F_*$ (dropping the ``$_*$'' on $F$ to avoid 
redundancy) and we write $\tlHM[k]{n}(F)$ for the $k$-th graded 
component of $\tlHM{n}(F)$.

By the previous proposition, there are two maps that we can
define between the graded pieces of $\tlHM{n}(F)$. Recognizing
$L\tlHM[k]{n}(F) \defeq L \tlHI{n + k}(F_k)$ and 
$\tlHM[k + 1]{n}(F) \defeq \tlHI{n + k + 1}(F_{k + 1})$, let
\[
\suspTL[k]{n} : L\tlHM[k]{n}(F) \to \tlHM[k + 1]{n}(F) 
\]
denote the composition
\begin{equation}\label{eq_susp_def}
L \tlHI{n + k}(F_k) \stackrel{\cong}{\to} 
   \tlHI{n + k + 1}L(F_k) 
   \xrightarrow{\;\tlHI{n + k + 1}(\susp_k)\;} 
   \tlHI{n + k + 1}(F_{k + 1})
\end{equation}
where $\susp_k: L(F_k) \to F_{k + 1}$ is the $k$-th suspension 
map of $(F_*, \deloop_*)$. Similarly, since we have that
$\tlHM[k - 1]{n}(F) \defeq \tlHI{n + k - 1}(F_{k - 1})$ and
$R \tlHM[k]{n}(F) \defeq R \tlHI{n + k}(F_{k})$, let
\[
\deloopTL[k]{n} : \tlHM[k - 1]{n}(F) \to R \tlHM[k]{n}(F) 
\]
be a map defined by
\begin{equation}\label{eq_deloop_def}
\tlHI{n + k - 1}(F_{k - 1})
   \xrightarrow{\;\tlHI{n + k - 1}(\deloop_k)\;} 
   \tlHI{n + k - 1} R(F_k) \stackrel{\cong}{\to} 
   R \tlHI{n + k}(F_k),
\end{equation}
where $\deloop_k: F_{k - 1} \to R(F_k)$ is the $k$-th delooping
of $(F_*, \deloop_*)$. Notice that, in this case, $\deloop_k$ is
an isomorphism for all $k$. Therefore, $\deloopTL[k]{n}$ is an 
isomorphism for all $k$. We claim that with the maps defined above,
$\tlHM{n}(F)$ is an object of $\HM$ (proved in Lem. 
\ref{lem_tlHM_is_functor} below). In fact, we have that:

\begin{thm}\label{thm_tlHM_corad}
For each integer $n$, $\tlHM{n}$ is a coradical of $\HM$.
\end{thm}

As a first step, we first establish that:

\begin{lem}\label{lem_tlHM_is_functor}
$\tlHM{n}$ is an endofunctor of $\HM$.
\end{lem}

\begin{proof}
Fix $(F_*, \deloop_*)$, we show that $\tlHM{n}(F)$ is an object of
$\HM$. We have already established the graded structure of 
$\tlHM{n}(F)$. We also have candidates for the suspension and 
delooping maps; these are $\suspTL{n}$ and $\deloopTL{n}$ 
respectively. It remains to show that $\suspTL[k]{n}$ and 
$\deloopTL[k + 1]{n}$ are adjoint in the sense that $\suspTL{n}$ is 
mapped to $\deloopTL{n}$ via the adjunction 
\[
\Phi: \homHM(L \tlHM[k]{n}(F), \tlHM[k + 1]{n}(F))
\to \homHM(\tlHM[k]{n}(F), R \tlHM[k + 1]{n}(F))
\]
given that $\susp_{k + 1}$ is adjoint to $\deloop_k$.

Let $\shift[k]{F_*}$ denote a shift in grading as given 
by $\shift[k]{F_*}_n = F_{n + k}$. Notice that 
$\tlHM{n}(\shift[k]{F}) = \shift[k]{\tlHM{n - k}(F)}$. Therefore, 
we may assume without loss of generality that $k = 0$. 

To simplify notations, let
\[
\lambda : L\tlHI{n} \to \tlHI{n + 1}L
\]
and 
\[
\rho : R\tlHI{n} \to \tlHI{n - 1}R
\]
be the natural isomorphisms given in Prop. \ref{prop_tl_L_R}. 
Let $L$ and $R$ be given as above,  we abuse notation and write $\Phi$ for the 
adjunction isomorphism
\[
\Phi: \homHI(L-, -) \to \homHI(-,R-)
\]
between any pair of objects in $\HI$.

Notice that for $f: L(F) \to G$ in $\HI$, $\Phi(f) = \eta R(f)$.
Since $\suspTL[0]{n}$ is given by
\[
L\tlHI{n}(F_0) \stackrel{\lambda}{\to} \tlHI{n + 1} L(F_0)
   \xrightarrow{\;\tlHI{n + 1}(\susp_0)\;} \tlHI{n + 1}(F_1),
\]
to show that $\suspTL[0]{n}$ is adjoint to $\deloopTL[0]{n}$ is to
verify that the following diagram commute:
\[
\begin{tikzcd}
\tlHI{n}(F_0) \arrow{rr}{\epsilon} \arrow[equals]{dd} &&
RL (\tlHI{n}F_0) \arrow{rr}{R(\tlHI{n + 1}(\susp) \lambda)} 
\arrow[dotted]{dd}{\rho R(\lambda)}&&
R \tlHI{n + 1} (F_1) \arrow[equals]{dd} \\
& (1) && (2) \\
\tlHI{n}(F_0) \arrow{rr}{\tlHI{n}(\epsilon)} &&
\tlHI{n} RL(X) \arrow{rr}{\rho \tlHI{n} R(\susp)} &&
R\tlHI{n}(F_1)
\end{tikzcd}
\]
where composition along the top row is precisely 
$\Phi(\suspTL[0]{n})$, and the bottom is precisely 
$\deloopTL[1]{n}$. The diagram breaks up into two squares,
labelled (1) and (2), along the map $\lambda \rho: RL 
\tlHI{n}(F_0) \to \tlHI{n} RL(F_0)$. We proceed by showing that 
each square is commutative.

\pfitem{Square (1) is commutative} : Applying the naturality of
$\eta^n$ to the map $\epsilon: F_0 \to RL(F_0)$, we obtain the 
following commutative diagram:
\begin{equation}\label{eq_LR_rear_face}
\begin{tikzcd}[column sep=large]
L^nR^n(F_0) \arrow{r}{L^nR^n\epsilon} \arrow{d}{\eta^n} &
L^nR^nRL(F_0) \arrow{d}{\eta^n(\epsilon)} \\
F_0 \arrow{r}{\epsilon} &
RL(F_0).
\end{tikzcd}
\end{equation}
Similarly, applying the naturality of $\epsilon$ to $\eta^n$, we
have
\begin{equation}\label{eq_LR_front_face}
\begin{tikzcd}[column sep=large]
L^nR^n(F_0) \arrow{d}{\epsilon} \arrow{r}{\eta^n} &
F_0 \arrow{d}{\epsilon} \\
RLL^nR^n(F_0) \arrow{r}{RL(\eta^n)} &
RL(F_0)
\end{tikzcd}
\end{equation}
which fits together in a ``cube'':
\[
\begin{tikzcd}[row sep=scriptsize, column sep=scriptsize]
& L^nR^n(F_0) \arrow[equals]{dl}\arrow{rr}\arrow{dd}{\eta} & & 
   L^nR^nRL(F_0) \arrow{dd} \\
L^nR^n(F_0) \arrow[crossing over]{rr}\arrow{dd}{\eta} & & RLL^nR^n(F_0) 
   \arrow[dotted]{ur} \\
& F_0 \arrow[equals]{dl}\arrow{rr} & & RL(F_0) 
   \arrow[equals]{dl} \\
F_0 \arrow{rr} & & RL(F_0) \arrow[crossing over, leftarrow]{uu}
\end{tikzcd}
\]
where the faces parallel to the page are precisely (from front to
rear) \eqref{eq_LR_front_face} and \ref{eq_LR_rear_face}
respectively. Since unit maps are an isomorphisms, if we let the 
dotted arrow be given by the composition
\[
RLL^nR^n(F_0) \stackrel{\epsilon^{-1}}{\to} L^nR^n(F_0) 
\xrightarrow{L^nR^n(\epsilon)} L^nR^nRL(F_0)
\]
then all faces of the cube are commutative.

At last, notice that the Square (1) form the ``cokernel'' of the
cube and is the bottom face of the lower cube in the following:
\[
\begin{tikzcd}[row sep=scriptsize, column sep=scriptsize]
& L^nR^n(F_0) \arrow[equals]{dl}\arrow{rr}\arrow{dd}{\eta} & & 
   L^nR^nRL(F_0) \arrow{dd} \\
L^nR^n(F_0) \arrow[crossing over]{rr}\arrow{dd}{\eta} & & 
   RLL^nR^n(F_0) \arrow{ur} \\
& F_0 \arrow[equals]{dl}\arrow{rr} \arrow{dd} & & RL(F_0) \arrow{dd}
   \arrow[equals]{dl} \\
F_0 \arrow{rr} \arrow{dd} & & 
   RL(F_0) \arrow[crossing over, leftarrow]{uu} \\
& \tlHI{n}(F_0) \arrow[equals]{dl}\arrow{rr} & & 
   \tlHI{n} RL(F_0) \arrow[leftarrow]{dl}\\
\tlHI{n}(F_0) \arrow{rr} & & 
   RL\tlHI{n}(F_0) \arrow[crossing over, leftarrow]{uu} 
\end{tikzcd}
\]
Here, each of the vertical faces of the bottom cube is commutative.
It follows that the Square (1) must also be commutative.

\pfitem{Square (2) is commutative} : Square (2) can be further 
subdivided into:
\[
\begin{tikzcd}[row sep=huge, column sep=huge]
RL\tlHI{n}(F_0) \arrow{r}{R(\lambda)} \arrow{d}{\rho R(\lambda)} &
R\tlHI{n}(LF_0) \arrow{ld}{\rho} \arrow{r}{R\tlHI{n}(\susp)} &
R\tlHI{n + 1}(F_1) \arrow{ld}{\rho} \arrow[equals]{d} \\
\tlHI{n} RL(F_0) \arrow{r}{\tlHI{n} R(f)} &
\tlHI{n} R(F_1) \arrow{r}{\rho^{-1}} &
R\tlHI{n + 1}(F_1).
\end{tikzcd}
\]
The commutativity of the triangles in the diagram are clear. The 
parallelogram in the center is commutative by the naturality of
$\rho$ applied to $\susp: LF_0 \to F_1$.

Finally, 
\pfitem{$\tlHM{n}$ is functorial} : let $f_*: (F_*,\susp_*) \to 
(G_*, \susp_*')$ be a map between homotopy modules. Let 
$\tlHM{n}(f)$ be a map of graded homotopy invariant sheaves with 
transfers where the map on the $k$-th associated graded is 
$\tlHM[k]{n}(f) \defeq \tlHI{n + k}(f_k)$. 

By naturality of $\rho : R\tlHI{n + 1} \to \tlHI{n} R$ and 
$\lambda: L\tlHI{n} \to \tlHI{n + 1}L$ and the above arguments, 
the following are commutative
\[
\begin{tikzcd}[column sep=12em, row sep=huge]
R\tlHI{n + k}(F_k) \arrow{r}{R \tlHI{n + k}(f_k)} 
   \arrow{d}{\tlHI{n + k - 1}R(\deloop) \rho} &
R\tlHI{n + k}(G_k) 
   \arrow{d}{\tlHI{n + k - 1}R(\deloop') \rho} \\
\tlHI{n + k - 1}(F_{k - 1}) 
   \arrow{r}{\tlHI{n + k - 1}(f_{k - 1})} &
\tlHI{n + k - 1}(G_{k - 1}) 
\end{tikzcd}
\]
\[
\begin{tikzcd}[column sep=12em, row sep=huge]
L\tlHI{n + k}(F_k) \arrow{r}{L \tlHI{n + k}(f_k)} 
   \arrow{d}{\tlHI{n + k + 1}L(\susp) \rho} &
L\tlHI{n + k}(G_k) 
   \arrow{d}{\tlHI{n + k + 1}L(\susp') \rho} \\
\tlHI{n + k + 1}(F_{k + 1}) 
   \arrow{r}{\tlHI{n + k + 1}(f_{k - 1})} &
\tlHI{n + k + 1}(G_{k + 1}).
\end{tikzcd}
\]

It is clear that $\tlHM{n}(f)$ is a map from $\tlHM{n}(F)$ to 
$\tlHM{n}(G)$ as homotopy modules. The fact that $\tlHM{n}$ 
respects composition follows from the functoriality of $\tlHI{*}$.
The lemma is established.
\end{proof}

Now we proceed with the proof of Theorem \ref{thm_tlHM_corad}:

\begin{proof}[Proof of Theorem:]

\pfitem{$\tlHM{n}$ is a quotient functor} : certainly $F_* \to 
\tlHM{n}(F)$ is surjective for each $n$ since it is a surjection 
at each degree. What we need to verify is that the degree-wise 
surjection gives rise to a map of homotopy modules. In particular, 
we need to verify that the following
\[
\begin{tikzcd}
L(F_k) \arrow{r}{\susp} \arrow{d} &
F_{k + 1} \arrow{d} \\
L \tlHI{n + k}(F_k) \arrow{r}{\susp} &
\tlHI{n + k - 1}(F_{k + 1}).
\end{tikzcd}
\]
is commutative.

To see this, notice that the above diagram fits as the outer 
square of the following
\[
\begin{tikzcd}
L(F_k) \arrow{r}{\susp} \arrow{d} &
F_{k + 1} \arrow{d} \\
\tlHI{n + k + 1} L(F_k) \arrow{r} \arrow{d}{\lambda^{-1}} &
\tlHI{n + k + 1} F_{k + 1} \arrow[equals]{d} \\
L \tlHI{n + k + 1}(F_k) \arrow{r}{\susp} &
\tlHI{n + k + 1} F_{k + 1}.
\end{tikzcd}
\]
Here, the top square commutes by the naturality of $\id \to 
\tlHI{n + k}$ applied to $\susp: L(F_k) \to F_{k + 1}$, and the 
bottom square commutes by definition. Indeed, the definition of 
the suspension map $\susp: L \tlHI{n + k}(F_k) \to 
\tlHI{n + k + 1}(F_{k + 1})$ is precisely given by the composition
$\lambda \tlHI{n + k + 1}(\susp)$.

The fact that $\tlHM{n}$ respects delooping follows precisely by
the duality of the suspension and delooping as established by the
preceding lemma.

\pfitem{$\tlHM{n}$ is a pre-coradical} : The kernel of $F_* \to 
\tlHM(F)$ is a homotopy module $K_*$ whose $k$-th graded term is 
$\ker (F_k \to \tlHI{n + k}(F_k)$. But $\tlHM{n}(K_*)$ is a 
coradical; hence, $\tlHM{n}(K_*) = \tlHI{n + k}(K_k) = 0$. That
is $\tlHM{n}(K) = 0$, as desired.

\pfitem{$\tlHM{n}$ is a right exact} : since $\tlHI{n + k}$ is 
right exact for each $k$, $\tlHM{n}$ for each associated graded 
term. It follows that $\tlHM{n}$ is right exact.
\end{proof}

The following corollary is a straightforward consequence of 
Theorem \ref{thm_corad_equiv_htt} and Theorem 
\ref{thm_tlHM_corad}.

\begin{cor}
There exists a strong filtration of $\HM$
\[
\cdots \subseteq \tgHM{i}\HM \subseteq \tgHM{i - 1}\HM \subseteq 
   \cdots \subseteq \HM
\]
and a strong cofiltration of $\HM$
\[
\cdots \subseteq \tlHM{i}\HM \subseteq \tlHM{i + 1} \subseteq 
   \cdots \subseteq \HM
\]
where $(F_*, \deloop_*)$ in $\tgHM{i}\HM$ is characterized by
$\tlHM{i}(F) = 0$, and $(F_*, \deloop_*)$ in $\tlHM{i}\HM$ is
characterized by $\tlHM{i}(F) = F_*$.
\end{cor}
\noproof

By \ref{cor_tt_ref_and_coref}, the reflection functors 
$\tgHM{n} : \HM \to \tgHM{n}\HM$ are defined by sending $(F_*, 
\deloop_*)$ in $\HM$ to the kernel of the unit map 
$(F_*, \deloop_*) \to \tlHM{n}(F)$. Write $\tgHM{n}(F)$ for this 
kernel. From this discussion, we have another characterization of 
the objects in $\tgHM{n}\HM$, which follows easily from 
\ref{thm_precorad_eq_tt}:

\begin{cor}\label{cor_tgHM_prop}
The objects in $\tgHM{n}\HM$ are those $(F_*, \deloop_*)$ for
which $\tgHM{n}(F) = (F_*, \deloop_*)$.
\end{cor}
\noproof

We conclude this section by two immediate applications of the
results above. The first is to show that there is a torsion 
filtration structure on the category of cycle modules (defined 
below) and the second is to describe the behavior of the torsion
filtration with respect to the tensor structure on both $\HI$
and $\HM$.

Recall from \cite{Rost96} the definition of a cycle module:

\begin{defn}\label{def_pre_cycmod}
Let $X$ be an arbitrary finite-type $k$-scheme, and let 
$\fields(X)$ be the class of all fields $F$ such that there
exists a map $\Spec F \to X$, and that $F$ is finitely generated
over $k$.

A \emph{cycle premodule} on $\fields(X)$ is an object function
that associated to each $F$ in $\fields(B)$ a $\Z$-graded abelian
group $M(F) = \prod_i M_i(K)$ with the following data:

\begin{enumerate}
\item[\textbf{D1.}] For each $\phi: F \to E$, there is a degree 0
map $\phi_*: M(F) \to M(E)$ called the \emph{restriction map 
associated to $\phi$}

\item[\textbf{D2.}] For each finite $\phi: F \to E$, there is a 
degree 0 map $\phi^*: M(E) \to M(F)$ called the \emph{corestriction
map associated to $\phi$}

\item[\textbf{D3.}] For each $F$, the group $M(F)$ is equipped
with a left $\milK_*(F)$-module, where $\milK_*(F)$ is the Milnor
$K$-ring of $F$.

\item[\textbf{D4.}] For any valuation $v$ of $F$, there exists 
maps $\res{v}: M(F) \to M(\resf(v))$ and $\Special{v}{\ideal{p}}$
called the \emph{residue} and \emph{specialization} respectively, 
where $\resf(v)$ is the residue field of $v$ and $\ideal{p}$ is 
a prime of $v$,
\end{enumerate}

which is subject to the following conditions:

\begin{enumerate}
\item[\textbf{R1a.}] For each $\phi: F \to E$ and $\psi: E \to L$,
$(\psi \comp \phi)_* = \psi_* \comp \phi_*$

\item[\textbf{R1b.}] For each finite $\phi: F \to E$ and $\psi: E
\to L$, $(\psi \comp \phi)^* = \phi^* \comp \psi^*$

\item[\textbf{R1c.}] For $\phi: F \to E$ and $\psi: E \to L$ with
$\phi$ finite, define $R = E \otimes_F L$, and let $\ideal{p}$
be any prime ideal of $R$. (As $R$ is Artin, let $l_p$ be the 
length of the localized ring $R_{(\ideal{p})}$), and 
$\phi_{\ideal{p}}: L \to R/{\ideal{p}}$ and 
$\psi_{\ideal{p}}: E \to R/{\ideal{p}}$ be natural maps.
\[
\psi_*\comp \phi^* = \sum_{\ideal{p}} l_p \cdot 
(\phi_{\ideal{p}})^* \comp (\psi_{\ideal{p}})_*.
\]

\item[\textbf{R2.}] For $\phi: F \to E$, $x \in \milK_*F$, $y \in
\milK_*E$, $\rho \in M(F)$, and $\mu \in M(E)$, then:

\item[\textbf{R2a.}] $\phi_*(x \cdot \rho) = \phi_*(x) \cdot 
\phi_*(\rho)$.

\item[\textbf{R2b.}] if $\phi$ is finite, $\phi^*(\phi_*(x) \cdot 
\mu) = x \cdot \phi^*(\mu)$.

\item[\textbf{R2c.}] if $\phi$ is finite, $\phi^*(y \cdot 
\phi_*(\rho)) = \phi^*(y) \cdot \rho$.

\item[\textbf{R3.}] For $\phi: F \to E$, $v$ a valuation on $E$
and $w$ and a valuation on $F$:

\item[\textbf{R3a.}] Suppose $w$ is a nontrivial restriction of 
$v$ with ramification index $e$. Let $\phib: \resf{w} \to 
\resf{v}$ be the induced map. Then:
\[
\res{v} \comp \phi_* = e \phib_* \comp \res{w}.
\]

\item[\textbf{R3b.}] Let $\phi$ be finite. For each valuation $v$ 
an extension of $w$ to $E$, let $\phi_v: \resf{w} \to \resf{v}$
be the induced map. Then
\[
\res{v} \comp \phi^* \sum_{v} \comp \res{v}.
\]

\item[\textbf{R3c.}] Suppose $v$ restricts to a trivial valuation
$w$ on $F$. Then
\[
\res{v} \comp \phi_* = 0
\]

\item[\textbf{R3d.}] Suppose $v$ again restricts to a trivial 
valuation, and let $\phib: F \to \resf{v}$ be the induced map, and
$\ideal{p}$ a prime of $v$. Then
\[
\Special{v}{\ideal{p}} \comp \phi_* = \phib_*
\]

\item[\textbf{R3e.}] Let $u$ be a $v$-unit, and $\rho \in M(F)$,
one has
\[
\res{v}(\{u\} \cdot \rho) = -\{\overline{u}\} \cdot \res{v}(\rho).
\]
\end{enumerate}
\end{defn}

For $X$ a $k$-scheme, let $\subsch{1}{X}$ denote the collection of 
codimension 1 subscheme. Write $\xi_X$ be the generic point of an
irreducible $X$ with $K_X = \O_X,\xi_X$. If $X$ is normal, then 
for $x \in \codim{1}{X}$, the local ring $\O_X,x$ is a valuation 
ring of $K_X$ with residue field $\resf{x}$. Write $M(x)$ for 
$M(\resf{x})$, and $\res{x}: M(\xi_X) \to M(x)$ for the 
restriction map.

Furthermore, for $x, y \in X$, let $Z$ be the closed subscheme 
determined by $x$, and $\overline{Z}$ be the normalization $Z$.
Define
\[
\ptres{x}{y}: M(x) \to M(y)
\]
by
\[
\ptres{x}{y} = 
\begin{cases}
0 & y \notin \subsch{1}{Z} \\
\sum_{z|y} \phi_{\resf{z},\resf{x}}^* \comp \res{z} & \textrm{otherwise}
\end{cases}
\]
In case $\ptres{x}{y}$ is nonzero, the sum is taken over all $z$
lying over $y \in \subsch{1}{Z}$, and
$\phi_{\resf{z},\resf{y}}^*$ is the corestriction map associated to 
the finite field extension $\resf{y} \to \resf{z}$.

\begin{defn}
A cycle module $M$ on $\fields(B)$ is a cycle premodule which
satisfies the following conditions:

\begin{enumerate}
\item[\textbf{(FD)}] \itemhead{Finite support of divisors.} 
$X$ be a normal scheme and $\rho \in M(\xi_X)$. Then $\res{x}: 
M(\xi_X) \to M(X)$ is 0 for all but finitely many $x \in 
\subsch{1}{X}$.

\item[\textbf{(C)}] \itemhead{Closedness.} If $X$ is an integral
local local of dimension 2 with closed point $x_0$, then the map 
from $M(\xi_X)$ to $M(x_0)$ given by
\[
\sum_{x \in \subsch{1}{X}} \ptres{x_0}{x} \comp \ptres{x}{\xi}
\]
is 0.
\end{enumerate}
\end{defn}

D\'eglise showed in \cite{DegModHom} that a homotopy module $(F_*, 
\deloop_*)$ gives rise to a unique cycle module $\assocCM{F_*}$,
and that this association defines an equivalence between the 
category of homotopy modules and cycle modules. (See 
\cite[3.7]{DegModHom}.)

Via this categorical equivalence, we have the following:
\begin{thm}
There exists a torsion filtration of $\CycMod$. In particular,
there exist a $\Z$-indexed descending strong filtration:
\[
\cdots \subseteq \tgHM{i}\CycMod \subseteq \tgHM{i - 1}\CycMod \subseteq 
   \cdots \subseteq \CycMod
\]
and a $\Z$-indexed increasing strong cofiltration 
\[
\cdots \subseteq \tlHM{i}\CycMod \subseteq \tlHM{i + 1} \subseteq 
   \cdots \subseteq \CycMod
\]
on $\CycMod$, such that for all integers $n$, the pair of full 
subcategories 
\[
(\tlHM{n}\CycMod, \tgHM{n}\CycMod)
\]
form a torsion theory on $\CycMod$.
\end{thm}

\begin{ex}\label{ex_milK}
A well-known example of a cycle module is Milnor $K$-theory,
$\milK_*$. Recall from \cite{MilK} that for a field $F$, the
Milnor $K$-theory of $F$ is the graded commutative ring given by
\[
\milK_*(F) \defeq T^*(F^*)/I
\]
where $T^*(F^*)$ denotes the tensor algebra of the multiplicative 
group $F^*$, and $I$ denotes the ideal generated by $a \tensor 
(1 - a)$ for all $a$ in $F^*$. The $n$-th Milnor $K$-theory of $F$ 
is the $n$-th graded piece of $\milK_*(F)$.

Via \cite[3.7]{DegModHom}, the homotopy module corresponding to 
$\milK_*$ is $(\Ox)^{\tensor n}$. Therefore, $\milK_n \in
\tlHM{n}\CycMod \cap \tgHM{n}\CycMod$.
\end{ex}

We now discuss the tensorial properties of the filtration. Let us
first consider the following notion:

\begin{defn}\label{def_graded_tensor}
Let $(\Cat{C}, \tensor, \Unit)$ be a monoidal category.
We say that $\Cat{C}$ is a \DEF{graded monoidal category (by
$(\phi_*\Cat{C}, \phi_*)$)} if $(\phi_*\Cat{C}, \phi_*)$ is a 
(weak) filtration of $\Cat{C}$ such that for all integers $n$
and $m$, $\phi_n\Cat{C} \tensor \phi_m\Cat{C} \subseteq
\phi_{n + m}\Cat{C}$.
\end{defn}

\begin{ex}
By construction, it is clear that $(\sgHI{*}\HI, \sgHI{*})$ 
defines a graded symmetric monoidal category on $\HI$ under
$\tHI$.
\end{ex}

We first show that $(\tgHI{*}\HI, \tgHI{*})$ define a graded
monoidal category on $\HI$. 

\begin{prop}\label{prop_tensor_and_tfilt_HI}
For $F$ in $\tgHI{n}\HI$ and $G$ in $\tgHI{m}\HI$, $F \tHI G$
is an object of $\tgHI{n + m}\HI$.
\end{prop}
\begin{proof}
Since $F \in \tgHI{n}\HI$, the counit $\epsilon: L^nR^n(F) \to F$ 
is surjective (Prop.  \ref{prop_THI_properties} (1)), and since 
the functor $- \tHI G$ is right exact, the map
\[
L^nR^n(F) \tHI G \xrightarrow{\epsilon_F \tHI G} F \tHI G.
\]
is also surjective. Similarly,
\[
L^nR^n(F) \tHI L^mR^m(G) \xrightarrow{L^nR^n(F) \tHI \epsilon_G}
L^nR^n(F) \tHI G
\]
is also surjective. Composing these two surjection, we have a 
surjection
\[
L^nR^n(F) \tHI L^mR^m(G) \to F \tHI G.
\]
But 
\[
L^nR^n(F) \tHI L^mR^m(G) = L^{n + m}(R^n(F) \tHI R^m(G)) \in
\sgHI{n + m}\HI.
\]
Therefore, $\tlHI{n + m}(L^nR^n(F) \tHI L^mR^m(G)) = 0$. However,
the functor $\tlHI{n + m}$ is right exact; in particular, 
\[
\tlHI{n + m}(L^nR^n(F) \tHI L^mR^m(G)) \to
   \tlHI{n + m}(F \tHI G)
\]
is a surjection. It follows that $\tlHI{n + m}(F \tHI G) = 0$ and
$F \tHI G \in \tgHI{n + m}\HI$ as desired.
\end{proof}

\begin{cor}\label{cor_graded_tensor_HI}
There exists a graded symmetric monoidal structure on $\HI$ by
$(\tgHI{*}\HI, \tgHI{*})$.
\end{cor}

We now show that the graded symmetric monoidal structure on $\HI$
can be extended to a graded symmetric monoidal structure on $\HM$.

\begin{prop}\label{prop_graded_mon_struct_HM}
Let $n$ and $m$ be arbitrary integers. For $F_* \defeq (F_*, 
\deloop_*)$ and $G_* \defeq (G_*,\deloop_*)$ in $\tgHM{n}\HM$ and 
$\tgHM{m}\HM$ respectively, $F_* \tHI G_*$ is an object of 
$\tgHM{n + m}$.
\end{prop}
\begin{proof}
By \ref{cor_tgHM_prop}, it suffices to show that $\tgHM{n + m}(F_*
\tHI G_*) = F_* \tHI G_*$.

Regard the objects of $\HM$ as pairs $(F, k)$, where $F \in \HI$
and $k$ is an arbitrary integer. It is straightforward to 
verify from the definition of $\tgHM{*}$ and \ref{prop_tl_L_R} 
that
\[
\tgHM{n}((F, k)) = \begin{cases}
(\tgHI{n - k}F, k) &\textrm{if }n > k\\
(F, k)             &\textrm{otherwise}.
\end{cases}
\]
Therefore, we are reduced to showing that for $F$ and $G$ in
$\tgHI{n}\HI$ and $\tgHI{m}\HI$ respectively, 
$\tgHI{n + m}(F \tHI G) = F \tHI G$. This follows directly from 
\ref{cor_graded_tensor_HI}.
\end{proof}

\begin{cor}\label{cor_graded_mon_struct_HM}
There exists a graded symmetric monoidal structure on $\HM$ by
$(\tgHI{*}\HM, \tgHM{*})$.
\end{cor}

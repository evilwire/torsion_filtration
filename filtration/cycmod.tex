\newpage
\section{Filtration on $\CycMod$}\label{sect_filtration_cycmod}

In this section, we extend the filtration structure defined on 
$\HI$ to the category of homotopy modules $\HM$. Via the 
categorical equivalence between $\CycMod$ and $\HM$ (Thm. 
\ref{thm_cycmod_eq_hm}), we obtain a filtration on the abelian 
category $\CycMod$.

Throughout the section, we adopt the same notation convention as 
in Section \ref{sect_cycmod_and_hm} and write $(F_*, \deloop_*)$ 
for a homotopy module where the $\deloop_n: F_{n} \to 
\RHI{(F_{n + 1})}$ denotes the $n$-th delooping map (see Def. 
\ref{def_hm}).

The following lemma will be key to extending the functors 
$\tlHI{n}$, $\tgHI{n}$ and $\sgHI{n}$ on $\HM$:

\begin{prop}\label{prop_tl_L_R}
For $F \in \HI$, 
\[
\LHI{(\tlHI{n} F)} \simeq \tlHI{n + 1}(\LHI{F})
\]
and
\[
\RHI{(\tlHI{n} F)} \simeq \tlHI{n - 1}(\RHI{F}).
\] 
Both isomorphisms are natural in $F$.
\end{prop}
\begin{proof}
For ease of notation, let $L$ be the functor $F \mapsto \LHI{F}$, 
and $R$ be given by $F \mapsto \RHI{F}$, and $(L, R)$ form an 
adjoint pair, such that $RL$ is naturally ismorphic to the 
identity (Prop. \ref{prop_unit_iso}).

The key is to prove that the diagram
\begin{equation}\label{eq_Ltl_com_diag}
\begin{tikzcd}
L^{n + 1}R^{n + 1}LF \arrow{r}{\eta^{n + 1}_{LF}} \arrow{d} &
LF \arrow[equals]{d} \\
L(L^{n}R^nF) \arrow{r}{L\eta^{n}_F} &
LF.
\end{tikzcd}
\end{equation}
is commutative. Here, $\eta^n$ denotes the counit $L^nR^n \to \id$,
and the vertical map $L^{n + 1}R^{n + 1}LF \to L(L^nR^n F)$ is 
given by the map $L^{n + 1}R^n \epsilon^{-1}$, where $\epsilon$ is
the unit $\id \to RL$ and an isomorphism.

To establish that the above diagram is commutative, we proceed by 
induction on $n$. The case $n = 0$ follows by the counit-unit 
adjunction. That is, the composition
\[
LF \stackrel{L\epsilon}{\to} LRLF \stackrel{\eta L}{\to} LF.
\]
is the identity. Therefore, $\eta L = L\epsilon^{-1}$, and the
following diagram commutes:
\[
\begin{tikzcd}
LRL F \arrow{r}{\eta L} \arrow{d}{L \epsilon^{-1}} &
LF \arrow[equals]{d} \\
LF \arrow[equals]{r} & LF.
\end{tikzcd}
\]

Now, suppose we have
\begin{equation}\label{eq_induct_hyp_tl_diag}
\begin{tikzcd}
L^{n}R^{n}LF \arrow{r}{\eta^{n}_{LF}} \arrow{d} &
LF \arrow[equals]{d} \\
L(L^{n - 1}R^{n - 1}F) \arrow{r}{L\eta^{n - 1}_F} &
LF.
\end{tikzcd}
\end{equation}
There exists a natural transformation $\eta': L^n R^n \to 
L^{n - 1} R^{n - 1}$. Applying this to $\epsilon^{-1}: RL F \to 
F$, we have the following commutative diagram
\[
\begin{tikzcd}
L^n R^n RL F \arrow{r}{\eta'_{RLF}} 
   \arrow{d}{L^nR^n \epsilon^{-1}} &
L^{n - 1}R^{n - 1} RLF 
   \arrow{d}{L^{n - 1}R^{n - 1}\epsilon^{-1}} \\
L^nR^n F \arrow{r}{\eta'_F} &
L^{n - 1}R^{n - 1} F,
\end{tikzcd}
\]
and applying $L$, we have
\[
\begin{tikzcd}
L^{n + 1} R^{n + 1} L F \arrow{r}{L\eta'_{RLF}} 
   \arrow{d}{L^{n + 1}R^n \epsilon^{-1}} &
L^{n}R^n LF \arrow{d}{L^nR^{n - 1}\epsilon^{-1}} \\
L^{n + 1}R^n F \arrow{r}{L\eta'_F} &
L^{n}R^{n - 1} F,
\end{tikzcd}
\]
which fits together with Diagram \ref{eq_induct_hyp_tl_diag} to 
give the following commutative diagram:
\[
\begin{tikzcd}
L^{n + 1} R^{n + 1} L F \arrow{r}{L\eta'_{RLF}} 
   \arrow{d}{L^{n + 1}R^n \epsilon^{-1}}
   \arrow[bend left]{rr}{\eta^{n + 1}_{LF}} &
L^{n}R^n LF \arrow{d}{L^nR^{n - 1}\epsilon^{-1}}
   \arrow{r}{\eta^{n}_{LF}} &
LF \arrow[equals]{d} \\
L^{n + 1}R^n F \arrow{r}{L\eta'_F} 
   \arrow[bend right]{rr}{L\eta^n_F} &
L^{n}R^{n - 1} F \arrow{r}{L\eta^{n - 1}_F} &
LF.
\end{tikzcd}
\]
Notice that the composition of the top and bottom horizontal 
arrows (indicated by the bent arrows) in the diagram above are 
precisely $\eta^{n + 1}_KF$ and $L \eta^n_F$ respectively, and
the claim is proved.

Returning to Diagram \ref{eq_Ltl_com_diag}, which we now know to 
be commutative, notice that the cokernel of the top row is 
$\tlHI{n + 1}LF$, and since $L$ is right exact, the cokernel of 
the bottom row is $L\tlHI{n} F$. By the Five Lemma, it is clear 
that $\tlHI{n + 1}LF \simeq L\tlHI{n} F$.

For the other isomorphism, apply similar arguments to the
diagram
\[
\begin{tikzcd}
L^{n - 1}R^{n - 1}(RF) \arrow{r}{\eta^n_{RF}} 
   \arrow{d}{\eta_{L^{n - 1}R^{n - 1}F}} &
RF \arrow[equals]{d} \\
R (L^n R^n F) \arrow{r}{R \eta^n} &
RF,
\end{tikzcd}
\]
using the fact that the functor $R$ is exact. The diagram itself
can be shown to be commutative by using similar arguments as in
the case for Diagram \ref{eq_Ltl_com_diag}.
\end{proof}

We can now define the cofiltration endofunctor on $\HM$. Let
$(F_*, \deloop_*)$ be an object of $\HM$, and write $\tlHM{n}(F_*)$ 
for the graded homotopy invariant sheaf with transfers where
\[
(\tlHM{n}(F_*))_k \defeq
\begin{cases}
   \tlHI{n + k}F_k & \textrm{if } n + k > 0 \\
   0               & \textrm{otherwise}.
\end{cases}
\]
For ease of notation, we write $\tlHM{n}(F)$ for the essential 
image of $F_*$ (dropping the ``$_*$'' on $F$ to avoid 
redundancy) and we write $\tlHM[k]{n}(F)$ for the $k$-th graded 
component of $\tlHM{n}(F)$.

By the previous proposition, there are two maps that we can
define between the graded pieces of $\tlHM{n}(F)$. Let
\[
\suspTL[k]{n} : \LHI{\tlHM[k]{n}(F)} \defeq 
   \LHI{\tlHI{n + k}(F_k)} \to \tlHM[k + 1]{n}(F) \defeq 
   \tlHI{n + k + 1}(F_{k + 1})
\]
denote the composition
\begin{equation}\label{eq_susp_def}
\LHI{\tlHI{n + k}(F_k)} \stackrel{\simeq}{\to} 
   \tlHI{n + k + 1}(\LHI{F_k}) 
   \xrightarrow{\;\tlHI{n + k + 1}(\susp_k)\;} 
   \tlHI{n + k + 1}(F_{k + 1})
\end{equation}
where $\susp_k: \LHI{F_k} \to F_{k + 1}$ is the $k$-th suspension 
map of $(F_*, \deloop_*)$. Let
\[
\deloopTL[k]{n} : \tlHM[k - 1]{n}(F) \defeq 
   \tlHI{n + k - 1}(F_{k - 1}) \to \RHI{\tlHM[k]{n}(F)} \defeq 
   \RHI{\tlHI{n + k}(F_{k})} 
\]
be given by
\begin{equation}\label{eq_deloop_def}
\tlHI{n + k - 1}(F_{k - 1})
   \xrightarrow{\;\tlHI{n + k - 1}(\deloop_k)\;} 
   \tlHI{n + k - 1}(\RHI{(F_k)}) \stackrel{\simeq}{\to} 
   \RHI{(\tlHI{n + k}(F_k))},
\end{equation}
where $\deloop_k: F_{k - 1} \to \RHI{(F_k)}$ is the $k$-th delooping
of $(F_*, \deloop_*)$. Notice that, in this case, $\deloop_k$ is
an isomorphism for all $k$. Therefore, $\deloopTL[k]{n}$ is an 
isomorphism for all $k$. We claim that with the maps defined above,
$\tlHM{n}(F)$ is an object of $\HM$ (proved in Lem. 
\ref{lem_tlHM_is_functor} below). In fact, we have that:

\begin{thm}\label{thm_tlHM_funct}
For each $n \in \Z$, $\tlHM{n}$ is a coradical of $\HM$.
\end{thm}

As a first step, we first establish that:

\begin{lem}\label{lem_tlHM_is_functor}
$\tlHM{n}$ is an endofunctor of $\HM$.
\end{lem}

\begin{proof}
Fix $(F_*, \deloop_*)$, we show that $\tlHM{n}(F)$ is an object of
$\HM$. We have already established the graded structure of 
$\tlHM{n}(F)$. We also have candidates for the suspension and 
delooping maps; these are $\suspTL{n}$ and $\deloopTL{n}$ 
respectively. It remains to show that $\suspTL[k]{n}$ and 
$\deloopTL[k + 1]{n}$ are adjoint in the sense that $\suspTL{n}$ is 
mapped to $\deloopTL{n}$ via the adjunction 
\[
\Phi: \homHM(\LHI{\tlHM[k]{n}(F)}, \tlHM[k + 1]{n}(F))
\to \homHM(\tlHM[k]{n}(F), \RHI{(\tlHM[k + 1]{n}(F))})
\]
given that $\susp_{k + 1}$ is adjoint to $\deloop_k$.

Recall that $\shift[k]{F_*}$ denotes a shift in grading as given by 
$\shift[k]{F_*}_n = F_{n + k}$. (See Sect. \ref{sect_cycmod_and_hm} 
for definition of $\shift[k]{F_*}$.) Noting that 
$\tlHM{n}(\shift[k]{F}) = \shift[k]{\tlHM{n - k}(F)}$, we may 
assume without loss of generality that $k = 0$. 

To simplify notations, let
\[
\lambda : L\tlHI{n} \to \tlHI{n + 1}L
\]
and 
\[
\rho : R\tlHI{n} \to \tlHI{n - 1}R
\]
be the natural isomorphisms given in Prop. \ref{prop_tl_L_R}. 
Following the convention in the proof of Prop. \ref{prop_tl_L_R}, 
let $L$ and $R$ denote the functors $F \mapsto \LHI{F}$ and $F 
\mapsto \RHI{F}$ respectively, and let $\epsilon^n : \id \to 
R^nL^n$ and $\eta^n: L^nR^n \to \id$ denote respectively the unit 
and counit maps; we abbreviate $\eta^1$ as $\eta$, and $\epsilon^1$
as $\epsilon$. Finally, we abuse notation and write $\Phi$ for the 
adjunction isomorphism
\[
\Phi: \homHI(L-, -) \to \homHI(-,R-)
\]
between any pair of objects in $\HI$.

Notice that for $f: L(F) \to G$ in $\HI$, $\Phi(f) = \eta R(f)$.
Since $\suspTL[0]{n}$ is given by
\[
L\tlHI{n}(F_0) \stackrel{\lambda}{\to} \tlHI{n + 1} L(F_0)
   \xrightarrow{\;\tlHI{n + 1}(\susp_0)\;} \tlHI{n + 1}(F_1),
\]
to show that $\suspTL[0]{n}$ is adjoint to $\deloopTL[0]{n}$ is to
verify that the following diagram commute:
\[
\begin{tikzcd}
\tlHI{n}(F_0) \arrow{rr}{\epsilon} \arrow[equals]{dd} &&
RL (\tlHI{n}F_0) \arrow{rr}{R(\tlHI{n + 1}(\susp) \lambda)} 
\arrow[dotted]{dd}{\lambda \rho}&&
R \tlHI{n + 1} (F_1) \arrow[equals]{dd} \\
& (1) && (2) \\
\tlHI{n}(F_0) \arrow{rr}{\tlHI{n}(\epsilon)} &&
\tlHI{n} RL(X) \arrow{rr}{\rho \tlHI{n} R(\susp)} &&
R\tlHI{n}(F_1)
\end{tikzcd}
\]
where composition along the top row is precisely 
$\Phi(\suspTL[0]{n})$, and the bottom is precisely 
$\deloopTL[1]{n}$. The diagram breaks up into two squares,
labelled (1) and (2), along the map $\lambda \rho: RL 
\tlHI{n}(F_0) \to \tlHI{n} RL(F_0)$. We proceed by showing that 
each square is commutative.

\pfitem{Square (1) is commutative} : Applying the naturality of
$\eta^n$ to the map $\epsilon: F_0 \to RL(F_0)$, we obtain the 
following commutative diagram:
\begin{equation}\label{eq_LR_rear_face}
\begin{tikzcd}[column sep=large]
L^nR^n(F_0) \arrow{r}{L^nR^n\epsilon} \arrow{d}{\eta^n} &
L^nR^nRL(F_0) \arrow{d}{\eta^n(\epsilon)} \\
F_0 \arrow{r}{\epsilon} &
RL(F_0).
\end{tikzcd}
\end{equation}
Similarly, applying the naturality of $\epsilon$ to $\eta^n$, we
have
\begin{equation}\label{eq_LR_front_face}
\begin{tikzcd}[column sep=large]
L^nR^n(F_0) \arrow{d}{\epsilon} \arrow{r}{\eta^n} &
F_0 \arrow{d}{\epsilon} \\
RLL^nR^n(F_0) \arrow{r}{RL(\eta^n)} &
RL(F_0)
\end{tikzcd}
\end{equation}
which fits together in a ``cube'':
\[
\begin{tikzcd}[row sep=scriptsize, column sep=scriptsize]
& L^nR^n(F_0) \arrow[equals]{dl}\arrow{rr}\arrow{dd}{\eta} & & 
   L^nR^nRL(F_0) \arrow{dd} \\
L^nR^n(F_0) \arrow[crossing over]{rr}\arrow{dd}{\eta} & & RLL^nR^n(F_0) 
   \arrow[dotted]{ur} \\
& F_0 \arrow[equals]{dl}\arrow{rr} & & RL(F_0) 
   \arrow[equals]{dl} \\
F_0 \arrow{rr} & & RL(F_0) \arrow[crossing over, leftarrow]{uu}
\end{tikzcd}
\]
where the faces parallel to the page are precisely (from front to
rear) Diagrams \ref{eq_LR_front_face} and \ref{eq_LR_rear_face}
respectively. Since unit maps are an isomorphisms, if we let the 
dotted arrow be given by the composition
\[
RLL^nR^n(F_0) \stackrel{\epsilon^{-1}}{\to} L^nR^n(F_0) 
\xrightarrow{L^nR^n(\epsilon)} L^nR^nRL(F_0)
\]
then all faces of the cube are commutative.

At last, notice that the Square (1) form the ``cokernel'' of the
cube, where Square (1) is precisely the bottom face of the bottom
cube in the following:
\[
\begin{tikzcd}[row sep=scriptsize, column sep=scriptsize]
& L^nR^n(F_0) \arrow[equals]{dl}\arrow{rr}\arrow{dd}{\eta} & & 
   L^nR^nRL(F_0) \arrow{dd} \\
L^nR^n(F_0) \arrow[crossing over]{rr}\arrow{dd}{\eta} & & 
   RLL^nR^n(F_0) \arrow{ur} \\
& F_0 \arrow[equals]{dl}\arrow{rr} \arrow{dd} & & RL(F_0) \arrow{dd}
   \arrow[equals]{dl} \\
F_0 \arrow{rr} \arrow{dd} & & 
   RL(F_0) \arrow[crossing over, leftarrow]{uu} \\
& \tlHI{n}(F_0) \arrow[equals]{dl}\arrow{rr} \arrow{dd} & & 
   \tlHI{n} RL(F_0) \arrow[leftarrow]{dl} \arrow{dd}\\
\tlHI(F_0) \arrow{rr} \arrow{dd} & & 
   RL\tlHI{n}(F_0) \arrow[crossing over, leftarrow]{uu} \\
& 0 & & 0 \\ 0 & & 0. \arrow[crossing over, leftarrow]{uu} \\
\end{tikzcd}
\]
Here, each of the vertical faces of the bottom cube is commutative.
It follows that the Square (1) must also be commutative.

\pfitem{Square (2) is commutative} : Square (2) can be further 
subdivided into:
\[
\begin{tikzcd}[row sep=huge, column sep=huge]
RL\tlHI{n}(F_0) \arrow{r}{R(\lambda)} \arrow{d}{\rho\lambda} &
R\tlHI{n}(LF_0) \arrow{ld}{\rho} \arrow{r}{R\tlHI{n}(\susp)} &
R\tlHI{n + 1}(F_1) \arrow{ld}{\rho} \arrow[equals]{d} \\
\tlHI{n} RL(F_0) \arrow{r}{\tlHI{n} R(f)} &
\tlHI{n} R(F_1) \arrow{r}{\rho^{-1}} &
R\tlHI{n + 1}(F_1).
\end{tikzcd}
\]
The commutativity of the triangles in the diagram are clear. The 
parallelogram in the center is commutative by the naturality of
$\rho$ applied to $\susp: LF_0 \to F_1$.

Finally, 
\pfitem{$\tlHM{n}$ is functorial} : let $f_*: (F_*,\susp_*) \to 
(G_*, \susp_*')$ be a map between homotopy modules. Let 
$\tlHM{n}(f)$ be a map of graded homotopy invariant sheaves with 
transfers where the map on the $k$-th associated graded is 
$\tlHM[k]{n}(f) \defeq \tlHI{n + k}(f_k)$. 

By naturality of $\rho : R\tlHI{n + 1} \to \tlHI{n} R$ and 
$\lambda: L\tlHI{n} \to \tlHI{n + 1}L$ and the above arguments, 
the following are commutative
\[
\begin{tikzcd}[column sep=12em, row sep=huge]
R\tlHI{n + k}(F_k) \arrow{r}{R \tlHI{n + k}(f_k)} 
   \arrow{d}{\tlHI{n + k - 1}R(\deloop) \rho} &
R\tlHI{n + k}(G_k) 
   \arrow{d}{\tlHI{n + k - 1}R(\deloop') \rho} \\
\tlHI{n + k - 1}(F_{k - 1}) 
   \arrow{r}{\tlHI{n + k - 1}(f_{k - 1})} &
\tlHI{n + k - 1}(G_{k - 1}) 
\end{tikzcd}
\]
\[
\begin{tikzcd}[column sep=12em, row sep=huge]
L\tlHI{n + k}(F_k) \arrow{r}{L \tlHI{n + k}(f_k)} 
   \arrow{d}{\tlHI{n + k + 1}L(\susp) \rho} &
L\tlHI{n + k}(G_k) 
   \arrow{d}{\tlHI{n + k + 1}L(\susp') \rho} \\
\tlHI{n + k + 1}(F_{k + 1}) 
   \arrow{r}{\tlHI{n + k + 1}(f_{k - 1})} &
\tlHI{n + k + 1}(G_{k + 1}).
\end{tikzcd}
\]

It is clear that $\tlHM{n}(f)$ is a map from $\tlHM{n}(F)$ to 
$\tlHM{n}(G)$ as homotopy modules. The fact that $\tlHM{n}$ 
respects composition follows from the functoriality of $\tlHI{*}$.
The lemma is established.
\end{proof}

Now we proceed with the proof of Theorem \ref{thm_tlHM_funct}:

\begin{proof}[Proof of Theorem:]

\pfitem{$\tlHM{n}$ is a quotient functor} : certainly $F_* \to 
\tlHM{n}(F)$ is surjective for each $n$ since it is a surjection 
at each degree. What we need to verify is that the degree-wise 
surjection gives rise to a map of homotopy modules. In particular, 
we need to verify that the following
\[
\begin{tikzcd}
L(F_k) \arrow{r}{\susp} \arrow{d} &
F_{k + 1} \arrow{d} \\
L \tlHI{n + k}(F_k) \arrow{r}{\susp} &
\tlHI{n + k - 1}(F_{k + 1}).
\end{tikzcd}
\]
is commutative.

To see this, notice that the above diagram fits as the outer 
square of the following
\[
\begin{tikzcd}
L(F_k) \arrow{r}{\susp} \arrow{d} &
F_{k + 1} \arrow{d} \\
\tlHI{n + k + 1} L(F_k) \arrow{r}{\lambda^{-1}} \arrow{d} &
\tlHI{n + k + 1} F_{k + 1} \arrow[equals]{d} \\
\tlHI{n + k + 1} L(F_k) \arrow{r}{\susp} &
\tlHI{n + k + 1} F_{k + 1}.
\end{tikzcd}
\]
Here, the top square commutes by the naturality of $\id \to 
\tlHI{n + k}$ applied to $\susp: L(F_k) \to F_{k + 1}$, and the
bottom square commutes by definition.

\pfitem{$\tlHM{n}$ is a pre-coradical} :

\pfitem{$\tlHM{n}$ is a right exact} :
\end{proof}

By Thm. \ref{thm_corad_equiv_htt}, the following is a 
straightforward consequence:

\begin{cor}
There exists a strong filtration of $\HM$
\[
\cdots \subseteq \tgHM{i}\HM \subseteq \tgHM{i - 1} \subseteq 
   \cdots \subseteq \HM
\]
and a strong cofiltration of $\HM$
\[
\cdots \subseteq \tlHM{i}\HM \subseteq \tlHM{i + 1} \subseteq 
   \cdots \subseteq \HM
\]
where $(F_*, \deloop_*) \in \tgHM{i}\HM$ is characterized by
$\tgHM{i}(F) = 0$, and $(F_*, \deloop_*) \in \tlHM{i}\HM$ is
characterized by $\tlHM{i}(F) = F_*$.
\end{cor}

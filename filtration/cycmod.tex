\newpage
\chapter{Filtration on $\CycMod$}\label{sect_filtration_cycmod}

In this chapter, we will extend the three filtrations on
$\HI$ to the Rost-D\'eglise category of homotopy modules $\HM$ 
(see Definition \ref{def_hm} below). To further simplify notation, 
in this chapter, let $L: \HI \to \HI$ denote the functor $F 
\mapsto \LHI{F}$, and let $R: \HI \to \HI$ denote the functor 
given by $F \mapsto \RHI{F}$. We write $\epsilon^n : \id \to 
R^nL^n$ and $\eta^n: L^nR^n \to \id$ for the unit and counit maps; 
we abbreviate $\eta^1$ as $\eta$, and $\epsilon^1$ as $\epsilon$.
The extension of these filtrations to $\HM$ is new.

\section{Torsion filtration on $\HM$}


Recall from \cite[1.17]{DegModHom} the following definition:

\begin{defn}\label{def_hm}
A \DEF{homotopy module} is a $\Z$-graded homotopy 
invariant sheaf with transfers $F_*$ such that for every $n$, 
there exists a map $\susp_n: \LHI{F_n} \to F_{n+1}$ such that the 
corresponding adjunction map $\deloop_n: F_n \to \RHI{(F_{n+1})}$
is an isomorphism.

We call the morphisms $\susp_n$ and $\deloop_n$ the \DEF{$n$-th 
suspensions} and the \DEF{$n$-th delooping} respectively.

Let $\HM$ denote the category of $\DEF{homotopy modules}$.
Ojects in $\HM$ will be represented by $(F_*, \deloop_*)$ where 
$F_*$ is the $\Z$-graded homotopy invariant sheaf with transfers, 
and $\deloop_*$ is the sequence of deloopings.
\end{defn}

There is a fully faithful functor $\spectHI: \HI \to \HM$ given
by $F \mapsto (F_*, \deloop_*)$ where
\[
F_n = \begin{cases}
F(n) &\textrm{if }n > 0 \\
F    &n = 0 \\
\RHI[|n|]F &\textrm{otherwise}
\end{cases}
\]
and the $n$-th delooping $F_n \to \RHI{(F_{n + 1})}$ is the unit
map for $n \geq 0$ and the identity for $n < 0$. Furthermore, 
$\spectHI$ has a right adjoint $\loopHM: \HM \to \HI$ given by 
$(F_*, \deloop_*) \mapsto F_0$ (see \cite[1.18]{DegModHom}). 
Since $\spectHI$ is fully faithful and admits a right adjoint, 
we can regard $\HI$ a full coreflective subcategory of $\HM$. The
torsion filtration on $\HI$, defined in the previous sections 
give rise to two $\N$-indexed weak filtrations of $\HM$. The goal 
is to extend these filtrations to a $\Z$-indexed strong filtration 
and a $\Z$-indexed cofiltration of $\HM$. In particular, we show
that there is a sequence of coradicals $\tlHM{n}$
on $\HM$ such that for nonnegative $n$, the restriction of 
$\tlHM{n}$ to $\HI$ is $\tlHI{n}$. In this case, the associated
torsion theories will define the extension of the torsion 
filtrations on $\HI$ to $\HM$.

The following proposition will be crucial to extending the 
functors $\tlHI{n}$:

\begin{prop}\label{prop_tl_L_R}
For $F$ in $\HI$, there are natural isomorphisms:
\[
L \tlHI{n}(F) \cong \tlHI{n + 1} L(F)
\]
and
\[
R\tlHI{n}(F) \cong \tlHI{n - 1}R(F).
\] 
\end{prop}
\begin{proof}
By Lemma \ref{lem_LRcommute}, the following diagram, natural in $F$, 
is commutative:
\begin{equation}\label{eq_Ltl_com_diag_HI}
\begin{tikzcd}[column sep=80pt]
L^{n + 1}R^{n + 1}L \arrow{r}{\cuHI^{n + 1}L} \arrow{d} &
L \arrow[equals]{d} \\
L(L^{n}R^n) \arrow{r}{L\cuHI^{n}} &
L.
\end{tikzcd}
\end{equation}
Here, $\cuHI^n$ denotes the counit $L^nR^n \to \id$,
and the vertical map $L^{n + 1}R^{n + 1}L(F) \to L(L^nR^n (F))$ is 
given by the map $L^{n + 1}R^n \eta^{-1}$, where $\eta$ is
the unit $\id \to RL$, which is an isomorphism.

The cokernel of $\cuHI^{n + 1}$ is $\tlHI{n + 1}L(F)$. Since $L$ 
is right exact, the cokernel of $L\epsilon^n$ is $L\tlHI{n}(F)$. 
By the Five Lemma, it is clear that $\tlHI{n + 1}L(F) \cong 
L\tlHI{n}(F)$. Since \eqref{eq_Ltl_com_diag_HI} is natural in $F$,
the isomorphism $\tlHI{n + 1}L \to L\tlHI{n}$ is natural as well.
By similar arguments, one can show that $R\tlHI{n}$ is naturally
isomorphic to $\tlHI{n - 1}R$ as well.
\end{proof}

We will now define the coradical on $\HM$. Let $(F_*, \deloop_*)$ 
be an object of $\HM$, and write $\tlHM{n}(F_*)$ for the graded 
homotopy invariant sheaf with transfers where
\[
(\tlHM{n}(F_*))_k \defeq
\begin{cases}
   \tlHI{n + k}(F_k) & \textrm{if } n + k > 0 \\
   0                 & \textrm{otherwise}.
\end{cases}
\]
For ease of notation, we write $\tlHM{n}(F)$ for the essential 
image of $F_*$ (dropping the ``$_*$'' on $F$ to avoid 
redundancy) and we write $\tlHM[k]{n}(F)$ for the $k$-th graded 
component of $\tlHM{n}(F)$.

By the previous proposition, there are two maps that we can
define between the graded pieces of $\tlHM{n}(F)$. Recognizing
$L\tlHM[k]{n}(F) \defeq L \tlHI{n + k}(F_k)$ and 
$\tlHM[k + 1]{n}(F) \defeq \tlHI{n + k + 1}(F_{k + 1})$, let
\[
\suspTL[k]{n} : L\tlHM[k]{n}(F) \to \tlHM[k + 1]{n}(F)
\]
denote the composition
\begin{equation}\label{eq_susp_def}
L \tlHI{n + k}(F_k) \stackrel{\cong}{\to} 
   \tlHI{n + k + 1}L(F_k) 
   \xrightarrow{\;\tlHI{n + k + 1}(\susp_k)\;} 
   \tlHI{n + k + 1}(F_{k + 1})
\end{equation}
where $\susp_k: L(F_k) \to F_{k + 1}$ is the $k$-th suspension 
map of $(F_*, \deloop_*)$. Similarly, since $\tlHM[k - 1]{n}(F) = 
\tlHI{n + k - 1}(F_{k - 1})$ and $R \tlHM[k]{n}(F) = 
R\tlHI{n + k}(F_{k})$, let
\[
\deloopTL[k]{n} : \tlHM[k - 1]{n}(F) \to R \tlHM[k]{n}(F) 
\]
denote the composition
\begin{equation}\label{eq_deloop_def}
\tlHI{n + k - 1}(F_{k - 1})
   \xrightarrow{\;\tlHI{n + k - 1}(\deloop_k)\;} 
   \tlHI{n + k - 1} R(F_k) \stackrel{\cong}{\to} 
   R \tlHI{n + k}(F_k),
\end{equation}
where $\deloop_k: F_{k - 1} \to R(F_k)$ is the $k$-th delooping
of $(F_*, \deloop_*)$. Notice that since $\deloop_k$ is an isomorphism 
for all $k$, $\deloopTL[k]{n}$ is also an isomorphism for 
all $k$. We claim that $\tlHM{n}(F)$, together with $\suspTL[k]{n}$ and
$\deloopTL[k]{n}$ as the suspension maps and delooping maps,
is an object of $\HM$, and that $\tlHM{n}$ is an endofunctor of $\HM$.
This is precisely the content of the following lemma:

\begin{lem}\label{lem_tlHM_is_functor}
$\tlHM{n}$ is an endofunctor of $\HM$.
\end{lem}

\begin{proof}
Fix $(F_*, \deloop_*)$, we show that $\tlHM{n}(F)$ is an object of
$\HM$. We have already established the graded structure of 
$\tlHM{n}(F)$. We also have candidates for the suspension and 
delooping maps; these are $\suspTL{n}$ and $\deloopTL{n}$ 
respectively. It remains to show that $\suspTL[k]{n}$ and 
$\deloopTL[k + 1]{n}$ are adjoint in the sense that $\suspTL{n}$ is 
mapped to $\deloopTL{n}$ via the adjunction 
\[
\Phi: \homHM(L \tlHM[k]{n}(F), \tlHM[k + 1]{n}(F))
\to \homHM(\tlHM[k]{n}(F), R \tlHM[k + 1]{n}(F))
\]
given that $\susp_{k + 1}$ is adjoint to $\deloop_k$.

Let $F_{k + *}$ denote a shift in grading as given 
by $(F_{k + *})_n = F_{n + k}$. Notice that $\tlHM[m]{n}(F_{k + *}) = 
\tlHM[m + k]{n - k}(F)$. Therefore, by replacing $F_*$ by 
$F_{-k + *}$, we may assume without loss of generality that 
$k = 0$. In this case, let $F = F_0$ and $G = F_1$.

To further simplify notations, let
\[
\lambda : L\tlHI{n} \to \tlHI{n + 1}L\;\;\;\textrm{ and }\;\;\;
   \rho : R\tlHI{n} \to \tlHI{n - 1}R
\]
be the natural isomorphisms given in Proposition \ref{prop_tl_L_R}. 
Let $L$ and $R$ be given as above,  we abuse notation and write $\Phi$ for the 
adjunction isomorphism
\[
\Phi: \homHI(L-, -) \to \homHI(-,R-)
\]
between any pair of objects in $\HI$.

Notice that for $s: L(F) \to G$, $\Phi(s) = \eta R(s)$.
Since $\suspTL[0]{n}$ is given by
\[
L\tlHI{n}(F) \stackrel{\lambda}{\to} \tlHI{n + 1} L(F)
   \xrightarrow{\;\tlHI{n + 1}(\susp_0)\;} \tlHI{n + 1}(G),
\]
to show that $\suspTL[0]{n}$ is adjoint to $\deloopTL[0]{n}$ is to
verify that the following diagram commute:
\[
\begin{tikzcd}
\tlHI{n}(F) \arrow{rr}{\epsilon} \arrow[equals]{dd} &&
RL\tlHI{n}(F) \arrow{rr}{R(\tlHI{n + 1}(\susp) \lambda)} 
\arrow[dotted]{dd}{\rho R(\lambda)}&&
R \tlHI{n + 1} (G) \arrow[equals]{dd} \\
& (1) && (2) \\
\tlHI{n}(F) \arrow{rr}{\tlHI{n}(\epsilon)} &&
\tlHI{n} RL(F) \arrow{rr}{\rho \tlHI{n} R(\susp)} &&
R\tlHI{n}(G)
\end{tikzcd}
\]
where composition along the top row is precisely 
$\Phi(\suspTL[0]{n})$, and the bottom is precisely 
$\deloopTL[1]{n}$. The diagram breaks up into two squares,
labelled (1) and (2), along the map $\lambda \rho: RL 
\tlHI{n}(F) \to \tlHI{n} RL(F)$. We proceed by showing that 
each square is commutative.

\pfitem{Square (1) is commutative} : Applying the naturality of
$\eta^n$ to the map $\epsilon: F \to RL(F)$, we obtain the 
following commutative diagram:
\begin{equation}\label{eq_LR_rear_face}
\begin{tikzcd}[column sep=large]
L^nR^n(F) \arrow{r}{L^nR^n\epsilon} \arrow{d}{\eta^n} &
L^nR^nRL(F) \arrow{d}{\eta^n(\epsilon)} \\
F \arrow{r}{\epsilon} &
RL(F).
\end{tikzcd}
\end{equation}
Similarly, applying the naturality of $\epsilon$ to $\eta^n$, we
have
\begin{equation}\label{eq_LR_front_face}
\begin{tikzcd}[column sep=large]
L^nR^n(F) \arrow{d}{\epsilon} \arrow{r}{\eta^n} &
F \arrow{d}{\epsilon} \\
RLL^nR^n(F) \arrow{r}{RL(\eta^n)} &
RL(F)
\end{tikzcd}
\end{equation}
which fits together in a ``cube'':
\[
\begin{tikzcd}[row sep=scriptsize, column sep=scriptsize]
& L^nR^n(F) \arrow[equals]{dl}\arrow{rr}\arrow{dd}{\eta} & & 
   L^nR^nRL(F) \arrow{dd} \\
L^nR^n(F) \arrow[crossing over]{rr}\arrow{dd}{\eta} & & RLL^nR^n(F) 
   \arrow[dotted]{ur} \\
& F \arrow[equals]{dl}\arrow{rr} & & RL(F) 
   \arrow[equals]{dl} \\
F \arrow{rr} & & RL(F) \arrow[crossing over, leftarrow]{uu}
\end{tikzcd}
\]
where the squares on the faces parallel to the page are precisely 
(from front to rear) \eqref{eq_LR_front_face} and 
\eqref{eq_LR_rear_face} respectively. Since unit maps are an 
isomorphisms, if we let the dotted arrow be given by the 
composition
\[
RLL^nR^n(F) \stackrel{\epsilon^{-1}}{\to} L^nR^n(F) 
\xrightarrow{L^nR^n(\epsilon)} L^nR^nRL(F)
\]
then all faces of the cube are commutative.

At last, notice that the Square (1) form the ``cokernel'' of the
cube and is the bottom face of the lower cube in the following:
\[
\begin{tikzcd}[row sep=scriptsize, column sep=scriptsize]
& L^nR^n(F) \arrow[equals]{dl}\arrow{rr}\arrow{dd}{\eta} & & 
   L^nR^nRL(F) \arrow{dd} \\
L^nR^n(F) \arrow[crossing over]{rr}\arrow{dd}{\eta} & & 
   RLL^nR^n(F) \arrow{ur} \\
& F \arrow[equals]{dl}\arrow{rr} \arrow{dd} & & RL(F) \arrow{dd}
   \arrow[equals]{dl} \\
F \arrow{rr} \arrow{dd} & & 
   RL(F) \arrow[crossing over, leftarrow]{uu} \\
& \tlHI{n}(F) \arrow[equals]{dl}\arrow{rr} & & 
   \tlHI{n} RL(F) \arrow[leftarrow]{dl}\\
\tlHI{n}(F) \arrow{rr} & & 
   RL\tlHI{n}(F) \arrow[crossing over, leftarrow]{uu} 
\end{tikzcd}
\]
Here, each of the vertical faces of the bottom cube is commutative.
It follows that the Square (1) must also be commutative.

\pfitem{Square (2) is commutative} : Square (2) can be further 
subdivided into:
\[
\begin{tikzcd}[row sep=huge, column sep=huge]
RL\tlHI{n}(F) \arrow{r}{R(\lambda)} \arrow{d}{\rho R(\lambda)} &
R\tlHI{n}(LF) \arrow{ld}{\rho} \arrow{r}{R\tlHI{n}(\susp)} &
R\tlHI{n + 1}(G) \arrow{ld}{\rho} \arrow[equals]{d} \\
\tlHI{n} RL(F) \arrow{r}{\tlHI{n} R(f)} &
\tlHI{n} R(G) \arrow{r}{\rho^{-1}} &
R\tlHI{n + 1}(G).
\end{tikzcd}
\]
The commutativity of the triangles in the diagram are clear. The 
parallelogram in the center is commutative by the naturality of
$\rho$ applied to $\susp: LF \to G$.

Finally, 
\pfitem{$\tlHM{n}$ is functorial} : let $f_*: (F_*,\deloop_*) \to 
(G_*, \deloop_*')$ be a map between homotopy modules. Let 
$\tlHM{n}(f)$ be a map of graded homotopy invariant sheaves with 
transfers where the map on the $k$-th associated graded is 
$\tlHM[k]{n}(f) \defeq \tlHI{n + k}(f_k)$. 

By naturality of $\rho : R\tlHI{n + 1} \to \tlHI{n} R$ and 
$\lambda: L\tlHI{n} \to \tlHI{n + 1}L$ and also by the above 
arguments, the following two squares are commutative
\[
\begin{tikzcd}[column sep=12em, row sep=huge]
R\tlHI{n + k}(F_k) \arrow{r}{R \tlHI{n + k}(f_k)} 
   \arrow{d}{\tlHI{n + k - 1}R(\deloop) \rho} &
R\tlHI{n + k}(G_k) 
   \arrow{d}{\tlHI{n + k - 1}R(\deloop') \rho} \\
\tlHI{n + k - 1}(F_{k - 1}) 
   \arrow{r}{\tlHI{n + k - 1}(f_{k - 1})} &
\tlHI{n + k - 1}(G_{k - 1}) 
\end{tikzcd}
\]
\[
\begin{tikzcd}[column sep=12em, row sep=huge]
L\tlHI{n + k}(F_k) \arrow{r}{L \tlHI{n + k}(f_k)} 
   \arrow{d}{\tlHI{n + k + 1}L(\susp) \rho} &
L\tlHI{n + k}(G_k) 
   \arrow{d}{\tlHI{n + k + 1}L(\susp') \rho} \\
\tlHI{n + k + 1}(F_{k + 1}) 
   \arrow{r}{\tlHI{n + k + 1}(f_{k - 1})} &
\tlHI{n + k + 1}(G_{k + 1}).
\end{tikzcd}
\]

It is clear that $\tlHM{n}(f)$ is a map from $\tlHM{n}(F)$ to 
$\tlHM{n}(G)$ as homotopy modules. The fact that $\tlHM{n}$ 
respects composition follows from the functoriality of $\tlHI{*}$.
The lemma is established.
\end{proof}

We now verify the main result of this section:

\begin{thm}\label{thm_tlHM_corad}
For each integer $n$, $\tlHM{n}$ is a coradical of $\HM$.
\end{thm}

\begin{proof}
\pfitem{$\tlHM{n}$ is a quotient functor} : certainly $F_* \to 
\tlHM{n}(F)$ is surjective for each $n$ since it is a surjection 
at each degree. What we need to verify is that the degree-wise 
surjection gives rise to a map of homotopy modules. In particular, 
we need to verify that the following diagram is commutative
\[
\begin{tikzcd}
L(F_k) \arrow{r}{\susp} \arrow{d} &
F_{k + 1} \arrow{d} \\
L \tlHI{n + k}(F_k) \arrow{r}{\susp} &
\tlHI{n + k - 1}(F_{k + 1}).
\end{tikzcd}
\]

To see this, notice that the above diagram fits as the outer 
square of the following diagram:
\[
\begin{tikzcd}
L(F_k) \arrow{r}{\susp} \arrow{d} &
F_{k + 1} \arrow{d} \\
\tlHI{n + k + 1} L(F_k) \arrow{r} \arrow{d}{\lambda^{-1}} &
\tlHI{n + k + 1} F_{k + 1} \arrow[equals]{d} \\
L \tlHI{n + k + 1}(F_k) \arrow{r}{\susp} &
\tlHI{n + k + 1} F_{k + 1}.
\end{tikzcd}
\]
Here, the top square commutes by the naturality of $\id \to 
\tlHI{n + k}$, and the bottom square commutes by definition. 
Indeed, the definition of the suspension map 
$\susp: L \tlHI{n + k}(F_k) \to \tlHI{n + k + 1}(F_{k + 1})$ is 
precisely given by the composition 
$\lambda \tlHI{n + k + 1}(\susp)$.

The fact that $\tlHM{n}$ respects delooping follows precisely by
the duality of the suspension and delooping as established by the
preceding lemma.

\pfitem{$\tlHM{n}$ is a pre-coradical} : The kernel of $F_* \to 
\tlHM{n}(F)$ is a homotopy module $K_*$ whose $k$-th graded term is 
\[
\mathrm{ker}(F_k \to \tlHI{n + k}(F_k)). 
\]
But $\tlHI{n}$ is a coradical; hence, $\tlHM{n}(K_*) = 
\tlHI{n + k}(K_k) = 0$. That is $\tlHM{n}(K) = 0$, as desired.

\pfitem{$\tlHM{n}$ is a right exact} : since $\tlHI{n + k}$ is 
right exact for each $k$, $\tlHM{n}$ for each associated graded 
term. It follows that $\tlHM{n}$ is right exact.
\end{proof}

The following corollary is a straightforward consequence of 
Theorems \ref{thm_corad_equiv_htt} and \ref{thm_tlHM_corad}.

\begin{cor}\label{cor_tor_filt_on_HM}
There exists a $\Z$-indexed torsion filtration on $\HM$. That 
is, there exists a $\Z$-indexed sequence of coradicals $\tlHM{i}$
such that the associated torsion subcategories, which are given
by $\Cat{F}_i = \TFHM{i}$
form an ascending strong cofiltration of $\HM$:
\[
\cdots \subseteq \TFHM{-1} \subseteq \TFHM{1} \subseteq \cdots 
   \subseteq \TFHM{i} \subseteq \TFHM{i + 1} \subseteq \cdots 
   \subseteq \HM
\]
and the associated torsionfree subcategories $\Cat{T}_i = \THM{i}$
form a descending strong filtration of $\HM$:
\[
\cdots \subseteq \THM{i} \subseteq \THM{i - 1} \subseteq 
   \cdots \subseteq \THM{0} \subseteq \THM{-1} \subseteq \cdots
   \cdots \subseteq \HM.
\]
\end{cor}

By Corollary \ref{cor_tt_ref_and_coref}, the coreflection functors 
$\tgHM{n} : \HM \to \THM{n}$ are defined by sending $(F_*, 
\deloop_*)$ in $\HM$ to the kernel of the unit map 
$(F_*, \deloop_*) \to \tlHM{n}(F)$. Write $\tgHM{n}(F)$ for this 
kernel. Therefore, we obtain another characterization of the objects 
in $\THM{n}$, which follows easily from Theorem
\ref{thm_precorad_eq_tt}:

\begin{cor}\label{cor_tgHM_prop}
An object $(F_*, \deloop_*)$ is in $\THM{n}$ if and only
if $\tgHM{n}(F) = (F_*, \deloop_*)$.
\end{cor}

We conclude this section with two immediate applications of the
results above. The first is to show that there is a torsion 
filtration structure on the category of cycle modules (defined 
below) and the second is to describe the behavior of the torsion
filtration with respect to the tensor structure on both $\HI$
and $\HM$.

We first apply Corollary \ref{cor_tor_filt_on_HM} directly to the
category of cycle modules. Recall from \cite{Rost96} the 
definition of a cycle module:

\begin{defn}\label{def_pre_cycmod}
Let $X$ be an arbitrary finite-type $k$-scheme, and let 
$\fields(X)$ be the class of all fields $F$ such that there
exists a map $\Spec E \to X$, and that $E$ is finitely generated
over $k$.

A \emph{cycle premodule} on $\fields(X)$ is an object function
that associated to each $E$ in $\fields(X)$ a $\Z$-graded abelian
group $M(E) = \prod_i M_i(E)$ with the following data:

\begin{enumerate}
\item[\textbf{D1.}] For each field extension $\phi: E' \to E$, 
there is a degree 0 map $\phi_*: M(E') \to M(E)$ called the 
\emph{restriction map associated to $\phi$}

\item[\textbf{D2.}] For each finite extension $\phi: E' \to E$, 
there is a degree 0 map $\phi^*: M(E) \to M(E')$ called the 
\emph{corestriction map associated to $\phi$}

\item[\textbf{D3.}] For each $E$ in $\fields(X)$, the group $M(E)$ 
is equipped with a left $\milK_*(E)$-module, where $\milK_*(E)$ is 
the Milnor $K$-ring of $E$.

\item[\textbf{D4.}] For any valuation $v$ of $E$ in $\fields(X)$, 
there exists maps $\res{v}: M(E) \to M(\resf{v})$ and 
$\Special{v}{\ideal{p}}$ called the \emph{residue} and 
\emph{specialization} respectively, where $\resf{v}$ is the 
residue field of $v$ and $\ideal{p}$ is a prime of $v$,
\end{enumerate}

which satisfy the following conditions:

\begin{enumerate}
\item[\textbf{R1a.}] For each field extension $\phi: E' \to E$ and 
field extension $\psi: E \to E''$, $(\psi \comp \phi)_* = \psi_* 
\comp \phi_*$

\item[\textbf{R1b.}] For each finite extension $\phi: E' \to E$ and 
finite extension $\psi: E \to E''$, $(\psi \comp \phi)^* = \phi^* 
\comp \psi^*$

\item[\textbf{R1c.}] For finite extension $\phi: E' \to E$ and any 
field extension $\psi: E' \to E''$ with $\phi$ finite, define $R = E 
\otimes_{E'} E''$, and let $\ideal{p}$ be any prime ideal of $R$. (As 
$R$ is Artin, let $l_p$ be the length of the local ring 
$R_{(\ideal{p})}$), and $\phi_{\ideal{p}}: E'' \to R/{\ideal{p}}$ 
and $\psi_{\ideal{p}}: E \to R/{\ideal{p}}$ be natural maps.
\[
\psi_*\comp \phi^* = \sum_{\ideal{p}} l_p \cdot 
(\phi_{\ideal{p}})^* \comp (\psi_{\ideal{p}})_*.
\]

\item[\textbf{R2.}] For any extension $\phi: E' \to E$, $x \in 
\milK_*(E')$, $y \in \milK_*(E)$, $\rho \in M(E')$, and $\mu \in 
M(E)$, then:

\item[\textbf{R2a.}] $\phi_*(x \cdot \rho) = \phi_*(x) \cdot 
\phi_*(\rho)$.

\item[\textbf{R2b.}] if $\phi$ is finite, $\phi^*(\phi_*(x) \cdot 
\mu) = x \cdot \phi^*(\mu)$.

\item[\textbf{R2c.}] if $\phi$ is finite, $\phi^*(y \cdot 
\phi_*(\rho)) = \phi^*(y) \cdot \rho$.

\item[\textbf{R3.}] For any field extension $\phi: E' \to E$, $v$ 
a valuation on $E$ and $w$ and a valuation on $E'$:

\item[\textbf{R3a.}] Suppose $w$ is a nontrivial restriction of 
$v$ with ramification index $e$. Let $\phib: \resf{w} \to 
\resf{v}$ be the induced map. Then:
\[
\res{v} \comp \phi_* = e \cdot \phib_* \comp \res{w}.
\]

\item[\textbf{R3b.}] Let $\phi$ be a finite extension, suppose
$w$ is an extension of $v$ to $E$. Let $\phi_v: \resf{w} \to 
\resf{v}$ be the induced map on the residue fields. Then
\[
\res{v} \comp \phi^* \sum_{v} \comp \res{v}.
\]

\item[\textbf{R3c.}] Suppose $v$ restricts to a trivial valuation 
on $E'$. Then
\[
\res{v} \comp \phi_* = 0
\]

\item[\textbf{R3d.}] Suppose $v$ restricts to a trivial valuation 
on $E'$, and let $\phib: F \to \resf{v}$ be the induced map on 
the residue fields. Let $\ideal{p}$ a prime of $v$. Then
\[
\Special{v}{\ideal{p}} \comp \phi_* = \phib_*
\]

\item[\textbf{R3e.}] Let $u$ be an element of $E$ such that 
$v(u) = 0$. Given $\rho$ in $M(F)$, one has
\[
\res{v}(\{u\} \cdot \rho) = -\{\overline{u}\} \cdot \res{v}(\rho).
\]
\end{enumerate}
\end{defn}

For $X$ a $k$-scheme, let $\subsch{1}{X}$ denote the collection of 
codimension 1 subscheme. Write $\xi_X$ be the generic point of an
irreducible $X$ with $K_X = \O_X,\xi_X$. If $X$ is normal, then 
for $x$ in $\subsch{1}{X}$, the local ring $\O_{x,X}$ is a valuation 
ring of $K_X$ with residue field $\resf{x}$. Write $M(x)$ for 
$M(\resf{x})$, and $\res{x}: M(\xi_X) \to M(x)$ for the 
restriction map.

Furthermore, for $x, y \in X$, let $Z$ be the closed subscheme 
determined by $x$, and $\overline{Z}$ be the normalization $Z$.
Define
\[
\ptres{x}{y}: M(x) \to M(y)
\]
by
\[
\ptres{x}{y} = 
\begin{cases}
0 & y \notin \subsch{1}{Z} \\
\sum_{z|y} \phi_{\resf{z},\resf{x}}^* \comp \res{z} & \textrm{otherwise}.
\end{cases}
\]
Here, following \cite{Rost96}, $z|y$ denotes the relation that $z$ 
lies over $y$. In particular, if $y \in \subsch{1}{Z}$, the sum 
is taken over all $z$ lying over $y \in \subsch{1}{Z}$. In this
case, $\phi_{\resf{z},\resf{y}}^*$ is the corestriction map 
associated to the finite field extension $\resf{y} \to \resf{z}$.

\begin{defn}
A cycle module $M$ on $\fields(X)$ is a cycle premodule which
satisfies the following conditions:

\begin{enumerate}
\item[\textbf{(FD)}] \itemhead{Finite support of divisors.} 
$X$ be a normal scheme and $\rho \in M(\xi_X)$. Then $\res{x}: 
M(\xi_X) \to M(X)$ is 0 for all but finitely many $x \in 
\subsch{1}{X}$.

\item[\textbf{(C)}] \itemhead{Closedness.} If $X$ is an integral
local local of dimension 2 with closed point $x_0$, then the map 
from $M(\xi_X)$ to $M(x_0)$ given by
\[
\sum_{x \in \subsch{1}{X}} \ptres{x_0}{x} \comp \ptres{x}{\xi}
\]
is 0.
\end{enumerate}
\end{defn}

D\'eglise showed in \cite{DegModHom} that a homotopy module $(F_*, 
\deloop_*)$ gives rise to a unique cycle module $\assocCM{F_*}$,
and that this association defines an equivalence between the 
category of homotopy modules and cycle modules (see 
\cite[3.7]{DegModHom}).

Via this categorical equivalence, we obtain the following
corollary: 
\begin{cor}\label{cor_tor_filt_on_CycMod}
There exists a $\Z$-indexed torsion filtration on $\CycMod$. That
is, there exists a $\Z$-indexed sequence of coradicals, which by abuse
of notation, we also represent by $\tlHM{i}$ such that the 
associated torsion subcategories $\TFCycMod{i}$ form an ascending 
strong cofiltration of $\CycMod$:
\[
\cdots \subseteq \TFCycMod{-1} \subseteq \TFCycMod{0} \subseteq \cdots 
   \subseteq \TFCycMod{i} \subseteq \cdots 
   \subseteq \CycMod
\]
and the associated torsionfree subcategories $\TCycMod{i}$
form a descending strong filtration of $\CycMod$:
\[
\cdots \subseteq \TCycMod{i} \subseteq 
   \cdots \subseteq \TCycMod{0} \subseteq \TCycMod{-1} \subseteq \cdots
   \cdots \subseteq \CycMod.
\]
\end{cor}

\begin{ex}\label{ex_milK}
A well-known example of a cycle module is Milnor $K$-theory,
$\milK_*$ (see \cite[1.4, 2.5]{Rost96}). Recall from \cite{MilK} that 
for a field $F$, the Milnor $K$-theory of $F$ is the graded 
commutative ring given by
\[
\milK_*(F) \defeq T^*(F^*)/I
\]
where $T^*(F^*)$ denotes the tensor algebra of the multiplicative 
group $F^*$, and $I$ denotes the ideal generated by $a \tensor 
(1 - a)$ for all $a$ in $F^*$. The $n$-th Milnor $K$-theory of $F$ 
is the $n$-th graded piece of $\milK_*(F)$.

Via \cite[3.7]{DegModHom}, the homotopy module corresponding to 
$\milK_*$ is $(\Ox)^{\tensor n}$. Therefore, $\milK_n \in
\TFCycMod{n} \cap \TCycMod{n}$.
\end{ex}

We now discuss the tensorial properties of the filtration. Let us
first consider the following notion:

\begin{defn}\label{def_graded_tensor}
Let $(\Cat{C}, \tensor, \Unit)$ be a monoidal category.
We say that $\Cat{C}$ is a \DEF{graded monoidal category by
a weak filtration $(\Cat{C}_*, \phi_*)$} of $\Cat{C}$ if for all 
integers $n$ and $m$, $\Cat{C}_n \tensor \Cat{C}_m \subseteq
\Cat{C}_{n + m}$.
\end{defn}

\begin{ex}
By construction, it is clear that $(\HI(n), \sgHI{*})$ 
defines a graded symmetric monoidal category on $\HI$ under
$\tHI$.
\end{ex}

We first show that $(\THI{*}, \tgHI{*})$ define a graded
monoidal category on $\HI$. We begin by proving the following
proposition:

\begin{prop}\label{prop_tensor_and_tfilt_HI}
For $F$ in $\THI{n}$ and $G$ in $\THI{m}$, $F \tHI G$
is an object of $\THI{n + m}$.
\end{prop}
\begin{proof}
Since $F$ is in $\THI{n}$, the counit $\epsilon: L^nR^n(F) \to F$ 
is surjective (Proposition \ref{prop_THI_properties} (1)), and since 
the functor $- \tHI G$ is right exact, the map
\[
L^nR^n(F) \tHI G \xrightarrow{\epsilon_F \tHI G} F \tHI G.
\]
is also surjective. Similarly,
\[
L^nR^n(F) \tHI L^mR^m(G) \xrightarrow{L^nR^n(F) \tHI \epsilon_G}
L^nR^n(F) \tHI G
\]
is also surjective. Composing these two surjection, we have a 
surjection
\[
L^nR^n(F) \tHI L^mR^m(G) \to F \tHI G.
\]
But 
\[
L^nR^n(F) \tHI L^mR^m(G) = L^{n + m}(R^n(F) \tHI R^m(G)) \in
\sgHI{n + m}\HI,
\]
which is an object in $\HI(n + m)$. Therefore, 
$\tlHI{n + m}(L^nR^n(F) \tHI L^mR^m(G)) = 0$. However, the functor 
$\tlHI{n + m}$ is right exact. In particular,
\[
\tlHI{n + m}(L^nR^n(F) \tHI L^mR^m(G)) \to
   \tlHI{n + m}(F \tHI G)
\]
is onto. It follows that $\tlHI{n + m}(F \tHI G) = 0$ and
$F \tHI G$ is an object in $\THI{n + m}$, as desired.
\end{proof}

The following is a direct consequence of the proposition above:

\begin{cor}\label{cor_graded_tensor_HI}
There exists a graded symmetric monoidal structure on $\HI$ by
$(\THI{*}, \tgHI{*})$.
\end{cor}

We now show that the graded symmetric monoidal structure on $\HI$
can be extended to a graded symmetric monoidal structure on $\HM$.
For the following, by abuse of notation, let $F_*$ denote the 
object $(F_*, \deloop_*)$ in $\HM$.

\begin{prop}\label{prop_graded_mon_struct_HM}
Let $n$ and $m$ be arbitrary integers, and fix $F_*$ in $\THM{n}$ 
and $G_*$ in $\THM{m}$. Then, $F_* \tHI G_*$ is an object of 
$\tgHM{n + m}$.
\end{prop}
\begin{proof}
By Corollary \ref{cor_tgHM_prop}, it suffices to show that $\tgHM{n + m}(F_*
\tHI G_*) = F_* \tHI G_*$.

Regard the objects of $\HM$ as pairs $(F, k)$, where $F \in \HI$
and $k$ is an arbitrary integer. It is straightforward to 
verify from the definition of $\tgHM{*}$ and Proposition \ref{prop_tl_L_R} 
that
\[
\tgHM{n}((F, k)) = \begin{cases}
(\tgHI{n - k}F, k) &\textrm{if }n > k\\
(F, k)             &\textrm{otherwise}.
\end{cases}
\]
Therefore, we are reduced to showing that for $F$ and $G$ in
$\THI{n}$ and $\THI{n}$ respectively, 
$\tgHI{n + m}(F \tHI G) = F \tHI G$. This follows directly from 
Corollary \ref{cor_graded_tensor_HI}.
\end{proof}

Finally, we obtain the following corollary as a direct consequence
of the proposition above:

\begin{cor}\label{cor_graded_mon_struct_HM}
There exists a graded symmetric monoidal structure on $\HM$ by
$(\THM{*}, \tgHM{*})$.
\end{cor}

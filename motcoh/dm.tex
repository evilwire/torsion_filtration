\newpage
\section{The Derived Category of Motives}\label{sect_dmeff_and_dm}

In this section, we define the derived category of motives $\DMeff$,
and show that it is equipped with an additive symmetric monoidal 
structure with a ``partial'' internal hom that will be used to 
construct the slice filtration in Section 
\ref{sect_slice_filt_dm}.

To do this, we first define the bounded above derived category 
$\DCat^-\ShCor$ of Nisnevich sheaves, and define $\DMeff$ to be
the localization of $\DCat^-\ShCor$ by a class of morphisms in
$\DCat^-\ShCor$ called $\A^1$-weak equivalence. We then show that
$\DMeff$ is in fact equivalent as a category to the subcategory
of $\DCat^-\ShCor$ with homotopy invariant cohomology. 

We also show that $\DCat^-\PST$ is equipped with tensor and 
internal hom operations on which induce a symmetric monoidal 
structure on $\DMeff$. For the remainder of the section, unless 
stated otherwise, all sheaves are Nisnevich sheaves. We will drop 
the ``Nis'', and simply write $\ShCor$ for the category of 
Nisnevich sheaves with transfers.

First consider the category $\PST$. By Yoneda, for $X \in \Sm$ and
$F \in \PST$,
\[
\hom_{\PST}(\Ztr(X), F) = F(X).
\]
It follows that $\Ztr(X)$ is projective for every $X \in \Sm$.
Since direct sums of projectives are projective, 
$\oplus_i \Ztr(X_i)$ is also projective for any arbitrary 
collection $\{Z_i\}$. Furthermore, for $F \in \PST$, there exists 
a surjection
\begin{equation}\label{eq_res_by_rep}
\bigoplus_{X} \bigoplus_{0 \neq x \in F(X)} \Ztr(X) 
   \stackrel{x}{\to} F.
\end{equation}
Hence, the category $\PST$ has enough projectives. Thus, we may 
define the bounded above derived category $\DCat[D]^-\PST$ of the 
abelian category $\PST$ as the homotopy category of chain 
complexes of projective object in $\PST$ (see 
\cite[Thm. 10.4.8]{WH}).

Similarly, we define the bounded derived category $\DCat^- \ShCor$
of Nisnevich sheaves with transfers. In particular, since $\ShCor$
has enough injectives (\cite[Thm. 13.1]{MVW}), again by 
\cite[Thm. 10.4.8]{WH}, we can define $\DCat^-\ShCor$ as the homotopy
category $\DCat[K]^-$ of bounded above complex of injective objects
in $\ShCor$. 

To simplify notations, for the remainder, we write $\DShCor$ for 
the derived category $\DCat^-\ShCor$ of Nisnevich sheaves with 
transfers. We now define $\DMeff$, the derived category of 
motives. To do so, we first need a notion of a thick subcategory.

\begin{defn}
A full additive subcategory $\Cat{W}$ of a derived category 
$\DCat$ is \DEF{thick} if it satisfies the following conditions:
\begin{enumerate}
\item if $A \to B \to C \to A[1]$ is a distinguished triangle, then
any two of $A, B, C$ is in $\Cat{W}$, then so is the third.

\item if $A$ and $B$ are objects in $\Cat{W}$, then $A \oplus B \in
\Cat{W}$.
\end{enumerate}
\end{defn}

If $\Cat{W}$ is a thick subcategory of a derived category $\DCat$,
then we can define a quaotient triangulated category 
$\DCat/\Cat{W}$. Let $\Cat{S}$ be the set of maps whose cone is in 
$\Cat{W}$. Then $\Cat{S}$ is a saturated multiplicative system in 
the sense that $\Cat{S}$ contains the identity, is closed under 
composition, and if $fg \in \Cat{S}$, then $f$ and $g$ are both in 
$\Cat{S}$. Define $\DCat/\Cat{W}$ to be the localization 
$\DCat[D] [\Cat{S}^{-1}]$.

\begin{defn}\label{def_DMeff}
Let $\Cat{W}_{\A}$ be the thick subcategory of $\DCat^-$ generated 
by the cones of $\Ztr(X \times \A^1) \to \Ztr(X)$ for every $X \in 
\Sm$, closed under direct sums that exist in $\DCat^-\ShCor$. 
Write $\Cat{S}_{\A}$ for the maps whose cone is in $\Cat{W}_{\A}$. 
We say $f \in \DShCor$ is an \DEF{$\A^1$-weak equivalence} if $f 
\in \Cat{S}_{\A}$.

We write $\DMeff$ for the localization 
$\DShCor [\Cat{S}_{\A}^{-1}]$. The category that we have
just defined is the derived category of motives.
\end{defn}

While we have defined $\DMeff$ as a localization of 
$\DCat^-$ by the $\A^1$-weak equivalences, we can identify
$\DMeff$ with a subcategory of $\DShCor$.

\begin{defn}
Let $F^* \in \DShCor$. We say that $F^*$ is \DEF{$\A^1$-local} if
$\hom_{\DShCor}(-, F^*)$ sends $\A^1$-weak equivalences to 
isomorphisms. We write $\ALocal$ for the full subcategory of 
$\A^1$-local objects in $\DShCor$.
\end{defn}

The following give a good characterization of the category 
$\ALocal$.

\begin{prop}[\cite{MVW} Prop. 14.8, Cor. 14.9]
Fix $F^*$ in $\DShCor$. Then $F^* \in \ALocal$ if and only if
$\nis(H^n F^*$ are homotopy invariant. In particular, we can
identify $\ALocal$ with the full subcategory of complexes in 
$\DShCor$ with homotopy invariant cohomology presheaves, which we 
represent by $\DCat^-_{\HI} \ShCor$.
\end{prop}

% define suslin C^*
\begin{defn}
For $F^*$ a bounded above complex of sheaves with transfers,
let $\suslC F^*$ denote the double complex $(\suslC F^*)^{-p, q} = 
\suslC[p] F^q$. The total complex $\Tot G^{*,*}$ is an object of 
$\DShCor$, which we represent by as $\suslC[] F^*$. 
\end{defn}

We make the following observations: since $F^*$ is bounded above,
by shifting sufficiently, we may assume that $F^*$ is concentrated
in strictly nonnegative degrees. Therefore, indexing the double
complex cohomologically, $\suslC F^*$ is third quadrant double 
complex. Filtering the total complex $\suslC[] F^*$ by row, we 
obtain a third quadrant spectral sequence converging to the 
cohomologies of $\suslC[] F^*$. Note by Lemma 
\ref{prop_susl_hom_is_hi} that the terms in the first page of the 
spectral sequence are all homotopy invariant. It follows that
$\suslC[] F \in \ALocal$.

The significance of the construction defined above is the 
following:

\begin{prop}
The functor $\suslC : \DShCor \to \ALocal$ is a left adjoint to
the inclusion of $\ALocal \into \DShCor$.
\end{prop}

\begin{proof}
There is a canonical map from $F^* \to \suslC[] F^*$,
given by the inclusion of $F_i = \suslC[0] F_i \into \oplus_{p + 
q = i} \suslC[-p] F^q$. This map is a $\A^1$-weak equivalence
(see \cite[Lemma 14.4]{MVW}). Therefore, for any $L^* \in \ALocal$
and $F^* \in \DShCor$,
\[
\hom_{\DShCor}(F^*, L^*) \cong \hom_{\DShCor}(\suslC[] F, L^*)
   = \hom_{\ALocal}(\suslC[] F^*, L^*).
\]
\end{proof}

There is a canonical functor $\pi: \DShCor \to \DMeff$, given
by sending an object of $\DShCor$ to its corresponding object in
$\DMeff$. Its restriction to $\ALocal$ defines a functor from
$\ALocal$ to $\DMeff$.

Furthermore, we can define a map from $\DMeff$ to $\ALocal$. 
Notice that if $F^*$ and ${F'}^*$ are $\A^1$-weak equivalent, then
transitivity implies that $\suslC[] F^*$ is $\A^1$-weak equivalent 
to $\suslC[] F^*$. It follows that the functor that sends $F^*$
to $\suslC[] F^*$ lifts to a functor from $\DMeff \to \ALocal$.
Let $\suslC$ denote the induced functor on $\DMeff$.

\begin{thm}
The functor $\pi: \ALocal \to \DMeff$ is an equivalence of 
category, with an quasi-inverse $\suslC$.
\end{thm}

\begin{proof}
The fact that $\pi$ is an equivalence is established in
\cite[Theorem 14.11]{MVW}. Furthermore, every $F^*$ in
$\DMeff$ is isomorphic (in $\DMeff$) to $\suslC[] F^*$, which
is in the essential image of $\pi$. For the second statement,
it suffices at this point to show that $\suslC \pi$ is naturally
isomorphic to the identity on $\ALocal$. This follows from the
fact that if $F^*$ is $\A^1$-local, then $\suslC[] F^*$ is 
isomorphic to $F^*$ (see \cite[Corollary 14.9]{MVW}).
\end{proof}

Recall from \cite[1.13]{MK} and \cite[8A.1]{MVW} the notion of a 
symmetric closed monoidal structure generalized to the setting of 
a triangulated category:

\begin{defn}
Let $\DCat[D]$ be a triangulated category. We say that $\DCat[D]$ 
is a \DEF{tensor triangulated category} if there exists a bifunctor
$- \tensor -$, together with two natural isomorphisms 
\[
\begin{tikzcd}
(M[1]) \tensor N \arrow{r}{l_{M, N}}&
(M \tensor N)[1] &
M \tensor (N[1]) \arrow{l}{r_{M, N}}
\end{tikzcd}
\]
such that $(\DCat[D], \tensor)$ satisfies the axioms of a 
symmetric monoidal category, and the following two conditions
hold
\begin{enumerate}
\item For any distinguished triangle $M' \to M \to M'' 
\stackrel{\delta}{\to} M'[1]$, and any $N \in \DCat[D]$,
the following triangles are distinguished
\[
\begin{tikzcd}
M' \tensor N \arrow{r} &
M \tensor N \arrow{r} &
M'' \tensor N \arrow{r}{l(\delta \tensor D)}&
(M' \tensor N)[1] \\
N \tensor M' \arrow{r} &
N \tensor M \arrow{r} &
N \tensor M'' \arrow{r}{r(D \tensor \delta)}&
(N \tensor M')[1]
\end{tikzcd}
\]

\item For any $M, N \in \DCat[D]$, the following anti-commutes,
i.e. $rl = -lr$:
\[
\begin{tikzcd}
M[1] \tensor N[1] \arrow{rr}{r}\arrow{dd}{l} && 
(M[1] \tensor N)[1] \arrow{dd}{l} \\
& -1 \\
(M \tensor D[1])[1] \arrow{rr}{r} &&
(M \tensor D)[2].
\end{tikzcd}
\]
\end{enumerate}

We say that $(\DCat, \tensor)$ is an \DEF{additive symmetric 
monoidal category} if 
\[
\left(\bigoplus M_i\right) \tensor N = \bigoplus_i (M_i \tensor N)
\]
for all $N \in \DCat$ and all family $\{M_i\}$ of objects of 
$\DCat$ such that $\oplus_i M_i \in \DCat$. Finally, we say that
$\DCat$ is \DEF{closed monoidal} if there exists an internal hom
bifunctor $\ihom(-,-)$ such that $F \tensor -$ is left adjoint to
$\ihom(-, F)$.
\end{defn}

For the remainder of this section, we show that $\DMeff$ is 
equipped with an additive symmetric monoidal structure with a
partially defined internal hom operation.

We show that $\DMeff$ is equipped with an additive closed 
symmetric monoidal structure. To do so, we first define the tensor 
and internal hom operators on $\PST$. The tensor structure will be 
determined by the following requirements: 

\begin{enumerate}
\item $\Ztr(X) \ttr \Ztr(Y) \defeq \Ztr(X \times Y)$, 

\item for each map $\phi \in \hom_{\PST}(\Ztr(X), \Ztr(Y)) = 
\Cor(Y, X)$ identified with the correspondence $W$, $\phi \tensor 
\Ztr(Z): \Ztr(X) \ttr \Ztr(Z) \to \Ztr(Y) \ttr \Ztr(Z)$ is the
correspondence $W \times Z$. 
\end{enumerate}

First, it is clear that we can extend $\ttr$ to arbitrary 
direct sums of representable presheaves.  

Next, for arbitrary $F, G \in \PST$, let $P^* \to F$ and $Q^* \to G$
be a resolutions by direct sums of representable functors of $F$ 
and $G$ respectively. We write $F \tL G$ for the total complex of 
the double complex $P^* \ttr Q^*$. By the Comparison Theorem 
\cite[Theorem 2.26]{WH}, any two projective resolutions are chain 
homotopy equivalent. Therefore, it is easy to see that up to chain 
homotopy equivalence, this is independent of the choice of $P^*$ 
and $Q^*$.

In particular, $H^0(F \tL G)$ is well-defined. Define the tensor 
operation on $\PST$ to be
\[
F \tensor G \defeq F \tL G,
\]
and define internal hom by
\[
\ihom(F, G): X \mapsto \hom_{\PST}(F \tensor \Ztr(X), G).
\]
These operations define a closed monoidal structure on $\PST$.
That is, for all $F \in \PST$, the functor $F \tL -$ is adjoint to 
$\ihom(-, F)$ (see \cite[Lemma 8.3]{MVW}).

\begin{rmk}
Take note that the $\tensor$ structure defined is \emph{not} the
usual tensor product on presheaves of abelian groups. In 
particular, $\Ztr(X)(Z) \tensor_{\Z} \Ztr(Y)(Z) \neq 
\Ztr(X \times Y)(Z)$, where $\tensor_{\Z}$ denotes the usual 
tensor product of abelian groups.
\end{rmk}

We now extend $\tensor$ to $\DCat[D]^-\PST$. To do so, let $F^*$ 
represent a bounded above complex of presheaves with transfers.
By \cite[10.5.6]{WH}, $F^*$ is quasi-isomorphic to a projective 
complex $P^*$. In fact, we may assume that $P^*$ is a complex such 
that $P^i$ is a direct sum of representable presheaves. We call 
$P^* \to F^*$ \DEF{a projective resolution (by representable 
presheaves)}.

Define $F^* \tL G^*$ to be the total complex associated with
$P^* \tensor Q^*$, where $P^*$ and $Q^*$ are projective 
resolutions of $F^*$ and $G^*$ respectively. Once again, up to
quasi-isomorphism, $F^* \tL G^*$ is defined independent of the 
choice of projective resolutions, and is itself a bounded above 
complex. 

In particular, $\tL$ is well-defined as a bi-functor on 
$\DCat[D]^-\PST$. Indeed, let $F^*$ and ${F'}^*,$ be two 
quasi-isomorphic bounded above complexes in $\PST$. Then for any 
bounded above $G^*$ in $\PST$, $F^* \tL G^* \cong {F'}^* \tL G^*$. 
(see \cite[Lemma 8.7]{MVW}) To show that $\DCat^-\PST$ is equipped
with a tensor triangulated structure, we make the following 
observation.

Let $\Cor^{\oplus}$ denote the closure under arbitrary direct 
sum of representable presheaves in $\PST$. This is an additive
category equipped with an additive symmetric monoidal structure.
By \cite[Prop. 8A.4]{MVW}, we see that the homotopy category 
$\DCat[K]^-(\Cor^{\oplus})$ is a tensor triangulated categegory.
By similar arguments as in \cite[10.4.8]{WH}, $\DCat[D]^-\PST$
is equivalent as a category to $\DCat[K]^-(\Cor^{\oplus})$.
It follows that $\DCat[D]^-\PST$ is also a tensor triangulated 
category.

Next, we show that the tensor structure is preserved under
Nisnevich sheafification.

\begin{defn}\label{def_shcor_tensor}
Let $F, G$ be Nisnevich sheaves with transfers. Define
$F \ttrNis G$ to be the Nisnevich sheafification of the presheaf
$F \ttr G$. That is,
\[
F \ttrNis G \defeq \nis(F \ttr G)
\]
where $\nis$ is the Nisnevich sheafification. We can extend 
$\ttrNis$ to chain complexes of Nisnevich sheaves. Let $F^*$ and 
$G^*$ be a chain complex of Nisnevich sheaves with transfers. 
Define $F^* \tLNis G^*$ to be the Nisnevich sheafification of the 
complex $F^* \tL G^*$:
\[
F^* \tLNis G^* \defeq \nis(F^* \tL G^*).
\]
This is well-defined up to quasi-isomorphism.
\end{defn}

\begin{rmk}
Fix $F^*$ and $G^*$ sheaves with transfers, and let $P^*$ and 
$Q^*$ be resolutions by representables of $F^*$ and $G^*$ 
respectively. Since $\nis$ is exact, $\nis(F^* \tL G^*) = 
\nis(\Tot(P^* \ttr Q^*)) = \Tot(P^* \ttrNis Q^*)$.
\end{rmk}

We claim that $(\DShCor, \tLNis)$ is an additive symmetric 
monoidal triangulated category. The proof depends on the following
lemma.

\begin{lem}[\cite{MVW} Prop. 8A.7]\label{lem_loc_and_tensor}
Let $\DCat$ be a tensor triangulated category, and let $\Cat{W}$
be a collection of maps in $\DCat$ that is closed under $-\tensor 
N$ for every $N \in \DCat$ i.e. if $M \to M'$ is in $\Cat{W}$ then 
so is $M \tensor N \to M' \tensor N$. Then the localization $\DCat 
[\Cat{W}^{-1}]$ is also a tensor triangulated category.
\end{lem}

To proceed, we note that if $F^*$ is quasi-isomorphic to ${F'}^*$,
then for all bounded above complex $G^*$ of sheaves with transfers, 
$F^* \tLNis G^*$ is quasi-isomorphic to ${F'}^* \tLNis G^*$ (see
\cite[Prop. 8.16]{MVW}). Therefore, $\tLNis$ is a well-defined 
bifunctor on $\DShCor$. Finally, observe that $\DShCor$ is 
obtained from $\DCat^-\PST$ by formally inverting morhisms of the 
form $F^* \to {F'}^*$ such that $\nis(F^*) \to \nis({F'}^*)$ is an 
quasi-isomorphism. The tensor triangulated structure now follows 
from the preceding lemma.

In fact, the same argument shows that $\DMeff$ is equipped with a
tensor triangulated structure. Recall that $\DMeff$ is obtained
from $\DShCor$ by formally inverting the $\A^1$-weak equivalences.
To show that $\tLNis$ induces a tensor structure on $\DMeff$, it
suffices to observe that if $\phi : F^* \to {F'}^*$ is an $\A^1$-weak 
equivalence, then for all $G^* \in \DShCor$, $\phi \tLNis G^* : 
F^* \tLNis G^* \to {F'}^* \tLNis G^*$ is an $\A^1$-weak 
equivalence, which is proven in \cite[Lemma 9.5]{MVW}.

It is straightforward to verify that the tensor operation, which
we represent by $\tDM$ is symmetric monoidal. We now define the 
internal hom operation for $\DMeff$.

\begin{defn}
Let $B^*$ be a bounded above complex of Nisnevich sheaves. Let $B^*
\to I^*$ be a Cartan-Eilenburg resolution. Such a resolution 
exists since the category of $\ShCor$ has enough injectives (see 
\cite[Prop.  6.19]{MVW}). For $X \in \Sm$, define 
$\sheaf{\rhom}(\Ztr(X), B)$ to be the complex of sheaves given by
\[
\sheaf{\rhom}(\Ztr(X), B)(U) = \rhom_{\DCat^-\ShCor}(\Ztr(X \times U),
I^*)
\]
We can extend $\sheaf{\rhom}$ in the first factor to the full 
subcategory of $\DShCor$ generated by $\Ztr(X)$.
\end{defn}

Likewise, we can define a similar operation for $\DMeff$. Rather,
we define

\begin{defn}
Let $B^*$ denote a 
\end{defn}

% make a remark about bounded cohomology dimension of the basefield
% in the case we are working with \'etale cohomology.

The following is a straightforward consequence of the definition:

\begin{lem}
For all $X \in \Sm$, and all $F^*, G^* \in \DCat^-\ShCor$, we have
the following adjunction
\[
\hom_{\DCat^-\ShCor}(F^*\tLNis \Ztr(X), G^*) \cong
\hom_{\DCat^-\ShCor}(F^*, \sheaf{\rhom}(\Ztr(X), G^*)).
\]
\end{lem}


\newpage
\chapter{The Derived Category of Motives}\label{sect_dmeff_and_dm}

In this chapter, we define the derived category of motives $\DMeff$,
and show that it is equipped with an additive symmetric monoidal 
structure with a partial internal hom (defined in Definition
\ref{def_tensor_triang_cat}) that will be used to construct the 
slice filtration in Section \ref{sect_slice_filt_dm} (see Remark 
\ref{rmk_partial_ihom}).

To do this, we first define the bounded above derived category 
$\DCat^-\ShCor$ of Nisnevich sheaves, and define $\DMeff$ to be
the localization of $\DCat^-\ShCor$ by a class of morphisms in
$\DCat^-\ShCor$ called $\A^1$-weak equivalences. We then show that
$\DMeff$ is in fact equivalent as a category to the subcategory
of $\DCat^-\ShCor$ with homotopy invariant cohomology. 

We also show that $\DCat^-\PST$ is equipped with tensor and 
internal hom operations on which induce a symmetric monoidal 
structure on $\DMeff$. For the remainder of the chapter, unless 
stated otherwise, all sheaves are Nisnevich sheaves. We will drop 
the ``Nis'', and simply write $\ShCor$ for the category of 
Nisnevich sheaves with transfers. This chapter is taken from
Lectures 8, 9 and 14 of \cite{MVW}.

\section{Derived Category of Motives}

First consider the category $\PST$. By Yoneda, for $X$ in $\Sm$ and
$F$ in $\PST$,
\[
\hom_{\PST}(\Ztr(X), F) = F(X).
\]
It follows that $\Ztr(X)$ is projective for every $X$ in $\Sm$.
Since direct sums of projectives are projective, 
$\oplus_i \Ztr(X_i)$ is also projective for any arbitrary 
collection $\{Z_i\}$. Furthermore, for $F$ in $\PST$, there exists 
a surjection
\begin{equation}\label{eq_res_by_rep}
\bigoplus_{X} \bigoplus_{0 \neq x \in F(X)} \Ztr(X) 
   \stackrel{x}{\to} F.
\end{equation}
Hence, the category $\PST$ has enough projectives. Thus, we may 
define the bounded above derived category $\DCat[D]^-\PST$ of the 
abelian category $\PST$ as the homotopy category of chain 
complexes of projective objects in $\PST$ (see 
\cite[10.4.8]{WH}).

Similarly, we define the bounded derived category $\DCat^- \ShCor$
of Nisnevich sheaves with transfers. In particular, since $\ShCor$
has enough injectives (\cite[13.1]{MVW}), again by 
\cite[10.4.8]{WH}, we can define $\DCat^-\ShCor$ as the homotopy
category $\DCat[K]^-$ of bounded above complex of injective objects
in $\ShCor$. 

To simplify notations, for the remainder, we write $\DShCor$ for 
the derived category $\DCat^-\ShCor$ of Nisnevich sheaves with 
transfers. We now define $\DMeff$, the derived category of 
motives. To do so, we first need a notion of a thick subcategory.

\begin{defn}
A full additive subcategory $\Cat{W}$ of a derived category 
$\DCat$ is \DEF{thick} if it satisfies the following conditions:
\begin{enumerate}
\item if $A \to B \to C \to A[1]$ is a distinguished triangle, then
any two of $A, B, C$ is in $\Cat{W}$, then so is the third.

\item if $A \oplus B$ is an object of $\Cat{W}$, then $A$ and $B$ 
are both objects of $\Cat{W}$.
\end{enumerate}
\end{defn}

If $\Cat{W}$ is a thick subcategory of a derived category $\DCat$,
then we can define a quotient triangulated category 
$\DCat/\Cat{W}$. Let $\Cat{S}$ be the set of maps whose cone is in 
$\Cat{W}$. Then $\Cat{S}$ is a saturated multiplicative system in 
the sense that $\Cat{S}$ contains the identity, is closed under 
composition, and if $fg \in \Cat{S}$, then $f$ and $g$ are both in 
$\Cat{S}$. Define $\DCat/\Cat{W}$ to be the localization 
$\DCat[D] [\Cat{S}^{-1}]$ (see \cite{Verd96}).

\begin{defn}\label{def_DMeff}
Let $\Cat{W}_{\A}$ be the thick subcategory of $\DCat^-$ generated 
by the cones of $\Ztr(X \times \A^1) \to \Ztr(X)$ for every $X$ in 
$\Sm$, and closed under direct sums that exist in $\DCat^-\ShCor$. 
Write $\Cat{S}_{\A}$ for the maps whose cone is in $\Cat{W}_{\A}$. 
We say that a map $f$ in $\DShCor$ is an \DEF{$\A^1$-weak 
equivalence} if $f \in \Cat{S}_{\A}$.

We write $\DMeff$ for the localization 
$\DShCor [\Cat{S}_{\A}^{-1}]$. The category that we have
just defined is the derived category of motives, whose objects are
called \DEF{motives}.
\end{defn}

While we have defined $\DMeff$ as a localization of 
$\DCat^-$ by the $\A^1$-weak equivalences, we can identify
$\DMeff$ with a subcategory of $\DShCor$.

\begin{defn}\label{def_ALocal}
Let $F^*$ be an object of $\DShCor$. We say that $F^*$ is 
\DEF{$\A^1$-local} if $\hom_{\DShCor}(-, F^*)$ sends $\A^1$-weak 
equivalences to isomorphisms. We write $\ALocal$ for the full 
subcategory of $\A^1$-local objects in $\DShCor$.
\end{defn}

Proposition \ref{prop_char_of_alocal} gives a good 
characterization of the category $\ALocal$.

\begin{prop}[\cite{MVW} Prop. 14.8, Cor. 14.9]\label{prop_char_of_alocal}
For $F^*$ in $\DShCor$, $F^* \in \ALocal$ if and only if
$\nis(H^n F^*)$ are homotopy invariant for all integer $n$. In 
particular, we can identify $\ALocal$ with the full subcategory of 
complexes in $\DShCor$ with homotopy invariant cohomology 
presheaves.
\end{prop}

% define suslin C^*
\begin{defn}
For $F^*$ a bounded above cochain complex of sheaves with transfers,
let $\suslC[] F^*$ denote the total complex of the double complex 
$\suslC[p] F^q$. It is an object of 
$\DShCor$.
\end{defn}

We make the following observations: since $F^*$ is bounded above,
by shifting sufficiently, we may assume that $F^*$ is concentrated
in strictly nonnegative degrees. Therefore, indexing the double
complex cohomologically, $\suslC F^*$ is a third quadrant double 
complex. Filtering the total complex $\suslC[] F^*$ by row, we 
obtain a third quadrant spectral sequence converging to the 
cohomologies of $\suslC[] F^*$. Since the cohomology of any complex
$\suslC G$ for a sheaf with transfers $G$ is homotopy invariant 
(see \cite[2.19]{MVW}) the terms in the first page of the spectral 
sequence are all homotopy invariant. It follows that $\suslC[] F^* 
\in \ALocal$.

The significance of the construction defined above is explained
in the next proposition:

\begin{prop}\label{prop_suslC_ALocal}
The functor $\suslC : \DShCor \to \ALocal$ is a left adjoint to
the inclusion of $\ALocal \into \DShCor$.
\end{prop}

\begin{proof}
There is a canonical map from $F^* \to \suslC[] F^*$,
given by the inclusion of $F_i = \suslC[0] F_i \into \oplus_{p + 
q = i} \suslC[-p] F^q$. This map is a $\A^1$-weak equivalence
(see \cite[14.4]{MVW}). Therefore, for any $L^*$ in $\ALocal$
and $F^*$ in $\DShCor$,
\[
\hom_{\DShCor}(F^*, L^*) \cong \hom_{\DShCor}(\suslC[] F^*, L^*)
   = \hom_{\ALocal}(\suslC[] F^*, L^*).
\qedhere
\]
\end{proof}

There is a canonical functor $\pi: \DShCor \to \DMeff$, given
by sending an object of $\DShCor$ to its corresponding object in
$\DMeff$. Its restriction to $\ALocal$ defines a functor from
$\ALocal$ to $\DMeff$.

Furthermore, we can define a map from $\DMeff$ to $\ALocal$. 
Notice that if $F^*$ and ${F'}^*$ are $\A^1$-weak equivalent, then
transitivity implies that $\suslC[] F^*$ is $\A^1$-weak equivalent 
to $\suslC[] F^*$. It follows that the functor that sends $F^*$
to $\suslC[] F^*$ lifts to a functor from $\DMeff \to \ALocal$.
Let $\suslC$ denote the induced functor on $\DMeff$.

\begin{thm}\label{thm_ALocal_eq_DMeff}
The functor $\pi: \ALocal \to \DMeff$ is an equivalence of 
categories, with quasi-inverse $\suslC$.
\end{thm}

\begin{proof}
The fact that $\pi$ is an equivalence is established in
\cite[14.11]{MVW}. Furthermore, given $M$ in
$\DMeff$, then $M$ is represented by some bounded above complex 
$F^*$. In turn, $F^*$ is isomorphic (in $\DMeff$) to $\suslC[] 
F^*$, which is in the essential image of $\pi$, and define
$\suslC M = \suslC[] F^*$. 

For the second statement, it suffices at this point to show that 
$\suslC \pi$ is naturally isomorphic to the identity on $\ALocal$. 
This follows from the fact that if $F^*$ is $\A^1$-local, then 
$\suslC[] F^*$ is isomorphic to $F^*$ (see \cite[14.9]{MVW}).
\end{proof}

\begin{ex}\label{ex_geo_obj}
An important class of examples are the \DEF{geometric objects}. 
Let $X$ be a smooth scheme. Then, we may regard $\Ztr(X)$ as a
chain complex of Nisnevich sheaf with transfers concentrated
in degree 0 (see Example \ref{ex_ZtrX}), and represents an object in
$\DShCor$, and also an object in $\DMeff$. We call the full
triangulated subcategory of $\DMeff$ generated by $\Ztr(X)$, as 
$X$ ranges over all smooth schemes, the \DEF{effective geometric 
motives}, which we represent by $\DMeffgm$. We write $M(X)$ for 
the class of $\Ztr(X)$ in $\DMeff$.

On the other hand, $\suslC \Ztr(X)$ represents an object in 
$\ALocal$, and we can similarly define the geometric object of 
$\ALocal$ as the full triangulated subcategory generated by
$\suslC \Ztr(X)$, for $X$ in $\Sm$. By Theorem 
\ref{thm_ALocal_eq_DMeff}, $\DMeffgm$ correspond precisely to
the geometric objects of $\ALocal$.
\end{ex}

\section{Triangulated Monoidal Structure on $\DMeff$}
\label{sect_TMS_DMeff}

Recall from \cite[1.13]{MK} and \cite[8A.1]{MVW} the notion of a 
symmetric closed monoidal structure generalized to the setting of 
a triangulated category:

\begin{defn}\label{def_tensor_triang_cat}
Let $\DCat[D]$ be a triangulated category. We say that $\DCat[D]$ 
is a \DEF{tensor triangulated category} if there exists a bifunctor
$- \tensor -$, together with two natural isomorphisms 
\[
\begin{tikzcd}
(M[1]) \tensor N \arrow{r}{l_{M, N}}&
(M \tensor N)[1] &
M \tensor (N[1]) \arrow{l}{r_{M, N}}
\end{tikzcd}
\]
such that $(\DCat[D], \tensor)$ satisfies the axioms of a 
symmetric monoidal category, and the following two conditions
hold
\begin{enumerate}
\item For any distinguished triangle $M' \to M \to M'' 
\stackrel{\delta}{\to} M'[1]$, and any $N$ in $\DCat[D]$,
the following triangles are distinguished
\[
\begin{tikzcd}
M' \tensor N \arrow{r} &
M \tensor N \arrow{r} &
M'' \tensor N \arrow{r}{l(\delta \tensor D)}&
(M' \tensor N)[1] \\
N \tensor M' \arrow{r} &
N \tensor M \arrow{r} &
N \tensor M'' \arrow{r}{r(D \tensor \delta)}&
(N \tensor M')[1]
\end{tikzcd}
\]

\item For any $M$ and $N$ in $\DCat[D]$, the following 
anti-commutes, i.e. $rl = -lr$:
\[
\begin{tikzcd}[row sep=small, column sep=small]
M[1] \tensor N[1] \arrow{rr}{r}\arrow{dd}{l} && 
(M[1] \tensor N)[1] \arrow{dd}{l} \\
& -1 \\
(M \tensor D[1])[1] \arrow{rr}{r} &&
(M \tensor D)[2].
\end{tikzcd}
\]
\end{enumerate}

We say that $(\DCat, \tensor)$ is an \DEF{additive symmetric 
monoidal category} if 
\[
\left(\bigoplus M_i\right) \tensor N = \bigoplus_i (M_i \tensor N)
\]
for all $N$ in $\DCat$ and all family $\{M_i\}$ of objects of 
$\DCat$ such that $\oplus_i M_i \in \DCat$. 

Finally, we say that $\DCat$ \DEF{has a partial internal hom} if 
there exists a full subcategory $\DCat^{\compact}$ of $\DCat$, 
containing $\Unit$ and a bifunctor $\ihom(-,-) : 
(\DCat^{\compact})^{op} \times \DCat \to \DCat$ such that for
all $F$ in $\DCat^{\compact}$, $F \tensor -$ is left adjoint to 
$\ihom(F, -)$. 

In this case, we call $\DCat^{\compact}$ the \DEF{\SemiInvertible 
object} of $\DCat$, and $\ihom$ the partial internal hom 
(bi)functor on $\DCat$. We call the pair $(\ihom, 
\DCat^{\compact})$ the \DEF{partial internal hom structure on 
$\DCat$}.
\end{defn}

Following \cite{MVW}, we show that $\DMeff$ is equipped with an 
additive symmetric monoidal structure with a partial internal hom. 
To do so, we first define the tensor and internal hom operators on 
$\PST$. The tensor structure will be determined by these 
requirements: 

\begin{enumerate}
\item $\Ztr(X) \ttr \Ztr(Y) \defeq \Ztr(X \times Y)$, 

\item for each map $\phi$ in $\hom_{\PST}(\Ztr(X), \Ztr(Y))$ which
equals $\Cor(Y, X)$ identified with the correspondence $W$, $\phi 
\tensor \Ztr(Z): \Ztr(X) \ttr \Ztr(Z) \to \Ztr(Y) \ttr \Ztr(Z)$ is 
the correspondence $W \times Z$. 
\end{enumerate}
\noindent It is clear that we can extend the bifunctor $\ttr$ to 
arbitrary direct sums of representable presheaves.  

Next, for arbitrary presheaves with transfers $F, G$, let $P^* \to 
F$ and $Q^* \to G$ be resolutions by direct sums of representable 
functors of $F$ and $G$ respectively. We write $F \tL G$ for the 
total complex of the double complex $P^* \ttr Q^*$. By the 
Comparison Theorem \cite[2.26]{WH}, any two projective resolutions 
are chain homotopy equivalent. Therefore, it is easy to see that 
up to chain homotopy equivalence, this is independent of the 
choice of $P^*$ and $Q^*$.

In particular, $H^0(F \tL G)$ is well-defined. Define the tensor 
operation on $\PST$ to be
\[
F \tensor G \defeq F \tL G,
\]
and define internal hom by
\[
\ihom(F, G): X \mapsto \hom_{\PST}(F \tensor \Ztr(X), G).
\]
These operations define a closed monoidal structure on $\PST$.
That is, for all $F$ in $\PST$, the functor $F \tL -$ is adjoint to 
$\ihom(-, F)$ (see \cite[8.3]{MVW}).

\begin{rmk}
Notice that the $\tensor$ structure defined is \emph{not} the
usual tensor product on presheaves of abelian groups. In 
particular, $\Ztr(X)(Z) \tensor_{\Z} \Ztr(Y)(Z) \neq 
\Ztr(X \times Y)(Z)$, where $\tensor_{\Z}$ denotes the usual 
tensor product of abelian groups.
\end{rmk}

We now extend $\tensor$ to $\DCat[D]^-\PST$. To do so, let $F^*$ 
represent a bounded above cochain complex of presheaves with 
transfers. By \cite[10.5.6]{WH}, $F^*$ is quasi-isomorphic to a 
projective complex $P^*$. In fact, we may assume that $P^*$ is a 
complex such that $P^i$ is a direct sum of representable 
presheaves. 

Define $F^* \tL G^*$ to be the total complex associated with
$P^* \tensor Q^*$, where $P^*$ and $Q^*$ are projective 
resolutions of $F^*$ and $G^*$ respectively. Notice that $F^* \tL 
G^*$ is defined up to quasi-isomorphism, and is independent 
of the choice of $P^*$ and $Q^*$.

In particular, $\tL$ is well-defined as a bi-functor on 
$\DCat[D]^-\PST$. Indeed, let $F^*$ and ${F'}^*,$ be two 
quasi-isomorphic bounded above complexes in $\PST$. Then for any 
bounded above $G^*$ in $\PST$, $F^* \tL G^* \cong {F'}^* \tL G^*$. 
(see \cite[8.7]{MVW}) To show that $\DCat^-\PST$ is equipped
with a tensor triangulated structure, we make the following 
observation.

Let $\Cor^{\oplus}$ denote the closure under direct 
sum of representable presheaves in $\PST$. This is an additive
category equipped with an additive symmetric monoidal structure.
By \cite[8A.4]{MVW}, we see that the homotopy category 
$\DCat[K]^-(\Cor^{\oplus})$ is a tensor triangulated categegory.
By similar arguments as in \cite[10.4.8]{WH}, $\DCat[D]^-\PST$
is equivalent as a category to $\DCat[K]^-(\Cor^{\oplus})$.
It follows that $\DCat[D]^-\PST$ is also a tensor triangulated 
category.

Next, we show that the tensor structure is preserved under
Nisnevich sheafification.

\begin{defn}\label{def_shcor_tensor}
Let $F, G$ be Nisnevich sheaves with transfers. Define
$F \ttrNis G$ to be the Nisnevich sheafification of the presheaf
$F \ttr G$. That is,
\[
F \ttrNis G \defeq \nis(F \ttr G)
\]
where $\nis$ is the Nisnevich sheafification. We can extend 
$\ttrNis$ to chain complexes of Nisnevich sheaves. Let $F^*$ and 
$G^*$ be bounded above cochain complexes of Nisnevich sheaves with transfers. 
Define $F^* \tLNis G^*$ to be the Nisnevich sheafification of the 
complex $F^* \tL G^*$:
\[
F^* \tLNis G^* \defeq \nis(F^* \tL G^*).
\]
This is well-defined up to quasi-isomorphism.
\end{defn}

\begin{rmk}
Fix $F$ and $G$ sheaves with transfers, and let $P^*$ and 
$Q^*$ be resolutions by representables of $F^*$ and $G^*$ 
respectively. Since $\nis$ is exact, $\nis(F \tL G) = 
\nis(\Tot(P^* \ttr Q^*)) = \Tot(P^* \ttrNis Q^*)$.
\end{rmk}

We claim that $(\DShCor, \tLNis)$ is an additive symmetric 
monoidal triangulated category. The proof depends on the following
lemma.

\begin{lem}[\cite{MVW} Prop. 8A.7]\label{lem_loc_and_tensor}
Let $\DCat$ be a tensor triangulated category, and let $\Cat{W}$
be a collection of maps in $\DCat$ that is closed under $-\tensor 
N$ for every $N$ in $\DCat$ i.e. if $M \to M'$ is in $\Cat{W}$ then 
so is $M \tensor N \to M' \tensor N$. Then the localization $\DCat 
[\Cat{W}^{-1}]$ is also a tensor triangulated category.
\end{lem}

To proceed, we note that if $F^*$ is quasi-isomorphic to ${F'}^*$,
then for every bounded above complex $G^*$ of sheaves with 
transfers, $F^* \tLNis G^*$ is quasi-isomorphic to 
${F'}^* \tLNis G^*$ (see \cite[8.16]{MVW}). Therefore, $\tLNis$ is 
a well-defined bifunctor on $\DShCor$. Finally, observe that 
$\DShCor$ is obtained from $\DCat^-\PST$ by formally inverting 
morhisms of the form $F^* \to {F'}^*$ such that $\nis(F^*) \to 
\nis({F'}^*)$ is an quasi-isomorphism. The tensor triangulated 
structure now follows from Lemma \ref{lem_loc_and_tensor}.

In fact, the same argument shows that $\DMeff$ is equipped with a
tensor triangulated structure. Recall that $\DMeff$ is obtained
from $\DShCor$ by formally inverting the $\A^1$-weak equivalences.
To show that $\tLNis$ induces a tensor structure on $\DMeff$, it
suffices to observe that if $\phi : F^* \to {F'}^*$ is an $\A^1$-weak 
equivalence, then for all $G^*$ in $\DShCor$, $\phi \tLNis G^* : 
F^* \tLNis G^* \to {F'}^* \tLNis G^*$ is an $\A^1$-weak 
equivalence, which is proven in \cite[9.5]{MVW}.

It is straightforward to verify that the tensor operation, which
we represent by $\tDM$ is symmetric monoidal. We now define the 
internal hom operation for $\DMeff$.

There also exists a tensor operation on $\ALocal$, which is 
different than the one defined on its parent category $\DShCor$.
For $F^*, G^*$ in $\ALocal$, we define $F^* \tAL G^*$ to be
\[
\suslC[]( F^* \tLNis G^* ).
\]
It is straightforward to verify that $(\ALocal, \tAL)$ satisfy
the conditions for a tensor triangulated category. In fact, the
categorical equivalence $\pi : \ALocal \to \DMeff$ established
in Theorem \ref{thm_ALocal_eq_DMeff} is an equivalence of tensor
triangulated categories (see \cite[14.11]{MVW}).

\begin{defn}
Let $B^*$ be a bounded above complex of Nisnevich sheaves. Let $B^*
\to I^*$ be a Cartan-Eilenberg resolution. Such a resolution 
exists since the category of $\ShCor$ has enough injectives (see 
\cite[6.19]{MVW}). For $X$ in $\Sm$, define 
$\sheaf{\rhom}(\Ztr(X), B)$ to be the complex of sheaves given by
\[
\sheaf{\rhom}(\Ztr(X), B)(U) = \rhom_{\DCat^-\ShCor}(\Ztr(X \times U),
I^*)
\]
We can extend $\sheaf{\rhom}$ in the first factor to the full 
subcategory of $\DShCor$ generated by the sheaves $\Ztr(X)$.
\end{defn}

The following lemma is a straightforward consequence of the above 
definition.

\begin{lem}\label{lem_ihom_tL_adjunction}
For all $X$ in $\Sm$, and all $F^*, G^*$ in $\DCat^-\ShCor$, we 
have the following adjunction
\[
\hom_{\DShCor}(F^*\tLNis \Ztr(X), G^*) \cong
\hom_{\DShCor}(F^*, \sheaf{\rhom}(\Ztr(X), G^*)).
\]
\end{lem}

We can define a similar operation for $\DMeff$ by
defining the internal hom for $\ALocal$, which induces an internal
hom structure on $\DMeff$ via the equivalence from $\ALocal$ to
$\DMeff$.

For the internal hom structure on $\ALocal$, notice that for all
smooth schemes $X$, the cochain complex $\sheaf{\rhom}(\Ztr(X), F^*)$ is $\A^1$-local if 
$F^*$ is (\cite[14.12]{MVW}). We now define $\ihomAL$. Fix $X$ in
$\Sm$, and $F^*$ be a bounded above $\A^1$-local complex. Then, for
any smooth scheme $U$,
\[
\ihomAL(\suslC \Ztr(X), F^*)(U) \defeq \rhom_{\ALocal}(\suslC 
   \Ztr(X \times U), F^*).
\]
Since $\suslC$ is adjoint to the inclusion of $\ALocal$ into 
$\DShCor$, it is clear that $\ihomAL$ is equal to $\sheaf{\rhom}$
restricted to $\ALocal$ in the second factor. Therefore, we have
\begin{align*}
\hom_{\ALocal}(F^* \tAL \suslC \Ztr(X), G^*) &\stackrel{(1)}{=}
\hom_{\ALocal}(\suslC[] F^* \tAL \suslC \Ztr(X), G^*) \\ 
&\stackrel{(2)}{=} \hom(F^* \tLNis \Ztr(X), G^*) \\
&\stackrel{(3)}{=} \hom(F^*, \sheaf{\rhom}(\Ztr(X), G^*) \\
&\stackrel{(4)}{=} \hom_{\ALocal}(F^*, \ihomAL(\suslC \Ztr(X), G^*)),
\end{align*}
where $F^*$ and $G^*$ are bounded above $\A^1$-local complexes,
and $X$ is an arbitrary smooth scheme. In the above, the first 
equality follows from the definition of $\tAL$, (2) and (4) 
follow from the adjunction introduced above (see Proposition
\ref{prop_suslC_ALocal}), and (3) follows from the adjunction
established in Lemma \ref{lem_ihom_tL_adjunction}. 

Via the categorical equivalence between $\ALocal$ and $\DMeff$,
there exists an internal hom functor on the geometric objects
of $\DMeff$. We write $\tDM$ and $\ihomDMf$ for the tensor and
partial internal hom operators respectively. We have just
established the proposition below:

\begin{prop}[\cite{MVW} 14.12]\label{prop_DMgm_monoidal}
Let $\tDM$ and $\ihomDMf$ be given as above. Then for all
$M$ in $\DMeffgm$, $- \tDM M$ is left adjoint to $\ihomDMf(M, -)$.
In particular, $\DMeffgm$ is an additive symmetric closed monoidal 
category.
\end{prop}

\begin{rmk}\label{rmk_partial_ihom}
Notice that $\sheaf{\rhom}$ and $\ihomAL$ do not define a closed
monoidal structure on their respective categories, as they are only
defined on the geometric objects.
\end{rmk}

\section{The motivic complex $\Z(n)$}
\label{sect_motivic_complex}

We now introduce an important set of objects in $\DMeff$. Let 
$\Ztr(\Gm)$ denote the cokernel of
\[
   \Z = \Ztr(\Spec \basefield) \to \Ztr(\A^1 - 0)
\]
given by $\basefield [x, x^{-1}] \to \basefield$, induced by
$x \mapsto 1$. Since $\basefield \to \basefield[x, x^{-1}]$
defines a splitting $\Ztr(\A^1 - 0) \cong \Ztr(\Gm) \oplus \Z$,
$\Ztr(\Gm)$ is also a Nisnevich sheaf with transfers. 

More generally, let $X$ be a smooth scheme, and let $x$ be a 
$\basefield$-point of $X$ represented by $\Spec k \to X$. We 
define the \DEF{pointed presheaf $\Ztr(X, x)$} as the 
cokernel of $x: \Z \to \Ztr(X)$, which defines a splitting of the 
structure map $\Ztr(X) \to \Z$. By the same reason as above, 
$\Ztr(X, x)$ is also a Nisnevich sheaf.

If $\{\Ztr(X_i, x_i) : i = 1,\dots, n\}$ is a collection of 
pointed schemes, we define their \DEF{wedge sum} $\bigwedge_i^n
\Ztr(X_i, x_i)$ to be
\[
\cok \left( \bigoplus_i \Ztr(X_1 \times \cdots \times \hat{X_i} 
   \times \cdots \times X_n) \xrightarrow{id \times \cdots \times 
      x_i \times \cdots \times id} \Ztr(X_1 \times \cdots \times 
         X_n)\right).
\]
By induction, $\bigwedge_i \Ztr(X_i, x_i)$ is a direct summand of
$\Ztr(X_1 \times \cdots \times X_n)$ (see \cite[2.13]{MVW}),
and defines a Nisnevich sheaf.

In particular, for each nonnegative integer $n$, we can define 
$\bigwedge_{i = 0}^n \Ztr(\A^1 - 0, 1)$, which we view as an 
object of $\DMeff$. In fact, these are geometric motives, i.e.
objects in the subcategory $\DMeffgm$. We write this object as
$\Z(n)$, which we call the $n$-th \DEF{motivic complex}. It is 
easy to see that $\Z(n) \tDM \Z(m) \cong \Z(n + m)$. 

\begin{rmk}
The careful reader may notice that in \cite{MVW}, the motivic 
complex $\Z(n)$ is defined to be $\suslC \bigwedge_i^n \Ztr(\A^1 
- 0, 1)$, and not $\bigwedge_i^n \Ztr(\A^1 - 0, 1)$ (see 
\cite[3.1]{MVW}). However, notice that in $\DMeff$, the
two definitions of $\Z(n)$ are identified. This is a 
straightforward consequence of the fact that $F \to \suslC F$ is
an $\A^1$-weak equivalence.
\end{rmk}

\begin{rmk}\label{rmk_mot_coh}
Nisnevich motivic cohomology with integer coefficients is defined
as
\[
\MH^{p, q}(X) = \homDMf(\Ztr(X), \Z(q)[p]).
\]
Notice that $\Z(1) \cong \O^*[-1]$ (\cite[4.1]{MVW}).
Therefore $\MH^{1,1}(X) = \O^*(X)$, and $\MH^{2,1} = \Pic(X)$.
Furthermore, $\MH^{n,n}(\Spec F) = \milK_n(F)$ 
(\cite[5.1]{MVW}).

More generally, we have
\[
\MH^{n, i}(X) \cong \CH^i(X, 2i - n)
\]
where $\CH^i(X, k)$ denote the $k$-th higher Chow group of $X$
(\cite[19.1]{MVW}).
\end{rmk}

\section{Cancellation Theorem}
\label{sect_cancellation}

We conclude this chapter with an important result, taken from
\cite[Corollary 4.10]{V02}. To simplify notations, for $M$ in 
$\DMeff$, we write $\tZ{M}$ for $M \tDM \Z(1)$, and $\hZ{M}$ for 
$\ihomDMf(\Z(1), M)$, and write $\tZ[n]{M}$ and $\hZ[n]{M}$ for 
the $n$-th iterations of applying $- \tDM \Z(1)$ and 
$\ihomDMf(\Z(1), -)$ respectively to $M$. 

As $\Z(n) \tDM \Z(1) = \Z(n + 1)$, the functor given by $M \mapsto 
\tZ[n]{M}$ is equal to the functor $- \tDM \Z(n)$. Since right 
adjoints of the same functor are naturally isomorphic, $M \mapsto 
\hZ[n]{M}$ is naturally isomorphic to $\ihomDMf(\Z(n), -)$. 
Furthermore, $\Z(1)$ is an object of $\DMeffgm$. Thus, $- \tDM 
\Z(1)$ is left adjoint to $\ihomDMf(\Z(1), -)$ by Proposition
\ref{prop_DMgm_monoidal}. More generally, $\Z(n)$ is an object
of $\DMeffgm$ for all positive integer $n$. It follows that
$-\tDM \Z(n)$ is left adjoint to $\ihomDMf(\Z(n), -)$.

\begin{thm}[Cancellation]\label{thm_dm_cancellation}
For any $M$ and $N$ in $\DMeff$,
\[
\homDMf(\tZ{M}, \tZ{N}) \cong \homDMf(M, N).
\]
In other words, tensoring with $\Z(1)$ is fully and faithful.
\end{thm}

This statement can likewise be interpreted for the category 
$\ALocal$. For $F^*$ and $G^*$ bounded above $\A^1$-local 
complexes, by abuse of notation, write $\tZ{F^*}$ for
$F \tLNis \suslC \Z(1)$. By Theorems \ref{thm_dm_cancellation}
and \ref{thm_ALocal_eq_DMeff}, 
\[
\hom_{\DShCor}(F^* \tLNis \suslC \Z(1), G^* \tLNis \suslC \Z(1))
\cong \hom_{\DShCor}(F^*, G^*).
\]
This is the version of the statement that we will use in
subsequent chapters. One important corollary of Theorem
\ref{thm_dm_cancellation} is the following:

\begin{cor}\label{cor_tZ_hZ_eq_id}
For each $M$ in $\DMeff$ and each nonnegative integer $n$, 
the counit map
\[
\hZ[n]{\tZ[n]{M}} \to M.
\]
is an isomorphism, natural in $M$.
\end{cor}
\begin{proof}
By Theorem \ref{thm_ALocal_eq_DMeff}, it suffices to verify the
statement for $\A^1$-local complexes. 

Let $F^*$ be the bounded above $\A^1$-local complex corresponding 
to $M$. Notice that by the Cancellation Theorem, reinterpreted for
$\A^1$-local complexes,
\[
\sheaf{\rhom}(\suslC \Z(1), \suslC \Z(1) \tLNis F^*)(U) = 
\rhom(\Ztr(U), F^*) = F^*(U)
\]
for all $U$ in $\Sm$. It follows that $\sheaf{\rhom}(\suslC\Z(1),
\suslC\Z(1) \tLNis F^*) \to F^*$ is an isomorphism. The
proposition now follows by induction on $n$.
\end{proof}

\newpage
\section{The Derived Category of Motives}\label{sect_dmeff_and_dm}

In this section, we define the derived category of motives $\DMeff$,
and show that it is equipped with a closed monoidal structure that
will be used to construct the slice filtration in Section 
\ref{sect_slice_filt_dm}.

To do this, we first define the bounded above derived category 
$\DShCor$ of Nisnevich sheaves, and show that the derived 
category is equipped with a tensor and internal hom functor. We 
then show that $\DMeff$ can be identified as the full subcategory 
of $\DShCor$ with homotopy invariant cohomology. The closed
monoidal structure on $\DMeff$ is therefore obtained from the
closed monoidal structure on $\DShCor$. For the remainder of the
section, unless stated otherwise, all sheaves are Nisnevich 
sheaves. We will drop the ``Nis'', and simply write $\ShCor$ for 
the category of Nisnevich sheaves with transfers.

First consider the category $\PST$. By Yoneda, for $X \in \Sm$ and
$F \in \PST$,
\[
\hom_{\PST}(\Ztr(X), F) = F(X).
\]
It follows that $\Ztr(X)$ is projective for every $X \in \Sm$.
Since direct sums of projectives are projective, 
$\oplus_i \Ztr(X_i)$ is also projective for any arbitrary 
collection $\{Z_i\}$. Furthermore, for $F \in \PST$, there exists 
a surjection
\begin{equation}\label{eq_res_by_rep}
\bigoplus_{X} \bigoplus_{0 \neq x \in F(X)} \Ztr(X) 
   \stackrel{x}{\to} F.
\end{equation}
Hence, the category $\PST$ has enough projectives. Thus, we may 
define the bounded above derived category $\DCat{D}^-\PST$ of the 
abelian category $\PST$, and a closed monoidal structure on 
$\DCat{D}^-\PST$. The existence of $\DCat{D}^-\PST$ follows from 
\cite{WH} 10.4.8. The tensor structure is obtained from the 
following construction. Define $\Ztr(X) \ttr \Ztr(Y) \defeq 
\Ztr(X \times Y)$. Furthermore, for each $\phi: \Ztr(X) \to 
\Ztr(Y)$, $\phi \in \hom_{\PST}(\Ztr(X), \Ztr(Y)) = \Cor(Y, X)$ 
and is given by $W$. Define $\phi \tensor \Ztr(Z): \Ztr(X) \ttr 
\Ztr(Z) \to \Ztr(Y) \ttr \Ztr(Z)$ to be $W \times Z$. It is clear 
that we can extend $\ttr$ to arbitrary direct sums of 
representable presheaves.

For arbitrary $F, G \in \PST$, let $P^* \to F$ and $Q^* \to G$
be a resolutions by direct sums of representable functors of $F$ 
and $G$ respectively. We write $F \tL G$ for the total complex of 
the double complex $P^* \ttr Q^*$. By \cite{WH} ---, any two 
projective resolutions are chain homotopy equivalent. Therefore, 
it is easy to see that up to chain homotopy equivalence, this is 
independent of the choice of $P^*$ and $Q^*$.

In particular, $H^0(F \tL G)$ is well-defined. Set the tensor and
internal hom bifunctors on $\PST$ to be
\[
F \tensor G \defeq F \tL G.
\]
and let 
\[
\ihom(F, G): X \mapsto \hom_{\PST}(F \tensor \Ztr(X), G).
\]
The following shows that $\tensor$ and $\ihom$ define a closed
monoidal structure on $\HI$:
\begin{lem}
For all $F \in \PST$, the functor $F \tL -$ if adjoint to 
$\ihom(-, F).$
\end{lem}
\begin{proof}
Fix $F, G,$ and $H \in \PST$, we show that
\[
\hom_{\PST}(G \tensor F, H) = \hom_{\PST}(G, \ihom(F, H))
\]

Since $\PST$ has enough projectives, it suffices to prove this 
for $F, G$ representable. Assume $F = \Ztr(X), G = \Ztr(Y)$,
then 
\begin{align*}
\hom_{\PST}(\Ztr(X), \ihom(\Ztr(Y), H)) &=
\ihom(\Ztr(Y), H)(X) \\
&= \hom_{\PST}(\Ztr(X) \tensor \Ztr(Y), H)
\end{align*}
where the first equality follows by Yoneda, and the second, by
the definition of $\ihom(\Ztr(Y), H)$.
\end{proof}

We now extend $\tensor$ to $\DCat{D}^-\PST$. To do so, let $F^*$ 
represent a bounded above complex of presheaves with transfers.
By \cite[10.something]{WH}, $F^*$ is quasi-isomorphic to a 
projective complex $P^*$. In fact, we may assume that $P^*$ is a 
complex such that $P^i$ is a direct sum of representable 
presheaves. We call $P^* \to F^*$ \DEF{a projective resolution (by
representable presheaves)}.

Define $F^* \tL G^*$ to be the total complex associated with
$P^* \tensor Q^*$, where $P^*$ and $Q^*$ are projective 
resolutions of $F^*$ and $G^*$ respectively. Once again, up to
quasi-isomorphism, $F^* \tL G^*$ is defined independent of the 
choice of projective resolutions, and is itself a bounded above 
complex. 

\begin{lem}[\cite{MVW} Lemma 8.7]
Let $F^*$ and ${F'}^*,$ be two quasi-isomorphic bounded above 
complexes in $\PST$. Then for any bounded above $G^*$ in $\PST$,
$F^* \tL G^* \cong {F'}^* \tL G^*$.
\end{lem}
\begin{proof}
This follows straightforwardly from the fact that any projective
resolution of $F^*$ is a projective resolution of ${F'}^*$.
\end{proof}

In particular, $\tL$ is well-defined as a bi-functor on 
$\DCat{D}^-\PST$.

\begin{defn}
Let $\DCat{D}$ be a triangulated category. We say that $\DCat{D}$ 
is a \DEF{tensor triangulated category} if there exists a bifunctor
$- \tensor -$, together with two natural isomorphisms 
\[
\begin{tikzcd}
(M[1]) \tensor N \arrow{r}{l_{M, N}}&
(M \tensor N)[1] &
M \tensor (N[1]) \arrow{l}{r_{M, N}}
\end{tikzcd}
\]
such that $(\DCat{D}, \tensor)$ satisfies the axioms of a 
symmetric monoidal category, and the following two conditions
hold
\begin{enumerate}
\item For any distinguished triangle $M' \to M \to M'' 
\stackrel{\delta}{\to} M'[1]$, and any $N \in \DCat{D}$,
the following triangles are distinguished
\[
\begin{tikzcd}
M' \tensor N \arrow{r} &
M \tensor N \arrow{r} &
M'' \tensor N \arrow{r}{l(\delta \tensor D)}&
(M' \tensor N)[1] \\
N \tensor M' \arrow{r} &
N \tensor M \arrow{r} &
N \tensor M'' \arrow{r}{r(D \tensor \delta)}&
(N \tensor M')[1]
\end{tikzcd}
\]

\item For any $M, N \in \DCat{D}$, the following anti-commutes,
i.e. $rl = -lr$:
\[
\begin{tikzcd}
M[1] \tensor N[1] \arrow{rr}{r}\arrow{dd}{l} && 
(M[1] \tensor N)[1] \arrow{dd}{l} \\
& -1 \\
(M \tensor D[1])[1] \arrow{rr}{r} &&
(M \tensor D)[2].
\end{tikzcd}
\]
\end{enumerate}

We say that $(\DCat{D}, \tensor)$ is an \DEF{additive symmetric 
monoidal category} if 
\[
(\bigoplus M_i) \tensor N = \bigoplus_i (M_i \tensor N)
\]
for all $N \in \DCat{D}$ and all family $\{M_i\}$ of
objects of $\DCat{D}$ such that $\oplus_i M_i \in \DCat{D}$.
\end{defn}

\begin{lem}\label{lem_tensor_add_cat}
Let $\Cat{A}$ be an additive category, equipped with a additive 
symmetric monoidal structure. Then the homotopy category 
$\DCat{K}^-(\Cat{A})$ of bounded above (co)chain complexes of
is a tensor triangulated category.
\end{lem}
\begin{proof}
\end{proof}

Let $\Cor^{\oplus}$ denote the closure under arbitrary direct 
sum of representable presheaves in $\PST$. This is an additive
category equipped with an additive symmetric monoidal structure.
By the previous lemma, we see that the homotopy category 
$\DCat{K}(\Cor^{\oplus})$ is a tensor triangulated categegory.
By similar arguments as in \cite[10.4.8]{WH}, $\DCat{D}^-\PST$
is equivalent as a category to $\DCat{K}^-(\Cor^{\oplus})$.
It follows that $\DCat{D}^-\PST$ is also a tensor triangulated 
category.

\begin{defn}\label{def_shcor_tensor}
Let $F, G$ be Nisnevich sheaves with transfers. Define
$F \ttrNis G$ to be the Nisnevich sheafification of the presheaf
$F \ttr G$. That is,
\[
F \ttrNis G \defeq \nis(F \ttr G)
\]
where $\nis$ is the Nisnevich sheafification. We can extend 
$\ttrNis$ to chain complexes of Nisnevich sheaves. Let $F^*$ and 
$G^*$ be a chain complex of Nisnevich sheaves with transfers. 
Define $F^* \tLNis G^*$ to be the Nisnevich sheafification of the 
complex $F^* \tL G^*$:
\[
F^* \tLNis G^* \defeq \nis(F^* \tL G^*).
\]
This is well-defined up to quasi-isomorphism.
\end{defn}

\begin{rmk}
Fix $F^*$ and $G^*$ sheaves with transfers, and let $P^*$ and 
$Q^*$ be resolutions by representables of $F^*$ and $G^*$ 
respectively. Since $\nis$ is exact, $\nis(F^* \tL G^*) = 
\nis(\Tot(P^* \ttr Q^*)) = \Tot(P^* \ttrNis Q^*)$.
\end{rmk}

We claim that $(\DCat{D}^-\ShCor, \tLNis)$ is an additive 
symmetric monoidal triangulated category. This is precisely the
content of Theorem \ref{thm_dshcor_tensor}. We first consider the 
following lemmas.

\begin{lem}[\cite{MWV} Prop. 8A.7]
Let $\DCat$ be a tensor triangulated category, and let $\Cat{W}$
be a collection of maps in $\DCat$ that is closed under $-\tensor N$
for every $N \in \DCat$ i.e. if $M \to M'$ is in $\Cat{W}$ then so is
$M \tensor N \to M' \tensor N$. Then the localization 
$\DCat[\Cat{W}^{-1}]$ is also a tensor triangulated category.
\end{lem}



\begin{thm}\label{thm_dshcor_tensor}
\end{thm}

\begin{defn}\label{def_z_n}
Let $\Ztr(\Gm)$ denote the cokernel of
\[
\Z = \Ztr(\Spec \basefield) \to \Ztr(\A^1 - 0)
\]
given by $\basefield [x, x^{-1}] \to \basefield$, given by
$x \mapsto 1$. Since $\basefield \to \basefield[x, x^{-1}]$
defines a splitting $\Ztr(\A^1 - 0) \simeq \Ztr(\Gm) \oplus \Z$.
Therefore, $\Ztr(\Gm)$ is also a Nisnevich (and \'etale) sheaf 
with transfers. 

More generally, let $\Ztr(\Gmn{n})$ denote cokernel of the map 
\[
\oplus_j \Ztr((\A^1 - 0)^{n - 1}) \xrightarrow{\;\;\sum \phi_j\;\;}
\Ztr((\A^1 - 0)^n)
\]
where $\phi_j$ is induced by the map
\[
\basefield[x_1^{\pm},\dots,x_n^{\pm}] \to \basefield[x_1^{\pm},
\dots, x_m^{\pm}]
\]
given by
\[
x_i \mapsto
\begin{cases}
x_i &\textrm{if } i < j \\
1   &\textrm{if } i = j \\
x_{i - 1} &\textrm{if } i > j.
\end{cases}
\]
Finally, for $n \geq 0$, let $\Z(n)$ denote the complex $\suslC 
\Ztr(\Gmn{n})[-n]$ indexed cohomologically. That is, 
\[
(\Z(n))^i = \begin{cases}
\suslC[n - i]\Ztr(\Gmn{n}) & \textrm{for }i \leq n \\
0 & \textrm{otherwise}
\end{cases}
\]
\end{defn}

%%%%%
%%%%%

\begin{defn}
Let $F \in \HI$. Let $\RHI{F}$ denote presheaf given by
\[
U \mapsto \cok\big( F(U \times \A^1) \to 
   F(U \times (\A^1 - 0))\big).
\]
We write $\RHI[(n + 1)]{F}$ for $\RHI{(\RHI[n]{F})}$. We call 
$\RHI{F}$ the contraction of $F$.
\end{defn}

\begin{prop}\label{prop_contract_is_exact}
The association $F \mapsto \RHI{F}$ is an exact functor from
the category of $\HI$ to $\HI$. (see \cite{MVW} Lecture 23, and
\cite{DegGenMot} Prop. 3.4.3)
\end{prop}

\begin{proof}
The inclusion of $X \simeq X \times \{1\}$ into $X \times 
(\A^1 - 0)$ induces a splitting
\[
F(X) \oplus \RHI{F}(X).
\]
It follows that $\RHI{F}$ is also homotopy invariant. Moreover,
if $F$ is a sheaf, then so is $\RHI{F}$.

Assume that we have an exact sequence of homotopy invariant 
sheaves
\[
0 \to F' \to F \to F'' \to 0.
\]
Write $\tilde{F}$ for the sheaf given by $U \mapsto F(U \times 
(\A^1 - 0))$.

By the Snake Lemma, it suffices to prove that
\[
0 \to \tilde{F}' \to \tilde{F} \to \tilde{F}'' \to 0
\]
is exact. For exactness, it suffices to show that the above 
sequence is exact at a Hensel local scheme $S$. That is, for
Hensel local scheme $S$

\end{proof}

\newpage
\section{The Derived Category of Motives}\label{sect_dmeff_and_dm}

In this section, we define the derived category of motives $\DMeff$,
and show that it is equipped with a closed monoidal structure that
will be used to construct the slice filtration in Section 
\ref{sect_slice_filt_dm}.

To do this, we first define the bounded above derived category 
$\DShCor$ of Nisnevich sheaves, and show that the derived 
category is equipped with a tensor and internal hom functor. We 
then show that $\DMeff$ can be identified as the full subcategory 
of $\DShCor$ with homotopy invariant cohomology. The closed
monoidal structure on $\DMeff$ is therefore obtained from the
closed monoidal structure on $\DShCor$. For the remainder of the
section, unless stated otherwise, all sheaves are Nisnevich 
sheaves. We will drop the ``Nis'', and simply write $\ShCor$ for 
the category of Nisnevich sheaves with transfers.

\begin{defn}\label{def_z_n}
Let $\Ztr(\Gm)$ denote the cokernel of
\[
\Z = \Ztr(\Spec \basefield) \to \Ztr(\A^1 - 0)
\]
given by $\basefield [x, x^{-1}] \to \basefield$, given by
$x \mapsto 1$. Since $\basefield \to \basefield[x, x^{-1}]$
defines a splitting $\Ztr(\A^1 - 0) \simeq \Ztr(\Gm) \oplus \Z$.
Therefore, $\Ztr(\Gm)$ is also a Nisnevich (and \'etale) sheaf 
with transfers.

More generally, let $\Ztr(\Gmn{n})$ denote cokernel of the map 
\[
\oplus_j \Ztr((\A^1 - 0)^{n - 1}) \xrightarrow{\;\;\sum \phi_j\;\;}
\Ztr((\A^1 - 0)^n)
\]
where $\phi_j$ is induced by the map
\[
\basefield[x_1^{\pm},\dots,x_n^{\pm}] \to \basefield[x_1^{\pm},
\dots, x_m^{\pm}]
\]
given by
\[
x_i \mapsto
\begin{cases}
x_i &\textrm{if } i < j \\
1   &\textrm{if } i = j \\
x_{i - 1} &\textrm{if } i > j.
\end{cases}
\]
Finally, for $n \geq 0$, let $\Z(n)$ denote the complex $\suslC 
\Ztr(\Gmn{n})[-n]$ indexed cohomologically. That is, 
\[
(\Z(n))^i = \begin{cases}
\suslC[n - i]\Ztr(\Gmn{n}) & \textrm{for }i \leq n \\
0 & \textrm{otherwise}
\end{cases}
\]
\end{defn}

%%%%%
%%%%%

\begin{defn}
Let $F \in \HI$. Let $\RHI{F}$ denote presheaf given by
\[
U \mapsto \cok\big( F(U \times \A^1) \to 
   F(U \times (\A^1 - 0))\big).
\]
We write $\RHI[(n + 1)]{F}$ for $\RHI{(\RHI[n]{F})}$. We call 
$\RHI{F}$ the contraction of $F$.
\end{defn}

\begin{prop}\label{prop_contract_is_exact}
The association $F \mapsto \RHI{F}$ is an exact functor from
the category of $\HI$ to $\HI$. (see \cite{MVW} Lecture 23, and
\cite{DegGenMot} Prop. 3.4.3)
\end{prop}

\begin{proof}
The inclusion of $X \simeq X \times \{1\}$ into $X \times 
(\A^1 - 0)$ induces a splitting
\[
F(X) \oplus \RHI{F}(X).
\]
It follows that $\RHI{F}$ is also homotopy invariant. Moreover,
if $F$ is a sheaf, then so is $\RHI{F}$.

Assume that we have an exact sequence of homotopy invariant 
sheaves
\[
0 \to F' \to F \to F'' \to 0.
\]
Write $\tilde{F}$ for the sheaf given by $U \mapsto F(U \times 
(\A^1 - 0))$.

By the Snake Lemma, it suffices to prove that
\[
0 \to \tilde{F}' \to \tilde{F} \to \tilde{F}'' \to 0
\]
is exact. For exactness, it suffices to show that the above 
sequence is exact at a Hensel local scheme $S$. That is, for
Hensel local scheme $S$

\end{proof}

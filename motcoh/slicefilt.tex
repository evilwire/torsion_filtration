\newpage
\section{Slice Filtration on $\DMeff$ and $\DM$}\label{sect_slice_filt_dm}

In this section, we construct a categorical filtration on $\DMeff$ 
and $\DM$.

\begin{defn}\label{def_cat_filtration}
Let $\Cat{A}$ be a category. A \DEF{weak filtration} of $\Cat{A}$ 
is a ($\Z$-indexed) sequence of subcategories
\[
\cdots \subseteq \phi_i\Cat{A} \subseteq \phi_{i - 1}\Cat{A} 
   \subseteq \cdots \subseteq \Cat{A}
\]
together with (co)reflection functors $\phi_i: \Cat{A} \to 
\phi_i\Cat{A}$ for each $i$ such that $\phi_i$ restricts to 
identity on $\phi_i\Cat{A}$.
\end{defn}

We now introduce the slice filtration on $\DMeff$. The results here
are based on the work of [HuKaSu]. Let us first fix a more 
convenient set of notations. For each $M \in \DMeff$, set $M(n) = 
l \tDM \Z(n)$, and $M_{-n} = \rhomDMf(\Z(n), M)$. Since $M \mapsto
M(n)$ is left adjoint to $M \mapsto M_{-n}$, we have a 
distinguished triangle 
\begin{equation}\label{slice_triangle_1}
M_{-n}(n) \to M \to M' \to M_{-n}(n)[1],
\end{equation}
for each $n \geq 0$, and some $M'$. 

\begin{prop}
Fixing $n \geq 0$, the $M'$ in the distinguished triangle in
(\ref{slice_triangle_1}) is defined up to unique isomorphism.

In this case, associating $M$ with $M'$ is functorial.
\end{prop}

We write $\slDM{n}$ for this functor, and $\sgDM{n}$ for the
functor $M \mapsto M_{-n}(n)$.

\begin{prop}
For each $n \geq 0$, the functors $\slDM{n}$ and $\sgDM{n}$ are
both triangulated. That is, if
\[
M' \to M \to M'' \to M'[1]
\]
is a distinguished triangle, so are
\[
\sgDM{n}M' \to \sgDM{n}M \to \sgDM{n}M'' \to \sgDM{n}M'[1]
\]
and
\[
\slDM{n}M' \to \slDM{n}M \to \slDM{n}M'' \to \slDM{n}M'[1].
\]
\end{prop}

Let $\sgDM{n}\DMeff$ be the full subcategory of objects of the 
type $M(n)$ where $M \in \DMeff$, and let $\DMeff_{n}$ be the full 
subcategory of $M \in \DMeff$ such that $M_{-n} = 0$.

Furthermore, we have two towers --- one ascending, one descending 
--- of subcategories:
\begin{equation}
\DMeff \subset \sgDM{1}\DMeff \subset \sgDM{2}\DMeff \subset \cdots \\
\end{equation}
\vskip 5pt
\begin{equation}
0 \subset \slDM{1}\DMeff \subset \slDM{2}\DMeff \subset \cdots \subset
\end{equation}

and from this, we have

\begin{prop}
For each $m, n$, with $m < n$, $M \in \DMeff$,

\begin{enumerate}
\item the following is a distinguished triangle
\begin{equation}
\sgDM{n}M \to \sgDM{m}M \to \slDM{n}\sgDM{m}M \to \sgDM{n}M[1].
\end{equation}

\item $\sgDM{n}\sgDM{m} = \sgDM{m}\sgDM{n} = \sgDM{n}$, and 
similarly $\slDM{m}\slDM{n} = \slDM{n}\slDM{m} = \slDM{m}$.

\item $\sgDM{m}\slDM{n} = \slDM{n}\sgDM{m}$.

\item $\sgDM{n}\slDM{m} = \slDM{m}\sgDM{n} = 0$.
\end{enumerate}
\end{prop}

\begin{defn}
For $M \in \DMeff$, we say $\slDM{n}\sgDM{n - 1}M$ is the 
\emph{$n$-th slice} of $M$, written as $\sliceDM{n}M$.
\end{defn}

\begin{prop}
$\sliceDM{n}$ is a triangulated endofunctor for each $n$. That is,
for any distinguished triangle
\[
M' \to M \to M'' \to M'[1]
\]
the following is also a triangle:
\[
\sliceDM{n}M' \to \sliceDM{n}M \to \sliceDM{n}M'' \to \sliceDM{n}M'[1]
\]
\end{prop}

\begin{prop}\label{DMeff_slice_fund_invariant}
For each $n$ and $M$, $\sliceDM{n}M = M'(n)[2n]$ for some $M' \in
\DMeff$.
\end{prop}

\begin{defn}
We call $M'$ in Prop. \ref{DMeff_slice_fund_invariant} the 
\emph{$n$-th fundamental invariant of $M$}, represented by
$\fIDM{n}M$.
\end{defn}

We now relate the slice filtration to the category of birational
motives.

\begin{defn}
A motive $M \in \DMeff$ is \emph{birational} if for any $X \in 
\Sm$, and $U$ a dense open subscheme,
\[
\rhomDMf(\suslC{U}, M) = \rhomDMf(\suslC{X}, M)
\]
We write $\BiMot$ for the full subcategory of birational motives.
\end{defn}

\begin{prop}
$M \in \BiMot$ if and only if $\rhomDMf(Z(1)[1], M) = 0$.
\end{prop}

\begin{prop}
For each $M \in \DMeff$, $\fIDM{n}M \in \BiMot$ for every $n$.
In particular, $\slDM{0}\DMeff = \BiMot$.
\end{prop}

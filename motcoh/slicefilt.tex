\newpage
\section{Slice Filtration on $\DMeff$ and $\DM$}\label{sect_slice_filt_dm}

In this section, we construct a sequence of subcategories on 
$\DMeff$ using the structure on $\DMeff$ defined in the previous
section. To be more precise, consider the following definition.

\begin{defn}\label{def_cat_filtration}
Let $\Cat{A}$ be a category. A \DEF{weak filtration} of $\Cat{A}$ 
is a ($\Z$-indexed) sequence of subcategories
\[
\cdots \subseteq \phi_i\Cat{A} \subseteq \phi_{i - 1}\Cat{A} 
   \subseteq \cdots \subseteq \Cat{A}
\]
together with (co)reflection functors $\phi_i: \Cat{A} \to 
\phi_i\Cat{A}$ for each $i$ such that $\phi_i$ restricts to 
the identity on $\phi_i\Cat{A}$.
\end{defn}

We show that there is a weak filtration on $\DMeff$, and extend 
this filtration to a localization of $\DMeff$. The construction
is based on the work of Voevodsky, Huber, and Kahn.

Before we begin, notice that $\Z(1)$ is an object of $\DMeffgm$.
Hence, $-\tDM \Z(1)$ and $\ihomDMf(\Z(1), -)$ are adjoint pairs
(see Prop. \ref{prop_DMgm_monoidal}). To simplify notations, for 
$M$ in $\DMeff$, we write $\tZ{M}$ for $M \tDM \Z(1)$, and 
$\hZ{M}$ for $\ihomDMf(\Z(1), M)$, and write $\tZ[n]{M}$ and 
$\hZ[n]{M}$ for the $n$-th iterations of applying $- \tDM \Z(1)$ 
and $\ihomDMf(\Z(1), -)$ respectively to $M$. 

Since $\Z(n) \tDM \Z(1) = \Z(n + 1)$, it is easy to see that the 
functor given $M \mapsto \tZ[n]{M}$ is equal to the functor $- 
\tDM \Z(n)$. By adjunction, $M \mapsto \hZ[n]{M}$ is equal to 
$\ihomDMf(\Z(n), -)$. We write $\LDMf{n}$ for the endofunctor $- 
\tDM \Z(n)$, and $\RDMf{n}$ for $\ihomDMf(\Z(n), -)$, and set 
$\LDMf{0}$ and $\RDMf{0}$ to be the identity functor. It is clear 
from the above that for each nonnegative integer $n$, $(\LDMf{n}, 
\RDMf{n})$ form a left-right adjoint pair.

We now introduce the slice filtration on $\DMeff$. For each 
$n \geq 0$, the counit $\LDMf{n}\RDMf{n} \to \id$ gives rise to a 
distinguished triangles of the form
\begin{equation}\label{eq_slice_triangle_1}
\tZ[n]{\hZ[n]{M}} \to M \to M' \to \tZ[n]{\hZ[n]{M}}[1],
\end{equation}
for some $M'$ in $\DMeff$. We call such triangles \DEF{the slice
triangle of $M$}.

By the Five Lemma (see \cite[10.2.2]{WH}), the distinguished 
triangle in \eqref{eq_slice_triangle_1} is defined up to unique 
isomorphism. In fact, associating $M$ with $M'$ is a triangulated 
functor. That is, if
\[
M' \to M \to M'' \to M'[1]
\]
is distinguished, then so are
\[
\sgDM{n}M' \to \sgDM{n}M \to \sgDM{n}M'' \to \sgDM{n}M'[1]
\]
and
\[
\slDM{n}M' \to \slDM{n}M \to \slDM{n}M'' \to \slDM{n}M'[1]
\]
(see \cite[Corollary 1.4]{HuKa}).

Following loc. cit., we write $\slDM{n}$ for the functor $M 
\mapsto M'$, and $\sgDM{n}$ for the functor $M \mapsto 
\tZ[n]{\hZ[n]{M}}$. Furthermore, let $\sgDM{n}\DMeff$ be the full
subcategory of objects of the form $\tZ[n]{M}$, and let 
$\slDM{n} \DMeff$ be the full subcategory of objects $M$ such 
that $M_{-n} = 0$.

Since $\tZ[n + 1]{M} = \tZ[n]{\tZ[1]{M}}$, and if $\hZ[n]{M} = 0$
then $\hZ[(n + 1)]{M} = 0$, we have two towers --- one descending, one 
ascending --- of subcategories:
\begin{equation}
\DMeff \supset \sgDM{1}\DMeff \supset \sgDM{2}\DMeff \supset \cdots \\
\end{equation}
\vskip 5pt
\begin{equation}
0 \subset \slDM{1}\DMeff \subset \slDM{2}\DMeff \subset \cdots \subset
\DMeff
\end{equation}

To show that these subcategories define two filtrations of 
$\DMeff$, we need to define reflection functors $\sgDM{n}\DMeff 
\to \DMeff$ and $\DMeff \to \slDM{n}\DMeff$ for each nonnegative 
integer $n$, such that their restriction to the target categories 
are the identity. In this case, the reflection functors are given 
by $\sgDM{n}$ and $\slDM{n}$ respectively (see loc. cit. Prop. 1.1 
and Corollary 1.4(ii)). All that remains is to see that $\sgDM{n}$ 
restricted to $\sgDM{n}\DMeff$ (resp. $\slDM{n}$ restricted to 
$\slDM{n}\DMeff$) are (naturally isomorphic to) the identity.

Since $\hZ[n]{M} = 0$ for $M$ in $\slDM{n}\DMeff$, $\sgDM{n} M = 
0$. The fact that $\slDM{n} M \cong M$ follows immediately from 
the slice triangle \eqref{eq_slice_triangle_1}. For $\sgDM{n}$, we 
first prove the following:

\begin{prop}
For each $M$ in $\DMeff$ and each nonnegative integer $n$, 
\[
\hZ[n]{\tZ[n]{M}} \cong M.
\]
\end{prop}
\begin{proof}
By Theorem \ref{thm_ALocal_eq_DMeff}, it suffices to verify the
statement for $\A^1$-local complexes. 

Let $F^*$ be the bounded above $\A^1$-local complex corresponding 
to $M$. Notice that by the Cancellation Theorem, reinterpreted for
$\A^1$-local complexes,
\[
\sheaf{\rhom}(\suslC \Z(1), \suslC \Z(1) \tLNis F^*)(U) = 
\rhom(\Ztr(U), F^*) = F^*(U)
\]
for all $U$ in $\Sm$. It follows that $\sheaf{\rhom}(\suslC\Z(1),
\suslC\Z(1) \tLNis F^*) \to F^*$ is an isomorphism, and the
proposition follows.
\end{proof}

To see that $\sgDM{n}$ restricts to the identity on $\sgDM{n} 
\DMeff$, notice that for all $M$ in $\sgDM{n} \DMeff$, $M = 
\tZ[n]{M'}$, and by the proposition above $\sgDM{n}M = 
\tZ[n]{\hZ[n]{\tZ[n]{M'}}} \cong \tZ[n]{M'} = M.$

We summarize the main results of the above discussion in the 
following proposition.

\begin{prop}\label{prop_slice_DMeff}
Let $\sgDM{n} \DMeff$ and $\slDM{n} \DMeff$ be given as above.
Then $\{\sgDM{n} \DMeff\}_{n \geq 0}$ and $\{\slDM{n} \DMeff\}_{n 
\geq 0}$ form respectively a descending and an ascending 
filtrations of $\DMeff$.

In this case, we have triangulated reflection functors 
\[
\sgDM{n}: \DMeff \to \sgDM{n}\DMeff
\] 
defined by $M \mapsto \tZ[n]{\hZ[n]{M}}$, and 
\[
\slDM{n}: \DMeff \to \slDM{n}\DMeff,
\]
which sends $M$ to $M'$ in the slice triangle given in 
\eqref{eq_slice_triangle_1}.
\end{prop}

\begin{defn}
We call the filtration defined by
\[
\DMeff \supset \sgDM{1}\DMeff \supset \sgDM{2}\DMeff \supset \cdots \\
\]
the \DEF{slice filtration} of $\DMeff$, and
\[
0 \subset \slDM{1}\DMeff \subset \slDM{2}\DMeff \subset \cdots \subset
\]
the \DEF{slice cofiltration} of $\DMeff$.
\end{defn}

We now define the slice functors. Before we proceed, we have the
following properties about the functors $\sgDM{n}$ and $\slDM{n}$.

\begin{prop}\label{prop_sDM_properties}
For all nonnegative integers $m, n$, such that $m < n$, and
for all $M$ in $\DMeff$,

\begin{enumerate}
\item the following is a distinguished triangle
\begin{equation}
\sgDM{n}\sgDM{m} M \to \sgDM{m}M \to \slDM{n}\sgDM{m}M \to 
   \sgDM{n}\sgDM{m} M[1].
\end{equation}
The triangle is functorial in $M$.

\item $\sgDM{n}\sgDM{m} = \sgDM{m}\sgDM{n} = \sgDM{n}$.

\item $\sgDM{m}\slDM{n} = \slDM{n}\sgDM{m}$.

\item $\sgDM{n}\slDM{m} = \slDM{m}\sgDM{n} = 0$.

\item $\slDM{m}\slDM{n} = \slDM{n}\slDM{m} = \slDM{m}$.

\item $\tZ[k]{(\sgDM{n}M)} = \sgDM{n + k}\tZ[k]{M}$ for all 
positive integers $k$.

\end{enumerate}
\end{prop}
\begin{proof}
Let $M$ be an arbitrary object of $\DMeff$.

\begin{enumerate}
\item The existence of the triangle follows from setting $M = 
\sgDM{n} M$ in the slice triangle in \eqref{eq_slice_triangle_1}. 
Functoriality in $M$ is immediate.

\item Since $\sgDM{n}M = \tZ[n]{\hZ[n]{M}}$, $\sgDM{m}\sgDM{n} = 
\sgDM{n}\sgDM{m} = \sgDM{n}$ is immediate.

\item Since $\sgDM{n}$ is a triangulated functor and by part (1),
we have
\[
\begin{tikzcd}
\sgDM{n} \sgDM{m} M \arrow{r} \arrow{d}{\cong} &
\sgDM{m} M \arrow{r} \arrow[equals]{d} &
\slDM{n} \sgDM{m} M \arrow{r}{+1} \arrow[dotted]{d} &
\slDM{n} \sgDM{m} M[1] \arrow{d}{\cong}\\
\sgDM{m} \sgDM{n} M \arrow{r} &
\sgDM{m} M \arrow{r} &
\sgDM{m} \slDM{n} M \arrow{r}{+1} &
\sgDM{m} \sgDM{n} M[1].
\end{tikzcd}
\]
It follows by the Five Lemma (\cite[10.2.2]{WH}) that 
\[
\sgDM{m}\slDM{n} M \simeq \slDM{n}\sgDM{m} M.
\]
Functoriality in $M$ of the rows implies that that this 
isomorphism is natural in $M$. That is, $\sgDM{m} \slDM{n}$ is 
naturally isomorphic to $\slDM{n} \sgDM{m}$.

\item Since $\hZ[n]{(\slDM{m})} = 0$, it is clear that
$\sgDM{n}\slDM{m} = 0$. On the other hand, by (2), $\sgDM{m}\sgDM{n}
= \sgDM{n}$. From the slice triangle in part (1)
\[
\sgDM{m} \sgDM{n} \to \sgDM{n} \to \slDM{m} \sgDM{n} \to sgDM{m} 
   \sgDM{n}[1]
\]
it follows that $\slDM{n} \sgDM{m} = 0$.

\item Applying the slice triangle \eqref{eq_slice_triangle_1} to
$\slDM{n} M$, we have
\[
\sgDM{m} \slDM{n} M \to \slDM{n} M \to \slDM{m} \slDM{n} M \to
\sgDM{m} \slDM{n} M[1].
\]
Applying $\slDM{n}$ to the slice triangle of $M$ gives:
\[
\slDM{n} \sgDM{m} M \to \slDM{n} M \to \slDM{n} \slDM{m} M \to
\slDM{n} \sgDM{m} M[1],
\]
whence, by (3) we have
\[
\begin{tikzcd}
\sgDM{m} \slDM{n} M \arrow{r} \arrow{d}{\cong} &
\slDM{n} M \arrow{r} \arrow[equals]{d} &
\slDM{m} \slDM{n} M \arrow{r}{+1} \arrow[dotted]{d} &
\sgDM{m} \slDM{n} M[1] \arrow{d}{\cong}\\
\slDM{n} \sgDM{m} M \arrow{r} &
\slDM{n} M \arrow{r} &
\slDM{n} \slDM{m} M \arrow{r}{+1} &
\slDM{n} \sgDM{m} M[1].
\end{tikzcd}
\]
The fact that $\slDM{n} \slDM{m} M \cong \slDM{m} \slDM{n} M$
follows from the Five Lemma. Naturality in $M$ now follows from
naturality in (1 - 3).

\item The case $k = 1$ is established in \cite[Cor. 1.4]{HuKa}.
The general case follows by induction on $k$.

\end{enumerate}
\end{proof}

Notice that, from the above, we have the following distinguished
triangle
\begin{equation}
\sgDM{n} M \to \sgDM{n - 1} M \to \slDM{n} \sgDM{n - 1} M
\to \sgDM{n} M[1].
\end{equation}

\begin{defn}
For $M$ in $\DMeff$, we say $\slDM{n}\sgDM{n - 1}M$ is the 
\emph{$n$-th slice} of $M$, written as $\sliceDM{n}M$. Since
$\slDM{n}$ and $\sgDM{n}$ are triangulated functors, so is 
$\sliceDM{n}$.
\end{defn}

By Prop. \ref{prop_sDM_properties} (3), $\sliceDM{n} =
\sgDM{n}\slDM{n - 1}$. In particular, the essential image of
$\sliceDM{n}$ is in $\sgDM{n}\DMeff$. That is, for all $M$ in 
$\DMeff$, $\slice{n}M = \tZ[n]{M'}$. In particular, the following
holds:

\begin{prop}\label{prop_DMeff_slice_fund_invariant}
For each $n$ and $M$, $\sliceDM{n}M = M'(n)[2n]$ for some $M'$ in
$\DMeff$.
\end{prop}

\begin{defn}\label{def_fI_DMeff}
We call $M'$ in Prop. \ref{prop_DMeff_slice_fund_invariant} the 
\emph{$n$-th fundamental invariant of $M$}, represented by
$\fIDM{n}M$.
\end{defn}

\begin{ex}\label{ex_sfilt_MPn}
It is clear that $\Z(n)$ is its own $n$-th slice. Furthermore,
since $M(\P^n) = \oplus_{i = 0}^n \Z(n)[2n]$ (see 
\cite[Corollary 15.5]{MVW}), it is easy to verify that 
\[
\slDM{k}M(\P^n) = \begin{cases}
M(\P^k) & \textrm{if }k \leq n \\
M(\P^n) & \textrm{otherwise}
\end{cases}
\]
and 
\[
\sgDM{k}M(\P^k) = \begin{cases}
\tZ[k]{M(\P^{n - k})} & \textrm{if }k \leq n\\
0 & \textrm{otherwise}.
\end{cases}
\]
Therefore, $\sliceDM{k} M(\P^n) = \Z(k)[2k]$ and thus the 
fundamental invariant of $M(\P^n)$ is $\fIDM{k}M(\P^n) = \Z$.
\end{ex}

\begin{defn}
Let $\DM$ be the category obtained from $\DMeff$ by inverting
the operation $M \mapsto \tZ{M}$. That is, the objects of $\DM$
are a pair $(M, n)$, where $M$ is an object of $\DMeff$, and $n$
is any integer, such that $(\tZ{M}, n) \cong (M, n + 1)$; the set 
of morphism between $(M, n)$ and $(M', n')$ is 
\[
\varprojlim_{k} \homDMf(\tZ[k + n]{M}, \tZ[k + n']{M'}).
\]
as $k$ ranges over all values for which $k + n$ and $k + n'$ are
positive. We write $\homDM((M, n), (M',n'))$ for the hom set of 
$(M, n)$ and $(M', n')$. 
\end{defn}

\begin{rmk}
By induction, we have that $(M, n) \cong (M \tDM \Z(n), 0)$, for 
any positive integer $n$ and all $M$ in $\DMeff$. In particular,
$(M, n) \cong (M', n')$ for $n \geq n'$, then $M \simeq 
\tZ[n - n']M'$.
\end{rmk}

\begin{rmk}\label{rmk_homs_in_DM}
By the Cancellation Theorem, 
\[
\homDMf(M, M') = \homDMf(\tZ[n]{M}, \tZ[n]{M'})
\]
for all positive integers. Therefore, the projective limit in the 
definition of $\homDM$ is a finite limit. That is, it suffices to 
take $k > |n| + |n'|$, say.

Furthermore, the localization functor $\DMeff \to \DM$, given by
sending $M$ in $\DMeff$ to $(M, 0)$ is fully faithful. Therefore, we 
can identify $\DMeff$ as the full subcategory of $\DM$.

Finally, let $R$ denote the association given by $(M, n) \mapsto
(M, n - 1)$. It is clear that $R$ is functorial. Now, for $\phi: 
M \to M'$ in $\DMeff$, $\phi$ induces a map $\phi_*: (M, 0)
\to (M', 0)$. Then $R^n\phi_* : (M, n) \to (M', n)$ is identified 
with $\phi$ in $\homDM((M, n), (M', n)) = \homDMf(M, M')$.
\end{rmk}

We now describe the slice filtration and cofiltration structure on 
$\DM$. We assume each subcategory in the slice filtration to be 
full, and we describe only the objects in these subcategories. 
First, if $k \geq 0$, let the objects of $\sgDM{k} \DM$ be those 
$(M, 0)$ for which $M \in \sgDM{k} \DMeff$. For $k < 0$, let the 
objects of $\sgDM{k} \DM$ consist of objects $(M, n)$ for which 
$n \geq k$. It is clear that we have a tower of subcategories

\begin{equation}\label{eq_DM_slice_filt}
\DM \supset \cdots \sgDM{-1}\DM \supset \sgDM{0}\DM = \DMeff 
   \supset \sgDM{1}\DM \supset \cdots.
\end{equation}

To show that this tower of subcategories constitutes a filtration
of $\DM$, we construct an extension of the functors $\sgDM{k}$.
Define, by convention, $\sgDM{n} M = M$ for all nonpositive 
integer $n$, and all $M$ in $\DMeff$. Then, define
\[
\sgDM{k}(M, n) \defeq (\sgDM{k - n}M, n).
\]
This definition is independent of the choice of $n$. Indeed, if
$(M, n) = (M', n')$ for some integer $n'$ less than $n$, say, then
$\tZ[n - n']{M} = M'$, and $\sgDM{k - n'}M' = \sgDM{k - n}M$ by
(6) of Prop. \ref{prop_sDM_properties}. Hence, $(\sgDM{k - n'}M', 
n') \cong (\sgDM{k - n}M, n)$.

We show that $\sgDM{k}$ is a triangulated functor which satisfy 
the condition of a filtration. We verify this claim in the 
Prop. \ref{prop_sDM_reflection}. We first prove the following
lemma:

\begin{lem}\label{lem_triangle_in_DM}
If $(M_1, n_1) \to (M_2, n_2) \to (M_3, n_3) \to (M_1, n_1)[1]$
is a distinguished triangle in $\DM$ such that $n_1 = n_2 = n$, 
then there exists some $M_4$ such that $(M_4, n) = (M_3, n_3)$.
\end{lem}
\begin{proof}
Let $\phi$ denote the map from $(M_1, n)$ to $(M_2, n)$. Then
$\phi$ is identified with some map $\phi': M_1 \to M_2$ in 
$\DMeff$. Let
\[
M_1 \to M_2 \to M \to M_1[1]
\]
be its corresponding triangle. Then, we have
\[
\begin{tikzcd}
(M_1, n) \arrow{r}{\phi} \arrow[equals]{d} &
(M_2, n) \arrow{r} \arrow[equals]{d} &
(M_3, n_3) \arrow{r} \arrow[dotted]{d} &
(M_1, n)[1] \arrow[equals]{d}\\
(M_1, n) \arrow{r}{\phi} &
(M_2, n) \arrow{r} &
(M, n) \arrow{r} &
(M_1, n)[+1].
\end{tikzcd}
\]
The claim now follows form the Five Lemma.
\end{proof}

\begin{prop}\label{prop_sDM_reflection}
Let $k$ be an arbitrary integer.

\begin{enumerate}
\item the association $(M, n) \mapsto \sgDM{n} (M, n)$ is 
a triangulated functor.

\item $\sgDM{k}$ is a right adjoint to the inclusion of 
$\sgDM{k}\DM$ into $\DM$.

\item the restriction of $\sgDM{k}$ to $\sgDM{k} \DM$ is the 
identity.
\end{enumerate}
\end{prop}
\begin{proof}
We restrict our attention to the full subcategory $\DM$ of objects 
$(M, n)$ for which $k > n$, (since otherwise, $\sgDM{k}$ acts as 
the identity, which is clearly satisfies the properties above).

In this case, (1 - 3) follows straightforwardly from the fact that
$\sgDM{k - n}$ satisfies (1 - 3) for $\DMeff$ (see the discussion
preceding Prop. \ref{prop_slice_DMeff}).
\end{proof}

Next, we define $\slDM{k} \DM$ to be the full subcategory of 
object $(M, n)$ in $\DM$ for which $\sgDM{k}(M, n) = 0$. It is
clear that we have the following tower of subcategories:
\begin{equation}\label{eq_DM_slice_cofilt}
\cdots \subset \slDM{0}\DM \subset \slDM{1}\DM \subset \slDM{2}\DM 
   \subset \cdots \subset \DM.
\end{equation}
As expected, the tower also defines a filtration of $\DM$, and
we define the reflection functors $\slDM{k} : \DM \to \slDM{k} \DM$.
Notice that for $(M, n)$ in $\DM$, we have the following
\begin{equation}\label{eq_DM_slice_triangle}
\sgDM{k} (M, n) \to (M, n) \to (M', n') \to \sgDM{k} (M, n)[1].
\end{equation}

Before we proceed, we define an extension of the endofunctor 
$\slDM{k}$ of $\DMeff$ to nonpositive integers $k$ as we did for 
$\sgDM{k}$. Set $\slDM{k} M = 0$ for $k \leq 0$.

\begin{lem}
For each integer $k$, the object $(M', n')$ in 
\eqref{eq_DM_slice_triangle} is defined up to unique isomorphism. 
In particular, $M' = \slDM{k - n} M$.
\end{lem}
\begin{proof}
By definition, $\sgDM{k}(M, n) = (\sgDM{k - n}M, n)$. It follows 
from Lemma \ref{lem_triangle_in_DM} that we may assume $n' = n$. 
The Lemma now follows from the uniqueness of $M'$.
\end{proof}

Let $\slDM{k}$ denote the association $(M, n) \mapsto (M', n') =
(\slDM{k - n}M, n)$. The following is a straightforward consequence
of the properties of the $\sgDM{k}$ of $\DMeff$:

\begin{prop}
For each integer $k$,
\begin{enumerate}
\item $\slDM{k}$ is a triangulated functor

\item $\slDM{k}$ defines a left adjoint to the inclusion of $\slDM{k}\DM$
into $\DM$.

\item the restriction of $\slDM{k}$ to $\slDM{k}\DM$ is the identity.
\end{enumerate}
\end{prop}

It follows that the towers of subcategories given in 
\eqref{eq_DM_slice_filt} and \eqref{eq_DM_slice_cofilt} 
respectively define a descending and an ascending filtrations on 
$\DM$.

We can also extend the definition of the fundamental invariants
$\fIDM{n}$ to negative integers $n$. Notice that for each $(M, n)$
in $\DM$, and each integer $k$, we have the slice triangle:
\[
\sgDM{k + 1}(M, n) \to \sgDM{k}(M, n) \to \slDM{k + 1}\sgDM{k}(M, n)
\to \sgDM{k + 1}(M, n)[1].
\]

\begin{defn}
We define $\slDM{k + 1}\sgDM{k}(M, n)$ be the the \DEF{$k$-th 
slice} of the slice filtration on $\DM$. Since both $\slDM{k + 1}$ 
and $\sgDM{k}$ are triangulated functors, so is $\slDM{k + 1}
\sgDM{k}$. We write $\sliceDM{k}$ for this functor. 

By similar arguments as in Prop. 
\ref{prop_DMeff_slice_fund_invariant}, we have that 
$\sliceDM{k}(M, n) = (M'[2k], k)$ for some $M'$ in $\DMeff$. We 
define the \DEF{$k$-th fundamental invariant} of $(M, n)$ to be
\[
\fIDM{k}(M, n) \defeq M'.
\]
\end{defn}

We conclude this section by discussing the relationship between the
tensor structure on $\DMeff$ and $\DM$ and their respective slice 
filtrations. First, we introduce a tensor product on $\DM$ as an 
extension of the one on $\DMeff$, which we represent by $\tDM$ as
well.

For $(M, n), (M'. n')$ in $\DM$, define 
\[
(M, n) \tDM (M', n') = (M \tDM M', n + n'). 
\]
Since $\tDM$ is additive and symmetric, so is its extension to 
$\DM$. In fact, $(\DM, \tDM)$ is an additive symmetric monoidal 
triangulated category.

For $\DMeff$, we have the following graded tensor structure:

\begin{prop}[\cite{HuKa} Corollary 1.6]\label{prop_tDM_sfilt_DMeff}
For nonnegative integers $n, n'$, there exists a unique natural
transformation $\sgDM{n} \tDM \sgDM{n'} \to \sgDM{n + n'}(- 
\tDM -)$ compatible with the tensor structure on $\DMeff$. That 
is, we have the following commutative square for each $M$ and
$M'$ in $\DMeff$:
\[
\begin{tikzcd}
\sgDM{n}(M) \tDM \sgDM{n'}(M') \arrow{r} \arrow{d} &
\sgDM{n + n'}(M \tDM M') \arrow{d} \\
M \tDM M' \arrow[equals]{r} &
M \tDM M'.
\end{tikzcd}
\]

Furthermore, $\sgDM{n} \tDM \sgDM{n'} = \sgDM{n + n'}$. 
\end{prop}

We can extend this result to $\DM$. The following is a
straightforward consequence of the preceding proposition.

\begin{cor}\label{cor_tDM_sfilt_DM}
For all integers $n, n'$, there exists a unique natural 
transformation of bifunctors on $\sgDM{n} \tDM \sgDM{n'} \to 
\sgDM{n + n'}$ compatible with the tensor structure of $\DM$.
\end{cor}

A corollary of the proposition above applies to the tensor
structure on the slices (and similarly, on the fundamental 
invariants) of the slice filtration.

\begin{cor}
For all integers nonnegative $n, n'$, there exists unique
natural transformations of bifunctors $\sliceDM{n} \tDM 
\sliceDM{n'} \to \sliceDM{n + n'}$ and $\fIDM{n} \tDM
\fIDM{n'} \to \fIDM{n + n'}$ compatible with the tensor structure
on $\DMeff$.

The natural transformations can be extended to natural 
transformations on the slice structure on $\DM$: we have natural 
transformations $\sliceDM{n} \tDM \sliceDM{n'} \to 
\sliceDM{n + n'}$ and $\fIDM{n} \tDM \fIDM{n'} \to \fIDM{n + n'}$
compatible with the tensor structure on $\DM$.
\end{cor}

\begin{rmk}
Notice that the fundamental invariants $\fIDM{k}$ of the slice 
filtration on $\DM$ always takes value in $\DMeff$. More 
specifically, the fundamental invariant always takes value in the 
full subcategory of birational motives defined in \cite{KaSu}. 
This is established for the fundamental invariants for $\DMeff$ 
in \cite[Section 2]{HuKa}, and can be extended directly to $\DM$.
\end{rmk}

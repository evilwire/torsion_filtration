\newpage
\chapter{Slice Filtration on $\DMeff$ and $\DM$}\label{sect_slice_filt_dm}

In this chapter, we construct a sequence of subcategories on 
$\DMeff$ using the tensor and the partial internal hom structure 
on $\DMeff$ defined in the previous chapter (see Section 
\ref{sect_TMS_DMeff}). In order to be more precise, we introduce
the following notion.

\begin{defn}\label{def_cat_filtration}
Let $\Cat{A}$ be a category. A descending \DEF{weak filtration} of 
$\Cat{A}$ is a ($\Z$-indexed) sequence of subcategories
\[
\Cat{A} \cdots \supseteq \Cat{A}_i \supseteq \Cat{A}_{i + 1}
   \supseteq \cdots \supseteq 
\]
together with coreflection functors $\phi_i: \Cat{A} \to \Cat{A}_i$
for each $i$ such that $\phi_i$ restricts to the identity on
$\Cat{A}_i$. One can similarly define ascending weak filtrations using
reflections $\Cat{A} \to \phi_n\Cat{A}$. We will represent a weak
filtration as $(\Cat{A}_*, \phi_*)$, where $\Cat{A}_i$ are the
subcategories and $\phi_i$ are the reflection/coreflections.

We say that a weak filtration $(\Cat{A}_*, \phi_*)$ is 
\DEF{degenerate} if all subcategories $\Cat{A}_n$ are equal. If
$\Cat{A}$ has a zero object, then we say that $(\Cat{A}_*, \phi_*)$
is \DEF{trivial} if each $\Cat{A}_n$ consists of only the zero object.
\end{defn}

\begin{rmk}
An $\N$-indexed descending weak filtration is just a $\Z$-indexed 
descending weak filtration such that $\Cat{A}_j = \Cat{A}$
for all $j \leq 0$. Likewise, an $\N$-indexed ascending 
weak filtration is an ascending weak filtration for which
$\Cat{A}_j = \Cat{A}_0$ for all negative $j$.
\end{rmk}

We show that there are two $\N$-indexed weak filtrations --- 
one ascending, one descending --- on $\DMeff$. The construction 
is based on the work of Voevodsky, Huber, and Kahn \cite{HuKa}. 
We then extend the weak filtration on $\DMeff$ to a $\Z$-indexed
weak filtration on $\DM$. Aside from Lemmas \ref{lem_LRcommute} 
and \ref{lem_LR_commute_LR} and Propositions 
\ref{prop_sgn_commute_sgm} and \ref{prop_sDM_properties}, the 
content from the first two sections is taken from 
\cite[\S 1]{HuKa}. The extensions of the filtrations to $\DM$ are 
new.

\section{Slice filtration on $\DMeff$}
\label{sect_slice_filtration_DMeff}
To simplify notations, following Chapter \ref{sect_dmeff_and_dm},
we write $\tZ[n]{M}$ for $M \tDM \Z(n)$, and $\hZ[n]{M}$ for 
$\ihomDMf(\Z(n), M)$. Furthermore, let $\LDMf{n}$ denote the
functor $- \tDM \Z(n)$, and $\RDMf{n}$ denote the functor 
$\ihomDMf(\Z(n), -)$. By convention, define $\LDMf{0}$ and 
$\RDMf{0}$ to be the identity functor. As we have noted in
the paragraph preceding Theorem \ref{thm_dm_cancellation}, 
$(\LDMf{n}, \RDMf{n})$ form an adjoint pair of triangulated 
functors for each natural numbers $n$.

We first describe the descending weak filtrations on $\DMeff$. Fix
a natural number $n$, let $\GFiltDM[n]{\DMeff}$ be the full 
subcategory of objects of the form $\tZ[n]{M}$ for some $M$ in 
$\DMeff$, and let $\LFiltDM[n]{\DMeff}$ be the full subcategory of 
objects $M$ such that $\hZ[n]{M} = 0$.

Since $\tZ[n + 1]{M} = \tZ[n]{\tZ[1]{M}}$ and $\hZ[n]{M} = 0$ 
implies $\hZ[(n + 1)]{M} = 0$, we have the following towers of 
subcategories:
\begin{equation}\label{eq_gslice_tower_DMeff}
\DMeff = \GFiltDM[0]{\DMeff} \supseteq \GFiltDM[1]{\DMeff} 
   \supseteq \GFiltDM[2]{\DMeff} \supseteq \cdots \supseteq 0
\end{equation}
\begin{equation}\label{eq_lslice_tower_DMeff}
0 = \LFiltDM[0]{\DMeff} \subseteq \LFiltDM[1]{\DMeff} \subseteq 
   \LFiltDM[2]{\DMeff} \subseteq \cdots \subseteq \DMeff
\end{equation}

For each $n$, $L^n R^n : \DMeff \to \GFiltDM[n]{\DMeff}$ is right
adjoint to the inclusion of $\GFiltDM[n]{\DMeff}$ into $\DMeff$
(see \cite[1.1]{HuKa}). Moreover, by Corollary 
\ref{cor_tZ_hZ_eq_id}, $R^nL^n \cong \id$. Since an object $M$ in 
$\GFiltDM[n]{\DMeff}$ is of the form $\tZ[n]{M'}$ for some $M'$ in 
$\DMeff$, and
\[
L^nR^n M = L^nR^n \tZ[n]{M'} \cong L^n M' = M,
\]
the functor $L^nR^n$ is naturally isomorphic to the identity on
$\GFiltDM[n]{\DMeff}$. 

\begin{defn}\label{def_sgDM}
Following \cite{HuKa}, let $\sgDM{n}$ denote the triangulated functor
$L^nR^n$. Thus, $\sgDM{n}M=\LHI[n]{\RHI[n]{M}}$. 
\end{defn}
\comment{
By the Cancellation Theorem \ref{thm_dm_cancellation}, the unit
$\LHI[n]{M} \to \LHI[n]{\RHI[n]{\LHI[n]{M}}}$ is a natural 
isomorphism. Since objects $\GFiltDM[n]{\DMeff}$ are the objects
of the form $\LHI[n]{M}$, it follows that $\sgDM{n}$ restricted
to $\GFiltDM[n]{\DMeff}$ is naturally isomorphic to the identity.
}

Furthermore, for each $M$ in $\DMeff$, there exists some $M'$ in 
$\DMeff$ such that there is a distinguished triangle:
\begin{equation}\label{eq_slice_triangle_1}
\sgDM{n}M \to M \to M' \to \sgDM{n}M[1].
\end{equation}
By \cite[1.4(i, ii)]{HuKa}, $M'$ is uniquely defined up to 
unique isomorphism, and there is a cohomological functor 
$\slDM{n}$ given by $M \mapsto M'$, which is the left adjoint to 
the inclusion of $\LFiltDM[n]{\DMeff}$ in $\DMeff$. If $M$ is in 
$\LFiltDM[n]{\DMeff}$, then $\sgDM{n} M = \LHI[n]{\RHI[n]{M}} = 
0$. Since \eqref{eq_slice_triangle_1} is distinguished, $M \cong 
\slDM{n} M$. To show that this isomorphism is natural in $M$, we 
will prove the stronger result that \eqref{eq_slice_triangle_1} is 
natural in $M$.

Fix a map $f: M \to M'$ in $\DMeff$. By the naturality of the 
counit $\epsilon: \sgDM{n} \to \id$, we have the following 
commutative square:
\[
\begin{tikzcd}
\sgDM{n}M \arrow{r}{\epsilon_M} \arrow{d}{\sgDM{n}f} &
M \arrow{d}{f} \\
\sgDM{n}M' \arrow{r}{\epsilon_{M'}} &
M'.
\end{tikzcd}
\]
Completing the rows of the square into distinguished triangles,
we have the following commutative diagram
\[
\begin{tikzcd}
\sgDM{n}M \arrow{r}{\epsilon_M} \arrow{d}{\sgDM{n}f} &
M \arrow{d}{f} \arrow{r} &
\slDM{n}M \arrow[dotted]{d} \arrow{r} &
\sgDM{n}M[1] \arrow{d} \\
\sgDM{n}M' \arrow{r}{\epsilon_{M'}} &
M' \arrow{r} &
\slDM{n}M' \arrow{r} &
\sgDM{n}M'[1]
\end{tikzcd}
\]
Since $\slDM{n}$ is left adjoint to the inclusion of 
$\LFiltDM[n]{\DMeff}$ into $\DMeff$,
\[
\homDMf(M, \slDM{n}M') \cong \homDMf(\slDM{n}M, \slDM{n}M').
\]
Hence, the induced map from $\slDM{n} M \to \slDM{n} M'$ (the 
dotted arrow in the diagram above) is $\slDM{n} f$. We summarize 
the main results of the above discussion in the following 
proposition:

\begin{prop}\label{prop_slice_DMeff}
The tower of subcategories in \eqref{eq_gslice_tower_DMeff} 
defines a descending weak filtration of $\DMeff$, where the
coreflection functors
\[
\sgDM{n}: \DMeff \to \GFiltDM[n]{\DMeff}
\] 
are defined by $M \mapsto \tZ[n]{\hZ[n]{M}}$.

Furthermore, the tower in \eqref{eq_lslice_tower_DMeff}
defines an ascending weak filtration of $\DMeff$, with
reflection functors 
\[
\slDM{n}: \DMeff \to \LFiltDM[n]{\DMeff},
\]
defined by sending $M$ to $M'$ in the triangle given in 
\eqref{eq_slice_triangle_1}.
\end{prop}

We call the pair of weak filtrations associated with the
towers in \eqref{eq_gslice_tower_DMeff} and 
\eqref{eq_lslice_tower_DMeff} the \DEF{slice filtration on 
$\DMeff$}.

Notice that, by replacing $M$ by $\sgDM{n} M$ in 
\eqref{eq_slice_triangle_1}, we get distinguished triangles for 
all positive integers $m$ and $n$, all of which are natural in 
$M$:
\begin{equation}\label{eq_slice_triangle_2}
\sgDM{n}\sgDM{m} M \to \sgDM{m}M \to \slDM{n}\sgDM{m}M \to 
   \sgDM{n}\sgDM{m} M[1].
\end{equation}

\section{Fundamental invariants of the slice filtration}

In this section, following \cite[1.4 (iv, v)]{HuKa}, we define
the slice and fundamental invariant functors associated to the
slice filtration on $\DMeff$. Before we do so, we will show that 
the functors $\sgDM{n}$ and $\slDM{n}$ satisfy a number of 
properties described in Proposition \ref{prop_sDM_properties}.
Lemmas \ref{lem_LRcommute} and \ref{lem_LR_commute_LR} and 
Propositions \ref{prop_sgn_commute_sgm} and 
\ref{prop_sDM_properties} are new.

We first digress to discuss two results from category theory.
For the following, let $L$ and $R$ be a pair of adjoint 
endofunctors on $\Cat{C}$, and suppose that the unit $\eta : \id 
\to RL$ is a natural isomorphism. Write $L^n$ and $R^n$ for the 
$n$-th iteration of $L$ and $R$ respectively. Since $L$ and $R$ 
are adjoint functors, so are $L^n$ and $R^n$. Write $\cuHI^n$ for 
the counit $L^n R^n \to \id$ and $\eta^n$ for the unit $\id \to 
R^n L^n$. In this case, $\eta^n$ is also a natural isomorphism for 
each positive integer $n$. 

\begin{lem}\label{lem_LRcommute}
For each positive integer $n$, the natural isomorphism 
$(L^{n + 1} R^n \eta)^{-1}: L^{n + 1}R^{n + 1} L \to
L(L^n R^n)$ fits into the following commutative diagram of
natural transformations:
\begin{equation}\label{eq_Ltl_com_diag}
\begin{tikzcd}[column sep=80pt]
L^{n + 1}R^{n + 1}L \arrow{r}{\epsilon^{n + 1}L} \arrow{d} &
L \arrow[equals]{d} \\
L(L^{n}R^n) \arrow{r}{L\epsilon^{n}} &
L.
\end{tikzcd}
\end{equation}
Dually, the natural isomorphism $\eta L^nR^{n + 1}:
L^nR^n R \to R(L^{n + 1}R^{n + 1})$ fits into the following
commutative diagram of natural transformations:
\begin{equation}\label{eq_Rtl_com_diag}
\begin{tikzcd}[column sep=80pt]
L^nR^nR \arrow{r}{\epsilon^n R} 
   \arrow{d}{\eta} &
R \arrow[equals]{d} \\
R (L^{n + 1} R^{n + 1}) \arrow{r}{R \epsilon^{n + 1}} &
R.
\end{tikzcd}
\end{equation}
\end{lem}
\begin{proof}
We first show that \eqref{eq_Ltl_com_diag} is commutative.
To do so, we proceed by induction on $n$. For the case $n = 0$, 
by the counit-unit adjunction, the following composition is
the identity transformation:
\[
L \stackrel{L\eta}{\to} LRL \stackrel{\epsilon L}{\to} 
   L.
\]
Therefore, $\epsilon L = L(\eta^{-1})$, and the
following diagram commutes:
\[
\begin{tikzcd}
LRL \arrow{r}{\epsilon L} \arrow{d}{L(\eta^{-1})} &
L \arrow[equals]{d} \\
L \arrow[equals]{r} & L.
\end{tikzcd}
\]

Now assume that for some integer $n$, the following diagram is 
commutative:
\begin{equation}\label{eq_induct_hyp_tl_diag}
\begin{tikzcd}[column sep=80pt]
L^{n}R^{n}L \arrow{r}{\epsilon^{n}L} \arrow{d}{L^nR^{n - 1}\eta^{-1}} &
L \arrow[equals]{d} \\
L(L^{n - 1}R^{n - 1}) \arrow{r}{L\epsilon^{n - 1}} &
L.
\end{tikzcd}
\end{equation}
Write $\epsilon'$ for the natural transformation $L^n \epsilon R^n:
L^n R^n \to L^{n - 1} R^{n - 1}$. Applying the naturality of 
$\epsilon'$ to the natural isomorphism $\eta^{-1}: RL \to \id$, we 
have the following commutative diagram
\[
\begin{tikzcd}[column sep=80pt]
L^n R^n RL \arrow{r}{\epsilon'RL} 
   \arrow{d}{L^nR^n \eta^{-1}} &
L^{n - 1}R^{n - 1} RL
   \arrow{d}{L^{n - 1}R^{n - 1}\eta^{-1}} \\
L^nR^n \arrow{r}{\epsilon'} &
L^{n - 1}R^{n - 1}.
\end{tikzcd}
\]
Now apply $L$ to the above, we have
\[
\begin{tikzcd}[column sep=80pt]
L^{n + 1} R^{n + 1} L \arrow{r}{L\epsilon'RL} 
   \arrow{d}{L^{n + 1}R^n \eta^{-1}} &
L^{n}R^n L \arrow{d}{L^nR^{n - 1}\eta^{-1}} \\
L^{n + 1}R^n \arrow{r}{L\epsilon'} &
L^{n}R^{n - 1},
\end{tikzcd}
\]
which fits together with \eqref{eq_induct_hyp_tl_diag} to give the 
following commutative diagram:
\[
\begin{tikzcd}[column sep=65pt]
L^{n + 1} R^{n + 1} L \arrow{r}{L\epsilon' RL} 
   \arrow{d}{L^{n + 1}R^n(\eta^{-1})} &
L^{n}R^n L \arrow{d}{L^nR^{n - 1}(\eta^{-1})}
   \arrow{r}{\epsilon^{n}} &
L \arrow[equals]{d} \\
L^{n + 1}R^n \arrow{r}{L\epsilon'} &
L^{n}R^{n - 1} \arrow{r}{L\epsilon^{n - 1}} &
L.
\end{tikzcd}
\]
Notice that $\epsilon^n \comp L\epsilon'R = \eta^{n + 1}$ and 
$\epsilon^{n - 1} \comp \eta' = \epsilon^{n}$. Therefore, in the 
diagram above, the composition of the two maps in the top row
is precisely $\epsilon^{n + 1} L$ and the composition of in the
bottom row is precisely $L \epsilon^n$. By induction, the 
commutativity of \eqref{eq_Ltl_com_diag} is established. The 
commutativity of \eqref{eq_Rtl_com_diag} follows by similar 
arguments.
\end{proof}

\begin{lem}\label{lem_LR_commute_LR}
For all positive integers $n$ and $m$, there exists a natural 
isomorphism $\tau : L^nR^nL^mR^m \to L^mR^mL^nR^n$ such that the 
following is a commutative diagram of natural transformations:
\begin{equation}\label{eq_lem_LR_commute_LR}
\begin{tikzcd}[column sep=75pt]
L^n R^n L^m R^m \arrow{r}{L^nR^n \epsilon^m} \arrow{d}{\tau} &
L^nR^n \arrow{d} \\
L^mR^m L^nR^n \arrow{r}{\epsilon^m L^nR^n} &
L^nR^n
\end{tikzcd}
\end{equation}
\end{lem}
\begin{proof}
We first consider the case $m \leq n$. By the counit-unit 
adjunction, the composition
\[
R^m \xrightarrow{\eta^m R^m}
R^m L^m R^m \xrightarrow{R^m \cuHI^m}
R^m
\]
is the identity transformation. Applying $L^nR^{n - m}$ to the 
above, we obtain the following commutative square:
\[
\begin{tikzcd}[column sep=75pt]
L^nR^{n - m}R^mL^mR^m \arrow{r}{L^nR^n\cuHI^m} 
   \arrow{d}{(L^nR^{n - m}\eta^m R^m)^{-1}} &
L^nR^{n - m}R^m \arrow[equals]{d} \\
L^nR^{n - m}R^m \arrow[equals]{r} &
L^nR^n.
\end{tikzcd}
\]

Similarly, by the unit-counit adjunction, the compositions
\[
L^m \xrightarrow{L^m \eta^m} L^m R^m L^m 
   \xrightarrow{\cuHI^m L^m} L^m
\]
is also the identity transformation. Applying the above to 
$L^{n - m}R^n$, we obtain the following commutative square:
\[
\begin{tikzcd}[column sep=75pt]
L^mL^{n - m}R^n \arrow[equals]{r} 
   \arrow{d}{(L^m\eta^n L^{n - m}R^n)} &
L^nR^n \arrow[equals]{d} \\
L^mR^mL^mL^{n - m}R^n \arrow{r}{\cuHI^m L^nR^n} &
L^mL^{n - m}R^n 
\end{tikzcd}
\]

Combining these squares, and setting 
\[
\tau \defeq (L^m\eta^n L^{n - m}R^n) \comp
(L^nR^{n - m}\eta^m R^m)^{-1},
\]
we obtain the commuting square \eqref{eq_lem_LR_commute_LR} for 
$n \geq m$. 

For the case $n < m$, iterating on the results of Lemma 
\ref{lem_LRcommute}, we have the following commutative diagrams: 
\begin{equation}\label{eq_R_CD_iter_LR}
\begin{tikzcd}[column sep=75pt]
R^nL^mR^m \arrow{r}{R^n \cuHI^m}
\arrow{d}{(\eta^nL^{m - n}R^{m - n})^{-1}} &
R^n \arrow[equals]{d} \\
L^{m - n}R^m \arrow{r}{\cuHI^{m - n}R^n} &
R^n 
\end{tikzcd}
\end{equation}
and
\begin{equation}\label{eq_L_CD_iter_LR}
\begin{tikzcd}[column sep=75pt]
L^{m + n}R^{m + n}L^n \arrow{r}{\cuHI^m L^n} 
   \arrow{d}{L^m R^{m - n}\eta^n} &
L^n \arrow[equals]{d} \\
L^nL^{m - n}R^{m - n} \arrow{r}{L^n\cuHI^{m - n}} &
L^n.
\end{tikzcd}
\end{equation}

Applying \eqref{eq_L_CD_iter_LR} to $R^n$ and $L^n$ to 
\eqref{eq_R_CD_iter_LR}, the resulting diagrams fit together to 
give
\[
\begin{tikzcd}[column sep=75pt]
L^nR^n L^mR^m \arrow{r}{L^nR^n \cuHI^m}
\arrow{d}{(L^n\eta^nL^{m - n}R^{m - n})^{-1}} &
L^nR^n \arrow[equals]{d} \\
L^nL^{m - n} R^{m - n}R^n \arrow{r} 
\arrow{d}{L^mR^{m - n}\eta^nR^n}&
L^nR^n \arrow[equals]{d} \\
L^mR^mL^nR^n \arrow{r}{\cuHI^m L^nR^n} &
L^nR^n.
\end{tikzcd}
\]
By setting
\[
\tau \defeq (L^n\eta^nL^{m - n}R^{m - n})^{-1} \comp
   L^mR^{m - n}\eta^nR^n,
\]
we see that \eqref{eq_lem_LR_commute_LR} is commutative, and the 
lemma is established.
\end{proof}

Applying Lemma \ref{lem_LR_commute_LR} to the pair of adjoint
functors $M \mapsto \LHI[n]{M}$ and $M \mapsto \RHI[n]{M}$ on
$\DMeff$, we obtain the following proposition:

\begin{prop}\label{prop_sgn_commute_sgm}
There is a natural isomorphism
$\sgDM{n}\sgDM{m} \stackrel{\tau}{\to} \sgDM{m}\sgDM{n}$
fitting into the following commutative diagram of natural
transformations:
\begin{equation}\label{eq_prop_sg_commute_sg}
\begin{tikzcd}[column sep=65pt]
\sgDM{n} \sgDM{m} \arrow{r}{\sgDM{n}\epsilon^m} \arrow{d}{\tau} 
&\sgDM{n} \arrow[equals]{d} \\
\sgDM{m} \sgDM{n} \arrow{r}{\epsilon^m \sgDM{n}}
&\sgDM{n},
\end{tikzcd}
\end{equation}
where $\epsilon^m : \sgDM{m} \to \id$ is the unit. Furthermore,
$\sgDM{n} \epsilon^m$ and $\epsilon^m \sgDM{n}$ are natural
isomorphisms.
\end{prop}

\begin{prop}\label{prop_sDM_properties}
For all nonnegative integers $m, n$, such that $m < n$, and
for all $M$ in $\DMeff$, there exists the following natural
isomorphisms:

\begin{enumerate}
\item $\sgDM{m}\slDM{n} \cong \slDM{n}\sgDM{m}$.

\item $\sgDM{n}\slDM{m} = \slDM{m}\sgDM{n} = 0$.

\item $\slDM{m}\slDM{n} \cong \slDM{n}\slDM{m} \cong \slDM{m}$.

\item $\tZ[k]{(\sgDM{n}M)} = \sgDM{n + k}\tZ[k]{M}$ for all 
positive integers $k$.

\end{enumerate}
\end{prop}
\begin{proof}
For part (1), apply the commutative diagram of functors 
\eqref{eq_prop_sg_commute_sg} in Proposition 
\ref{prop_sgn_commute_sgm} to an object $M$ of $\DMeff$, and 
extend the rows to triangles. We obtain the following commutative 
diagram:
\[
\begin{tikzcd}
\sgDM{n} \sgDM{m} M \arrow{r} \arrow{d}{\cong} &
\sgDM{m} M \arrow{r} \arrow[equals]{d} &
\slDM{n} \sgDM{m} M \arrow{r}{+1} \arrow[dotted]{d} &
\sgDM{n} \sgDM{m} M[1] \arrow{d}{\cong}\\
\sgDM{m} \sgDM{n} M \arrow{r} &
\sgDM{m} M \arrow{r} &
\sgDM{m} \slDM{n} M \arrow{r}{+1} &
\sgDM{m} \sgDM{n} M[1].
\end{tikzcd}
\]
By the Five Lemma (\cite[10.2.2]{WH}), we have that
\[
\sgDM{m}\slDM{n} M \cong \slDM{n}\sgDM{m} M.
\]
Since the rows are functorial in $M$ and the isomorphism
$\sgDM{n} \sgDM{m} M \to \sgDM{m}\sgDM{n} M$ is natural,
for a given map $f: M \to M'$, the induced maps 
$\slDM{n}\sgDM{m}(f)$ and $\sgDM{m}\slDM{n}(f)$ fit into the 
following commutative square:
\[
\begin{tikzcd}[column sep=huge]
\slDM{n} \sgDM{m} M \arrow{r}{\slDM{n} \sgDM{m}(f)} \arrow{d}{\cong}
&\slDM{n} \sgDM{m} M' \arrow{d}{\cong} \\
\sgDM{m}\slDM{n} M \arrow{r}{\sgDM{m} \slDM{n}(f)} &
\sgDM{m}\slDM{n} M'
\end{tikzcd}
\]
Therefore, $\sgDM{m} \slDM{n}$ is naturally isomorphic to 
$\slDM{n} \sgDM{m}$.

Since $\hZ[n]{(\slDM{m})} = 0$, it is clear that
$\sgDM{n}\slDM{m} = 0$. On the other hand, by Proposition 
\ref{prop_sgn_commute_sgm}, $\sgDM{m}\sgDM{n} = \sgDM{n}$. From 
the following functorial distinguished triangle
\[
\sgDM{m} \sgDM{n} \to \sgDM{n} \to \slDM{m} \sgDM{n} \to \sgDM{m} 
   \sgDM{n}[1]
\]
it follows that $\slDM{n} \sgDM{m} = 0$, which proves (2).

For (3), apply the slice triangle \eqref{eq_slice_triangle_1} to
$\slDM{n} M$ to obtain:
\[
\sgDM{m} \slDM{n} M \to \slDM{n} M \to \slDM{m} \slDM{n} M \to
\sgDM{m} \slDM{n} M[1].
\]
Applying $\slDM{n}$ to the slice triangle of $M$ gives:
\[
\slDM{n} \sgDM{m} M \to \slDM{n} M \to \slDM{n} \slDM{m} M \to
\slDM{n} \sgDM{m} M[1],
\]
and by part (2), there exists a natural isomorphism $\sgDM{m} \slDM{n} M
\to \slDM{n} \sgDM{m}$ which fits into the following commutative
diagram:
\[
\begin{tikzcd}
\sgDM{m} \slDM{n} M \arrow{r} \arrow{d}{\cong} &
\slDM{n} M \arrow{r} \arrow[equals]{d} &
\slDM{m} \slDM{n} M \arrow{r}{+1} \arrow[dotted]{d} &
\sgDM{m} \slDM{n} M[1] \arrow{d}{\cong}\\
\slDM{n} \sgDM{m} M \arrow{r} &
\slDM{n} M \arrow{r} &
\slDM{n} \slDM{m} M \arrow{r}{+1} &
\slDM{n} \sgDM{m} M[1].
\end{tikzcd}
\]
The fact that $\slDM{n} \slDM{m} M \cong \slDM{m} \slDM{n} M$
follows from the Five Lemma. Naturality in $M$ now follows from
part (1).

For part (4), the case $k = 1$ is established in 
\cite[1.4(v)]{HuKa}. The general case follows by induction on 
$k$. \qedhere
\end{proof}

Setting $m = n - 1$ in \eqref{eq_slice_triangle_2}, we obtain the
following functorial distinguished triangle:

\begin{equation}\label{eq_slice_triangle_3}
\sgDM{n} \to \sgDM{n - 1} \to \slDM{n} \sgDM{n - 1}
\to \sgDM{n}[1].
\end{equation}

\begin{defn}\label{def_slice_functors_DMeff}
For $M$ in $\DMeff$ and positive integer $n$, we say 
$\slDM{n}\sgDM{n - 1}M$ is the \emph{$n$-th slice} of $M$, written 
as $\sliceDM{n}M$. Since $\slDM{n}$ and $\sgDM{n}$ are 
triangulated functors, so is $\sliceDM{n}$. We define the $0$-th
slice functor to be $\slDM{0}$.
\end{defn}

By Proposition \ref{prop_sDM_properties}(1), $\sliceDM{n} \cong
\sgDM{n}\slDM{n - 1}$. In particular, the image of
$\sliceDM{n}$ is in $\GFiltDM[n]{\DMeff}$. That is, for each $M$ in 
$\DMeff$, there exists some $M'$ such that $\sliceDM{n}M \cong 
\tZ[n]{M'}$. Setting $M'' \defeq M'[-2n]$, we obtain the following 
proposition:

\begin{prop}[\cite{HuKa}, 1.4(v)]
\label{prop_DMeff_slice_fund_invariant}
For each $n$ and $M$, $\sliceDM{n}M = \tZ[n]{M''}[2n]$ for some 
unique $M''$ in $\DMeff$. The object $M''$ is defined up to
unique isomorphism.
\end{prop}

\begin{defn}\label{def_DM_fund_invariant}
Following \loccit, we call $M''$ in Proposition
\ref{prop_DMeff_slice_fund_invariant} the \DEF{$n$-th fundamental 
invariant of $M$}, which we write as $\fIDM{n}M$. For each positive 
$n$, $\fIDM{n}$ is an endofunctor on $\DMeff$.
\end{defn}

\begin{ex}\label{ex_sfilt_MPn}
It is clear that $\Z(n)$ is its own $n$-th slice. Furthermore,
since $M(\P^n) = \oplus_{i = 0}^n \Z(n)[2n]$ (see 
\cite[15.5]{MVW}), it is easy to verify that 
\[
\slDM{k}M(\P^n) = \begin{cases}
M(\P^k) & \textrm{if }k \leq n \\
M(\P^n) & \textrm{otherwise}
\end{cases}
\]
and 
\[
\sgDM{k}M(\P^k) = \begin{cases}
\tZ[k]{M(\P^{n - k})} & \textrm{if }k \leq n\\
0 & \textrm{otherwise}.
\end{cases}
\]
Therefore, $\sliceDM{k} M(\P^n) = \Z(k)[2k]$, and the $k$-th
fundamental invariant of $M(\P^n)$ is $\fIDM{k}M(\P^n) = \Z$,
for $k = 0,1,\dots,n$.
\comment{
Lastly, for $A$ an abelian variety, then by -, $X \mapsto 
\hom_{\Sm}(X, A)$ defines a homotopy invariant sheaf with 
transfers, which we also represent by $A$. By
\cite[]{}, the $0$-th slice and the $0$-th fundamental invariant
of $A$ is given by 
}
\end{ex}

\section{Slice filtration on $\DM$}
\label{sect_slice_filt_on_DM}

In this section, we will extend the slice filtration on $\DMeff$ 
to $\Z$-indexed filtrations on $\DM$. Recall from \cite[14.2]{MVW} 
the following definition of the category $\DM$.

\begin{defn}
Let $\DM$ be the category obtained from $\DMeff$ by inverting
the operation $M \mapsto \tZ{M}$. That is, the objects of $\DM$
are pairs $(M, n)$, where $M$ is an object of $\DMeff$, and $n$
is any integer, such that $(\tZ{M}, n) \cong (M, n + 1)$; the set 
of morphism between $(M, n)$ and $(M', n')$ is 
\[
\varinjlim_{k} \homDMf(\tZ[k + n]{M}, \tZ[k + n']{M'}).
\]
as $k$ ranges over all integer values for which $k + n$ and 
$k + n'$ are positive. We write $\homDM((M, n), (M',n'))$ for the 
hom set of $(M, n)$ and $(M', n')$. 
\end{defn}

By induction, we have that $(M, n) \cong (M \tDM \Z(n), 0)$, for 
any positive integer $n$ and all $M$ in $\DMeff$. In particular,
if $(M, n) \cong (M', n')$ for $n \geq n'$, then $M \cong 
\tZ[n - n']{M'}$.

Furthermore, by the Cancellation Theorem (Theorem 
\ref{thm_dm_cancellation}), 
\[
\homDMf(M, M') = \homDMf(\tZ[n]{M}, \tZ[n]{M'})
\]
for all positive integers. Therefore, the colimit in the 
definition of $\homDM$ is a finite limit. That is, it suffices to 
take $k > |n| + |n'|$, say.

By the Cancellation Theorem \ref{thm_dm_cancellation}, the
localization functor $\spectDM: \DMeff \to \DM$, given by sending $M$
in $\DMeff$ to $(M, 0)$ is fully faithful. Therefore, we can identify
$\DMeff$ as a full subcategory of $\DM$. 

We will now give a description of the subcategories in the slice
filtration on $\DM$. Each subcategory in the slice filtration 
will be full, and we describe only the objects in these 
subcategories. For each integer $k$, let the objects of 
$\GFiltDM[k]{\DM}$ consist of objects $(M, n)$ for which $n \geq 
k$. As defined, $\GFiltDM[n + 1]{\DM} \subseteq \GFiltDM[n]{\DM}$ 
and therefore, we have the following tower of subcategories:
\begin{equation}\label{eq_DM_slice_filt}
\DM \supseteq \cdots \GFiltDM[-1]{\DM} \supseteq \GFiltDM[0]{\DM}
   \supseteq \GFiltDM[1]{\DM} \supseteq \cdots.
\end{equation}
Notice that for $n \geq 0$, $(M, n) \cong (\tZ[n]{M}, 0)$. 
Therefore, if $M \cong M'(n)$ for some $M'$ in $\DMeff$, $(M, 0) 
\cong (M', n)$ in $\DM$. Conversely, if $(M, 0)$ is in 
$\GFiltDM[n]{\DM}$, then $(M, 0) \cong (M', n)$ for some $M'$ in
$\DMeff$. Hence, $M \cong M'(n)$ in $\DMeff$. It follows that the 
image of $\GFiltDM[n]{\DMeff}$ under $\spectDM$ coincides with 
$\GFiltDM[n]{\DM}$, when $n \geq 0$. In Definition 
\ref{def_sgDM_DM}, we define a way to associate every object
$(M, n)$ in $\DM$ with an object $\sgDM{0}(M, n)$ in 
$\GFiltDM[0]{\DM}$, and in Proposition \ref{prop_sDM_reflection},
we show that $\sgDM{0}$ is right adjoint to $\Sigma^\infty$.
Therefore, we can realize $\DMeff$ as a coreflective subcategory
of $\DM$.

To show that this tower of subcategories constitutes a weak 
filtration of $\DM$, we must construct an extension of the 
functors $\sgDM{k}$ of Definition \ref{def_sgDM}.  By convention, 
for all $M$ in $\DMeff$, define $\sgDM{n} M$ to be $M$ for all $n 
\leq 0$. 

\begin{defn}\label{def_sgDM_DM}
For integers $n$ and $k$, we set
\[
\sgDM{k}(M, n) \defeq (\sgDM{k - n}M, n).
\]
This definition preserves isomorphisms. Indeed, if
$(M, n) \cong (M', n')$ for some integer $n'$ less than $n$, say, 
then $\tZ[n - n']{M} = M'$, and $\sgDM{k - n'}M' \cong 
\LHI[n - n']{\sgDM{k - n}M}$ by (4) of Proposition 
\ref{prop_sDM_properties}. Hence, $(\sgDM{k - n'}M', n') \cong 
(\sgDM{k - n}M, n)$. 
\end{defn}

We want to show that the $\sgDM{k}$ are triangulated functors from 
$\DM$ to $\GFiltDM[k]{\DM}$ that make $(\GFiltDM[k]{\DM}, 
\sgDM{k})$ into a weak filtration. We will verify this claim in 
the Proposition \ref{prop_sDM_reflection}. Let us first prove the 
following lemma:

\begin{lem}\label{lem_triangle_in_DM}
If $(M_1, n) \to (M_2, n) \to (M_3, n_3) \to (M_1, n)[1]$
is a distinguished triangle in $\DM$, then there exists some $M$ 
such that $(M, n) \cong (M_3, n_3)$.
\end{lem}
\begin{proof}
Let $\phi$ denote the map from $(M_1, n)$ to $(M_2, n)$. Then
$\phi$ is identified with some map $\phi': M_1 \to M_2$ in 
$\DMeff$. Complete $\phi'$ to a triangle:
\[
M_1 \to M_2 \to M \to M_1[1].
\]
Then, we have
\[
\begin{tikzcd}
(M_1, n) \arrow{r}{\phi} \arrow[equals]{d} &
(M_2, n) \arrow{r} \arrow[equals]{d} &
(M_3, n_3) \arrow{r} \arrow[dotted]{d} &
(M_1, n)[1] \arrow[equals]{d}\\
(M_1, n) \arrow{r}{\phi} &
(M_2, n) \arrow{r} &
(M, n) \arrow{r} &
(M_1, n)[+1].
\end{tikzcd}
\]
The claim now follows from the Five Lemma.
\end{proof}

\begin{prop}\label{prop_sDM_reflection}
Let $k$ be an arbitrary integer.

\begin{enumerate}
\item $(M, n) \mapsto \sgDM{k} (M, n)$ defines
a triangulated functor.

\item $\sgDM{k}$ is a right adjoint to the inclusion of 
$\GFiltDM[k]{\DM}$ into $\DM$.

\item the restriction of $\sgDM{k}$ to $\GFiltDM[k]{\DM}$ is 
naturally isomorphic to the identity.
\end{enumerate}
\end{prop}
\begin{proof}
If $k \leq n$, then $\sgDM{k}(M, n) = (M, n)$,
and by definition $(M, n)$ is an object of $\GFiltDM[k]{\DM}$.
On the other hand, if $k > n$, then as defined, $\sgDM{k}(M, n) = 
(\sgDM{k - n}M, n)$. By \cite[1.1]{HuKa}, $\sgDM{k - n}M$ is in 
$\GFiltDM[k - n]{\DMeff}$. Hence, $M \cong M'(k - n)$. Therefore,
$\sgDM{k}(M, n) \cong (M'(k - n), n) \cong (M', k)$. This shows
that $\sgDM{k}(M, n)$ is always an object of $\GFiltDM[k]{\DM}$.

Consider a map $f: (M, n) \to (M', n')$. Since we have already
shown that $\sgDM{k}$ preserves isomorphisms, by replacing either
$(M, n)$ or $(M', n')$ by an isomorphic object, we may assume that 
$n = n'$, and $f$ comes from a map $g: M \to M'$ in $\DMeff$. 
Define $\sgDM{k}(f)$ to be the map given by $\sgDM{k - n}{g}$ in 
$\DMeff$. This definition preserves the identity map, 
isomorphisms, and composition. It follows that $\sgDM{k}$ is a 
functor on $\DM$ whose image lies in $\GFiltDM[k]{\DM}$.

Given a triangle,
\[
(M', n') \to (M, n) \to (M'', n'') \to (M', n')[1]
\]
we may assume without loss of generality that $n = n' = n''$,
and that this distinguished triangle comes from the distinguished
triangle in $\DMeff$:
\[
M' \to M \to M'' \to M'[1]
\]
Since $\sgDM{k - n}$ is a triangulated functor on $\DMeff$ (see 
Definition \ref{def_sgDM}), it follows that
\[
\sgDM{k - n}M' \to \sgDM{k - n}M \to \sgDM{k - n}M'' \to
\sgDM{k - n}M'[1]
\]
is a distinguished triangle in $\DMeff$. Thus, we have the 
following distinguished triangle in $\DM$:
\[
\sgDM{k}(M', n) \to \sgDM{k}(M, n) \to \sgDM{k}(M'', n)
   \to \sgDM{k}(M', n)[1].
\]
Therefore, $\sgDM{k}$ is a triangulated functor, which
proves part (1) of the proposition.

For part (2), let $(M, n)$ be an object of $\DM$, and
$(M', n')$ be an object of $\GFiltDM[k]{\DM}$. By replacing
$(M', n')$ with an isomorphic object, we may assume that
$n' = k$. In the case $n > k$, notice that $\sgDM{k}(M, n) = 
(M, n)$, and the adjunction relation is trivially satisfied.
Otherwise, for some suitably large integer $l$, we have the 
following equality:
\[
\homDM((M', k), (M, n)) = \homDMf(M'(l + k), M(l + n)).
\]
Since $M'(l + k) \in \GFiltDM[k + l]{\DMeff}$ and $\sgDM{k + l}$
is right adjoint to the inclusion of $\GFiltDM[l + k]{\DMeff}$ into
$\DMeff$,
\[
\homDMf(M'(l + k), M(l + n)) \cong \homDMf(M'(l + k), \sgDM{k + l}M(l + n)).
\]
Notice that by Proposition \ref{prop_sDM_properties}(4),
$\sgDM{k + l}M(n + l) \cong (\sgDM{k - n}M)(n + l)$. Therefore,
\begin{align*}
\homDM((M', k), (M, n)) &\cong \homDMf(M'(l + k), (\sgDM{k - n}M)(l + n)) \\
&\cong \homDMf((M', k), \sgDM{k}(M, n)).
\end{align*}
Since the isomorphism is functorial in both $(M, n)$ and 
$(M', k)$, it follows that $\sgDM{k}$ is right adjoint to
the inclusion of $\GFiltDM[k]{\DM}$ into $\DM$.

For part (3), if $(M, n)$ is an object of $\GFiltDM[k]{\DM}$,
then $(M, n) \cong (M', k)$ for some $M'$. Furthermore, as defined,
$\sgDM{k}(M', k) = (M', k)$. As this isomorphism is natural in
$(M, n)$, we have just established part (3).
\end{proof}

Next, we define $\LFiltDM[k]{\DM}$ to be the full subcategory of 
objects $(M, n)$ in $\DM$ for which $\sgDM{k}(M, n) = 0$. Since
$\sgDM{k}(M, n) = (\sgDM{k - n}M, n) = 0$ implies that 
$\sgDM{k + 1}(M, n) = (\sgDM{k + 1 - n}M, n) = 0$, 
$\LFiltDM[k + 1]{\DM}$ is a subcategory of $\LFiltDM[k]{\DM}$, and
we obtain the following tower of subcategories:
\begin{equation}\label{eq_DM_slice_cofilt}
0 \subseteq \cdots \subseteq \LFiltDM[0]{\DM} \subseteq 
   \LFiltDM[1]{\DM} \subseteq \LFiltDM[2]{\DM} \subseteq \cdots 
   \subseteq \DM.
\end{equation}
As expected, the tower also defines a filtration of $\DM$, and
we define the reflection functors $\slDM{k} : \DM \to 
\LFiltDM[k]{\DM}$. Notice that for $(M, n)$ in $\DM$, we have the 
following triangle
\begin{equation}\label{eq_DM_slice_triangle}
\sgDM{k} (M, n) \to (M, n) \to (M', n') \to \sgDM{k} (M, n)[1].
\end{equation}

Before we proceed, we define an extension of the endofunctor 
$\slDM{k}$ of $\DMeff$ to nonpositive integers $k$ as we did for 
$\sgDM{k}$. Set $\slDM{k} M = 0$ for $k \leq 0$.

\begin{lem}
For each integer $k$, the object $(M', n')$ in 
\eqref{eq_DM_slice_triangle} is defined up to unique isomorphism.
In particular, $(M', n')$ is uniquely isomorphic to 
$(\slDM{k - n} M, n)$.
\end{lem}
\begin{proof}
  By definition, $\sgDM{k}(M, n) = (\sgDM{k - n}M, n)$. By Lemma
  \ref{lem_triangle_in_DM} we may assume $n' = n$, and $\sgDM{k -
    n} M \to M \to M' \to \sgDM{k - n}M[1]$ is a distinguished
  triangle in $\DMeff$. Since $M'$ is uniquely defined up to unique
  isomorphism (see \cite[1.3(i)]{HuKa}), $M' \cong \slDM{k - n} M$,
  and $(M', n') \cong (\slDM{k - n}M, n)$ as claimed.
\end{proof}

Copying the proof of $\cite[1.3]{HuKa}$, for each integer $k$ we 
obtain functors $\slDM{k}$, defined by sending $(M, n) \mapsto 
(\slDM{k - n}, n)$. These functors satisfy the following 
properties:

\begin{prop}\label{prop_slDM_functor}
For each integer $k$,
\begin{enumerate}
\item $\slDM{k}$ is a triangulated functor

\item the image of $\slDM{k}$ is $\LFiltDM[k]{\DM}$ and $\slDM{k}$ 
defines a left adjoint to the inclusion of $\LFiltDM[k]{\DM}$ into 
$\DM$.

\item the restriction of $\slDM{k}$ to $\LFiltDM[k]{\DM}$ is naturally
isomorphic to the identity.

\item If $k > 0$, the restriction of $\slDM{k}$ to $\DMeff$ is the
functor $\slDM{k}$ of Proposition \ref{prop_slice_DMeff}.
\end{enumerate}
\end{prop}

It follows that the towers of subcategories given in 
\eqref{eq_DM_slice_filt} and \eqref{eq_DM_slice_cofilt} 
respectively define a descending and an ascending filtration on 
$\DM$.

\section{Extending the fundamental invariants}

We can also extend the definition of the fundamental invariants
$\fIDM{k}$ to negative integers $k$. Notice that for each $(M, n)$
in $\DM$, and each integer $k$, we have the slice triangle:
\[
\sgDM{k + 1}(M, n) \to \sgDM{k}(M, n) \to \slDM{k + 1}\sgDM{k}(M, n)
\to \sgDM{k + 1}(M, n)[1].
\]

\begin{defn}
Set $\sliceDM{k}(M, n) \defeq \slDM{k + 1}\sgDM{k}(M, n)$. We call
this functor the \DEF{$k$-th slice} on $\DM$. Since both $\slDM{k + 1}$
and $\sgDM{k}$ are triangulated, so is $\sliceDM{k}$.

By similar arguments as in Proposition 
\ref{prop_DMeff_slice_fund_invariant}, we have that 
$\sliceDM{k}(M, n) \cong (M''[2k], k)$ for some $M''$ in $\DMeff$,
which is unique up to unique isomorphism. We 
define the \DEF{$k$-th fundamental invariant} of $(M, n)$ to be
\[
\fIDM{k}(M, n) \defeq M''.
\]
For each integer $k$, $\fIDM{k}$ is a functor from $\DM$ to
$\DMeff$.

Notice that for $k \geq 0$, if $(M, n)$ is in $\DMeff$, the definition
of $\sliceDM{k}$ recovers the $k$-th slice functor in Definition
\ref{def_slice_functors_DMeff} by Proposition
\ref{prop_slDM_functor}. Similarly, $\fIDM{k}$ is an extension of the
$k$-th fundamental invariant on $\DMeff$.
\end{defn}

We conclude this section by discussing the relationship between the
tensor structure on $\DMeff$ and $\DM$ and their respective slice
filtrations. First, we introduce a tensor product on $\DM$ extending
the one on $\DMeff$, which we represent by $\tDM$ as well.

For $(M, n), (M'. n')$ in $\DM$, define 
\[
(M, n) \tDM (M', n') = (M \tDM M', n + n'). 
\]
As shown in \cite[15.8]{MVW} the cyclic permutation of 
$\Z(1)^{\tensor 3}$ is the identity in $\DMeff$, by 
\cite[8.4A12]{MVW} the triangulated category $\DM$ together with 
$\tDM$ defines $(\DM, \tDM)$ is an additive symmetric monoidal 
triangulated category.

For $\DMeff$, we have the following graded tensor structure:

\begin{prop}[\cite{HuKa} 1.6]\label{prop_tDM_sfilt_DMeff}
For nonnegative integers $n, n'$, there exists a unique natural
isomorphism $\eta: \sgDM{n} \tDM \sgDM{n'} \to \sgDM{n + n'}(- 
\tDM -)$ compatible with the tensor structure on $\DMeff$. That 
is, we have the following commutative square for each $M$ and
$M'$ in $\DMeff$:
\[
\begin{tikzcd}
\sgDM{n}(M) \tDM \sgDM{n'}(M') \arrow{r}{\eta} \arrow{d} &
\sgDM{n + n'}(M \tDM M') \arrow{d} \\
M \tDM M' \arrow[equals]{r} &
M \tDM M'.
\end{tikzcd}
\]
\end{prop}

We can extend this result to $\DM$. The following is a
straightforward consequence of Proposition 
\ref{prop_tDM_sfilt_DMeff}.

\begin{cor}\label{cor_tDM_sfilt_DM}
For all integers $n, n'$, there exists a unique natural 
transformation of bifunctors on $\sgDM{n} \tDM \sgDM{n'} \to 
\sgDM{n + n'}(- \tDM -)$ compatible with the tensor structure of 
$\DM$.
\end{cor}

A corollary of Proposition \ref{prop_tDM_sfilt_DMeff} applies to 
the tensor structure on the slices (and similarly, on the 
fundamental invariants) of the slice filtration.

\begin{cor}
For all integers nonnegative $n, n'$, there exists unique
natural transformations of bifunctors $\sliceDM{n} \tDM 
\sliceDM{n'} \to \sliceDM{n + n'}(- \tDM -)$ and $\fIDM{n} \tDM
\fIDM{n'} \to \fIDM{n + n'}(- \tDM -)$ compatible with the tensor 
structure on $\DMeff$.

The natural transformations can be extended to natural 
transformations on the slice structure on $\DM$: we have natural 
transformations $\sliceDM{n} \tDM \sliceDM{n'} \to 
\sliceDM{n + n'}(- \tDM -)$ and $\fIDM{n} \tDM \fIDM{n'} \to 
\fIDM{n + n'}(- \tDM -)$ compatible with the tensor structure on 
$\DM$.
\end{cor}
\begin{proof}
The existence of natural transformations $\sliceDM{n} \tDM 
\sliceDM{n'} \to \sliceDM{n + n'}(- \tDM -)$ and $\fIDM{n} \tDM 
\fIDM{n'} \to \fIDM{n + n'}(- \tDM -)$ on $\DMeff$ is proven
in \cite[1.6]{HuKa}.

To show that the natural transformations are also defined on 
$\DM$, fix integers $n, n'$, and let $(M, m)$ and $(M', m')$ be 
two objects in $\DM$. Since $(M',m') \cong (\tZ[k]M', m' - k)$, 
we may assume without loss of generality that $m = m' < 
\min(n, n', n + n')$. In this case, notice that
\[
\sliceDM{n}(M, m) = (\sliceDM{n - m}M, m) 
\textrm{ and }
\sliceDM{n'}(M', m) = (\sliceDM{n' - m}M', m),
\]
and
\[
\fIDM{n}(M, m) = \sliceDM{n - m}(M)[-n]
\textrm{ and }
\fIDM{n'}(M, m) = \sliceDM{n' - m}(M)[-n']
\]

Define
\[
\sliceDM{n}(M, m) \tDM \sliceDM{n'}(M',m)
\to \sliceDM{n + n'}((M, m) \tDM (M',m))
\]
to be
\[
(\sliceDM{n - m}(M) \tDM \sliceDM{n' - m}, 2m) \to
   (\sliceDM{n + n' - 2m}(M \tDM M'), 2m)
\]
and
\[
\fIDM{n}(M, m) \tDM \fIDM{n'}(M',m)
\to \fIDM{n + n'}((M, m) \tDM (M',m))
\]
to be
\[
\fIDM{n - m}M[-n] \tDM \fIDM{n' - m}M'[-n']
\to \fIDM{n + n' - 2m}M[-(n + n')].
\]
Both maps are independent of the choice of $m$. Naturality in 
$(M, m)$ and $(M', m)$ follows from the naturality in $M$ and 
$M'$.
\end{proof}

\begin{rmk}
Notice that the fundamental invariants $\fIDM{k}$ of the slice 
filtration on $\DM$ always take value in $\DMeff$. More 
specifically, the fundamental invariants always take value in the 
full subcategory of birational motives defined in \cite{KaSu}. 
This is established for the fundamental invariants for $\DMeff$ 
in \cite[Section 2]{HuKa}, and can be extended directly to $\DM$.
\end{rmk}

\newpage
\chapter{Slice Filtration on $\DMeff$ and $\DM$}\label{sect_slice_filt_dm}

In this chapter, we construct a sequence of subcategories on 
$\DMeff$ using the structure on $\DMeff$ defined in the previous
chapter. To be more precise, consider the following definition.

\begin{defn}\label{def_cat_filtration}
Let $\Cat{A}$ be a category. A descending \DEF{weak filtration} of 
$\Cat{A}$ is a ($\Z$-indexed) sequence of subcategories
\[
\Cat{A} \cdots \supseteq \Cat{A}_i \supseteq \Cat{A}_{i + 1}
   \supseteq \cdots \subseteq 
\]
together with (co)reflection functors $\phi_i: \Cat{A} \to 
\Cat{A}_i$ for each $i$ such that $\phi_i$ restricts to 
the identity on $\Cat{A}_i$. One can similarly define
ascending weak filtrations.
\end{defn}

\begin{rmk}
An $\N$-indexed descending weak filtration is just a $\Z$-indexed 
descending weak filtration such that $\Cat{A}_j = \Cat{A}$
for all negative integers $j$. Likewise, an $\N$-indexed ascending 
weak filtration is an ascending weak filtration for which
$\Cat{A}_j = \Cat{A}_0$ for all negative $j$.
\end{rmk}

We show that there are two $\N$-indexed weak filtrations --- 
one ascending, one descending --- on $\DMeff$. The construction 
is based on the work of Voevodsky, Huber, and Kahn \cite{HuKa}. 
We then extend the weak filtration on $\DMeff$ to a $\Z$-indexed
weak filtration on $\DM$. Aside from Proposition 
\ref{prop_sDM_properties}, all content from the first section
is taken from \cite[\S 1]{HuKa}. The extension of the filtrations 
to $\DM$ are new.

\section{Slice filtration on $\DMeff$}
To simplify notations, following Chapter \ref{sect_dmeff_and_dm},
we write $\tZ[n]{M}$ for $M \tDM \Z(n)$, and $\hZ[n]{M}$ for 
$\ihomDMf(\Z(n), M)$. Furthermore, let $\LDMf{n}$ denote the
functor $- \tDM \Z(n)$, and $\RDMf{n}$ denote the functor 
$\ihomDMf(\Z(n), -)$. By convention, define $\LDMf{0}$ and 
$\RDMf{0}$ to be the identity functor. As we have noted at the end 
of Chapter \ref{sect_dmeff_and_dm}, $(\LDMf{n}, \RDMf{n})$ form an 
adjoint pair of triangulated functors for each natural numbers $n$.

We first describe the descending weak filtrations on $\DMeff$. Fix
an natural number $n$, let $\GFiltDM[n]{\DMeff}$ be the full 
subcategory of objects of the form $\tZ[n]{M}$ for some $M$ in 
$\DMeff$, and let $\LFiltDM[n]{\DMeff}$ be the full subcategory of 
objects $M$ such that $\hZ[n]{M} = 0$.

Since $\tZ[n + 1]{M} = \tZ[n]{\tZ[1]{M}}$ and $\hZ[n]{M} = 0$ 
implies $\hZ[(n + 1)]{M} = 0$, we have the following towers of 
subcategories:
\begin{equation}\label{eq_gslice_tower_DMeff}
\DMeff = \GFiltDM[0]{\DMeff} \supseteq \GFiltDM[1]{\DMeff} 
   \supseteq \GFiltDM[2]{\DMeff} \supseteq \cdots \supseteq 0
\end{equation}
\begin{equation}\label{eq_lslice_tower_DMeff}
0 = \LFiltDM[0]{\DMeff} \subseteq \LFiltDM[1]{\DMeff} \subseteq 
   \LFiltDM[2]{\DMeff} \subseteq \cdots \subseteq \DMeff
\end{equation}

For each $n$, $L^n R^n : \DMeff \to \GFiltDM[n]{\DMeff}$ is right
adjoint to the inclusion of $\GFiltDM[n]{\DMeff}$ into $\DMeff$
(see \cite[1.1]{HuKa}). Moreover, by Corollary 
\ref{cor_tZ_hZ_eq_id}, $R^nL^n \cong \id$. Since an object $M$ in 
$\GFiltDM[n]{\DMeff}$ is of the form $\tZ[n]{M'}$ for some $M'$ in 
$\DMeff$, and
\[
L^nR^n M = L^nR^n \tZ[n]{M'} \cong L^n M' = M,
\]
the functor $L^nR^n$ is naturally isomorphic to the identity on
$\GFiltDM[n]{\DMeff}$. Following \cite{HuKa}, let $\sgDM{n}$
denote the triangulated functor $L^nR^n$.

Furthermore, for each $M$ in $\DMeff$, we have a distinguished 
triangle
\begin{equation}\label{eq_slice_triangle_1}
\sgDM{n}M \to M \to M' \to \sgDM{n}M[1]
\end{equation}
for some $M'$ in $\DMeff$, where $\tZ[n]{\hZ[n]{M}} \to M$ is the 
counit map. Since $\sgDM{n}$ is right adjoint to inclusion,
\[
\homDMf(\sgDM{n}M, M[1]) \cong \homDMf(\sgDM{n}M, \sgDM{n}M[1])
\]
and therefore,
\[
\homDMf(\sgDM{n}M, M') = 0.
\]
It follows that $M'$ is uniquely defined up to unique isomorphism.
By the naturality of $\sgDM{n} \to \id$, it is clear that
\eqref{eq_slice_triangle_1} is functorial in $M$, and therefore,
$M \mapsto M'$ defines a triangulated functor, which is the left 
adjoint to the inclusion of $\LFiltDM[n]{\DMeff}$ into $\DMeff$. 
Following the notation in \cite{HuKa}, let $\slDM{n}$ denote this 
functor. As in the case for $\sgDM{n}$, $\slDM{n}$ is naturally 
isomorphic to the identity on $\LFiltDM[n]{\DMeff}$ (see 
\cite[1.4i, ii]{HuKa}). 

We summarize the main results of the above discussion in the 
following proposition:

\begin{prop}\label{prop_slice_DMeff}
The tower of subcategories in \eqref{eq_gslice_tower_DMeff} 
defines a descending weak filtration of $\DMeff$, where the
reflection functors
\[
\sgDM{n}: \DMeff \to \GFiltDM[n]{\DMeff}
\] 
are defined by $M \mapsto \tZ[n]{\hZ[n]{M}}$.

Furthermore, the tower in \eqref{eq_lslice_tower_DMeff}
defines an ascending weak filtration of $\DMeff$, with
reflection functors 
\[
\slDM{n}: \DMeff \to \LFiltDM[n]{\DMeff},
\]
defined by sending $M$ to $M'$ in the triangle given in 
\eqref{eq_slice_triangle_1}.
\end{prop}

We call the pair of weak filtrations associated with the
towers in \eqref{eq_gslice_tower_DMeff} and 
\eqref{eq_lslice_tower_DMeff} the \DEF{slice filtration on 
$\DMeff$}.

\section{Fundamental invariants of the slice filtration}

Before we define the slice functors, we have the following 
properties about the functors $\sgDM{n}$ and $\slDM{n}$.
This proposition is new.

\begin{prop}\label{prop_sDM_properties}
For all nonnegative integers $m, n$, such that $m < n$, and
for all $M$ in $\DMeff$,

\begin{enumerate}
\item the following is a distinguished triangle
\begin{equation*}
\sgDM{n}\sgDM{m} M \to \sgDM{m}M \to \slDM{n}\sgDM{m}M \to 
   \sgDM{n}\sgDM{m} M[1].
\end{equation*}
The triangle is functorial in $M$.

\item $\sgDM{n}\sgDM{m} = \sgDM{m}\sgDM{n} = \sgDM{n}$.

\item $\sgDM{m}\slDM{n} = \slDM{n}\sgDM{m}$.

\item $\sgDM{n}\slDM{m} = \slDM{m}\sgDM{n} = 0$.

\item $\slDM{m}\slDM{n} = \slDM{n}\slDM{m} = \slDM{m}$.

\item $\tZ[k]{(\sgDM{n}M)} = \sgDM{n + k}\tZ[k]{M}$ for all 
positive integers $k$.

\end{enumerate}
\end{prop}
\begin{proof}
\begin{enumerate}
\item The existence of the triangle follows from setting $M = 
\sgDM{n} M$ in the slice triangle in \eqref{eq_slice_triangle_1}. 
Functoriality in $M$ is immediate.

\item Since $\sgDM{n}M = \tZ[n]{\hZ[n]{M}}$, $\sgDM{m}\sgDM{n} = 
\sgDM{n}\sgDM{m} = \sgDM{n}$ is immediate.

\item Since $\sgDM{n}$ is a triangulated functor and by part (1),
we have
\[
\begin{tikzcd}
\sgDM{n} \sgDM{m} M \arrow{r} \arrow{d}{\cong} &
\sgDM{m} M \arrow{r} \arrow[equals]{d} &
\slDM{n} \sgDM{m} M \arrow{r}{+1} \arrow[dotted]{d} &
\slDM{n} \sgDM{m} M[1] \arrow{d}{\cong}\\
\sgDM{m} \sgDM{n} M \arrow{r} &
\sgDM{m} M \arrow{r} &
\sgDM{m} \slDM{n} M \arrow{r}{+1} &
\sgDM{m} \sgDM{n} M[1].
\end{tikzcd}
\]
It follows by the Five Lemma (\cite[10.2.2]{WH}) that 
\[
\sgDM{m}\slDM{n} M \cong \slDM{n}\sgDM{m} M.
\]
Functoriality in $M$ of the rows implies that that this 
isomorphism is natural in $M$. That is, $\sgDM{m} \slDM{n}$ is 
naturally isomorphic to $\slDM{n} \sgDM{m}$.

\item Since $\hZ[n]{(\slDM{m})} = 0$, it is clear that
$\sgDM{n}\slDM{m} = 0$. On the other hand, by (2), $\sgDM{m}\sgDM{n}
= \sgDM{n}$. From the slice triangle in part (1)
\[
\sgDM{m} \sgDM{n} \to \sgDM{n} \to \slDM{m} \sgDM{n} \to \sgDM{m} 
   \sgDM{n}[1]
\]
it follows that $\slDM{n} \sgDM{m} = 0$.

\item Applying the slice triangle \eqref{eq_slice_triangle_1} to
$\slDM{n} M$, we have
\[
\sgDM{m} \slDM{n} M \to \slDM{n} M \to \slDM{m} \slDM{n} M \to
\sgDM{m} \slDM{n} M[1].
\]
Applying $\slDM{n}$ to the slice triangle of $M$ gives:
\[
\slDM{n} \sgDM{m} M \to \slDM{n} M \to \slDM{n} \slDM{m} M \to
\slDM{n} \sgDM{m} M[1],
\]
whence, by (3) we have
\[
\begin{tikzcd}
\sgDM{m} \slDM{n} M \arrow{r} \arrow{d}{\cong} &
\slDM{n} M \arrow{r} \arrow[equals]{d} &
\slDM{m} \slDM{n} M \arrow{r}{+1} \arrow[dotted]{d} &
\sgDM{m} \slDM{n} M[1] \arrow{d}{\cong}\\
\slDM{n} \sgDM{m} M \arrow{r} &
\slDM{n} M \arrow{r} &
\slDM{n} \slDM{m} M \arrow{r}{+1} &
\slDM{n} \sgDM{m} M[1].
\end{tikzcd}
\]
The fact that $\slDM{n} \slDM{m} M \cong \slDM{m} \slDM{n} M$
follows from the Five Lemma. Naturality in $M$ now follows from
naturality in (1 - 3).

\item The case $k = 1$ is established in \cite[Cor. 1.4]{HuKa}.
The general case follows by induction on $k$.

\end{enumerate}
\end{proof}

Notice that, from the above, we have the following distinguished
triangle
\begin{equation}
\sgDM{n} M \to \sgDM{n - 1} M \to \slDM{n} \sgDM{n - 1} M
\to \sgDM{n} M[1].
\end{equation}

\begin{defn}\label{def_slice_functors_DMeff}
For $M$ in $\DMeff$, we say $\slDM{n}\sgDM{n - 1}M$ is the 
\emph{$n$-th slice} of $M$, written as $\sliceDM{n}M$. Since
$\slDM{n}$ and $\sgDM{n}$ are triangulated functors, so is 
$\sliceDM{n}$.
\end{defn}

By Proposition \ref{prop_sDM_properties} (3), $\sliceDM{n} =
\sgDM{n}\slDM{n - 1}$. In particular, the essential image of
$\sliceDM{n}$ is in $\GFiltDM[n]{\DMeff}$. That is, for all $M$ in 
$\DMeff$, $\sliceDM{n}M = \tZ[n]{M'}$. In particular, the following
holds:

\begin{prop}[\cite{HuKa}, 1.4(v)]
\label{prop_DMeff_slice_fund_invariant}
For each $n$ and $M$, $\sliceDM{n}M = M'(n)[2n]$ for some $M'$ in
$\DMeff$.
\end{prop}

Following \cite{HuKa}, we call $M'$ in Proposition
\ref{prop_DMeff_slice_fund_invariant} the \emph{$n$-th fundamental 
invariant of $M$}, which we write as $\fIDM{n}M$.

\begin{ex}\label{ex_sfilt_MPn}
It is clear that $\Z(n)$ is its own $n$-th slice. Furthermore,
since $M(\P^n) = \oplus_{i = 0}^n \Z(n)[2n]$ (see 
\cite[15.5]{MVW}), it is easy to verify that 
\[
\slDM{k}M(\P^n) = \begin{cases}
M(\P^k) & \textrm{if }k \leq n \\
M(\P^n) & \textrm{otherwise}
\end{cases}
\]
and 
\[
\sgDM{k}M(\P^k) = \begin{cases}
\tZ[k]{M(\P^{n - k})} & \textrm{if }k \leq n\\
0 & \textrm{otherwise}.
\end{cases}
\]
Therefore, $\sliceDM{k} M(\P^n) = \Z(k)[2k]$, and the $k$-th
fundamental invariant of $M(\P^n)$ is $\fIDM{k}M(\P^n) = \Z$,
for $k = 0,1,\dots,n$.
\end{ex}

\section{Slice filtration on $\DM$}

In this section, we will extend the slice filtration on $\DMeff$ 
to $\Z$-indexed filtrations on $\DM$.

\begin{defn}
Let $\DM$ be the category obtained from $\DMeff$ by inverting
the operation $M \mapsto \tZ{M}$. That is, the objects of $\DM$
are pairs $(M, n)$, where $M$ is an object of $\DMeff$, and $n$
is any integer, such that $(\tZ{M}, n) \cong (M, n + 1)$; the set 
of morphism between $(M, n)$ and $(M', n')$ is 
\[
\varinjlim_{k} \homDMf(\tZ[k + n]{M}, \tZ[k + n']{M'}).
\]
as $k$ ranges over all integer values for which $k + n$ and 
$k + n'$ are positive. We write $\homDM((M, n), (M',n'))$ for the 
hom set of $(M, n)$ and $(M', n')$. 
\end{defn}

\begin{rmk}
By induction, we have that $(M, n) \cong (M \tDM \Z(n), 0)$, for 
any positive integer $n$ and all $M$ in $\DMeff$. In particular,
$(M, n) \cong (M', n')$ for $n \geq n'$, then $M \cong 
\tZ[n - n']{M'}$.
\end{rmk}

\begin{rmk}\label{rmk_homs_in_DM}
By the Cancellation Theorem (Theorem \ref{thm_dm_cancellation}), 
\[
\homDMf(M, M') = \homDMf(\tZ[n]{M}, \tZ[n]{M'})
\]
for all positive integers. Therefore, the colimit in the 
definition of $\homDM$ is a finite limit. That is, it suffices to 
take $k > |n| + |n'|$, say.

Furthermore, the localization functor $\spectDM: \DMeff \to \DM$, 
given by sending $M$ in $\DMeff$ to $(M, 0)$ is fully faithful by
Theorem \ref{thm_dm_cancellation}. Therefore, we can identify 
$\DMeff$ as the full subcategory of $\DM$. The functor $\spectDM$ 
admits a right adjoint, which we represent by $\loopDM$. This 
functor is defined by 
\[
\loopDM(M, n) = \begin{cases}
\LHI[n]{M} &\textrm{for }n \geq 0 \\
0 &\textrm{for }n < 0.
\end{cases}
\]
\end{rmk}

We will now give a description of the subcategories in the slice
filtration on $\DM$. Each subcategory in the slice filtration 
will be full, and we describe only the objects in these 
subcategories. For each integer $k$, let the objects of 
$\GFiltDM[k]{\DM}$ consist of objects $(M, n)$ for which $n \geq 
k$. As defined, $\GFiltDM[n + 1]{\DM} \subseteq \GFiltDM[n]{\DM}$ 
and therefore, we have the following tower of subcategories:
\begin{equation}\label{eq_DM_slice_filt}
\DM \supseteq \cdots \GFiltDM[-1]{\DM} \supseteq \GFiltDM[0]{\DM}
   \supseteq \GFiltDM[1]{\DM} \supseteq \cdots.
\end{equation}
Notice that for $n \geq 0$, $(M, n) \cong (\tZ[n]{M}, 0)$. 
Therefore, via $\loopDM: \DM \to \DMeff$, we may identify 
$\GFiltDM[n]{\DM}$ with $\GFiltDM[n]{\DMeff}$.

To show that this tower of subcategories constitutes a weak 
filtration of $\DM$, we construct an extension of the functors 
$\sgDM{k}$.  By convention, for all $M$ in $\DMeff$, define 
$\sgDM{n} M$ to be $M$ for all nonpositive integer $n$, and set
\[
\sgDM{k}(M, n) \defeq (\sgDM{k - n}M, n).
\]
This definition is independent of the choice of $n$. Indeed, if
$(M, n) = (M', n')$ for some integer $n'$ less than $n$, say, then
$\tZ[n - n']{M} = M'$, and $\sgDM{k - n'}M' = \sgDM{k - n}M$ by
(6) of Proposition \ref{prop_sDM_properties}. Hence, 
$(\sgDM{k - n'}M', n') \cong (\sgDM{k - n}M, n)$.

We show that $\sgDM{k}$ is a triangulated functor which satisfy 
the condition of a filtration. We verify this claim in the 
Proposition \ref{prop_sDM_reflection}. We first prove the following
lemma:

\begin{lem}\label{lem_triangle_in_DM}
If $(M_1, n) \to (M_2, n) \to (M_3, n_3) \to (M_1, n)[1]$
is a distinguished triangle in $\DM$, then there exists some $M$ 
such that $(M, n) \cong (M_3, n_3)$.
\end{lem}
\begin{proof}
Let $\phi$ denote the map from $(M_1, n)$ to $(M_2, n)$. Then
$\phi$ is identified with some map $\phi': M_1 \to M_2$ in 
$\DMeff$. Complete $\phi'$ to a triangle:
\[
M_1 \to M_2 \to M \to M_1[1].
\]
Then, we have
\[
\begin{tikzcd}
(M_1, n) \arrow{r}{\phi} \arrow[equals]{d} &
(M_2, n) \arrow{r} \arrow[equals]{d} &
(M_3, n_3) \arrow{r} \arrow[dotted]{d} &
(M_1, n)[1] \arrow[equals]{d}\\
(M_1, n) \arrow{r}{\phi} &
(M_2, n) \arrow{r} &
(M, n) \arrow{r} &
(M_1, n)[+1].
\end{tikzcd}
\]
The claim now follows from the Five Lemma.
\end{proof}

\begin{prop}\label{prop_sDM_reflection}
Let $k$ be an arbitrary integer.

\begin{enumerate}
\item the association $(M, n) \mapsto \sgDM{n} (M, n)$ is 
a triangulated functor.

\item $\sgDM{k}$ is a right adjoint to the inclusion of 
$\GFiltDM[k]{\DM}$ into $\DM$.

\item the restriction of $\sgDM{k}$ to $\GFiltDM[k]{\DM}$ is 
naturally isomorphic to the identity.
\end{enumerate}
\end{prop}
\begin{proof}
We restrict our attention to the full subcategory $\DM$ of objects 
$(M, n)$ for which $k > n$, (since otherwise, $\sgDM{k}$ acts as 
the identity, which is clearly satisfies the properties above). In 
this case, (1 - 3) follows straightforwardly from the fact that
$\sgDM{k - n}$ satisfies (1 - 3) for $\DMeff$ (see the discussion
preceding Proposition \ref{prop_slice_DMeff}).
\end{proof}

Next, we define $\LFiltDM[k]{\DM}$ to be the full subcategory of 
objects $(M, n)$ in $\DM$ for which $\sgDM{k}(M, n) = 0$. It is
clear that we have the following tower of subcategories:
\begin{equation}\label{eq_DM_slice_cofilt}
0 \subseteq \cdots \subseteq \LFiltDM[0]{\DM} \subseteq 
   \LFiltDM[1]{\DM} \subseteq \LFiltDM[2]{\DM} \subseteq \cdots 
   \subseteq \DM.
\end{equation}
As expected, the tower also defines a filtration of $\DM$, and
we define the reflection functors $\slDM{k} : \DM \to 
\LFiltDM[k]{\DM}$. Notice that for $(M, n)$ in $\DM$, we have the 
following triangle
\begin{equation}\label{eq_DM_slice_triangle}
\sgDM{k} (M, n) \to (M, n) \to (M', n') \to \sgDM{k} (M, n)[1].
\end{equation}

Before we proceed, we define an extension of the endofunctor 
$\slDM{k}$ of $\DMeff$ to nonpositive integers $k$ as we did for 
$\sgDM{k}$. Set $\slDM{k} M = 0$ for $k \leq 0$.

\begin{lem}
For each integer $k$, the object $(M', n')$ in 
\eqref{eq_DM_slice_triangle} is defined up to unique isomorphism. 
In particular, $M' = \slDM{k - n} M$.
\end{lem}
\begin{proof}
By definition, $\sgDM{k}(M, n) = (\sgDM{k - n}M, n)$. It follows 
from Lemma \ref{lem_triangle_in_DM} that we may assume $n' = n$. 
The Lemma now follows from the uniqueness of $M'$ (see 
\cite[1.3(i)]{HuKa}).
\end{proof}

Let $\slDM{k}$ denote the association $(M, n) \mapsto (M', n') =
(\slDM{k - n}M, n)$. The following is a straightforward consequence
of the properties of the $\sgDM{k}$ of $\DMeff$:

\begin{prop}\label{prop_slDM_functor}
For each integer $k$,
\begin{enumerate}
\item $\slDM{k}$ is a triangulated functor

\item $\slDM{k}$ defines a left adjoint to the inclusion of 
$\LFiltDM[k]{\DM}$ into $\DM$.

\item the restriction of $\slDM{k}$ to $\LFiltDM[k]{\DM}$ is naturally
isomorphic to the identity.
\end{enumerate}
\end{prop}

It follows that the towers of subcategories given in 
\eqref{eq_DM_slice_filt} and \eqref{eq_DM_slice_cofilt} 
respectively define a descending and an ascending filtrations on 
$\DM$.

\section{Extending the fundamental invariants}

We can also extend the definition of the fundamental invariants
$\fIDM{k}$ to negative integers $k$. Notice that for each $(M, n)$
in $\DM$, and each integer $k$, we have the slice triangle:
\[
\sgDM{k + 1}(M, n) \to \sgDM{k}(M, n) \to \slDM{k + 1}\sgDM{k}(M, n)
\to \sgDM{k + 1}(M, n)[1].
\]

\begin{defn}
We define $\slDM{k + 1}\sgDM{k}(M, n)$ be the the \DEF{$k$-th 
slice} of the slice filtration on $\DM$. Since both $\slDM{k + 1}$ 
and $\sgDM{k}$ are triangulated functors, so is $\slDM{k + 1}
\sgDM{k}$. We write $\sliceDM{k}$ for this functor. 

By similar arguments as in Proposition 
\ref{prop_DMeff_slice_fund_invariant}, we have that 
$\sliceDM{k}(M, n) = (M'[2k], k)$ for some $M'$ in $\DMeff$. We 
define the \DEF{$k$-th fundamental invariant} of $(M, n)$ to be
\[
\fIDM{k}(M, n) \defeq M'.
\]
Notice that for $k \geq 0$, if $(M, n)$ is in $\DMeff$, the 
definition of $\sliceDM{k}$ recovers the $k$-th slice functor
in Definition \ref{def_slice_functors_DMeff}. Similarly,
$\fIDM{k}$ is an extension of the $k$-th fundamental invariant
on $\DMeff$.
\end{defn}

We conclude this section by discussing the relationship between the
tensor structure on $\DMeff$ and $\DM$ and their respective slice 
filtrations. First, we introduce a tensor product on $\DM$ as an 
extension of the one on $\DMeff$, which we represent by $\tDM$ as
well.

For $(M, n), (M'. n')$ in $\DM$, define 
\[
(M, n) \tDM (M', n') = (M \tDM M', n + n'). 
\]
Since $\tDM$ is additive and symmetric, so is its extension to 
$\DM$. In fact, $(\DM, \tDM)$ is an additive symmetric monoidal 
triangulated category (see \cite[8.4A12]{MVW}).

For $\DMeff$, we have the following graded tensor structure:

\begin{prop}[\cite{HuKa} 1.6]\label{prop_tDM_sfilt_DMeff}
For nonnegative integers $n, n'$, there exists a unique natural
isomorphism $\eta: \sgDM{n} \tDM \sgDM{n'} \to \sgDM{n + n'}(- 
\tDM -)$ compatible with the tensor structure on $\DMeff$. That 
is, we have the following commutative square for each $M$ and
$M'$ in $\DMeff$:
\[
\begin{tikzcd}
\sgDM{n}(M) \tDM \sgDM{n'}(M') \arrow{r}{\eta} \arrow{d} &
\sgDM{n + n'}(M \tDM M') \arrow{d} \\
M \tDM M' \arrow[equals]{r} &
M \tDM M'.
\end{tikzcd}
\]
\end{prop}

We can extend this result to $\DM$. The following is a
straightforward consequence of Proposition 
\ref{prop_tDM_sfilt_DMeff}.

\begin{cor}\label{cor_tDM_sfilt_DM}
For all integers $n, n'$, there exists a unique natural 
transformation of bifunctors on $\sgDM{n} \tDM \sgDM{n'} \to 
\sgDM{n + n'}(- \tDM -)$ compatible with the tensor structure of 
$\DM$.
\end{cor}

A corollary of Proposition \ref{prop_tDM_sfilt_DMeff} applies to 
the tensor structure on the slices (and similarly, on the 
fundamental invariants) of the slice filtration.

\begin{cor}
For all integers nonnegative $n, n'$, there exists unique
natural transformations of bifunctors $\sliceDM{n} \tDM 
\sliceDM{n'} \to \sliceDM{n + n'}(- \tDM -)$ and $\fIDM{n} \tDM
\fIDM{n'} \to \fIDM{n + n'}(- \tDM -)$ compatible with the tensor 
structure on $\DMeff$.

The natural transformations can be extended to natural 
transformations on the slice structure on $\DM$: we have natural 
transformations $\sliceDM{n} \tDM \sliceDM{n'} \to 
\sliceDM{n + n'}(- \tDM -)$ and $\fIDM{n} \tDM \fIDM{n'} \to 
\fIDM{n + n'}(- \tDM -)$ compatible with the tensor structure on 
$\DM$.
\end{cor}
\begin{proof}
The existence of natural transformations $\sliceDM{n} \tDM 
\sliceDM{n'} \to \sliceDM{n + n'}(- \tDM -)$ and $\fIDM{n} \tDM 
\fIDM{n'} \to \fIDM{n + n'}(- \tDM -)$ on $\DMeff$ is proven
in \cite[1.6]{HuKa}.

To show that the natural transformations are also defined on 
$\DM$, fix integers $n, n'$, and let $(M, m)$ and $(M', m')$ be 
two objects in $\DM$. Since $(M',m') \cong (\tZ[k]M', m' - k)$, 
we may assume without loss of generality that $m = m' < 
\min(n, n', n + n')$. In this case, notice that
\[
\sliceDM{n}(M, m) = (\sliceDM{n - m}M, m) 
\textrm{ and }
\sliceDM{n'}(M', m) = (\sliceDM{n' - m}M', m),
\]
and
\[
\fIDM{n}(M, m) = \sliceDM{n - m}(M)[-n]
\textrm{ and }
\fIDM{n'}(M, m) = \sliceDM{n' - m}(M)[-n']
\]

Define
\[
\sliceDM{n}(M, m) \tDM \sliceDM{n'}(M',m)
\to \sliceDM{n + n'}((M, m) \tDM (M',m))
\]
by
\[
(\sliceDM{n - m}(M) \tDM \sliceDM{n' - m}, 2m) \to
   (\sliceDM{n + n' - 2m}(M \tDM M'), 2m)
\]
and
\[
\fIDM{n}(M, m) \tDM \fIDM{n'}(M',m)
\to \fIDM{n + n'}((M, m) \tDM (M',m))
\]
by
\[
\fIDM{n - m}M[-n] \tDM \fIDM{n' - m}M'[-n']
\to \fIDM{n + n' - 2m}M[-(n + n')].
\]
Both maps are independent of the choice of $m$. Naturality in 
$(M, m)$ and $(M', m)$ follows from the naturality in $M$ and 
$M'$.
\end{proof}

\begin{rmk}
Notice that the fundamental invariants $\fIDM{k}$ of the slice 
filtration on $\DM$ always take value in $\DMeff$. More 
specifically, the fundamental invariants always take value in the 
full subcategory of birational motives defined in \cite{KaSu}. 
This is established for the fundamental invariants for $\DMeff$ 
in \cite[Section 2]{HuKa}, and can be extended directly to $\DM$.
\end{rmk}

\newpage
\section{Slice Filtration on $\DMeff$ and $\DM$}\label{sect_slice_filt_dm}

In this section, we construct a sequence of subcategories on 
$\DMeff$ using the structure on $\DMeff$ defined in the previous
section. To be more precise, consider the following definition.

\begin{defn}\label{def_cat_filtration}
Let $\Cat{A}$ be a category. A \DEF{weak filtration} of $\Cat{A}$ 
is a ($\Z$-indexed) sequence of subcategories
\[
\cdots \subseteq \phi_i\Cat{A} \subseteq \phi_{i - 1}\Cat{A} 
   \subseteq \cdots \subseteq \Cat{A}
\]
together with (co)reflection functors $\phi_i: \Cat{A} \to 
\phi_i\Cat{A}$ for each $i$ such that $\phi_i$ restricts to 
the identity on $\phi_i\Cat{A}$.
\end{defn}

We show that there is a weak filtration on $\DMeff$, and extend 
this filtration to a localization of $\DMeff$. The construction
is based on the work of Voevodsky, Huber, and Kahn.

Before we begin, notice that $\Z(1)$ is an object of $\DMeffgm$.
Hence, $-\tDM \Z(1)$ and $\ihomDMf(\Z(1), -)$ are adjoint pairs
(see 
To simplify notations, for $M$ in $\DMeff$, we write $\tZ{M}$ 
for $M \tDM \Z(1)$, and $\hZ{M}$ for the $\ihomDMf(\Z(1), M)$,
and write $\tZ[n]{M}$ and $\hZ[n]{M}$ for the $n$-th iterations
of applying $- \tDM \Z(1)$ and $\ihomDMf(\Z(1), -)$ respectively
to $M$. 

Since $\Z(n) \tDM \Z(1) = \Z(n + 1)$, it is easy to see that the 
functor given $M \mapsto \tZ[n]{M}$ is equal to the functor $- 
\tDM \Z(n)$. By adjunction, $M \mapsto \hZ[n]{M}$ is equal to 
$\ihomDMf(\Z(n), -)$. We write $\LDMf{n}$ for the endofunctor $- 
\tDM \Z(n)$, and $\RDMf{n}$ for $\ihomDMf(\Z(n), -)$, and set 
$\LDMf{0}$ and $\RDMf{0}$ to be the identity functor. It is clear 
from the above that for each nonnegative integer $n$, $(\LDMf{n}, 
\RDMf{n})$ form a left-right adjoint pair.

We now introduce the slice filtration on $\DMeff$. For each 
$n \geq 0$, the counit $\LDMf{n}\RDMf{n} \to \id$ gives rise to a 
distinguished triangles of the form
\begin{equation}\label{eq_slice_triangle_1}
\tZ[n]{\hZ[n]{M}} \to M \to M' \to \tZ[n]{\hZ[n]{M}}[1],
\end{equation}
for some $M' \in \DMeff$.

By the Five Lemma (see \cite[10.2.2]{WH}), the distinguished 
triangle in \eqref{eq_slice_triangle_1} is defined up to unique 
isomorphism. In fact, associating $M$ with $M'$ is a triangulated 
functor. That is, if
\[
M' \to M \to M'' \to M'[1]
\]
is distinguished, then so are
\[
\sgDM{n}M' \to \sgDM{n}M \to \sgDM{n}M'' \to \sgDM{n}M'[1]
\]
and
\[
\slDM{n}M' \to \slDM{n}M \to \slDM{n}M'' \to \slDM{n}M'[1]
\]
(see \cite[Corollary 1.4]{HuKa}).

Following loc. cit., we write $\slDM{n}$ for the functor $M 
\mapsto M'$, and $\sgDM{n}$ for the functor $M \mapsto 
\tZ[n]{\hZ[n]{M}}$. Furthermore, let $\sgDM{n}\DMeff$ be the full
subcategory of objects of the form $\tZ[n]{M}$, and let 
$\slDM{n} \DMeff$ be the full subcategory of $M \in \DMeff$ such 
that $M_{-n} = 0$.

Since $\tZ[n + 1]{M} = \tZ[1]{\tZ[n]{M}}$, and if $\hZ[n]{M} = 0$
then $\hZ[n + 1]{M} = 0$, we have two towers --- one descending, one 
ascending --- of subcategories:
\begin{equation}
\DMeff \supset \sgDM{1}\DMeff \supset \sgDM{2}\DMeff \supset \cdots \\
\end{equation}
\vskip 5pt
\begin{equation}
0 \subset \slDM{1}\DMeff \subset \slDM{2}\DMeff \subset \cdots \subset
\DMeff
\end{equation}

To show that these subcategories define a filtration and 
cofiltration, we need to define reflection functors 
$\sgDM{n}\DMeff \to \DMeff$ and $\DMeff \to \slDM{n}\DMeff$ for 
each nonnegative integer $n$, such that their restriction to the 
target categories are the identity. In fact, the reflection 
functors are given by $\sgDM{n}$ and $\slDM{n}$ respectively (see 
loc. cit. Prop. 1.1 and Corollary 1.4(ii)). All that remains is 
to see that $\sgDM{n}$ restricted to $\sgDM{n}\DMeff$ (resp. 
$\slDM{n}$ restricted to $\slDM{n}\DMeff$) are (naturally 
isomorphic to) the identity.

Since $\hZ[n]{M} = 0$ for $M$ in $\slDM{n}\DMeff$, $\sgDM{n} M = 
0$. The fact that $\slDM{n} M \cong M$ follows immediately from 
the slice triangle \eqref{eq_slice_triangle_1}. For $\sgDM{n}$, we 
first prove the following:

\begin{prop}
For each $M$ in $\DMeff$ and each nonnegative integer $n$, 
\[
\hZ[n]{\tZ[n]{M}} \cong M.
\]
\end{prop}
\begin{proof}
By Theorem \ref{thm_ALocal_eq_DMeff}, it suffices to verify the
statement for $\A^1$-local complexes. 

Let $F^*$ be the bounded above $\A^1$-local complex corresponding 
to $M$. Notice that by the Cancellation Theorem, reinterpreted for
$\A^1$-local complexes,
\[
\sheaf{\rhom}(\suslC \Z(1), \suslC \Z(1) \tLNis F^*)(U) = 
\rhom(\Ztr(U), F^*) = F^*(U)
\]
for all $U$ in $\Sm$. It follows that $\sheaf{\rhom}(\suslC\Z(1),
\suslC\Z(1) \tLNis F^*) \to F^*$ is an isomorphism, and the
proposition follows.
\end{proof}

To see that $\sgDM{n}$ restricts to the identity on $\sgDM{n} 
\DMeff$, notice that for all $M$ in $\sgDM{n} \DMeff$, $M = 
\tZ[n]{M'}$, and by the proposition above $\sgDM{n}M = 
\tZ[n]{\hZ[n]{\tZ[n]{M'}}} \cong \tZ[n]{M'} = M.$

We summarize the main results of the above discussion in the 
following proposition.

\begin{prop}\label{prop_slice_DMeff}
Let $\sgDM{n} \DMeff$ and $\slDM{n} \DMeff$ be given as above.
Then $\{\sgDM{n} \DMeff\}_{n \geq 0}$ and $\{\slDM{n} \DMeff\}_{n 
\geq 0}$ form respectively a descending and ascending filtration
$\DMeff$.

In this case, we have triangulated reflection functors $\sgDM{n}: 
\DMeff \to \sgDM{n}\DMeff$ given by $M \mapsto \tZ[n]{\hZ[n]{M}}$, 
and $\slDM{n}: \DMeff \to \slDM{n}\DMeff$, which sends $M$ to $M'$
in the slice triangle given in \eqref{eq_slice_triangle_1}.
\end{prop}

\begin{defn}
We call the filtration defined by
\[
\DMeff \supset \sgDM{1}\DMeff \supset \sgDM{2}\DMeff \supset \cdots \\
\]
the \DEF{slice filtration} of $\DMeff$, and
\[
0 \subset \slDM{1}\DMeff \subset \slDM{2}\DMeff \subset \cdots \subset
\]
the \DEF{slice cofiltration} of $\DMeff$.
\end{defn}

For $M$ in $\sgDM{n} \DMeff$, consider the slice triangle

\begin{prop}
For each $m, n$, with $m < n$, $M \in \DMeff$,

\begin{enumerate}
\item the following is a distinguished triangle
\begin{equation}
\sgDM{n}M \to \sgDM{m}M \to \slDM{n}\sgDM{m}M \to \sgDM{n}M[1].
\end{equation}

\item $\sgDM{n}\sgDM{m} = \sgDM{m}\sgDM{n} = \sgDM{n}$, and 
similarly $\slDM{m}\slDM{n} = \slDM{n}\slDM{m} = \slDM{m}$.

\item $\sgDM{m}\slDM{n} = \slDM{n}\sgDM{m}$.

\item $\sgDM{n}\slDM{m} = \slDM{m}\sgDM{n} = 0$.
\end{enumerate}
\end{prop}

\begin{defn}
For $M \in \DMeff$, we say $\slDM{n}\sgDM{n - 1}M$ is the 
\emph{$n$-th slice} of $M$, written as $\sliceDM{n}M$.
\end{defn}

\begin{prop}
$\sliceDM{n}$ is a triangulated endofunctor for each $n$. That is,
for any distinguished triangle
\[
M' \to M \to M'' \to M'[1]
\]
the following is also a triangle:
\[
\sliceDM{n}M' \to \sliceDM{n}M \to \sliceDM{n}M'' \to \sliceDM{n}M'[1]
\]
\end{prop}

\begin{prop}\label{DMeff_slice_fund_invariant}
For each $n$ and $M$, $\sliceDM{n}M = M'(n)[2n]$ for some $M' \in
\DMeff$.
\end{prop}

\begin{defn}
We call $M'$ in Prop. \ref{DMeff_slice_fund_invariant} the 
\emph{$n$-th fundamental invariant of $M$}, represented by
$\fIDM{n}M$.
\end{defn}

We now relate the slice filtration to the category of birational
motives.

\begin{defn}
A motive $M \in \DMeff$ is \emph{birational} if for any $X \in 
\Sm$, and $U$ a dense open subscheme,
\[
\rhomDMf(\suslC{U}, M) = \rhomDMf(\suslC{X}, M)
\]
We write $\BiMot$ for the full subcategory of birational motives.
\end{defn}

\begin{prop}
$M \in \BiMot$ if and only if $\rhomDMf(Z(1)[1], M) = 0$.
\end{prop}

\begin{prop}
For each $M \in \DMeff$, $\fIDM{n}M \in \BiMot$ for every $n$.
In particular, $\slDM{0}\DMeff = \BiMot$.
\end{prop}

\newpage
\chapter{Homotopy Invariant Sheaves with Transfers}\label{sect_hist}

In this chapter, we define the notion of homotopy invariant
Nisnevich sheaves with transfers. In order to do so, we need to
introduce the category of correspondences and presheaves with 
transfers. 

For the remainder of the thesis, let $\basefield$ be a perfect 
field, and let $\Sm$ denote the category of smooth separated 
finite type $\basefield$-schemes. The material in this chapter is 
taken from Lecture 2 and 6 of \cite{MVW}.

\section{Sheaves with Transfers}

\begin{defn}\label{def_cor}
Let $X, Y$ be smooth separated finite type $\basefield$-schemes. 
An \DEF{elementary correspondence from $X$ to $Y$} is an 
irreducible closed subset $W$ of $X \times Y$ such that the 
projection to $X$ from the associated integral subscheme $\Sch{W}$ 
is finite and surjective onto a component of $X$.

Let \DEF{$\Cor_{\basefield}(X, Y)$} (or simply $\Cor(X, Y)$ in the 
case when the base field $k$ is understood) denote the free 
abelian group generated by the elementary correspondences from 
$X$ to $Y$. Elements of $\Cor(X, Y)$ are called \DEF{finite 
correspondences from $X$ to $Y$}.
\end{defn}

\begin{ex}
In the case when $X$ is an integral scheme over $\basefield$, the 
graph of any morphism $\phi: X \to Y$ defines an elementary 
correspondence from $X$ to $Y$. 

In the case where $X = Y = \Spec{L}$, where $L/\basefield$ is a 
Galois extension, the elementary correspondences are precisely the 
graphs of the automorphisms in the Galois group $G \defeq \Gal(L, 
\basefield)$. In this case, $\Cor_{\basefield}(X, Y) = \Z[G]$.
\end{ex}

Let $\Cor_{\basefield}$ be the collection of objects and morphisms
where the objects of $\Cor_k$ are smooth separated finite type
$\basefield$-schemes and whose morphisms between any two objects $X$
and $Y$ in $\Cor_k$ are $\Cor(X, Y)$. We claim that
$\Cor_{\basefield}$ forms a category. The main missing piece here is a
description of the composition of morphisms. This is achieved by
pulling back correspondences as given by the following result from
\cite[1.7]{MVW}.

\begin{lem}\label{lem_cor_composition}
Let $V$ be an elementary correspondence in $\Cor(X, Y)$
and and $W$ be an elementary correspondence in $\Cor(Y, Z)$. 
Writing
\[
V = \sum_i n_i V_i, \;\;\textrm{and}\;\; W = \sum_j m_i W_i
\]
where $V_i$ and $W_j$ are elementary correspondences, then
the pushforward of
\[
\sum_{i, j} n_im_j (V_i \times Z) \cap (X \times W_j)
\]
in along $X \times Y \times Z \to X \times Z$ defines a finite 
correspondence from $X$ to $Z$.
\end{lem}

\begin{defn}\label{def_pst}
A \DEF{presheaf with transfers} is a contravariant functor from 
$\Cor_k$ to abelian groups (or $R$-modules for some commutative 
ring $R$). A map between presheaves $F$ and $G$ is a natural 
transformation from $F$ to $G$. Let $\PST_{\basefield}$ (or simply 
$\PST$ in the case when there is no ambiguity about the basefield 
$\basefield$) denote the category of presheaves with transfers. 
Notice that $\PST$ has a natural structure of an abelian category.
\end{defn}

\begin{rmk}
The term ``with transfers'' comes from the existence of 
\DEF{transfer maps}. For $F$ in $\PST$, and a finite surjective
morphism $\phi : W \to X$ of smooth schemes, there exists a
map $\phi_*: F(W) \to F(X)$ induced by the graph of $\phi$,
regarded as an elementary correspondence from $W$ to $X$.
We call $\phi^*$ the \DEF{transfer map}. Notice that $\phi^*$
is in the ``opposite direction'' as the induced maps between
sections.
\end{rmk}

% 1. define etale topology
\begin{defn}\label{def_groth_top} (\cite[II.1.3]{SGA4})
A \DEF{Grothendieck pre-topology} on a category $\Cat{C}$ is a
collection $\Cat{U}$ of \DEF{covering families} indexed by the objects
of $\Cat{C}$. Here, for each $X$ in $\Cat{C}$, a covering family of
$X$ is a collection of sets of morphisms $\{U_\alpha \to X\}_\alpha$
called \DEF{covers of $X$}. Together, the covering families satisfy
the following axioms:

\begin{enumerate}
\item for every map $Y \to X$ in $\Cat{C}$ and every cover 
$\{U_\alpha \to X\}$ of $X$, the pullback $Y \times U_\alpha \to 
Y$ exists for every $\alpha$, and $\{U_\alpha \times_X Y \to Y\}$ 
is a cover of $Y$.

\item If $\{U_\alpha \to X\}$ is a cover of $X$ and for each 
$\alpha$, $\{V_{\alpha\beta} \to U_\alpha\}$ is a cover of 
$U_\alpha$, then $\{V_{\alpha\beta} \to X\}$ obtained via 
composition is a cover of $X$.

\item If $X' \to X$ is an isomorphism, then $\{X' \to X\}$ is a 
cover of $X$.
\end{enumerate}
\end{defn}

\begin{rmk}
The notion of Grothendieck pre-topology generalizes the notion of a
topology on a space $X$. Specifically, regarding a topology of $X$ as
a category $\Cat{T}_X$ where the objects are open subsets of $X$ and
the morphisms are inclusion maps, then the collections $\{V_i
\subset V\}$ of all covers of $V$, as $V$ ranges over all open 
subsets of $X$ satisfy the axioms of Definition \ref{def_groth_top} and 
define a Grothendieck pre-topology on $\Cat{T}_X$.
\end{rmk}

\begin{defn}
For let $S \defeq \{\phi_\alpha: U_\alpha \to X\}$ be a collection 
of morphisms between schemes. We say that $S$ is \DEF{jointly 
surjective} if $\bigcup_{\phi_\alpha \in S} \phi_\alpha(U_\alpha)
= X$.
\end{defn}

\begin{rmk}
For each $X$, consider the collection $\Cat{U}_X$ of jointly 
surjective sets of open immersions $\{U_\alpha \to X\}$. Then
$\Cat{U}_X$ as $X$ ranges over all finite type $k$-schemes 
form a Grothendieck pre-topology $\Cat{U}$ on the category 
$\SchCat_k$ of finite type $k$-schemes called the \DEF{large 
Zariski site on $k$-schemes.}
\end{rmk}

We are interested in two other important Grothendieck 
pre-topologies on $\Sm$. They are the \'etale site and the 
Nisnevich site, which we define below. Recall that a morphism 
$\phi: X \to Y$ is \DEF{\'etale} if $\phi$ is a flat and 
unramified. (See \cite[\S 1.3]{Milne}.)

\begin{defn}\label{def_sites}
The \DEF{large \'etale site} on $\Sm$, is the Grothendieck
pre-topology given by a jointly surjective \'etale morphisms
$\{U_\alpha \to X\}.$

The \DEF{large Nisnevich site} on $\Sm$ is the Grothendieck 
pre-topology such that every cover of $X$ is an \'etale cover 
$\{U_\alpha \to X\}$ such that for every $x$ in $X$, there exists 
some $\phi_\alpha: U_\alpha \to X$ and $y$ in $U_\alpha$ such that 
$\phi_\alpha(y) = x$ and the induced map $k(x) \to k(y)$ is an 
isomorphism.

Let $\EtSite$ and $\NisSite$ denote respectively the \'etale 
and Nisnevich site of smooth schemes over $k$.
\end{defn}

Since open immersions are \'etale, a jointly surjective collection 
of open immersions is both an \'etale cover and a Nisnevich cover. 
In this sense, the Zariski topology is coarser than the Nisnevich 
topology, which, in turn, is coarser than the \'etale topology
on $\Sm$.

\begin{defn}\label{def_etale_sheaf}\label{def_nis_sheaf}
An \DEF{\'etale sheaf with transfers} (resp. \DEF{Nisnevich sheaf 
with transfers}) $F$ is a presheaf with transfers that is also
an \'etale (resp. Nisnevich) sheaf. That is, $F$ satisfies the
following coherence conditions:
\begin{enumerate}
\item for each \'etale (resp. Nisnevich) cover $\{U_\alpha \to 
X\}$, the following sequence is exact:
\[
0 \to F(X) \to \prod_\alpha F(U_\alpha) \to \prod_{\alpha, \beta} 
   F(U_\alpha \times_X U_\beta)
\]
where the map $\prod_\alpha F(U_\alpha) \to \prod_{\alpha, \beta} 
F(U_\alpha \times_X U_\beta)$ is given by the first and second 
projections from $U_\alpha \times_X U_\beta$ to $U_\alpha$ and
$U_\beta$ respectively for each $\alpha, \beta$.

\item for each $U, V$, $F(U \dU V) = F(U) \oplus F(V)$.
\end{enumerate}

\noindent We write $\ShEtCor$ (resp. $\ShNisCor$) for the 
subcategory of \'etale (resp. Nisnevich) sheaves with transfers.
\end{defn}

Since the category of sheaves on any locale is well-powered
(see \cite[2.3.7]{Bo}), the category of \'etale sheaves 
with transfers is also well-powered. So is the category of 
Nisnevich sheaves with transfers.

It is clear from the definition and the discussion 
following Definition \ref{def_sites} that an \'etale sheaf is also 
a Nisnevich sheaf, and an Nisnevich sheaf is a Zariski sheaf.

Our focus will be on Nisnevich sheaves with transfers, and here 
are some prominent examples.

\begin{ex}\label{ex_Z_O_Ostar}
The constant sheaf $\Z$, the structure sheaf $\O$, and the sheaf
of global units $\Ox$ satisfy the conditions given in Definition
\ref{def_etale_sheaf}, and are both \'etale and Nisnevich sheaves.
To see that $\Z, \O,$ and $\Ox$ are \'etale and Nisnevich sheaves
with transfers, we must specify what the transfer structures are
on each respective sheaf.

For this purpose, assume that $X$ and $Y$ are integral schemes
in $\Sm$, and $W$ is an elementary correspondence from $X$ to $Y$. 
Then, $W$ is given by an integral scheme finite over $X$, which we 
also represent by $W$. Let $F$ and $L$ be the function fields of 
$X$ and $W$ respectively. Then, $L$ is an $n$-dimensional
$F$-vector space, for some positive integer $n$. The induced 
map $\Z(X) \to \Z(Y)$ is given by the composition
\[
\Z(Y) \to \Z(W) \stackrel{n}{\to} \Z(X).
\]

For the others, let $tr: L \to F$ and $N: L \to F$ denote the 
trace and norm maps respectively. It is straightforward to see 
that $tr$ and $N$ restrict to maps from $\O(W) \to \O(X)$. In
the case of $N$, it is easy to see that $N(f^{-1}) = N(f)^{-1}$.
Therefore, $N$ restricts to a map $\Ox(W) \to \Ox(X)$.

Hence, the map $\O(Y) \to \O(X)$ is given by the composition
\[
\O(Y) \to \O(W) \stackrel{tr}{\to} \O(X),
\]
and, the map $\O^*(Y) \to \O^*(X)$ is given by
\[
\O^*(Y) \to \O^*(W) \stackrel{N}{\to} \O^*(X).
\]
\end{ex}

\begin{ex}\label{ex_ZtrX}
A large class of Nisnevich sheaves with transfers are the 
representable sheaves. For each $X$ in $\Sm$, write $\Ztr(X)$
for the sheaf which associates to each $U$ the abelian group
$\Cor(U, X)$. To see that $\Ztr(X)$ is a Nisnevich sheaf, it
suffices to show that it is an \'etale sheaf. In particular,
for each $X$ in $\Sm$, $\Ztr(X)$ satisfies the coherence 
conditions given in Definition \ref{def_nis_sheaf}. The statement that
$\Ztr(X)$ is an \'etale sheaf is proven in \cite[6.2]{MVW}.
\end{ex}

We state the following results without proof, taken from 
\cite{MVW}. Let $\etale$ (resp. $\nis$) denote the \'etale
(resp. Nisnevich) sheafification of a (general) presheaf
on $\Sm$. (See \cite[3.1.1]{Tamme}.) Furthermore, for a presheaf 
with transfers $F$, let $\Etale{F}$ (resp. $\NisSh{F}$) denote the 
\'etale (resp.  Nisnevich) sheafification of $F$. 

\begin{prop}\label{prop_et_and_nis_sheafification}
\begin{enumerate}
\item For $F$ a presheaf with transfers, $\Etale{F}$ admits a 
unique transfers structure. That is, $\Etale{F}$ is an \'etale 
sheaf with transfers.

% 6.17, 6.18
The functor $\etale$ restricted to $\PST$ defines a left adjoint
to the inclusion of $\ShEtCor$ into $\PST$. 

Likewise, for $F$ in $\PST$, $\NisSh{F}$ is a Nisnevich sheaf with
transfers, and $\nis$ restricted to $\PST$ defines a left adjoint 
to the inclusion of $\ShNisCor$ into $\PST$. 

\item Both $\ShEtCor$ and $\ShNisCor$ are abelian 
subcategories of $\PST$ with enough injectives.
% 6.19
\end{enumerate}
\end{prop}
\begin{proof}
For the statements about \'etale sheaves with transfers, see 
\cite[6.17, 6.18 and 6.19]{MVW}. The arguments in the proofs of the
statements about \'etale sheaves can easily be extended to proofs
for the Nisnevich sheaves.
\end{proof}

\section{Homotopy invariant sheaves with transfers}

We now introduce the notion of homotopy invariant sheaves (defined
below), which will play a central role in the subsequent chapters.

\begin{defn}
A presheaf $F$ is \DEF{homotopy invariant} if the map 
\[
F(X) \to F(X \times \A^1)
\]
induced by the projection $X \times \A^1 \to X$ is an isomorphism.
We write $\HIPST$ for the category of homotopy invariant 
presheaves with transfers.

Similarly, we define homotopy invariant sheaves, and write $\HIEt$ 
(resp. $\HINis$) for the full subcategory of homotopy invariant 
\'etale sheaves (resp. Nisnevich sheaves) with transfers. We will 
simply write $\HI$ when the underlying pre-topology is understood.

Since $\ShEtCor$ and $\ShNisCor$ are both well-powered, so are 
$\HIEt$ and $\HINis$.
\end{defn}

\begin{rmk}
If $F$ is a homotopy invariant presheaf with transfers, then
$\Etale{F}$ and $\NisSh{F}$ are homotopy invariant sheaves under
the \'etale and Nisnevich topologies respectively. Together with
Proposition \ref{prop_et_and_nis_sheafification}, we have the following
commutative diagram detailing the subcategories of presheaves on
$\Sm$ and their reflection functors:
\[
\begin{tikzcd}[column sep=large, row sep=large]
\HIEt \arrow[hookrightarrow]{r} \arrow[hookrightarrow]{d} &
\ShEtCor \arrow[hookrightarrow]{r} \arrow[hookrightarrow]{d} &
\ShEt \arrow[hookrightarrow]{d} \\
\HINis \arrow[hookrightarrow]{r} \arrow[hookrightarrow]{d} 
\arrow[bend left]{u}{\etale}&
\ShNisCor \arrow[hookrightarrow]{r} \arrow[hookrightarrow]{d} 
\arrow[bend left]{u}{\etale}&
\ShNis \arrow[hookrightarrow]{d} 
\arrow[bend left]{u}{\etale}\\
\HIPST \arrow[hookrightarrow]{r} \arrow[bend left]{u}{\nis} &
\PST \arrow[hookrightarrow]{r} \arrow[bend left]{u}{\nis}&
\PSh \arrow[bend left]{u}{\nis}
\end{tikzcd}
\]
where $\PSh$ denotes the presheaves on $\Sm$. In the diagram above,
the reflection functors in the first two columns are the restrictions
of the reflection functors in the right-most column.
\end{rmk}

Of the three sheaves mentioned in Example \ref{ex_Z_O_Ostar}, $\Z$
and $\Ox$ are homotopy invariant sheaves, and $\O$ is not. In fact,
we can define a large class of homotopy invariant presheaves with
transfers with the following construction:

\begin{constr}\label{constr_suslin_C}
Let $F$ be a presheaf with transfers. Let $\Delta^n$ denote
\[
\Spec \basefield [x_0,\dots,x_n]/\big(1 - \sum_{i} x_i \big).
\]
Notice that for each $i$ in $\{0,...,n\}$, there exists a map 
$\partial_{n, i} : \Delta^{n - 1} \to \Delta^{n}$ induced by
\[
\basefield [x_0,\dots,x_n]/\big(\sum_{i} x_i - 1 \big)
\to \basefield [x_0,\dots,x_{n - 1}]/\big(\sum_{i} x_i - 1 \big)
\]
given by 
\[
x_j \mapsto
\begin{cases}
x_j & \textrm{if }j < i \\
0 & \textrm{if }j = i \\
x_{j - 1} &\textrm{otherwise}.
\end{cases}
\]
In particular, $\Delta^{\bullet}$ is a cosimplicial scheme, and 
$F( - \times \Delta^\bullet)$ is a simplicial presheaf with
transfers. Let $\suslC F$ be the associated cochain complex. That
is $(\suslC F(X))^{-n} \defeq \suslC[n] F(X) = F(X \times 
\Delta^n)$, and the chain map is given by
\[
\partial_{n}^* \defeq \sum_{i = 0}^{n} (-1)^i \partial_{n,i}^*.
\]
\end{constr}

Clearly, if $F$ is a homotopy invariant presheaf, then the complex 
$\suslC F$ is exact except at degree $0$. In particular, the
inclusion of $F$ as a cochain complex concentrated in degree $0$ into
$\suslC F$ is a quasi-isomorphism of cochain complexes of presheaves.
In general, for $F$ in $\PST$, write $H^n \suslC F$ for the 
contravariant functor $U \mapsto H^n \suslC F(U)$. Then $H^n \suslC F$
is homotopy invariant for all $n$ (\cite[2.19]{MVW}).

If $F$ is a sheaf with transfers, then $\suslC[n] F$ is also a sheaf
with transfers for all positive $n$. Therefore, $\suslC F$ is a
cochain complex of sheaves with transfers. In particular, for all
$X$ in $\Sm$, $\suslC \Ztr(X)$ is a cochain complex of sheaves.

\begin{defn}\label{def_hSX}
We write $\hS{X}^{\mathrm{\acute{e}t}}$ (resp., $\hS{X}^{\Nis}$) 
for the etale (resp., Nisnevich) sheaf associated to $H^0 \suslC 
\Ztr(X)$. In the case where the pre-topology is understood, we will
omit the superscript, and simply write $\hS{X}$ for the associated
sheaf.
\end{defn}

\begin{rmk}
Recall that two morphisms $f, g: X \to Y$ in $\Cor$ are 
\DEF{$\A^1$-homotopic} if there exists some $h$ in $\Cor(X \times 
\A^1, Y)$ such that $h|_{X \times 0} = f$ and $h|_{X \times 1} = 
g$. We say that $f: X \to Y$ is an \DEF{$\A^1$-homotopy 
equivalence} if there exists a $g: Y \to X$ so that $fg$ is 
homotopic to the identity on $Y$, and $gf$ is homotopic to the 
identity on $X$.

If $X$ and $Y$ are homotopy equivalent, it is not true in general 
that $\Ztr(X)$ is isomorphic to $\Ztr(Y)$. For example, $\Z$ is
obviously not isomorphic to $\Ztr(\A^1)$. However, $\suslC \Ztr(X)$
is quasi-isomorphic to $\suslC \Ztr(Y)$ (see \cite[2.26]{MVW}).
Therefore, $\hS{X}$ and $\hS{Y}$ are isomorphic sheaves with 
transfers.
\end{rmk}

\begin{rmk}
We note that all results of this chapter hold for $\PST(R)$, which
are presheaves with transfers with values in $R$-modules, where $R$
is some commutative unital ring. In particular, $\Rtr(X)$ is an
\'etale/Nisnevich sheaf, for every $X$ in $\Sm$.
\end{rmk}

We conclude this chapter with an endofunctor on the category $\HI$
that will play an important role in the construction of filtrations on
$\HI$.

\begin{defn}\label{def_contract}
Let $F$ be a homotopy invariant presheaf with transfers. Write
$\RHI{F}(X)$ for the cokernel of $F(X \times \A^1) \to F(X \times 
(\A^1 - 0))$. It is easy to see that $\RHI{F}$ is again a presheaf
with transfers which we call \DEF{the contraction of $F$}. We will 
write $\RHI[n + 1]{F}$ for $\RHI{(\RHI[n]{F})}$.
\end{defn}

If $F$ is homotopy invariant, then $\RHI{F}$ is also a homotopy 
invariant. Furthermore, $F(X \times (\A^1 - 0))$ splits into 
$F(X) \oplus \RHI{F}(X)$. Thus, if $F$ is a sheaf, then $\RHI{F}$ 
is also a sheaf. In fact, $F \mapsto \RHI{F}$ defines an
endofunctor on the category of homotopy invariant sheaves with 
transfers.

\begin{prop}[\cite{DegGenMot} 3.4.3]
\label{prop_contract_is_exact}
The functor $F \mapsto \RHI{F}$ is exact.
\end{prop}

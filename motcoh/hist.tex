\newpage
\section{Homotopy Invariant Sheaves with Transfers}\label{sect_hist}

Let $\basefield$ be a perfect field, and let $\Sm$ denote the 
category of smooth separated finite type $\basefield$-schemes.

\begin{defn}\label{def_cor}
Let $X, Y \in \Sm$. An \DEF{elementary correspondence from $X$ to 
$Y$} is an irreducible closed subset $W$ of $X \times Y$ such that 
the projection to $X$ from the associated integral subscheme 
$\Sch{W}$ is finite and surjective.

Let \DEF{$\Cor(X, Y)$} denote the free abelian group generated by 
the elementary correspondences from $X$ to $Y$. Elements of 
$\Cor(X, Y)$ are called \DEF{finite correspondences from $X$ to 
$Y$}.
\end{defn}

Let $\Cor_{\basefield}$ be a collection of objects and morphisms 
where 
\[
\Obj{Cor_k} = \Obj{\Sm}
\]
and
\[
\hom_{\Cor_k}(X, Y) = \Cor(X, Y).
\]
We claim that $\Cor_{\basefield}$ form a category. The main 
missing piece here is to the composition of morphisms. This is 
achieved by pulling back correspondences as given by the following 
lemma:

\begin{lem}\label{lem_cor_composition}
Let $V \in \Cor(X, Y)$ and $W \in \Cor(Y, Z)$ be elementary 
correspondences. Writing
\[
V = \sum_i n_i V_i, \;\;textrm{and}\;\; W = \sum_j m_i W_i
\]
where $V_i$ and $W_j$ are elementary correspondences, then
the pushforward of
\[
\sum_{i, j} n_im_j (V_i \times Z) \cap (X \times W_j)
\]
in $X \times Z$ defines a finite correspondence from $X$ to $Z$.
\end{lem}
\begin{proof}(See \cite{MVW} Lemma 1.7.)
% reduce to elementary correspondences,
Notice that $\Cor(X, Y) = \sum_i \Cor(X, Y_i)$ where $Y_i$ are the 
connected components of $Y$, and similarly for $X$. Therefore, 
without loss of generality, we may assume that $V$ and $W$ are 
elementary correspondences, and such that $X$ and $Y$ are 
connected.

Form the pullback $\Sch{V} \times_Y \Sch{W}$ of $\Sch{V}$ and
$\Sch{W}$ in the following:
\[
\begin{tikzcd}
\Sch{V} \times_Y \Sch{W} \arrow{r} \arrow{d}
&\Sch{W} \arrow{d} \\
\Sch{V} \arrow{r}& Y.
\end{tikzcd}
\]
Let $i: \Sch{V} \times_Y \Sch{W} \to X \times Y \times Z$ be the
evident map, and consider its image $T$ in $X \times Y \times Z$. 
We claim that each irreducible component $T_i$ of $T$ is finite 
and surjective over $X$. 

Indeed, since $T$ is the scheme theoretic intersection 
of $\Sch{V} \times Z$ and $X \times \Sch{W}$, $T_i$ is the image 
of an irreducible component of $\Sch{V} \times_Y \Sch{W}$. 
However, each irreducible component of $\Sch{V} \times_Y \Sch{W}$ 
maps finitely and surjectively to $\Sch{V}$. Since composition of
finite surjections is a finite surjection, it follows that each
such component maps finitely surjectively to $X$. Furthermore, 
since the image under of an irreducible subscheme finite over $X$ is 
also finite over $X$ (\cite{Hart} Ex. II.4.4 and \cite{EGA3} 
4.4.2), it follows that $T_i$ is finite and surjective over $X$.

To finish, consider the projection of $X \times Y \times Z \to 
X \times Z$ of $X$ schemes. The image of each $T_i$ is again finite
and surjective over $X$. It follows that the image of $T$ in $X 
\times Z$ which given precisely by the integral scheme associated 
to
\[
\sum_{i, j} (V \times Z) \cap (X \times W)
\]
defines an elementary correspondence from $X$ to $Z$.
\end{proof}

\begin{defn}\label{def_pst}
A \DEF{presheaf with transfers} is a contravariant functor from 
$\Cor_k$ to the abelian groups (or $R$-modules for some 
commutative ring $R$). A map between presheaves $F$ and $G$ is a 
natural transformation from $F$ to $G$. Let $\PST_{\basefield}$ (or
simply $\PST$ in the case when the basefield $\basefield$ is 
understood) denote the category of presheaves with transfers.
\end{defn}

\begin{rmk}
We wish to highlight the distinction between a presheaf in the 
sense of \cite{Hart} and a presheaf with transfers defined above.
The significance
\end{rmk}

% 1. define etale topology

\begin{defn}\label{def_etale_sheaf}
An \DEF{\'Etale sheaf with transfers} $F$ is a $\PST$ that is also
an \'etale sheaf. That is,
\begin{enumerate}
\item for each Nisnevich cover $U \to X$, the following sequence 
is exact:
\[
0 \to F(X) \to F(U) \to F(U \times_X U)
\]
where the map $F(U) \to F(U \times_X U)$ is given by $p_2^* - 
p_1^*$ for $p_i$ the $i$-th projection map from $U \times_X U
\to U$.

\item for each $U, V$, $F(U \dU V) = F(U) \oplus F(V)$.
\end{enumerate}
\end{defn}


% 3. prove that Z_tr(X) is an etale sheaf with transfers
% 4. define Z_tr of pairs

% 5. define Nisnevich topology

\begin{defn}\label{def_nis_sheaf}
A \DEF{Nisnevich sheaf} $F$ is a $\PST$ such that $F$ is also a
sheaf on the Nisnevich topos (i.e. satisfying the conditions of
Def. \ref{def_etale_sheaf}
\end{defn}

\begin{rmk}
It is clear from the definition that a Nisnevich sheaf is also
an \'Etale sheaf.
\end{rmk}
% define homotopy invariant sheaves with transfers
% define the bounded below derived category of sheaves

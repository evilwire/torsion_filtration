\newpage
\section{Homotopy Invariant Sheaves with Transfers}\label{sect_hist}

Let $\basefield$ be a perfect field, and let $\Sm$ denote the 
category of smooth separated finite type $\basefield$-schemes.

\begin{defn}\label{def_cor}
Let $X, Y \in \Sm$. An \DEF{elementary correspondence from $X$ to 
$Y$} is an irreducible closed subset $W$ of $X \times Y$ such that 
the projection to $X$ from the associated integral subscheme 
$\Sch{W}$ is finite and surjective.

Let \DEF{$\Cor_{\basefield}(X, Y)$} (or simply $\Cor(X, Y)$ in the 
case when the base field $k$ is understood) denote the free 
abelian group generated by the elementary correspondences from 
$X$ to $Y$. Elements of $\Cor(X, Y)$ are called \DEF{finite 
correspondences from $X$ to $Y$}.
\end{defn}

\begin{ex}
In the case when $X$ is an integral scheme over $\basefield$, the 
graph of any morphism $\phi: X \to Y$ defines an elementary 
correspondence from $X$ to $Y$. 

In the case where $X = Y = \Spec{L}$, where $L/\basefield$ is a 
Galois extension, the elementary correspondences are precisely the 
graphs of the automorphisms in the Galois group $G \defeq \Gal(L, 
\basefield)$. In this case, $\Cor_{\basefield}(X, Y) = \Z[G]$.
\end{ex}

Let $\Cor_{\basefield}$ be a collection of objects and morphisms 
where 
\[
\Obj{Cor_k} = \Obj{\Sm}
\]
and
\[
\hom_{\Cor_k}(X, Y) = \Cor(X, Y).
\]
We claim that $\Cor_{\basefield}$ forms a category. The main 
missing piece here is to the composition of morphisms. This is 
achieved by pulling back correspondences as given by the following 
lemma:

\begin{lem}\label{lem_cor_composition}
Let $V \in \Cor(X, Y)$ and $W \in \Cor(Y, Z)$ be elementary 
correspondences. Writing
\[
V = \sum_i n_i V_i, \;\;\textrm{and}\;\; W = \sum_j m_i W_i
\]
where $V_i$ and $W_j$ are elementary correspondences, then
the pushforward of
\[
\sum_{i, j} n_im_j (V_i \times Z) \cap (X \times W_j)
\]
in $X \times Z$ defines a finite correspondence from $X$ to $Z$.
\end{lem}
\begin{proof}(See \cite{MVW} Lemma 1.7.)
% reduce to elementary correspondences,
Notice that $\Cor(X, Y) = \sum_i \Cor(X, Y_i)$ where $Y_i$ are the 
connected components of $Y$, and similarly for $X$. Therefore, 
without loss of generality, we may assume that $V$ and $W$ are 
elementary correspondences, and such that $X$ and $Y$ are 
connected.

Form the pullback $\Sch{V} \times_Y \Sch{W}$ of $\Sch{V}$ and
$\Sch{W}$ in the following:
\[
\begin{tikzcd}
\Sch{V} \times_Y \Sch{W} \arrow{r} \arrow{d}
&\Sch{W} \arrow{d} \\
\Sch{V} \arrow{r}& Y.
\end{tikzcd}
\]
Let $i: \Sch{V} \times_Y \Sch{W} \to X \times Y \times Z$ be the
evident map, and consider its image $T$ in $X \times Y \times Z$. 
We claim that each irreducible component $T_i$ of $T$ is finite 
and surjective over $X$. 

Indeed, since $T$ is the scheme theoretic intersection 
of $\Sch{V} \times Z$ and $X \times \Sch{W}$, $T_i$ is the image 
of an irreducible component of $\Sch{V} \times_Y \Sch{W}$. 
However, each irreducible component of $\Sch{V} \times_Y \Sch{W}$ 
maps finitely and surjectively to $\Sch{V}$. Since composition of
finite surjections is a finite surjection, it follows that each
such component maps finitely surjectively to $X$. Furthermore, 
since the image under of an irreducible subscheme finite over $X$ is 
also finite over $X$ (\cite{Hart} Ex. II.4.4 and \cite{EGA3} 
4.4.2), it follows that $T_i$ is finite and surjective over $X$.

To finish, consider the projection of $X \times Y \times Z \to 
X \times Z$ of $X$ schemes. The image of each $T_i$ is again finite
and surjective over $X$. It follows that the image of $T$ in $X 
\times Z$ which is given precisely by the integral scheme associated 
to
\[
\sum_{i, j} (V \times Z) \cap (X \times W)
\]
defines an elementary correspondence from $X$ to $Z$.
\end{proof}

\begin{defn}\label{def_pst}
A \DEF{presheaf with transfers} is a contravariant functor from 
$\Cor_k$ to the abelian groups (or $R$-modules for some 
commutative ring $R$). A map between presheaves $F$ and $G$ is a 
natural transformation from $F$ to $G$. Let $\PST_{\basefield}$ (or
simply $\PST$ in the case when there is no ambiguity about the 
basefield $\basefield$) denote the category of presheaves with 
transfers. Notice that $\PST$ has a natural structure of an abelian
category.
\end{defn}

\begin{rmk}
The term ``with transfers'' comes from the existence of 
\DEF{transfer maps}. For $F \in \PST$, and a finite surjective
morphism $\phi : W \to X$ of smooth schemes, there exists a
map $\phi_*: F(W) \to F(X)$ induced by the graph of $\phi$,
regarded as an elementary correspondence from $W$ to $X$.
We call $\phi^*$ the \DEF{transfers map}. Notice that $\phi^*$
is in the ``opposite direction'' as the induced maps between
sections.
\end{rmk}

% 1. define etale topology
\begin{defn}\label{def_groth_top}
Recall the \DEF{Grothendieck pre-topology} of a category $\Cat{C}$ 
is a distinguished collection $\Cat{U}$ of \DEF{covering families} 
indexed by the objects of $\Cat{C}$. Here, for each $X \in \Cat{C}$,
a covering family of $X$ is a collection of sets of morphisms 
$\{U_\alpha \to X\}_\alpha$ called \DEF{covers of $X$}. Together, 
the covering families satisfy the following axioms:

\begin{enumerate}
\item for every map $Y \to X$ in $\Cat{C}$ and every cover $\{U 
\to X\}$ of $X$, the pullback $Y \times U_\alpha \to Y$ exists 
for every $\alpha$, and the collection $\{U_\alpha \times_X Y \to
Y\}$ is a cover of $Y$.

\item If $\{U_\alpha \to X\}$ is a cover of $X$ and for each 
$\alpha$, $\{V_{\alpha\beta} \to U_\alpha\}$ is a cover of 
$U_\alpha$, then $\{V_\alpha\beta \to X\}$ obtained via 
composition is a cover of $X$.

\item If $X' \to X$ is an isomorphism, then $\{X' \to X\}$ is a 
cover of $X$.
\end{enumerate}
\end{defn}

\begin{rmk}
Grothendieck pretopology generalizes the notion of a topology on
a space $X$. Specifically, regarding a topology of $X$ as a 
category $\Cat{T}_X$ where the objects are open subsets of $X$ and 
the morphisms are inclusion maps, then the collections $\{V_i 
\subset V\}$ of all covers of $V$, as $V$ ranges over all open 
subsets of $X$ satisfy the axioms of Def. \ref{def_groth_top} and 
define a Grothendieck topology on the category $\Cat{T}_X$.
\end{rmk}

\begin{defn}
For let $S \defeq \{\phi_\alpha: U_\alpha \to X\}$ be a collection 
of morphisms between schemes. We say that $S$ is \DEF{jointly 
surjective} if $\bigcup_{\phi_\alpha \in S} \phi_\alpha(U_\alpha)
= X$.
\end{defn}

\begin{rmk}
For each $X$, consider the collection $\Cat{U}_X$ of jointly 
surjective sets of open immersions $\{U_\alpha \to X\}$. Then
$\Cat{U}_X$ as $X$ ranges over all finite type $k$ schemes 
form a Grothendieck pre-topology $\Cat{U}$ on the category 
$\SchCat_k$ of finite type $k$ schemes called the \DEF{large 
Zariski site on $k$-schemes.}
\end{rmk}

We are interested in two other important Grothendieck 
pre-topologies on $\Sm$. They are the \'etale site and the 
Nisnevich site, which we define below.

\begin{defn}
Recall that $\phi: X \to Y$ is \DEF{\'etale} if $\phi$ is a flat
and unramified. (See \cite{Milne} \S 3.)

The \DEF{large \'etale site} on $\Sm$, is the Grothendieck 
pre-topology given by jointly surjective \'etale morphisms 
$\{U_\alpha \to X\}.$

The \DEF{large Nisnevich site} on $\Sm$ is the Grothendieck 
pre-topology such that every cover of $X$ is an \'etale cover 
$\{U_\alpha \to X\}$ such that for every $x \in X$, there exists 
some $\phi_\alpha: U_\alpha \to X$ and $y \in U_\alpha$ such that 
$\phi_\alpha(y) = x$ and the induced map $k(x) \to k(y)$ is an 
isomorphism.

Let $\EtSite$ and $\NisSite$ denote respectively the \'etale 
and Nisnevich site of smooth schemes over $k$.
\end{defn}

\begin{rmk}\label{rmk_comp_of_tops}
Since open immersions are \'etale, a jointly surjective collection 
of open immersions is both an \'etale cover and a Nisnevich cover. 
In this sense, the Zariski topology is coarser than the Nisnevich 
topology, which, in turn, is coarser than the \'etale topology
on $\Sm$.
\end{rmk}

\begin{defn}\label{def_etale_sheaf}\label{def_nis_sheaf}
An \DEF{\'etale sheaf with transfers} (resp. \DEF{Nisnevich sheaf 
with transfers}) $F$ is a $\PST$ that is also
an \'etale (resp. Nisnevich) sheaf. That is, $F$ satisfies the
following coherence conditions:
\begin{enumerate}
\item for each \'etale (resp. Nisnevich) cover $\{U_\alpha \to 
X\}$, the following sequence is exact:
\[
0 \to F(X) \to \prod_\alpha F(U_\alpha) \to \prod_{\alpha, \beta} 
   F(U_\alpha \times_X U_\beta)
\]
where the map $\prod_\alpha F(U_\alpha) \to \prod_{\alpha, \beta} 
F(U_\alpha \times_X U_\beta)$ is given by the first and second 
projections from $U_\alpha \times_X U_\beta$ to $U_\alpha$ and
$U_\beta$ respectively for each $\alpha, \beta$.

\item for each $U, V$, $F(U \dU V) = F(U) \oplus F(V)$.
\end{enumerate}
\end{defn}

since the category of sheaves of any locale is well-powered, the 
category of \'etale sheaves with transfers is also well-powered. 
So is the category of Nisnevich sheaves with transfers.

\begin{rmk}
It is clear from the definition and Remark \ref{rmk_comp_of_tops}
that an \'etale sheaf is also a Nisnevich sheaf.
\end{rmk}

Our focus will be on the Nisnevich sheaves with transfers, and 
here are some prominent examples 

\begin{ex}\label{ex_Z_O_Ostar}
The constant sheaf $\Z$, the structure sheaf $\O$, and the sheaf
of global units $\Ox$ satisfy the conditions given in Def. 
\ref{def_etale_sheaf}, and are both \'etale and Nisnevich sheaves.
To see that $\Z, \O,$ and $\Ox$ are \'etale and Nisnevich sheaves
with transfers, we must specify what the transfer structures are
on each respective sheaf.

For this purpose, it assume that $X, Y \in \Sm$ integral, and
$W$ an elementary correspondence from $X$ to $Y$. Then, $W$ is
given by an integral scheme finite over $X$, which we also 
represent by $W$. Let $F$ and $L$ be the function fields of $X$ 
and $W$ respectively. Then, $L$ is a dimensional $n$ $F$-vector 
space, where $n$ is some positive integer. The induced map $\Z(X) 
\to \Z(Y)$ is given by the composition
\[
\Z(Y) \to \Z(W) \stackrel{n}{\to} \Z(X).
\]

For the others, let $tr: L \to F$ and $N: L \to F$ denote the 
trace and norm maps respectively. It is straightforward to see 
that $tr$ and $N$ restricts to maps from $\O(W) \to \O(X)$. In
the case of $N$, it is easy to see that $N(f^{-1}) = N(f)^{-1}$,
and $N$ restricts to a map $\Ox(W) \to \Ox(X)$.

Hence, the map $\O(Y) \to \O(X)$ is given by the composition
\[
\O(Y) \to \O(W) \stackrel{tr}{\to} \O(X),
\]
and, the map $\O^*(Y) \to \O^*(X)$ is given by
\[
\O^*(Y) \to \O^*(W) \stackrel{N}{\to} \O^*(X).
\]
\end{ex}

\begin{ex}
A large class of Nisnevich sheaves with transfers are the 
representable sheaves. For exach $X \in \Sm$, write $\Ztr(X)$
for the sheaf which associates to each $U$ the abelian group
$\Cor(U, X)$. To see that $\Ztr(X)$ is a Nisnevich sheaf, it
suffices to show that it is an \'etale sheaf. In particular,
for each $X \in \Sm$, $\Ztr(X)$ satisfies the coherence 
conditions given in \ref{def_nis_sheaf}. The statement that
$\Ztr(X)$ is an \'etale sheaf is proven in \cite{MVW}, Lemma 6.2.
We give a short summary of the arguments for the convenience of
the reader.

Since $\Cor(U \dU V, X) = \Cor(U, X) \oplus \Cor(V, X)$, condition 
(2) is satisfied. For (1), assume we have an \'etale cover given
by a surjective \'etale map $\pi: U \to Y$. In particular $\pi$ is
flat, and so is $U \times X \to Y \times X$. Injectivity of
\[
\Ztr(Y, X) \to \Ztr(U, X)
\]
follow from uniqueness of flat pullback of cycles. Without loss
of generality, we may assume that $U$ and $Y$ are integral, with
function fields $F$ and $L$ respectively. By faithfully flat 
descent, we may reduce the argument to verifying exactness at
the stalk of the generic point. This follows from the fact that
\[
\Cor(S, X) \to \Cor(T, X) \twoto \Cor(T \times_S T, X)
\]
is an equalizer sequence.
\end{ex}

We now introduce a class of sheaves that will be the focus of
later sections.

\begin{defn}
A presheaf $F$ is \DEF{homotopy invariant} if the map 
\[
F(X) \to F(X \times \A^1)
\]
induced by the projection $X \times \A^1 \to X$ is an isomorphism.
Similarly, we define homotopy invariant sheaves.
\end{defn}

We write $\HI_{\Nis}$ (or simply $\HI$) for the full subcategory 
of homotopy invariant Nisnevich sheaves with transfers. By
\cite{Bo} Corollary 2.3.7, the category of $\HI$ is well-powered.

Of the three sheaves mentioned in Example \ref{ex_Z_O_Ostar}, $\Z$
and $\Ox$ are homotopy invariant sheaves, and $\O$ is not. In 
general, we have the following result for identifying homotopy 
invariance:

\begin{lem}
A sheaf $F$ is homotopic if and only if
\[
i_0^* = i_1^* : F(X \times \A^1) \to F(X)
\]
for all $X$, where $i_0^*$ and $i_1^*$ are induced by the 
inclusion of $X$ into $X \times \A^1$ as $X \times 0$ and $X 
\times 1$ respectively.
\end{lem}
\begin{proof}
If $F$ is homotopy invariant, then it is clear that $i_0^* = 
i_1^*$.

As projection $\pi: X \times \A^1 \to X$ emits a section (via $i_0$, 
say), the induced map $\pi^*: F(X) \to F(X \times \A^1)$ emits a
splitting. Therefore, to show that $\pi^*$ is an isomorphism, it
suffices to show that $\pi^*$ is a surjection.

Consider the following commutative diagram
\[
\begin{tikzcd}[row sep=huge, column sep=huge]
{}&
X \times \A^1&
X \arrow{l}{i_0}\\
X \times \A^1 \arrow[equals]{ru} \arrow{r}{i_1 \times id}&
X \times \A^1 \times \A^1 \arrow{u}{id \times m} &
X \times \A^1 \arrow{l}{i_0 \times id} \arrow{u}{\pi},
\end{tikzcd}
\]
where $m: \A^1 \times \A^1 \to \A^1$ is the multiplication map
given by $\basefield [t] \to \basefield [x] \otimes \basefield 
[y]$, $t \mapsto x \otimes 1 + 1 \otimes y$. Applying $F$, we
have
\[
\begin{tikzcd}[row sep=huge, column sep=huge]
{}&
F(X \times \A^1) \arrow{r}{i_0^*}\arrow[equals]{ld}
   \arrow{d}{(id \times m)^*}&
F(X) \arrow{d}{\pi^*} \\
F(X \times \A^1)&
F(X \times \A^1 \times \A^1) \arrow{l}{(i_1 \times id)^*} 
   \arrow{r}{(i_0 \times id)^*}&
X \times \A^1,
\end{tikzcd}
\]
But $\pi^*i_0^* = (i_0 \times id)^*(id \times m)^* =
(i_1 \times id)^*(id \times m)^* = id$. Thus $\pi^*$ is a 
surjection, and $F$ is homotopy invariant as desired.
\end{proof}

We now give a way to construct a large class of homotopy invariant
presheaves with transfers.

\begin{constr}\label{constr_suslin_C}
Let $F$ be a presheaf with transfers. Let $\Delta^n$ denote
\[
\Spec \basefield [x_0,\dots,x_n]/\big(1 - \sum_{i} x_i \big).
\]
Notice that for each $i = 0,..., n$, there exists a map 
$\partial_{n, i} : \Delta^{n - 1} \to \Delta^{n}$ induced by
\[
\basefield [x_0,\dots,x_n]/\big(\sum_{i} x_i - 1 \big)
\to \basefield [x_0,\dots,x_{n - 1}]/\big(\sum_{i} x_i - 1 \big)
\]
given by $x_j \mapsto x_j (j \neq i)$ and $x_i \mapsto 0$. These
are the algebraic analogue of face maps in simplicial theory.

Write $\suslC[n] F$ for the presheaf given by $U \mapsto
F(U \times \Delta^n)$. Notice that $\suslC$ preserves transfers 
and sheaf structure. Furthermore, by the above, for each $i = 0,
\dots, n$, there exists a map of abelian groups
\[
\partial_{n,i}^*(X) : \suslC[n] F(X) \to \suslC[n - 1] F(X)
\]
induced by $\partial_i$ and is functorial in $X$. In particular, 
for each $i$, there exists a map of presheaves $\partial^*_n : 
\suslC[n + 1] F \to \suslC[n] F$, and set 
\[
\partial_{n}^* \defeq \sum_{i = 0}^{n} (-1)^i \partial_{n,i}^*.
\]
It is easy to see that $\partial_n^* \comp \partial_{n - 1}^* = 0$ 
and we have a chain complex $\suslC F$ of presheaves with 
transfers:
\[
\cdots \stackrel{\partial_n^*}{\to} \suslC[n - 1] F 
\stackrel{\partial_{n - 1}^*}{\to} \cdots \suslC[1] F \to
\suslC[0] F \to 0.
\]
\end{constr}

For the remainder of the section, we prove a number of important
results regarding $\suslC F$. We begin with

\begin{prop}\label{prop_susl_hom_is_hi}
$H_*\suslC F : U \mapsto H_*\suslC F(U)$ are homotopy invariant.
\end{prop}

\begin{proof}
This follows straightforwardly from the following lemma.
\end{proof}

\begin{lem}\label{lem_hi_implies_i0_i1_homotopic}
Let $F$ be a presheaf. The the maps $i_0^*, i_1^* : 
\suslC F(X \times \A^1) \to \suslC F(X)$ are chain homotopic.
\end{lem}
\begin{proof}
Fix $F \in \PST$. We construct a homotopy contraction from
\[
s_n : \suslC[n + 1] F(X) \to \suslC[n] F(X \times \A^1).
\]
To do this, let $v_j$ denote the $j$-th vertex of $\Delta^n$. That
is, the closed point associated to the maximal ideal of
\[
\basefield [x_0,\cdots,x_{n}]/\left(1 - \sum_{i} x_i \right)
\]
generated by $x_i$ for $i \neq j$, and $x - j$.

Let $\theta_i: \Delta^{n + 1} \to \Delta^n \times \A^1$ be the map
uniquely determined by
\[
\theta_i(v_j) = \begin{cases}
(v_j, 0) & \textrm{if }j \leq i \\
(v_{j - 1}, 1) & \textrm{otherwise}
\end{cases}
\]
Define $h_i : \suslC[n] F(X \times \A^1) \to \suslC[n + 1] F(X)$ 
to be the map induced by $id \times \theta_i : X \times 
\Delta^{n + 1} \to X \times \Delta^n \times \A^1$. It is 
straightforward to verify that $i_0^* = \partial_{n + 1}h_n$,
$i_1^* = \partial_0 h_0$, and $s_n = \sum_i (-1)^i h_i$ is the 
desired homotopy contraction.
\end{proof}

\begin{defn}
Two morphisms $f, g: X \to Y$ in $\Cor$ are \DEF{$\A^1$-homotopic} 
if there exists some $h \in \Cor(X \times \A^1, Y)$ such that 
$h|_{X \times 0} = f$ and $h|_{X \times 1} = g$. We say that $f: X 
\to Y$ is an \DEF{$\A^1$-homotopy equivalence} if there exists a 
$g: Y \to X$ so that $fg$ is homotopic to the identity on $Y$, and
$gf$ is homotopic to the identity on $X$.
\end{defn}

Notice that although $X$ is homotopy equivalent to $X \times \A^1$,
$\Ztr(X)$ is not isomorphic to $\Ztr(X \times \A^1)$. For example,
the sections of these sheaves are not the same for $S = \Spec 
\basefield$. However, we have the following

\begin{prop}[\cite{MVW} Lemma 2.26]\label{prop_a1_hom_implies_hom}
If $f: X \to Y$ is an $\A^1$-homotopy equivalence with homotopy
inverse $g$, then $f_*: \suslC \Ztr(X) \to \suslC \Ztr(Y)$ is a 
chain homotopy equivalence with chain homotopy inverse $g_*$.
\end{prop}

We begin with the following Lemma.

\begin{lem}
For any $X \in \Sm$, the canonical projection
\[
\pi: X \times \A^1 \to X
\]
induces a chain homotopy equivalence
\[
\pi_*: \suslC \Ztr(X \times \A^1) \to \suslC \Ztr(X).
\]
\end{lem}

\begin{proof}
Notice first that if $F$ is a presheaf of abelian groups and for
every smooth scheme $U$, there is a homomorphism of sheaves $h:
F \to F(- \times \A^1)$ such that
\begin{equation}\label{eq_contr_cond}
\begin{tikzcd}[row sep=huge, column sep=huge]
{}&
F(U) \arrow{ld}{0} \arrow{d}{h(U)} \arrow{rd}{id} \\
F(U) &
F(U \times \A^1) \arrow{l}{F(i_0)} \arrow{r}{F(i_1)} &
F(U),
\end{tikzcd}
\end{equation}
then $C_*F$ is chain contractible. (See \cite{MVW}, Lemma 2.22.) 
For this, notice that $h$ induces morphism of chain complex 
$\suslC h : \suslC F(U) \to \suslC F(U \times \A^1)$ that is 
natural in $U$. To show that $\suslC F$ is contractible, we show 
that $\suslC F(U)$ is chain contractible for every $U$. However, 
by Lemma \ref{lem_hi_implies_i0_i1_homotopic}, $i_1^* \suslC h$ 
is chain homotopic to $0 = i_0^* \suslC h$.

Next, let $F$ be the cokernel of $\Ztr(i_0): \Ztr(X) \to \Ztr(X 
\times \A^1)$. Since $\suslC \Ztr(X \times \A^1) \simeq \suslC
\Ztr(X) \oplus \suslC F$, it suffices to show that $\suslC F$
is chain contractible. We do this by showing that there exists
$h: F \to F(- \times \A^1)$ satisfying the commutativity condition 
at the start of this proof.

To proceed, for each $U$, consider the map $H_U: \Cor(U, X \times 
\A^1) \to \Cor(U \times \A^1, X \times \A^1)$, given by the 
composition
\[
\Cor(U, X \times \A^1) \to
\Cor(U \times \A^1, X \times \A^1 \times \A^1) \to
\Cor(U \times \A^1, X \times \A^1).
\]
It is clear that $H_U$ restricted to a map $\Cor(U, X \times 0)
\to \Cor(U \times \A^1, X \times 0)$. Identifying $X \times 0$
with $X$, we have the following diagram
\[
\begin{tikzcd}[row sep=huge]
\Cor(U, X) \arrow{r}{\Ztr(i_0)} \arrow{d}{H_U} &
\Cor(U, X \times \A^1) \arrow{r} \arrow{d}{H_U} &
F(U) \arrow{r} \arrow[dotted]{d}{h_U}&
0 \\
\Cor(U \times \A^1, X \times \A^1) \arrow{r}{\Ztr(i_0)} &
\Cor(U \times \A^1, X \times \A^1) \arrow{r} &
F(U \times \A^1) \arrow{r} &
0.
\end{tikzcd}
\]
In particular, we have an induced map $h_U: F(U) \to F(U \times 
\A^1)$. It is easy to see that this is natural in $U$, and defines 
a map $h : F \to F( - \times \A^1)$. We show that $h$ has the
desired properties.

First, let $U = X \times \A^1$. In this case, it is easy to see
that $i_0^* H_{X \times \A^1} : \Cor(X \times \A^1, X \times \A^1) 
\to \Cor( X \times \A^1, X \times \A^1)$ sends $id$ to the 
composition $X \times \A^1 \stackrel{\pi}{\to} X 
\stackrel{i_0}{\to} X \times \A^1$, whose image in $F(X \times 
\A^1)$ is 0. That is, $F(i_0)h_{X \times \A^1}(id) = 0.$ 
Similarly, $F(i_1)h_{X \times \A^1}(id) = \overline{id}$, where
$\overline{id}$ is the image of $id \in \Cor(X \times \A^1, X 
\times \A^1)$. 

In general, every element $\overline{f} \in F(U)$ is the image
of $id \in \Cor(X \times \A^1, X \times \A^1)$ via some 
correspondence $f \in \Cor(U, X \times \A^1)$. Applying this
to the above, we have $F(i_0)h_U(\overline{f}) = 0$ and $F(i_1)
h_U(\overline{f}) = \overline{f}$. It follows that $F$ satisfies
the condition given in Display \ref{eq_contr_cond}, and the lemma
follows.
\end{proof}

We now resume the proof of the proposition.

\begin{proof}[Proof of Prop. \ref{prop_a1_hom_implies_hom}]
The condition that $X$ and $Y$ are homotopy equivalent implies
that there exists some $h$ for which the following diagram is
commutative:
\[
\begin{tikzcd}[column sep=huge, row sep=huge]
{}& X \\
X \arrow{ru}{gf} \arrow{r}{i_0} &
X \times \A^1 \arrow{u}{h} &
X \arrow{l}{i_1} \arrow{lu}{id}
\end{tikzcd}
\]
Applying $\suslC \Ztr$, we have
\[
\begin{tikzcd}[column sep=huge, row sep=huge]
{}& X \\
X \arrow{ru}{g_*f_*} \arrow{r}{(i_0)_*} &
X \times \A^1 \arrow{u}{h_*} &
X \arrow{l}{(i_1)_*} \arrow{lu}{id}
\end{tikzcd}
\]
By Lemma \ref{lem_hi_implies_i0_i1_homotopic}, $(i_0)_*$ and 
$(i_1)_*$ are homotopy equivalences, and are homotopy inverses to 
$p_* : \suslC \Ztr(X \times \A^1) \to \suslC \Ztr(X)$. But 
$h_* (i_1)_* = id$ (triangle on the right), and it follows that 
$h_*$ is homotopy equivalent to $p^*$, whence $g_*f_* \simeq id$.

Applying the above arguments for $Y$, we see that $f_*g_* = id$.
It follows that $f_*$ and $g_*$ are mutual homotopy inverses, and
define homotopy equivalences between $\suslC \Ztr(X)$ and $\suslC
\Ztr(Y)$.
\end{proof}

We conclude this section with an important class of chain
complexes.

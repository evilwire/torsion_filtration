\newpage
\section{Homotopy Invariant Sheaves with Transfers}\label{sect_hist}

Let $\basefield$ be a perfect field, and let $\Sm$ denote the 
category of smooth separated finite type $\basefield$-schemes.

\begin{defn}\label{def_cor}
Let $X, Y \in \Sm$. An \DEF{elementary correspondence from $X$ to 
$Y$} is an irreducible closed subset $W$ of $X \times Y$ such that 
the projection to $X$ from the associated integral subscheme 
$\Sch{W}$ is finite and surjective.

Let \DEF{$\Cor_{\basefield}(X, Y)$} (or simply $\Cor(X, Y)$ in the 
case when the base field $k$ is understood) denote the free 
abelian group generated by the elementary correspondences from 
$X$ to $Y$. Elements of $\Cor(X, Y)$ are called \DEF{finite 
correspondences from $X$ to $Y$}.
\end{defn}

\begin{ex}
In the case when $X$ is an integral scheme over $\basefield$, the 
graph of any morphism $\phi: X \to Y$ defines an elementary 
correspondence from $X$ to $Y$. 

In the case where $X = Y = \Spec{L}$, where $L/\basefield$ is a 
Galois extension, the elementary correspondences are precisely the 
graphs of the automorphisms in the Galois group $G \defeq \Gal(L, 
\basefield)$. In this case, $\Cor_{\basefield}(X, Y) = \Z[G]$.
\end{ex}

Let $\Cor_{\basefield}$ be a collection of objects and morphisms 
where 
\[
\Obj{Cor_k} = \Obj{\Sm}
\]
and
\[
\hom_{\Cor_k}(X, Y) = \Cor(X, Y).
\]
We claim that $\Cor_{\basefield}$ forms a category. The main 
missing piece here is to the composition of morphisms. This is 
achieved by pulling back correspondences as given by the following 
lemma:

\begin{lem}\label{lem_cor_composition}
Let $V \in \Cor(X, Y)$ and $W \in \Cor(Y, Z)$ be elementary 
correspondences. Writing
\[
V = \sum_i n_i V_i, \;\;\textrm{and}\;\; W = \sum_j m_i W_i
\]
where $V_i$ and $W_j$ are elementary correspondences, then
the pushforward of
\[
\sum_{i, j} n_im_j (V_i \times Z) \cap (X \times W_j)
\]
in $X \times Z$ defines a finite correspondence from $X$ to $Z$.
\end{lem}
\begin{proof}(See \cite{MVW} Lemma 1.7.)
% reduce to elementary correspondences,
Notice that $\Cor(X, Y) = \sum_i \Cor(X, Y_i)$ where $Y_i$ are the 
connected components of $Y$, and similarly for $X$. Therefore, 
without loss of generality, we may assume that $V$ and $W$ are 
elementary correspondences, and such that $X$ and $Y$ are 
connected.

Form the pullback $\Sch{V} \times_Y \Sch{W}$ of $\Sch{V}$ and
$\Sch{W}$ in the following:
\[
\begin{tikzcd}
\Sch{V} \times_Y \Sch{W} \arrow{r} \arrow{d}
&\Sch{W} \arrow{d} \\
\Sch{V} \arrow{r}& Y.
\end{tikzcd}
\]
Let $i: \Sch{V} \times_Y \Sch{W} \to X \times Y \times Z$ be the
evident map, and consider its image $T$ in $X \times Y \times Z$. 
We claim that each irreducible component $T_i$ of $T$ is finite 
and surjective over $X$. 

Indeed, since $T$ is the scheme theoretic intersection 
of $\Sch{V} \times Z$ and $X \times \Sch{W}$, $T_i$ is the image 
of an irreducible component of $\Sch{V} \times_Y \Sch{W}$. 
However, each irreducible component of $\Sch{V} \times_Y \Sch{W}$ 
maps finitely and surjectively to $\Sch{V}$. Since composition of
finite surjections is a finite surjection, it follows that each
such component maps finitely surjectively to $X$. Furthermore, 
since the image under of an irreducible subscheme finite over $X$ is 
also finite over $X$ (\cite{Hart} Ex. II.4.4 and \cite{EGA3} 
4.4.2), it follows that $T_i$ is finite and surjective over $X$.

To finish, consider the projection of $X \times Y \times Z \to 
X \times Z$ of $X$ schemes. The image of each $T_i$ is again finite
and surjective over $X$. It follows that the image of $T$ in $X 
\times Z$ which is given precisely by the integral scheme associated 
to
\[
\sum_{i, j} (V \times Z) \cap (X \times W)
\]
defines an elementary correspondence from $X$ to $Z$.
\end{proof}

\begin{defn}\label{def_pst}
A \DEF{presheaf with transfers} is a contravariant functor from 
$\Cor_k$ to the abelian groups (or $R$-modules for some 
commutative ring $R$). A map between presheaves $F$ and $G$ is a 
natural transformation from $F$ to $G$. Let $\PST_{\basefield}$ (or
simply $\PST$ in the case when there is no ambiguity about the 
basefield $\basefield$) denote the category of presheaves with 
transfers. Notice that $\PST$ has a natural structure of an abelian
category.
\end{defn}

\begin{rmk}
The term ``with transfers'' comes from the existence of 
\DEF{transfer maps}. For $F \in \PST$, and a finite surjective
morphism $\phi : W \to X$ of smooth schemes, there exists a
map $\phi_*: F(W) \to F(X)$ induced by the graph of $\phi$,
regarded as an elementary correspondence from $W$ to $X$.
We call $\phi^*$ the \DEF{transfers map}. Notice that $\phi^*$
is in the ``opposite direction'' as the induced maps between
sections.
\end{rmk}

% 1. define etale topology
\begin{defn}\label{def_groth_top}
Recall the \DEF{Grothendieck pre-topology} of a category $\Cat{C}$ 
is a distinguished collection $\Cat{U}$ of \DEF{covering families} 
indexed by the objects of $\Cat{C}$. Here, for each $X \in \Cat{C}$,
a covering family of $X$ is a collection of sets of morphisms 
$\{U_\alpha \to X\}_\alpha$ called \DEF{covers of $X$}. Together, 
the covering families satisfy the following axioms:

\begin{enumerate}
\item for every map $Y \to X$ in $\Cat{C}$ and every cover $\{U 
\to X\}$ of $X$, the pullback $Y \times U_\alpha \to Y$ exists 
for every $\alpha$, and the collection $\{U_\alpha \times_X Y \to
Y\}$ is a cover of $Y$.

\item If $\{U_\alpha \to X\}$ is a cover of $X$ and for each 
$\alpha$, $\{V_{\alpha\beta} \to U_\alpha\}$ is a cover of 
$U_\alpha$, then $\{V_\alpha\beta \to X\}$ obtained via 
composition is a cover of $X$.

\item If $X' \to X$ is an isomorphism, then $\{X' \to X\}$ is a 
cover of $X$.
\end{enumerate}
\end{defn}

\begin{rmk}
Grothendieck pretopology generalizes the notion of a topology on
a space $X$. Specifically, regarding a topology of $X$ as a 
category $\Cat{T}_X$ where the objects are open subsets of $X$ and 
the morphisms are inclusion maps, then the collections $\{V_i 
\subset V\}$ of all covers of $V$, as $V$ ranges over all open 
subsets of $X$ satisfy the axioms of Def. \ref{def_groth_top} and 
define a Grothendieck topology on the category $\Cat{T}_X$.
\end{rmk}

\begin{defn}
For let $S \defeq \{\phi_\alpha: U_\alpha \to X\}$ be a collection 
of morphisms between schemes. We say that $S$ is \DEF{jointly 
surjective} if $\bigcup_{\phi_\alpha \in S} \phi_\alpha(U_\alpha)
= X$.
\end{defn}

\begin{rmk}
For each $X$, consider the collection $\Cat{U}_X$ of jointly 
surjective sets of open immersions $\{U_\alpha \to X\}$. Then
$\Cat{U}_X$ as $X$ ranges over all finite type $k$ schemes 
form a Grothendieck pre-topology $\Cat{U}$ on the category 
$\SchCat_k$ of finite type $k$ schemes called the \DEF{large 
Zariski site on $k$-schemes.}
\end{rmk}

We are interested in two other important Grothendieck 
pre-topologies on $\Sm$. They are the \'etale site and the 
Nisnevich site, which we define below.

\begin{defn}
Recall that $\phi: X \to Y$ is \DEF{\'etale} if $\phi$ is a flat
and unramified. (See \cite{Milne} \S 3.)

The \DEF{large \'etale site} on $\Sm$, is the Grothendieck 
pre-topology given by jointly surjective \'etale morphisms 
$\{U_\alpha \to X\}.$

The \DEF{large Nisnevich site} on $\Sm$ is the Grothendieck 
pre-topology such that every cover of $X$ is an \'etale cover 
$\{U_\alpha \to X\}$ such that for every $x \in X$, there exists 
some $\phi_\alpha: U_\alpha \to X$ and $y \in U_\alpha$ such that 
$\phi_\alpha(y) = x$ and the induced map $k(x) \to k(y)$ is an 
isomorphism.

Let $\EtSite$ and $\NisSite$ denote respectively the \'etale 
and Nisnevich site of smooth schemes over $k$.
\end{defn}

\begin{rmk}\label{rmk_comp_of_tops}
Since open immersions are \'etale, a jointly surjective collection 
of open immersions is both an \'etale cover and a Nisnevich cover. 
In this sense, the Zariski topology is coarser than the Nisnevich 
topology, which, in turn, is coarser than the \'etale topology
on $\Sm$.
\end{rmk}

\begin{defn}\label{def_etale_sheaf}\label{def_nis_sheaf}
An \DEF{\'etale sheaf with transfers} (resp. \DEF{Nisnevich sheaf 
with transfers}) $F$ is a $\PST$ that is also
an \'etale (resp. Nisnevich) sheaf. That is, $F$ satisfies the
following coherence conditions:
\begin{enumerate}
\item for each \'etale (resp. Nisnevich) cover $\{U_\alpha \to 
X\}$, the following sequence is exact:
\[
0 \to F(X) \to \prod_\alpha F(U_\alpha) \to \prod_{\alpha, \beta} 
   F(U_\alpha \times_X U_\beta)
\]
where the map $\prod_\alpha F(U_\alpha) \to \prod_{\alpha, \beta} 
F(U_\alpha \times_X U_\beta)$ is given by the first and second 
projections from $U_\alpha \times_X U_\beta$ to $U_\alpha$ and
$U_\beta$ respectively for each $\alpha, \beta$.

\item for each $U, V$, $F(U \dU V) = F(U) \oplus F(V)$.
\end{enumerate}
\end{defn}

since the category of sheaves of any locale is well-powered, the 
category of \'etale sheaves with transfers is also well-powered. 
So is the category of Nisnevich sheaves with transfers.

% 3. prove that Z_tr(X) is an etale sheaf with transfers
% 4. define Z_tr of pairs

% 5. define Nisnevich topology

\begin{rmk}
It is clear from the definition and Remark \ref{rmk_comp_of_tops}
that an \'etale sheaf is also a Nisnevich sheaf.
\end{rmk}

Our focus will be on the Nisnevich sheaves with transfers, and 
here are some prominent examples 

\begin{ex}\label{ex_Z_O_Ostar}
The constant sheaf $\Z$, the structure sheaf $\O$, and the sheaf
of global units $\Ox$ satisfy the conditions given in 
\end{ex}

\begin{ex}
A large class of Nisnevich sheaves with transfers are the 
representable sheaves. For exach $X \in \Sm$, write $\Ztr(X)$
for the sheaf which associates to each $U$ the abelian group
$\Cor(U, X)$. To see that $\Ztr(X)$ is a Nisnevich sheaf, it
suffices to show that it is an \'etale sheaf. In particular,
for each $X \in \Sm$, $\Ztr(X)$ satisfies the coherence 
conditions given in \ref{def_nis_sheaf}. The statement that
$\Ztr(X)$ is an \'etale sheaf is proven in \cite{MVW}, Lemma 6.2.
We give a short summary of the arguments for the convenience of
the reader.

Since $\Cor(U \dU V, X) = \Cor(U, X) \oplus \Cor(V, X)$, condition 
(2) is satisfied. For (1), assume we have an \'etale cover given
by a surjective \'etale map $\pi: U \to Y$. In particular $\pi$ is
flat, and so is $U \times X \to Y \times X$. Injectivity of
\[
\Ztr(Y, X) \to \Ztr(U, X)
\]
follow from uniqueness of flat pullback of cycles. Without loss
of generality, we may assume that $U$ and $Y$ are integral, with
function fields $F$ and $L$ respectively. By faithfully flat 
descent, we may reduce the argument to verifying exactness at
the stalk of the generic point. This follows from the fact that
\[
\Cor(S, X) \to \Cor(T, X) \twoto \Cor(T \times_S T, X)
\]
is an equalizer sequence.
\end{ex}

\begin{defn}
A presheaf $F$ is \DEF{homotopy invariant} if the map 
\[
F(X) \to F(X \times \A^1)
\]
induced by the projection $X \times \A^1 \to X$ is an isomorphism.

We write $\HI_{\Nis}$ (or simply $\HI$) for the full subcategory 
of homotopy invariant Nisnevich sheaves with transfers.
\end{defn}

Of the three sheaves mentioned in Example \ref{ex_Z_O_Ostar}, $\Z$
and $\Ox$ are homotopy invariant sheaves, and $\O$ is not.


For the following section, the following will be an important 
functor on the category of $\HI$.

\begin{defn}
Let $F \in \HI$. Let $\RHI{F}$ denote presheaf given by
\[
U \mapsto \cok\big( F(U \times \A^1) \to 
   F(U \times (\A^1 - 0))\big).
\]
We write $\RHI[n + 1]{F}$ for $\RHI{(\RHI[n]{F})}$. We call 
$\RHI{F}$ the contraction of $F$.
\end{defn}

\begin{prop}\label{prop_contract_is_exact}
The association $F \mapsto \RHI{F}$ is an exact functor from
the category of $\HI$ to $\HI$.
\end{prop}
\begin{proof}

\end{proof}

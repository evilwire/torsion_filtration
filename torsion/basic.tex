\newpage
\section{Coradicals and Torsion Theory}\label{sect_torsion_theory}

In this section, we develop the basics of torsion theory in a 
categorical setting. The concepts and results here closely follow
those of \cite{BJV} and \cite{DTor}, except we develop the theory
from the dual perspective of the coradicals. The ideas are not new;
neither is the methodology. 

For the following, let $\Cat{A}$ be a well-powered abelian category.
That is, for every object $A \in \Cat{A}$, the collection of 
subobjects of $A$ form a set.

The following notions will be important to us:

\begin{defn}\label{def_coradical}
Let $\phi: \Cat{A} \to \Cat{A}$ be an endofunctor. 
\begin{enumerate}
\item We say that $\phi$ is a \emph{quotient functor} if there 
exists a natural surjection $\eta: \id \to \phi$. 

That is, for every $f: A \to B$, the following diagram commutes.
\[
\begin{tikzcd}
A \arrow{r}{f} \arrow{d}{\id_A} &
B \arrow{d}{\id_B} \\
\phi A \arrow{r}{\phi f} 
&\phi B
\end{tikzcd}
\]

\item We say that $\phi$ is \emph{idempotent} if $\phi^2 = \phi$.

\item a quotient functor $\phi$ is a \emph{pre-coradical} if for all
$A \in \Cat{A}$, $\phi$ applied to the kernel of the surjection 
$A \to \phi(A)$ is $0$.

\item finally, a pre-coradical is a \emph{coradical} if $\phi$ is
right exact.
\end{enumerate}
\end{defn}



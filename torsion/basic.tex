\chapter{Coradicals and Torsion Theory}\label{sect_torsion_theory}

In this chapter, we develop the basics of torsion theory in a 
categorical setting. The concepts and results here closely follow
those of \cite{BJV} and \cite{DTor}, except we develop the theory
from the dual perspective of coradicals. The ideas are not new;
neither is the methodology. We have included proofs of all results 
in this chapter for the convenience of the reader.

\section{Coradicals}
\label{sect_coradicals}

For the remainder of the chapter, let $\Cat{A}$ be a cocomplete 
well-powered abelian category. That is, for every object $A$ in 
$\Cat{A}$, the collection of subobjects of $A$ forms a set.  

\begin{defn}
For a given subcategory $\Cat{C}$ of an abelian category $\Cat{A}$, 
and an object $A$ in $\Cat{A}$, we say $A_{\Cat{C}}$ is a
\emph{largest $\Cat{C}$-subobject of $A$} if $A_{\Cat{C}}$ is a 
subobject of $A$ belonging to $\Cat{C}$ such that for all 
subobjects $B$ in $\Cat{C}$, the monomorphism $B \into A$ factors
through $A_{\Cat{C}}$. That is, for every diagram
\[
\begin{tikzcd}
{} &
B \arrow[hook]{d}{i} \arrow[dotted]{ld}{f} \\
A_{\Cat{C}} \arrow[hook]{r}{j} & A
\end{tikzcd}
\]
where $B$ is in $\Cat{C}$, there exists a map $B \stackrel{f}{\to} 
A_{\Cat{C}}$ such that $jf = i$.

We say a subcategory $\Cat{C}$ of $\Cat{A}$ is \DEF{reflective}
(resp., \DEF{coreflective}) if the inclusion of $\Cat{C}$ into
$\Cat{A}$ admits a left (resp., right) adjoint $\phi$, which we
call the \DEF{reflection} (resp., \DEF{coreflection}).

If every $A$ in $\Cat{A}$ has a largest $\Cat{C}$ subobject, the 
choice of $A_{\Cat{C}}$ for each $A$ determines a right adjoint to 
the inclusion of $\Cat{C}$ in $\Cat{A}$, making $\Cat{C}$ a 
coreflective subcategory of $\Cat{A}$. 
\end{defn}

The assumption that $\Cat{A}$ is cocomplete and well-powered will 
be crucial for the following result.

\begin{prop}\label{prop_max_subobj}
For a cocomplete well-powered abelian category $\Cat{A}$ and
any full subcategory $\Cat{C}$ of $\Cat{A}$, closed under sums 
and quotients in $\Cat{A}$, any $A$ in $\Cat{A}$ has a largest 
$\Cat{C}$-subobject.
\end{prop}
\begin{proof}
Let $A$ be an object of $\Cat{A}$, and let $\{C_i\}$ be the set 
of subobjects of $A$ in $\Cat{C}$. Write $A_{\Cat{C}}$ for the
image of $\oplus_{i} C_i$ in $A$. Since $\Cat{C}$ is closed under
sums and quotients, $A_{\Cat{C}}$ is the desired maximal subobject 
of $A$ in $\Cat{C}$.
\end{proof}

We now define some key notions in torsion theory.

\begin{defn}\label{def_coradical}
\begin{enumerate}
\item A \DEF{quotient functor} is an endofunctor $\phi : \Cat{A} 
\to \Cat{A}$ together with a natural epimorphism $\eta: \id \to 
\phi$. That is, for every $f: A \to B$, the following diagram commutes.
\[
\begin{tikzcd}
A \arrow{r}{f} \arrow{d}{\eta_A} &
B \arrow{d}{\eta_B} \\
\phi(A) \arrow{r}{\phi(f)} 
&\phi(B)
\end{tikzcd}
\]
We will often drop the reference to $\eta$. 

\item We say that $\phi$ is \DEF{idempotent} if the natural 
epimorphism is the identity on the essential image of $\phi$.
That is, $\eta_{\phi(A)}: \phi(A) \to \phi^2(A)$ is a natural 
isomorphism.

\item A quotient functor $\phi$ is a \DEF{pre-coradical} if for all
$A$ in $\Cat{A}$, $\phi$ applied to the kernel of the epimorphism 
$A \to \phi(A)$ is $0$.

\item Finally, a pre-coradical $\phi$ is a \DEF{coradical} if $\phi$ is
right exact.
\end{enumerate}
\end{defn}

\begin{rmk}
Notice that quotient functors always take epimorphisms to 
epimorphisms. However, pre-coradicals are not always right exact.
If $0 \to A' \to A \to A'' \to 0$ is an exact sequence, and $\phi$
is a pre-coradical, exactness can fail at $\phi(A)$.
\end{rmk}

\begin{defn}\label{def_radical}
Let $\phi: \Cat{A} \to \Cat{A}$ be an endofunctor of an abelian 
category $\Cat{C}$. We say that $\phi$ is a \DEF{pre-radical} if 
there exists a natural monomorphism $\phi \to \id$ (in which case, we
say that $\phi$ is a \DEF{subobject functor}) such that 
$\phi(A/\phi(A)) = 0$ for all $A$. If $\phi$ is also left-exact, then
$\phi$ is a \DEF{radical}.
\end{defn}

\begin{ex}
Let $\Ab$ be the category of abelian groups, and let $G$ an
abelian group, written additively. We write $G_{tor}$ for the 
torsion subgroup of $G$, and we write $\phi(G)$ for $G/G_{tor}$.
The quotient functor $\phi$ is a pre-coradical, but is not a 
coradical. To see this, consider the following short exact 
sequence
\[
0 \to \Z \stackrel{2}{\to} \Z \to \Z/2 \to 0
\]
in the category of $\Ab$. Applying $\phi$, we have
\[
0 \to \Z \stackrel{2}{\to} \Z \to 0 \to 0
\]
which is not exact in the middle. On the other hand, it is easy to 
see that the functor $G \mapsto G_{tor}$ is a radical.

More generally, let $R$ be a commutative ring, and $S$ be a 
multiplicatively closed set. For an $R$ module $M$, let ${}_SM$
be the submodule $M$ of elements annihilated by $S$. We write
$\phi(M)$ for $M/{}_SM$. Then $\phi$ defines a pre-coradical on
the category of $R$-modules. As in the case for abelian groups,
the functor $\phi$ is not a coradical.
\end{ex}

Torsion theory is usually developed for radicals, which 
are coradicals in the opposite category of $\Cat{A}$. However, 
throughout this chapter, we mostly consider statements for
(pre-)coradicals. We leave the dual statements to the 
reader to formulate or look up in \cite{DTor} or 
\cite[Section 1.2]{BJV}.

\begin{prop}\label{prop_idempotence}
Any right exact quotient functor $\phi$ of an abelian category 
$\Cat{A}$ is idempotent. In particular, any coradical is 
idempotent (\CF \cite[I2.2]{BJV}).
\end{prop}
\begin{proof}
Fix $A$ in $\Cat{A}$, and let $\eta$ denote the natural epimorphism 
associated to the quotient functor $\phi$. Let $K$ be the kernel 
of $\eta_A: A \to \phi(A)$, and consider the sequence
\[
0 \to K \to A \to \phi(A) \to 0.
\]
Applying $\phi$, which is right exact, we have
\[
\phi(K) \to \phi(A) \to \phi^2(A) \to 0.
\]
Thus, $\phi^2(A)$ is the cokernel of $\phi(K) \to \phi(A)$. Moreover,
we have the following commutative diagram:
\[
\begin{tikzcd}
K \arrow{r} \arrow{d}{\eta_K} &
A \arrow{d}{\eta_A} \\
\phi(K) \arrow{r} &
\phi(A)
\end{tikzcd}
\]
and, since $K \to \phi(K)$ is an epimorphism,
\begin{align*}
\phi^2(A) &= \cok (\phi(K) \to \phi(A)) \\
          &= \cok (K \to \phi(K) \to \phi(A)) \\
          &= \cok (K \to A \to \phi(A)).
\end{align*}
But $K \to A \to \phi(A)$ is the $0$ map. Therefore, $\phi^2(A) = 
\phi(A)$ as desired.
\end{proof}

In addition to being dual notions, there is a one-to-one 
correspondence between idempotent pre-radicals and
idempotent pre-coradicals:

\begin{prop}\label{prop_rad_eq_corad}
Let $\phi$ be an idempotent pre-coradical of an abelian category
$\Cat{A}$, and $\eta$ be its 
corresponding natural epimorphism. Write $\kappa(A)$ for $\ker (A 
\stackrel{\eta_A}{\to} \phi(A))$. Then $\kappa$ is a pre-radical.

Dually, if $\psi$ is an idempotent pre-radical with natural 
injection $\epsilon$. Writing $\gamma(A) = \cok \epsilon_A$, we 
have that $\gamma$ is a pre-coradical.
\end{prop}
\begin{proof}
It suffices to prove this statement for the idempotent 
pre-coradicals, as the statement for pre-radical is 
the dual assertion. We proceed as follows:

The fact that $\kappa$ is functorial follows from the naturality 
of $\eta$. Moreover, it is clear that $\kappa$ is a subobject 
functor. To see that $\kappa$ is also a pre-radical, we need to 
show that $\kappa(A/\kappa(A)) = 0$ for all $A$ in $\Cat{A}$. Fix 
such an $A$, and notice that $A/\kappa(A) = \phi(A)$. Then we have 
the associated short exact sequence:
\[
0 \to \kappa(\phi(A)) \to \phi(A) \to \phi^2(A) \to 0.
\]
But $\phi(A) \to \phi^2(A)$ is the identity. It follows that 
$\kappa(\phi(A)) = \kappa(A/\kappa(A)) = 0$.

Next, consider the following short exact sequence associated to 
$\kappa(A)$:
\[
0 \to \kappa(\kappa(A)) \to \kappa(A) 
   \stackrel{\eta_{\kappa(A)}}{\to} \phi(\kappa(A)) \to 0.
\]
Since $\phi$ is a pre-coradical, we have we have that
\[
\phi(\kappa(A)) = \phi(\ker(A \to \phi(A))) = 0.
\]
It follows that $\kappa^2(A) = \kappa(A)$, and $\kappa$ is 
idempotent. The proposition follows.
\end{proof}

\begin{prop}
Let $\phi$ be a pre-coradical of an abelian category $\Cat{A}$. Suppose $B$ is a 
quotient of $\phi(A)$, and let $K$ be the kernel of the 
composition $A \to \phi(A) \to B$. Then $\phi(K)$ is isomorphic 
to the kernel of the epimorphism $\phi(A) \to B$ (\CF \cite[I2.3]{BJV}).
\end{prop}
\begin{proof}
Let $\eta$ denote the natural epimorphism associated to the 
quotient functor $\phi$, and let $f$ denote the epimorphism given 
by the composition $A \to \phi(A) \to B$.

Consider the exact sequence $0 \to K \to A \to B \to 0$. We claim
that $\phi(K) \to \phi(A) \to B \to 0$ is exact and fits into the 
following commutative diagram:
\[
\begin{tikzcd}
0 \arrow{r} &
K \arrow{r}{g} \arrow[twoheadrightarrow]{d}{\eta_K} &
A \arrow{r}{f} \arrow[twoheadrightarrow]{d}{\eta_A} &
B \arrow{r} \arrow[equal]{d} &
0 \\
& \phi(K) \arrow{r}{\phi(g)} &
\phi(A) \arrow{r} &
B \arrow{r} &
0.
\end{tikzcd}
\]
Notice that $\eta_K: K \to \phi(K)$ is epi. Therefore, the cokernel of
$\phi(g)$ is the cokernel of $K \to A \to \phi(A)$.  But the
epimorphism $A \to B$ factors through $\phi(A)$. It follows that $\cok
\phi(g) = B$. The rest of the claim now follows.

Let $L$ be the kernel of $\eta_A$. We claim that $L$ is also the 
kernel of the $\eta_K$. Since $f$ factors through $A \to \phi(A)$,
there exists a map from $L$ to $K$, which we call $h$. Applying 
the Snake Lemma to the following commutative diagram:
\[
\begin{tikzcd}
0 \arrow{r} &
L \arrow{r} \arrow{d}{h} &
A \arrow{r} \arrow[equals]{d} &
\phi(A) \arrow{r}\arrow[twoheadrightarrow]{d} &
0 \\
0 \arrow{r} &
K \arrow{r} &
A \arrow{r} &
B \arrow{r} &
0
\end{tikzcd}
\]
we have that $L$ is a subobject of $K$. Let $L'$ be the kernel of
$K \to \phi(K)$. We claim that $L$ is isomorphic to $L'$. Notice
that we have the following commutative diagram:
\begin{equation}\label{eq_corad_commuting_sq1}
\begin{tikzcd}
0 \arrow{r} &
L' \arrow{r}{i'} \arrow[dotted]{d}{j} &
K \arrow{r}{\eta_K} \arrow{d}{g} &
\phi(K) \arrow{r}\arrow{d}{\phi(g)} &
0 \\
0 \arrow{r} &
L \arrow{r}{i}\arrow[hookrightarrow]{ru}{h} &
A \arrow{r}{\eta_A} &
\phi(A) \arrow{r} &
0.
\end{tikzcd}
\end{equation}
Since $\eta_A \comp g \comp i' = \phi(g) \comp \eta_K \comp i' = 
0$, there exists a map $j$ from $L'$ to $L$ (dotted arrow in 
\eqref{eq_corad_commuting_sq1}) such that $ij = gi'$. Applying the 
Snake Lemma to \eqref{eq_corad_commuting_sq1}, we see that $j$ is 
injective.

By the naturality of $\eta$, we also have the following 
commutative square:
\[
\begin{tikzcd}
L \arrow{r}{h} \arrow{d}{\eta_L}&
K \arrow{d}{\eta_K} \\
\phi(L) \arrow{r} &
\phi(K).
\end{tikzcd}
\]
Since $\phi$ is a pre-coradical, $\phi(L) = 0$. Therefore,
$\eta_K \circ h = 0$. Thus, there exists a map $j': L \to L'$
such that $j\comp j' = \id_{L}$ and $j'\comp j = \id_{L'}$.
It follows that $L \cong L'$. Applying the Snake Lemma to
\eqref{eq_corad_commuting_sq1}, we see that $\phi(K) \to
\phi(A)$ is injective, as desired.
\end{proof}

\section{Torsion theories and coradicals}
\label{sect_tt_and_corads}

\begin{defn}
A \emph{torsion theory} for an abelian category $\Cat{A}$ is a 
pair $(\Cat{T}, \Cat{F})$ of full subcategories,
called the \emph{torsion subcategory} and the \emph{torsion-free 
subcategory} respectively, where the objects of $\Cat{T}$ are
the objects $T$ such that $\hom_{\Cat{A}}(T, F) = 0$ for every $F$
in $\Cat{F}$ and the objects of $\Cat{F}$ are the objects $F$ such 
that $\hom_{\Cat{A}}(T, F) = 0$ for every object $T$ in $\Cat{T}$.
\end{defn}

Certainly $0 \in \Cat{T} \cap \Cat{F}$. Therefore, neither 
subcategory is empty. We also have the following characterization
of the torsion and torsionfree subcategories. 

\begin{prop}\label{prop_tt_properties}
Suppose $\Cat{T}$ and $\Cat{F}$ are two full subcategories of a
cocomplete well-powered abelian category $\Cat{A}$. Then 
$\Cat{T}$ is the torsion subcategory of a torsion theory of 
$\Cat{A}$ if and only if $\Cat{T}$ is closed under extensions, 
direct sums and quotients.

Dually, $\Cat{F}$ is a torsionfree subcategory of a torsion theory
of $\Cat{A}$ if and only if $\Cat{F}$ is closed under extensions, 
direct products, and subobjects. (\CF \cite[I2.6]{BJV})
\end{prop}

\begin{proof}
It suffices to verify the statement for torsion subcategories.
Suppose $\Cat{T}$ is a torsion subcategory with $\Cat{F}'$ its 
corresponding torsionfree subcategory. 

\pfitem{Closed under quotients:} suppose $T$ is an object of 
$\Cat{T}$. For any epimorphism $T \to T'$, we have 
\[
0 \to \hom_{\Cat{A}}(T', F) \to \hom_{\Cat{A}}(T, F)
\]
for any $F$ in $\Cat{F}'$. However, $\hom_{\Cat{A}}(T, F) = 0$.
Therefore, $\hom_{\Cat{A}}(T', F) = 0$ for all $F$, and $Y$ is
in $\Cat{T}$.

\emph{Closed under sums:} suppose $\{T_i\}_{i \in I}$ is a 
collection of objects of $\Cat{T}$. We have
\[
\hom_{\Cat{A}}( \oplus_{i \in I} T_i, F) = \prod_{i \in I}
\hom_{\Cat{A}}( T_i, F) = 0
\]
for all $F$ in $\Cat{F}'$. It follows that $\oplus_{i \in I} T_i$
is an object of $\Cat{T}$.

\pfitem{Closed under extensions:} Suppose 
\[
0 \to T' \to A \to T'' \to 0
\]
is an exact sequence in $\Cat{A}$ with $T', T'' \in \Cat{T}$.
Then for any $F$ in $\Cat{F}$, 
\[
0 \to \hom_{\Cat{A}}(T'', F) \to \hom_{\Cat{A}}(A, F) \to
\hom_{\Cat{A}}(T', F).
\]
Since $\hom_{\Cat{A}}(T'', F) = \hom_{\Cat{A}}(T', F) = 0$,
it follows that $\hom_{\Cat{A}}(A, F) = 0$ for all $F$. Therefore, 
$A$ is in $\Cat{T}$.

Conversely, suppose $\Cat{T}$ is closed under extensions, direct 
sums and quotients. Let $\Cat{F}'$ be the full subcategory of $F$ 
such that $\hom_{\Cat{A}}(T, F) = 0$ for all $T$ in $\Cat{T}$, and let 
$\Cat{T}'$ be the full subcategory of $\Cat{A}$ whose objects are 
all $T'$ such that $\hom_{\Cat{A}}(T', F) = 0$ for all $F$ in 
$\Cat{F}$. We claim that $\Cat{T'} = \Cat{T}.$

Clearly, $\Cat{T}$ is a full subcategory of $\Cat{T'}$. Let $T$ be
an object of $\Cat{T'}$. By Proposition \ref{prop_max_subobj}, there 
exists a maximal $\Cat{T}$-subobject of $T$, which we represent by
$T_{\Cat{T}}$. We show that $T/T_{\Cat{T}}$ is an object of 
$\Cat{F}'$, and therefore it must be 0.

Suppose not. Then there exists some $T'$ in $\Cat{T}$ with a 
nonzero map $f: T' \to T/T_{\Cat{T}}$. Since $f(T')$ is an object 
in $\Cat{T}$, replacing $T'$ by its image in $T/T_{\Cat{T}}$, we 
may assume without loss of generality that $f$ is monic.

Pull back $T \to T/T_{\Cat{T}}$ by $f$, and we have:
\[
\begin{tikzcd}
0 \arrow{r} &
T_{\Cat{T}} \arrow{r}\arrow[equals]{d} &
P \arrow{r}{p} \arrow{d}{i} &
T' \arrow{r} \arrow[hook]{d}{f} &
0 \\
0 \arrow{r} &
T_{\Cat{T}} \arrow{r} &
T \arrow{r} &
T/T_{\Cat{T}} \arrow{r} &
0
\end{tikzcd}
\]
As $i$ is a pullback of a monomorphism, $i$ is itself monic.  As $T
\to T/T_\Cat{T}$ is epimorphic, so is $p$. Furthermore, $\ker p =
T_{\Cat{T}}$. Since $T_{\Cat{T}}$ and $T'$ are both in $\Cat{T}$, it
follows that $P$ must be in $\Cat{T}$ as well. However, $T'$ is
nontrivial, contradicting the maximality of $T_{\Cat{T}}$. Thus,
$T/T_{\Cat{T}} \in \Cat{F}$, and $T \in \Cat{T}$.
\end{proof}

\begin{prop}\label{prop_tt_suff_cond}
Let $(\Cat{T}, \Cat{F})$ be a pair of full subcategories of
a cocomplete well-powered abelian category
$\Cat{A}$. Then $(\Cat{T}, \Cat{F})$ is
a torsion theory if and only if the following conditions hold:
\begin{enumerate}
\item the only common object of $\Cat{T}$ and $\Cat{F}$ is 0.

\item for every $A$ in $\Cat{A}$, there exists a subobject 
$A_{\Cat{T}}$ of $A$ in $\Cat{T}$ such that $A/A_{\Cat{T}}$ 
is an object of $\Cat{F}$.
\end{enumerate}
(\CF \cite[I2.7]{BJV})
\end{prop}
\begin{proof}
\noindent $\Rightarrow$: Suppose $A$ is $\Cat{T} \cap \Cat{F}$.
Then $\hom_{\Cat{A}}(A,A) = 0$, so the identity is the zero map, and
$A = 0$. Now, for $A$ in $\Cat{A}$, let $A_{\Cat{T}}$ be its maximal
$\Cat{T}$ subobject. By the same reasoning as in the previous
proposition, $A/A_{\Cat{T}}$ is an object of $\Cat{F}$.

\vskip 10pt
\noindent $\Leftarrow$: suppose $\Cat{T}, \Cat{F}$ satisfy the 
condition of the proposition, and there is some $A$ in $\Cat{A}$ 
such that for all $F$ in $\Cat{F}$, $\hom_{\Cat{A}}(A, F) = 0$. Let
$A_{\Cat{T}}$ denote the $\Cat{T}$-subobject in Condition (2) 
associated to $A$. Since $A/A_{\Cat{T}} \in \Cat{F}$, $A \to 
A/A_{\Cat{T}}$ is the zero map. Hence, $A = A_{\Cat{T}}$, and
$A$ is in $\Cat{T}$. Similarly, if $F \in \Cat{F}$, then the 
inclusion $F_{\Cat{T}} \to F$ is the zero map, and hence 
$F/F_{\Cat{T}} = F$ which is in $\Cat{F}$.
\end{proof}

\begin{prop}\label{prop_tt_to_corad}
  Let $(\Cat{T}, \Cat{F})$ be a torsion theory for a cocomplete
  well-powered abelian category $\Cat{A}$.  Sending $A$ in $\Cat{A}$
  to its largest $\Cat{T}$-subobject $A_{\Cat{T}}$ defines an
  idempotent pre-radical.

Dually, sending $A$ to $A/A_{\Cat{T}}$ defines an idempotent 
pre-coradical (\CF \cite[I2.8]{BJV}).
\end{prop}
\begin{proof}
In this case, it is easier to prove the statement for idempotent 
pre-radicals. Let $\kappa$ denote the association defined by $A 
\mapsto A_{\Cat{A}}$ for $A$ in $\Cat{A}$.

To see that $\kappa$ is a functor, let $f: A \to B$ be any 
morphism. The image of $\kappa(A)$ in $B$ under $f$ is in 
$\Cat{T}$. By the maximality of $\kappa(B)$, there exists a map 
$g: f(\kappa(A)) \to \kappa(B)$, and define the map $\kappa(f)$ 
to be the composition of $g f|_{\kappa(A)}$.

It is clear from the construction that $\kappa$ is a subobject 
functor. Since $\kappa(A) \in \Cat{T}$, it is clear that the 
largest suboboject of $\kappa(A)$ is itself: hence $\phi^2(A)) = 
\phi(A)$. By the maximality of $\kappa(A)$, $A/\kappa(A) \in 
\Cat{F}$, and 
\[
\kappa(A/\kappa(A)) = 0.
\]

The dual statement follows from Proposition \ref{prop_rad_eq_corad}.
\end{proof}

\begin{rmk}
Since a coradical $\phi$ is left adjoint to the inclusion of its 
associated torsionfree subcategory $\Cat{F}$ in $\Cat{A}$ and its 
associated idempotent pre-radical $\kappa$ is right adjoint to the
inclusion of the torsion subcategory $\Cat{T}$ in $\Cat{A}$, $\Cat{T}$
is a coreflective subcategory, and $\Cat{F}$ is a reflective 
subcategory of $\Cat{A}$.
\end{rmk}

\begin{thm}\label{thm_precorad_eq_tt}
Let $\Cat{A}$ be a cocomplete well-powered abelian category.
There is a one-to-one correspondence between isomorphism 
classes of idempotent pre-coradicals of $\Cat{A}$ and torsion theories for
$\Cat{A}$. If $\phi$ is a pre-coradical, and $\eta$ is its 
associated natural epimorphism, then the torsion theories are 
defined by
\begin{align*}
\Cat{T} &= \{ T |\; \phi(T) = 0 \} \\
\Cat{F} &= \{ F |\; \eta_F: F \to \phi(F)
                  \textrm{ is an isomorphism}\}.
\end{align*}
(\CF \cite[I2.9]{BJV}).
\end{thm}
\begin{proof}
Obtaining an idempotent pre-coradical from a torsion theory is
established by Proposition \ref{prop_tt_to_corad}. Therefore, it suffices
to show that $(\Cat{T}, \Cat{F})$ as given in the statement of the
theorem defines a torsion theory on $\Cat{A}$, and that the associations
define quasi-inverses of one another.

To do this, we appeal to Proposition \ref{prop_tt_suff_cond}. It is 
clear that the only object common to both $\Cat{T}$
and $\Cat{F}$ is $0$. So we need only to show that for every $A$ 
in $\Cat{A}$, there exists $T$ in $\Cat{T}$ such that $A/T \in 
\Cat{F}$.

Fix $A$ in $\Cat{A}$. Since $\phi$ is idempotent, $\phi(A)$ is 
in $\Cat{F}$. Since $\phi$ is a pre-coradical, the kernel of $A
\to \phi(A)$ is in $\Cat{T}$.
\end{proof}

As we have mentioned in the paragraph preceding Proposition
\ref{prop_idempotence}, there is a result corresponding to Theorem 
\ref{thm_precorad_eq_tt} for radicals: the isomorphism classes of 
idempotent pre-radical $\kappa$ are in one-to-one correspondence 
with torsion theories on $\Cat{A}$. For a given pre-radical 
$\kappa$ with natural inclusion $\epsilon$, the associated torsion 
theory is defined by 
\begin{align*}
\Cat{T} &= \{ T | \epsilon_F: \kappa(T) \to T 
                  \textrm{ is an isomorphism}\} \\
\Cat{F} &= \{ F | \kappa(F) = 0 \}.
\end{align*}
In fact, we have the following.

\begin{cor}\label{cor_tt_ref_and_coref}
Let $\phi$ be an idempotent pre-coradical, and let $\kappa$ be the 
idempotent pre-radical associated to $\phi$ (see Proposition
\ref{prop_rad_eq_corad}). Then the torsion theory defined by $\phi$
in Theorem \ref{thm_precorad_eq_tt} is the same as the one for
$\kappa$ as defined above.

Moreover, $\phi$ is left adjoint to the inclusion $\Cat{F} \to 
\Cat{A}$ and $\kappa$ is right adjoint to the inclusion $\Cat{T} 
\to \Cat{A}$.
\end{cor}
\begin{proof}
The only thing left to verify is that $\phi$ defines a left 
adjoint to the inclusion of $\Cat{F}$ into $\Cat{A}$ and $\kappa$ 
defines a right adjoint to the inclusion of $\Cat{T}$ into 
$\Cat{A}$. We verify the statement only for $\phi$ and leave the 
latter to the reader.  

For $\phi$, let $A$ be an object of $\Cat{A}$, and let $F$ be an 
object of $\Cat{F}$. Consider the short exact sequence
\[
0 \to \kappa(A) \to A \to \phi(A) \to 0.
\]
Applying $\hom_{\Cat{A}}(-, F)$, we have the exact sequence
\[
0 \to \hom_{\Cat{A}}(\phi(A), F) \to \hom_{\Cat{A}}(A, F) \to
\hom_{\Cat{A}}(\kappa(A), F)
\]
Since $\kappa(A) \in \Cat{T}$ (Theorem \ref{thm_precorad_eq_tt}) 
and $\phi(A) \in \Cat{F}$, $\hom_{\Cat{A}}(\kappa(A), F) = 0$, and
\[
\hom_{\Cat{A}}(\phi(A), F) = \hom_{\Cat{F}}(\phi(A), F) \cong
   \hom_{\Cat{A}}(A, F)
\]
as desired.
\end{proof}

It should be evident from Theorem \ref{thm_precorad_eq_tt} that isomorphism
classes of
coradicals are not in one-to-one correspondence with torsion theories,
although they do give rise to unique torsion theories. We now
characterize the properties of the torsion theories that arise from
coradicals.

\begin{defn}
Let $(\Cat{T}, \Cat{F})$ be a torsion theory on $\Cat{A}$. We say 
that $(\Cat{T}, \Cat{F})$ is \DEF{hereditary} if $\Cat{T}$ is 
closed with respect to subobjects. That is, if $A \into B$ is an 
monomorphism in $\Cat{A}$ such that $B \in \Cat{T}$, then $A \in 
\Cat{T}$.

Dually, we say that $(\Cat{T}, \Cat{F})$ is \DEF{cohereditary} if
$\Cat{F}$ is closed under quotients.
\end{defn}

\begin{thm}\label{thm_corad_equiv_htt}
Let $\Cat{A}$ be a cocomplete well-powered abelian category.
There is a one-to-one correspondence between isomorphism 
classes of coradicals of $\Cat{A}$ and cohereditary torsion 
theories on $\Cat{A}$ (\CF \cite[I2.12]{BJV}).
\end{thm}
\begin{proof}

  \pfitem{From coradicals to cohereditary torsion theories} : Let
  $\phi$ be a coradical with natural epimorphisms $\eta$, and
  $(\Cat{T}, \Cat{F})$ the associated torsion theory, given by Theorem
  \ref{thm_precorad_eq_tt}.

We need only to show that $\Cat{F}$ defined by $\{A | \eta: A \to 
\phi(A) \textrm{ is an isomorphism}\}$ is closed under quotients.
Let $f: F \to A$ be an epimorphism with $F$ in $\Cat{F}$,
and write $K$ for the kernel of $f$. It follows from Proposition
\ref{prop_tt_properties} that $\Cat{F}$ is closed under 
subobjects. Hence, $K \in \Cat{F}$. Furthermore, by the right 
exactness of $\phi$, we have
\[
\begin{tikzcd}
0 \arrow{r} & 
K \arrow{r}{i} \arrow[equals]{d} &
F \arrow{r}{f} \arrow[equals]{d} &
A \arrow{r} \arrow{d} &
0 \\
& \phi(K) \arrow{r}{\phi(i) = i} &
\phi(F) \arrow{r}{\phi(f)} &
\phi(A) \arrow{r} &
0,
\end{tikzcd}
\]
whence $A = \phi(A)$ as desired. It follows that $A \in \Cat{F}$.

\pfitem{From cohereditary torsion theory to coradicals} : Let
$(\Cat{T}, \Cat{F})$ be a cohereditary torsion theory on $\Cat{A}$,
and let $\phi$ be its associated idempotent pre-coradical given by 
Theorem \ref{thm_precorad_eq_tt}. We need to show that $\phi$ is right 
exact.

We begin by demonstrating that, for an epimorphism $f: A \to B$ in
$\Cat{A}$, $\phi(B)$ is isomorphic to the push-out $P$ of $f$ and the
natural epimorphism $\eta_A: A \to \phi(A)$, as in the following
diagram:
\[
\begin{tikzcd}
A \arrow{r} \arrow{d} & B \arrow[dotted]{d}\\
\phi(A) \arrow[dotted]{r} & P.
\end{tikzcd}
\]
Since $\phi$ is left adjoint to inclusion, we have that
\[
\hom_{\Cat{A}}(\phi(B), F) = \hom_{\Cat{F}}(\phi(B), F) = 
   \hom_{\Cat{A}}(B, F)
\]
for all $F$ in $\Cat{F}$. In particular, for all $F$ in $\Cat{F}$, 
and epimorphisms $B \onto F$, there exists a unique map $\phi(B) 
\to F$ making the following diagram commutative
\[
\begin{tikzcd}
B \arrow[twoheadrightarrow]{rd}{\eta_B} \arrow{rr} & & F \\
&\phi(B) \arrow[dotted]{ru}.
\end{tikzcd}
\]

Now, since $P$ is the push-out, and $\phi(B)$ fits into the 
following commutative diagram
\begin{equation}\label{eq_corad_pushout}
\begin{tikzcd}
A \arrow{r} \arrow{d}{\eta_A} & B \arrow{d}{\eta_B} \\
\phi(A) \arrow{r} &\phi(B)
\end{tikzcd}
\end{equation}
there exists an unique map $P \to \phi(B)$. Furthermore, the map 
$\phi(A) \to P$ is an epimorphism since it is the push-out of the 
epimorphism $A \to B$. Since $\phi(A) \in \Cat{F}$, which is closed 
under quotients, it follows that $P \in \Cat{F}$. The map $B \to 
P$ is also an epimorphism because it is the push-out of the 
epimorphism $A \to \phi(A)$. It follows by the previous point that 
there exists an unique map $\phi(B) \to P$.

Since both maps are unique, it follows that each map is an 
isomorphism and is the inverse of the other. Furthermore,
Diagram \eqref{eq_corad_pushout} is a push-out diagram.

To complete the proof that $\phi$ is right exact, we need only to 
show that for an exact sequence 
\[
0 \to A' \stackrel{f}{\to} A \stackrel{g}{\to} A'' \to 0
\] 
in $\Cat{A}$, $\phi(A'')$ is the cokernel of $\phi(A') \to 
\phi(A)$. Consider the commutative diagram
\[
\begin{tikzcd}
0 \arrow{r} & 
A' \arrow{r}{f} \arrow{d}{\eta_{A'}} &
A \arrow{r}{g} \arrow{d}{\eta_A}&
A'' \arrow{r}\arrow{d}{\eta_{A''}} &
0 \\
& \phi(A') \arrow{r}{\phi(f)} &
\phi(A) \arrow{r}{\phi(g)} \arrow{rd}{h} &
\phi(A'') \arrow{r} &
0 \\
& & & \cok \phi(f) \arrow[dotted]{u}{p} \arrow{r} &0,
\end{tikzcd}
\]
where the top row is exact. Since the composition $\phi(A') \to 
\phi(A) \to \phi(A'')$ is 0, there exists an unique map $\cok 
\phi(f) \stackrel{p}{\to} \phi(A'')$ (shown as the dotted arrow in 
the diagram above), such that $ph = \phi(g)$.

However, we also have that $h \comp \eta_A \comp f = h \comp 
\phi(f) \comp \eta_{A'} = 0$. It follows that there exists a map 
$A'' \to \cok \phi(f)$ (represented by the dotted arrow in the 
following diagram) such that the following diagram is commutative:
\[
\begin{tikzcd}
Y \arrow{r}{g}\arrow{d}{\eta_Y} & Z \arrow[dotted]{d} \\
\phi(Y) \arrow{r} &\cok \phi(f).
\end{tikzcd}
\]
But $\phi(A'')$, as a push-out, admits an unique map
$\phi(A'') \stackrel{p'}{\to} \cok \phi(f)$. Once again, since the 
maps defined between $\phi(A'')$ and $\cok \phi(f)$ are unique with
respect to $\phi(g)$ and $h$, it follows that $p$ and $p'$ are 
isomorphisms and define inverses of one another. This concludes 
the theorem.
\end{proof}

\begin{rmk}
Notice that if $\phi$ is a coradical, then 
$\Cat{F}$ is a \emph{Serre subcategory} of $\Cat{A}$. In particular,
$\Cat{F}$ is an abelian category. In the case 
when $\Cat{A}$ has ``enough $\Cat{F}$-covers'', then the torsion 
subcategory $\Cat{T}$ is precisely the localization of $\Cat{A}$ 
by $\Cat{F}$ in the sense of \cite{Swan}, and the associated 
idempotent radical $\kappa$ is an exact radical.
\end{rmk}

\newpage
\section{Coradicals and Torsion Theory}\label{sect_torsion_theory}

In this section, we develop the basics of torsion theory in a 
categorical setting. The concepts and results here closely follow
those of \cite{BJV} and \cite{DTor}, except we develop the theory
from the dual perspective of the coradicals. The ideas are not new;
neither is the methodology. We have included the proof for 
convenience of the reader.

For the following, let $\Cat{A}$ be a well-powered abelian category.
That is, for every object $A \in \Cat{A}$, the collection of 
subobjects of $A$ form a set.

The following notions will be important to us:

\begin{defn}\label{def_coradical}
Let $\phi: \Cat{A} \to \Cat{A}$ be an endofunctor. 
\begin{enumerate}
\item We say that $\phi$ is a \emph{quotient functor} if there 
exists a natural surjection $\eta: \id \to \phi$. 

That is, for every $f: A \to B$, the following diagram commutes.
\[
\begin{tikzcd}
A \arrow{r}{f} \arrow{d}{\eta_A} &
B \arrow{d}{\eta_B} \\
\phi A \arrow{r}{\phi f} 
&\phi B
\end{tikzcd}
\]

\item We say that $\phi$ is \emph{idempotent} if $\phi^2 = \phi$.

\item A quotient functor $\phi$ is a \emph{pre-coradical} if for all
$A \in \Cat{A}$, $\phi$ applied to the kernel of the surjection 
$A \to \phi(A)$ is $0$.

\item Finally, a pre-coradical is a \emph{coradical} if $\phi$ is
right exact.
\end{enumerate}
\end{defn}

Torsion theory is usually developed for radical functors, which 
are the dual of a coradical. Formally:

\begin{defn}\label{def_radical}
Let $\phi: \Cat{A} \to \Cat{A}$ be an endofunctor of an abelian 
category $\Cat{A}$. We say that $\phi$ is a \emph{pre-radical} if 
there exists a natural monomorphism $\phi \to \id$ (in which case, we
say that $\phi$ is a \emph{subobject functor})such that 
$\phi(A/\phi(A)) = 0$ for all $A$. $\phi$ is a \emph{radical} if 
$\phi$ is additionally left exact.
\end{defn}

\begin{rmk}
In the case where $\Cat{A}$ is small, via Mitchell embedding 
there exists an embedding of $\Cat{A}$ as a subcategory of $R$-mod
for some suitable ring $R$. In this case, the subobject functors, 
quotient functors, idempotence, (pre-)radicals, and 
(pre-)coradicals corresponds to their counterparts in the ring 
theoretic setting.
\end{rmk}

\begin{rmk}
Notice that being a quotient and being a subobject are dual 
notions. All the proofs in this section for coradicals have an 
corresponding dual, which proves a parallel statement for radicals.
Throughout this section, we consider only statements for 
(pre-)coradicals functors. We leave the dual statements to the 
reader to formulate or look up in \cite{DTor}.
\end{rmk}

\begin{prop}
Any right exact quotient functor $\phi$ is idempotent. 
\end{prop}
\begin{proof}
Fix $A \in \Cat{A}$, and let $\eta$ denote the natural surjection 
associated to the quotient functor $\phi$. Let $K$ be the kernel 
of $\eta_A: A \to \phi(A)$, and consider the sequence
\[
0 \to K \to A \to \phi(A) \to 0.
\]
Applying $\phi$, which is right exact, we have
\[
\phi(K) \to \phi(A) \to \phi^2(A) \to 0.
\]
Namely, $\phi^2(A)$ is the cokernel of $\phi(K) \to \phi(A)$.

Moreover, we have the following commutative diagram:
\[
\begin{tikzcd}
K \arrow{r} \arrow{d}{\eta_K} &
A \arrow{d}{\eta_A} \\
\phi(K) \arrow{r} \arrow{d} &
\phi(A) \arrow{d} \\
0 & 0
\end{tikzcd}
\]
and notice that since $K \to \phi(K)$ is a surjection,
\begin{align*}
\phi^2(A) &= \cok (\phi(K) \to \phi(A)) \\
          &= \cok (K \to \phi(K) \to \phi(A)) \\
          &= \cok (K \to A \to \phi(A)).
\end{align*}
But $K \to A \to \phi(A)$ is the $0$ map. Therefore, $\phi^2(A) = 
\phi(A)$ as desired.
\end{proof}

\begin{prop}
Let $\phi$ be a pre-coradical of $\Cat{A}$. Suppose $B$ is a 
quotient of $\phi(A)$, and let $K$ be the kernel of the 
composition $A \to \phi(A) \to B$. Then $\phi(K)$ is isomorphic 
to the kernel of the surjection $\phi(A) \to B$.
\end{prop}
\begin{proof}
Let $\eta$ denote the natural surjection associated to the 
quotient functor $\phi$, and let $f$ denote the surjection given 
by the composition $A \to \phi(A) \to B$.

Consider the exact sequence $0 \to K \to A \to B \to 0$. We claim
that $\phi(K) \to \phi(A) \to B \to 0$ is exact and fits into the 
following commutative diagram:
\[
\begin{tikzcd}
0 \arrow{r} &
K \arrow{r}{g} \arrow[twoheadrightarrow]{d}{\eta_K} &
A \arrow{r}{f} \arrow[twoheadrightarrow]{d}{\eta_A} &
B \arrow{r} \arrow[equal]{d} &
0 \\
& \phi(K) \arrow{r}{\phi f} &
\phi(A) \arrow{r} &
B \arrow{r} &
0
\end{tikzcd}
\]
In particular, we show that $\cok \phi f = B$. Indeed, we have
that $\eta_K: K \to \phi(K)$ is surjective. Therefore,
\begin{align*}
\cok \phi f  &= \cok K \to \phi(K) \to \phi(A) \\
             &= \cok K \to A \to \phi(A)
\end{align*}
But the map from $A \to B$ factors through $\phi(A)$. It follows
that $\cok \phi f = B$.

Now consider the following commutative diagram:
\[
\begin{tikzcd}
{}
&\phi(K) \arrow{r} \arrow[dotted]{d}{h}
&\phi(A) \arrow{r} \arrow[equals]{d}
&B \arrow{r} \arrow[equals]{d}
&0\\
0 \arrow{r}
& K' \arrow{r}
&\phi(A) \arrow{r}
&B \arrow{r}
&0
\end{tikzcd}
\]
where $K'$ is the kernel of $\phi(A) \to B$, and the dotted arrow
follows from the fact that $K'$ is a kernel. Call this map $h$. 
By the Snake Lemma, the map $\phi(K) \to K'$ is surjective.

Furthermore, we have the following commutative diagram, with
exact rows:
\[
\begin{tikzcd}
0 \arrow{r} &
K \arrow{r} \arrow{d} &
A \arrow{r} \arrow{d} &
B \arrow{r} \arrow{d} &
0 \\
0 \arrow{r} &
K' \arrow{r} &
\phi(A) \arrow{r} &
B \arrow{r} &
0.
\end{tikzcd}
\]
Applying Snake Lemma once again, we see that the kernel of $K \to 
K'$ is isomorphic to the kernel $A \to \phi(A)$. Let $L$ denote
this kernel.

Furthermore, $h$ fits into the following commutative diagram:
\[
\begin{tikzcd}
0 \arrow{r} &
L \arrow{r} \arrow{d} &
K \arrow{r} \arrow{d} &
K' \arrow{r} \arrow[equals]{d} \arrow[dotted]{ld} &
0 \\
& \phi(L) \arrow{r} &
\phi(K) \arrow{r}{h} &
K' \arrow{r} &
0
\end{tikzcd}
\]
where the top row is exact. Since $\phi$ is a pre-coradical, 
$\phi(L) = 0$. Thus, the composition $L \to K \to \phi(K)$ is also
$0$. However, $K'$ is the cokernel, and therefore, there exists a
map $K' \to \phi(K)$ (represented by the dotted arrow in the above 
diagram). It follows that $h$ is an isomorphism as desired.
\end{proof}

\begin{defn}
Let $\Cat{A}$ be an abelian category. A \emph{torsion theory} for
$\Cat{A}$ is a pair $(\Cat{T}, \Cat{F})$ of full subcategories 
called the \emph{torsion subcategory} and \emph{torsion-free 
subcategory} respectively, where the objects of $\Cat{T}$ are
objects $X$ such that $\hom_{\Cat{A}}(X, Y) = 0$ for every $Y
\in \Cat{F}$ and objects of $\Cat{F}$ are objects $Y$ such that
$\hom_{\Cat{A}}(X, Y) = 0$ for every object $X \in \Cat{T}$.
\end{defn}

Certainly $0 \in \Cat{T} \cap \Cat{F}$. Therefore, neither 
subcategories are empty.

\begin{prop}
Let $\Cat{A}$ be a well-powered abelian category with a torsion 
theory, and $\Cat{T}$ and $\Cat{F}$ are two nonempty full 
subcategories. Then, $\Cat{T}$ is a torsion subcategory of 
$\Cat{A}$ if and only if $\Cat{T}$ is closed under extensions, 
direct sums and quotients. Dually, $\Cat{F}$ is a torsionfree 
subcategory of $\Cat{A}$ if and only if $\Cat{F}$ is closed under 
extensions, direct products, and submodules.
\end{prop}


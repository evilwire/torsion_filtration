\newpage
\section{Coradicals and Torsion Theory}\label{sect_torsion_theory}

In this section, we develop the basics of torsion theory in a 
categorical setting. The concepts and results here closely follow
those of \cite{BJV} and \cite{DTor}, except we develop the theory
from the dual perspective of the coradicals. The ideas are not new;
neither is the methodology. We have included the proof for 
convenience of the reader.

For the following, let $\Cat{A}$ be a well-powered abelian category.
That is, for every object $A \in \Cat{A}$, the collection of 
subobjects of $A$ form a set.

The following notions will be important to us:

\begin{defn}\label{def_coradical}
Let $\phi: \Cat{A} \to \Cat{A}$ be an endofunctor. 
\begin{enumerate}
\item We say that $\phi$ is a \DEF{quotient functor} if there 
exists a natural surjection $\eta: \id \to \phi$. 

That is, for every $f: A \to B$, the following diagram commutes.
\[
\begin{tikzcd}
A \arrow{r}{f} \arrow{d}{\eta_A} &
B \arrow{d}{\eta_B} \\
\phi A \arrow{r}{\phi f} 
&\phi B
\end{tikzcd}
\]

\item We say that $\phi$ is \DEF{idempotent} if the natural 
surjection is the identity on the essential image of $\phi$.
That is, if we set $\eta: \id \to \phi$ to be natural surjection,
then $\eta_{\phi(A)}: \phi(A) \to \phi^2(A)$ is the identity map.

\item A quotient functor $\phi$ is a \DEF{pre-coradical} if for all
$A \in \Cat{A}$, $\phi$ applied to the kernel of the surjection 
$A \to \phi(A)$ is $0$.

\item Finally, we say $\phi$ is a \DEF{coradical} if $\phi$ is
right exact.
\end{enumerate}
\end{defn}

\begin{rmk}
Notice that quotient functors always take surjections to 
surjections. However, pre-coradicals are not always right exact.
If $0 \to A' \to A \to A'' \to 0$ is an exact sequence, and $\phi$
is a pre-coradical, exactness can fails at $\phi(A)$.
\end{rmk}

Torsion theory is usually developed for radical functors, which 
are the dual of a coradical. Formally:

\begin{defn}\label{def_radical}
Let $\phi: \Cat{A} \to \Cat{A}$ be an endofunctor of an abelian 
category $\Cat{A}$. We say that $\phi$ is a \DEF{radical} if 
there exists a natural monomorphism $\phi \to \id$ (in which case, we
say that $\phi$ is a \DEF{subobject functor}) such that 
$\phi(A/\phi(A)) = 0$ for all $A$. 
\end{defn}

\begin{rmk}
In the case where $\Cat{A}$ is small, via Mitchell embedding 
there exists an embedding of $\Cat{A}$ as a subcategory of $R$-mod
for some suitable ring $R$. In this case, the subobject functors, 
quotient functors, idempotence, radicals, and coradicals 
corresponds to their counterparts in the ring theoretic setting.
\end{rmk}

\begin{rmk}\label{rmk_rad_corad_dual}
Notice that quotient and subobject functors are dual notions. So 
are radical and coradicals. In fact, all the results in this 
section for coradicals have an corresponding dual, i.e. a parallel 
statement for radicals. Throughout this section, we mostly 
consider statements for only (pre-)coradicals functors. We leave 
some dual statements to the reader to formulate or look up in 
\cite{DTor}.
\end{rmk}

\begin{prop}\label{prop_idempotence}
Any right exact quotient functor $\phi$ is idempotent. In 
particular, any coradical is idempotent.
\end{prop}
\begin{proof}
Fix $A \in \Cat{A}$, and let $\eta$ denote the natural surjection 
associated to the quotient functor $\phi$. Let $K$ be the kernel 
of $\eta_A: A \to \phi(A)$, and consider the sequence
\[
0 \to K \to A \to \phi(A) \to 0.
\]
Applying $\phi$, which is right exact, we have
\[
\phi(K) \to \phi(A) \to \phi^2(A) \to 0.
\]
Namely, $\phi^2(A)$ is the cokernel of $\phi(K) \to \phi(A)$.

Moreover, we have the following commutative diagram:
\[
\begin{tikzcd}
K \arrow{r} \arrow{d}{\eta_K} &
A \arrow{d}{\eta_A} \\
\phi(K) \arrow{r} \arrow{d} &
\phi(A) \arrow{d} \\
0 & 0
\end{tikzcd}
\]
and notice that since $K \to \phi(K)$ is a surjection,
\begin{align*}
\phi^2(A) &= \cok (\phi(K) \to \phi(A)) \\
          &= \cok (K \to \phi(K) \to \phi(A)) \\
          &= \cok (K \to A \to \phi(A)).
\end{align*}
But $K \to A \to \phi(A)$ is the $0$ map. Therefore, $\phi^2(A) = 
\phi(A)$ as desired.
\end{proof}

In addition to being dual notions, coradicals (or idempotent 
pre-radicals) are related to idempotent pre-coradicals in the
following way:

\begin{prop}\label{prop_rad_eq_corad}
Let $\phi$ be an idempotent pre-coradical, and $\eta$ be its 
corresponding natural surjection. Write $\kappa(X) = \ker (X 
\stackrel{\eta_X}{\to} \phi(X))$. Then $\kappa$ is a radical.

Dually, if $\psi$ is an idempotent pre-radical with natural 
injection $\epsilon$. Writing $\gamma(X) = \cok \epsilon_X$, we 
have that $\gamma$ is a coradical.
\end{prop}
\begin{proof}
It suffices to prove this statement for the idempotent 
pre-coradicals, as the arguments for pre-radical can be 
straightforwardly obtained by dualizing. In this case, let $\phi$, 
$\eta$ and $\kappa$ be given as above. We proceed as follows:

\pfitem{$\kappa$ is functorial}: This follows from the naturality 
of $\eta$. Indeed, we need only to define $\kappa$ for morphisms. 
Consider $f: X \to Y$ a map in $\Cat{A}$. From naturality of 
$\eta$, we have the following commutative diagram:
\[
\begin{tikzcd}
0 \arrow{r} &
\kappa(X) \arrow{r} \arrow[dotted]{d} &
X \arrow{r}{\eta_X} \arrow{d}{f} &
\phi(X) \arrow{r} \arrow{d}{\phi f} &
0 \\
0 \arrow{r} &
\kappa(Y) \arrow{r} &
Y \arrow{r}{\eta_Y} &
\phi(Y) \arrow{r} &
0.
\end{tikzcd}
\]
By the universality of $\kappa(Y)$ as the kernel, we have the 
induced map $\kappa(X) \to \kappa(Y)$ (shown as the dotted arrow 
in the preceding diagram). We define $\kappa f$ to be this induced
map. To finish, we need to show that for $f, g$ morphisms in 
$\Cat{A}$, $\kappa(fg) = \kappa(f)\kappa(g)$. This follows from
the uniqueness of induced map from the kernel.

\pfitem{$\kappa$ is a pre-radical}: From the above, it is clear 
that $\kappa$ is a subobject functor. To see that $\kappa$ is also
a pre-radical, we show that $\kappa(X/\kappa(X)) = 0$ for all $X 
\in \Cat{A}$. Fix such an $X$, and notice that $X/\kappa(X) = 
\phi(X)$. Then we have the associated short exact sequence:
\[
0 \to \kappa(\phi(X)) \to \phi(X) \to \phi^2(X) \to 0.
\]
But $\phi(X) \to \phi^2(X)$ is the identity. It follows that 
$\kappa(\phi(X)) = \kappa(X/\kappa(X)) = 0$.

\pfitem{$\kappa$ is idempotent}: Fix $X \in \Cat{A}$, and consider
the following short exact sequence associated to $\kappa(X)$:
\[
0 \to \kappa(\kappa(X)) \to \kappa(X) 
   \stackrel{\eta_{\kappa(X)}}{\to} \phi(\kappa(X)) \to 0.
\]
However, $\phi$ is a pre-radical, and therefore, 
\[
\phi(\kappa(X)) = \phi(\ker(X \to \phi(X))) = 0.
\]
It follows that $\kappa^2(X) = \kappa(X)$, and the proposition 
follows.
\end{proof}

\begin{prop}
Let $\phi$ be a pre-coradical of $\Cat{A}$. Suppose $B$ is a 
quotient of $\phi(A)$, and let $K$ be the kernel of the 
composition $A \to \phi(A) \to B$. Then $\phi(K)$ is isomorphic 
to the kernel of the surjection $\phi(A) \to B$.
\end{prop}
\begin{proof}
Let $\eta$ denote the natural surjection associated to the 
quotient functor $\phi$, and let $f$ denote the surjection given 
by the composition $A \to \phi(A) \to B$.

Consider the exact sequence $0 \to K \to A \to B \to 0$. We claim
that $\phi(K) \to \phi(A) \to B \to 0$ is exact and fits into the 
following commutative diagram:
\[
\begin{tikzcd}
0 \arrow{r} &
K \arrow{r}{g} \arrow[twoheadrightarrow]{d}{\eta_K} &
A \arrow{r}{f} \arrow[twoheadrightarrow]{d}{\eta_A} &
B \arrow{r} \arrow[equal]{d} &
0 \\
& \phi(K) \arrow{r}{\phi f} &
\phi(A) \arrow{r} &
B \arrow{r} &
0
\end{tikzcd}
\]
In particular, we show that $\cok \phi f = B$. Indeed, we have
that $\eta_K: K \to \phi(K)$ is surjective. Therefore,
\begin{align*}
\cok \phi f  &= \cok K \to \phi(K) \to \phi(A) \\
             &= \cok K \to A \to \phi(A)
\end{align*}
But the map from $A \to B$ factors through $\phi(A)$. It follows
that $\cok \phi f = B$.

Now consider the following commutative diagram:
\[
\begin{tikzcd}
{}
&\phi(K) \arrow{r} \arrow[dotted]{d}{h}
&\phi(A) \arrow{r} \arrow[equals]{d}
&B \arrow{r} \arrow[equals]{d}
&0\\
0 \arrow{r}
& K' \arrow{r}
&\phi(A) \arrow{r}
&B \arrow{r}
&0
\end{tikzcd}
\]
where $K'$ is the kernel of $\phi(A) \to B$, and the dotted arrow
follows from the fact that $K'$ is a kernel. Call this map $h$. 
By the Snake Lemma, the map $\phi(K) \to K'$ is surjective.

Furthermore, we have the following commutative diagram, with
exact rows:
\[
\begin{tikzcd}
0 \arrow{r} &
K \arrow{r} \arrow{d} &
A \arrow{r} \arrow{d} &
B \arrow{r} \arrow{d} &
0 \\
0 \arrow{r} &
K' \arrow{r} &
\phi(A) \arrow{r} &
B \arrow{r} &
0.
\end{tikzcd}
\]
Applying Snake Lemma once again, we see that the kernel of $K \to 
K'$ is isomorphic to the kernel $A \to \phi(A)$. Let $L$ denote
this kernel.

Furthermore, $h$ fits into the following commutative diagram:
\[
\begin{tikzcd}
0 \arrow{r} &
L \arrow{r} \arrow{d} &
K \arrow{r} \arrow{d} &
K' \arrow{r} \arrow[equals]{d} \arrow[dotted]{ld} &
0 \\
& \phi(L) \arrow{r} &
\phi(K) \arrow{r}{h} &
K' \arrow{r} &
0
\end{tikzcd}
\]
where the top row is exact. Since $\phi$ is a pre-coradical, 
$\phi(L) = 0$. Thus, the composition $L \to K \to \phi(K)$ is also
$0$. However, $K'$ is the cokernel, and therefore, there exists a
map $K' \to \phi(K)$ (represented by the dotted arrow in the above 
diagram). It follows that $h$ is an isomorphism as desired.
\end{proof}

\begin{defn}
Let $\Cat{A}$ be an abelian category. A \emph{torsion theory} for
$\Cat{A}$ is a pair $(\Cat{T}, \Cat{F})$ of full subcategories 
called the \emph{torsion subcategory} and \emph{torsion-free 
subcategory} respectively, where the objects of $\Cat{T}$ are
objects $X$ such that $\hom_{\Cat{A}}(X, Y) = 0$ for every $Y
\in \Cat{F}$ and objects of $\Cat{F}$ are objects $Y$ such that
$\hom_{\Cat{A}}(X, Y) = 0$ for every object $X \in \Cat{T}$.
\end{defn}

Certainly $0 \in \Cat{T} \cap \Cat{F}$. Therefore, neither 
subcategories are empty. We also have the following characterization
for the torsion and torsionfree subcategories.

\begin{prop}
Let $\Cat{A}$ be a well-powered abelian category, and $\Cat{T}$ 
and $\Cat{F}$ are two nonempty full subcategories. Then, $\Cat{T}$ 
is a torsion subcategory of $\Cat{A}$ if and only if $\Cat{T}$ is 
closed under extensions, direct sums and quotients. 

Dually, $\Cat{F}$ is a torsionfree subcategory of $\Cat{A}$ if and 
only if $\Cat{F}$ is closed under extensions, direct products, and 
submodules.
\end{prop}

\begin{proof}
It suffices to verify the statement for torsion subcategories.

Suppose $\Cat{T}$ is a torsion subcategory with $\Cat{F}'$ its 
corresponding torsionfree subcategory. 

\pfitem{Closed under quotients:} suppose $X \in \Cat{T}$. For any 
surjection $X \to Y \to 0$, we have 
\[
0 \to \hom_{\Cat{A}}(Y, F) \to \hom_{\Cat{A}}(X, F)
\]
for any $F \in \Cat{F}'$. However, $\hom_{\Cat{A}}(X, F) = 0$.
Therefore, $\hom_{\Cat{A}}(Y, F) = 0$ for all $F$, and $Y \in
\Cat{T}$.

\emph{Closed under sums:} suppose $\{X_i\}_{i \in I}$ is a 
collection of objects of $\Cat{T}$. We have
\[
\hom_{\Cat{A}}( \oplus_{i \in I} X_i, F) = \prod_{i \in I}
\hom_{\Cat{A}}( X_i, F) = 0
\]
for all $F \in \Cat{F}'$. It follows that $\oplus_{i \in I} X_i$
is an object of $\Cat{T}$.

\pfitem{Closed under extensions:} Suppose 
\[
0 \to X' \to X \to X'' \to 0
\]
is an exact sequence in $\Cat{A}$ with $X', X'' \in \Cat{T}$.
Then for any $F \in \Cat{F}$, 
\[
0 \to \hom_{\Cat{A}}(X'', F) \to \hom_{\Cat{A}}(X, F) \to
\hom_{\Cat{A}}(X', F).
\]
Since $\hom_{\Cat{A}}(X'', F) = \hom_{\Cat{A}}(X', F) = 0$,
it follows that $\hom_{\Cat{A}}(X, F) = 0$ for all $F$, and
$X \in \Cat{T}$.

Conversely, suppose $\Cat{T}$ is closed under extensions, direct 
sums and quotients. Let $\Cat{F}'$ be the full subcategory of $Y$ 
such that $\hom_{\Cat{A}}(X, Y) = 0$ for all $X \in \Cat{T}$, and 
$\Cat{T}'$ be the full subcategory of $X'$ such that 
$\hom_{\Cat{A}}(X', Y) = 0$ for all $Y \in \Cat{F}$. We proceed
by showing that $\Cat{T'} = \Cat{T}.$

We show that $\Cat{T} = \Cat{T'}$. Certainly $\Cat{T}$ is a full
subcategory of $\Cat{T'}$. Fix $X \in \Cat{T'}$. Then there exists
a maximal $\Cat{T}$ subobject of $X$. Indeed, let $S = \{X_i| i 
\in I\}$ be the set of subobjects of $X$ in $\Cat{T'}$. Setting 
$X' = \displaystyle \oplus_{i \in I} X_i$, we see that $X''$ is 
an object of $\Cat{T}$, whose image in $X$ is a quotient of 
$X''$, and therefore, is also in $\Cat{T}$. 

Let $\tilde{X}$ be this maximal subobject. We proceed by showing 
that $X/\tilde{X}$ is an object of $\Cat{F}'$, and hence is 0.
Suppose not. Then there exists some $X' \in \Cat{T}$ with a 
nonzero map $f: X' \to X/\tilde{X}$. Since $f(X') \in \Cat{T}$,
replacing $X'$ by its image in $X/\tilde{X}$, we may assume
without loss of generality that $f$ is injective.

Pullback $X \to X/\tilde{X}$ by $f$, and we have:
\[
\begin{tikzcd}
0 \arrow{r} &
\tilde{X} \arrow{r}\arrow[equals]{d} &
P \arrow{r}{p} \arrow{d}{i} &
X' \arrow{r} \arrow[hook]{d} &
0 \\
0 \arrow{r} &
\tilde{X} \arrow{r} &
X \arrow{r} &
X/\tilde{X} \arrow{r} &
0
\end{tikzcd}
\]
As $i$ is a pullback of an injection, $i$ is itself injective.
Similarly, as $X \to X/\tilde{X}$ is surjective, so is $p$.
Furthermore, $\ker p = \tilde{X}$. Since $\tilde{X}$ and $X'$
are both in $\Cat{T}$, it follows that $P$ must, too. However,
$X'$ is nontrivial, contradicting the maximality of $\tilde{X}$.
Thus, $X/\tilde{X} \in \Cat{F}$, and $X \in \Cat{T}$.
\end{proof}

\begin{prop}\label{prop_tt_suff_cond}
Let $(\Cat{T}, \Cat{F})$ be a pair of non-empty full subcategories
of a well-powered abelian category $\Cat{A}$. Then $(\Cat{T}, 
\Cat{F})$ is a torsion theory if and only if the following 
conditions hold:
\begin{enumerate}
\item the only common objects of $\Cat{T}$ and $\Cat{F}$ is 0.

\item for every $X \in \Cat{A}$, there exists $Y$ such that $Y \in 
\Cat{T}$ and $X/Y \in \Cat{F}$.
\end{enumerate}
\end{prop}
\begin{proof}
\noindent $\Rightarrow$: Fix $X \in \Cat{T}$ and $X \in \Cat{F}$, 
then $\hom_{\Cat{A}}(X,X) = 0$, so the identity is the 0 map, and
$X = 0$.

Now fix $X \in \Cat{A}$, and let $\{X_i\;:\;i \in I\}$ be the set
of $\Cat{T}$ subobjects of $X$. The image in $X$ of the direct 
sum of the subobjects is an object of $\Cat{T}$, the quotient by 
which is an object in $\Cat{F}$.
\vskip 10pt
\noindent $\Leftarrow$, suppose $\Cat{T}, \Cat{F}$ satisfy the 
condition of the proposition. Suppose we have $X \in \Cat{A}$ such 
that for all $Y \in \Cat{Y}$, $\hom_{\Cat{A}}(X, Y) = 0$.
Since $X/FX \in \Cat{F}$, $X \to X/FX$ is the zero map. Hence, $X 
= FX \in \Cat{T}$. Similarly, if $Y \in \Cat{F}$, then the 
inclusion $FY \to Y$ is the zero map, and hence $Y/FY = Y \in 
\Cat{F}.$
\end{proof}

\begin{defn}
Let $\Cat{C}$ be a subcategory of an abelian category; fix $X \in
\Cat{A}$. We say $Y$ is the \emph{largest $\Cat{C}$-subobject of 
$X$} if $Y \in \Cat{C}$ is a subobject of $X$ that factors all 
monomorphisms from objects in $\Cat{C}$ to $X$. That is, for 
every diagram
\[
\begin{tikzcd}
{} &
Y \arrow[hook]{d}{i} \arrow[dotted]{ld}{g} \\
Z \arrow[hook]{r}{j} & X
\end{tikzcd}
\]
where $Z \in \Cat{C}$, there exists a map $Z \stackrel{f}{\to} Y$ 
such that $gj = i$.
\end{defn}

\begin{por}
For any well-powered abelian category $\Cat{A}$ and any full 
subcategory $\Cat{C}$ closed under sums and quotients, any $X \in 
\Cat{A}$ has a largest $\Cat{C}$-subobject.
\end{por}

\begin{prop}\label{prop_tt_to_corad}
Let $(\Cat{T}, \Cat{F})$ be a torsion theory on $\Cat{A}$. Then
the functor $\phi: \Cat{A} \to \Cat{T}$ that sends $X \in \Cat{A}$
to its quotient by its largest $\Cat{T}$-subobject is an 
idempotent pre-coradical.

Dually, the functor that sends $X$ to its largest 
$\Cat{T}$-subobject is an idempotent pre-radical.
\end{prop}
\begin{proof}
In this case, it is easier to prove the statement for idempotent 
pre-radicals.

Let $f: X \to Y$ be any morphism and $\phi(X) 
\stackrel{i_{\phi(X)}}{\to} X$ and 
$\phi(Y)\stackrel{i_{\phi(Y)}}{\to} Y$ be the $\Cat{T}$-subobjects 
of $X$ and $Y$ respectively. Consider the image of $\phi(X)$ under 
$f i_{\phi(X)}$, which is a $\Cat{T}$ subobject of $Y$. Hence, 
there exists a map $\im i_{\phi(X)}f \to \phi(Y)$, and define the 
map $\phi f$ to be the composition of $\phi(X) \to \im 
i_{\phi(X)}f \to \phi(Y)$.

It is clear from the construction that $\phi$ is a subobject functor.
Since $\phi(X) \in \Cat{T}$, it is clear that the largest suboboject
of $\phi(X)$ is itself: hence $\phi^2(X)) = \phi(X)$. By the 
maximality of $\phi(X)$, we see that $X/\phi(X) \in \Cat{F}$, and
certainly, 
\[
\phi(X/\phi(X)) = 0.
\]

The dual statement follows from Prop. \ref{prop_rad_eq_corad}.
\end{proof}

\begin{thm}\label{thm_corad_implies_tt}
Let $\Cat{A}$ be a well-powered Abelian category. Then there exists
a one-to-one correspondence between idempotent pre-coradicals and
torsion theories on $\Cat{A}$ whose full subcategories are defined 
by
\begin{align*}
\Cat{T} &= \{ X \in \Cat{A} : \phi(X) = 0 \} \\
\Cat{F} &= \{ X \in \Cat{A} : \phi(X) = X \}.
\end{align*}
\end{thm}
\begin{proof}
Obtaining an idempotent pre-coradical from a torsion theory is
established by Prop. \ref{prop_tt_to_corad}. Therefore, it suffices
to show that $(\Cat{T}, \Cat{F})$ as given in the statement of the
theorem defines a torsion theory on $\Cat{A}$.

To do this, we appeal to Prop. \ref{prop_tt_suff_cond}. In 
particular, we show that the only object common to both $\Cat{T}$
and $\Cat{F}$ is $0$, and that for every $X \in \Cat{A}$, there 
exists $X \in \Cat{T}$ such that $X/Y \in \Cat{F}$.

This is clear, since if $X \in \Cat{T}$ and $X \in \Cat{F}$, then
$X = \phi(X) = 0$. Furthermore, given $X \in \Cat{A}$, since 
$\phi$ is idempotent, $\phi(X) \in \Cat{F}$. Since $\phi$ is a 
pre-coradical, $\ker(X \to \phi(X)) \in \Cat{T}$.
\end{proof}

As we have mentioned in Remark \ref{rmk_rad_corad_dual}, there is
a result corresponding to Theorem \ref{thm_corad_implies_tt} for 
radicals: every idempotent pre-radical $\kappa$ is in one-to-one 
correspondence with torsion theories on $\Cat{A}$ defined by
\begin{align*}
\Cat{T} &= \{ X \in \Cat{A} : \kappa(X) = X \} \\
\Cat{F} &= \{ X \in \Cat{A} : \kappa(X) = 0 \}.
\end{align*}
In fact, we have the following.

\begin{cor}
Let $\phi$ be an idempotent pre-coradical, and let $\kappa$ be the 
idempotent pre-radical associated to $\phi$ (see Prop. 
\ref{prop_rad_eq_corad}). Then the torsion theory defined by $\phi$
in Theorem \ref{thm_corad_implies_tt} is the same as the one for
$\kappa$ as defined above.

In this case, $\phi$ is left adjoint to the inclusion $\Cat{F} \to 
\Cat{A}$ and $\kappa$ is right adjoint to the inclusion $\Cat{T} 
\to \Cat{A}$.
\end{cor}
\begin{proof}
The only thing left to verify is that $\phi$ defines a left 
adjoint to the inclusion of $\Cat{F}$ into $\Cat{A}$ and $\kappa$ 
defines a right adjoint to the inclusion of $\Cat{T}$ into 
$\Cat{A}$. We verify the statement only for $\phi$ and leave the 
latter to the reader.  

For $\phi$, fix $X \in \Cat{A}$, and let $Y \in \Cat{F}$. 
Consider the short exact sequence
\[
0 \to \kappa(X) \to X \to \phi(X) \to 0.
\]
Applying $\hom_{\Cat{A}}(-, Y)$, we have the exact sequence
\[
0 \to \hom_{\Cat{A}}(\phi(X), Y) \to \hom_{\Cat{A}}(X, Y) \to
\hom_{\Cat{A}}(\kappa(X), Y)
\]
Since $\kappa(X) \in \Cat{T}$ (Theorem 
\ref{thm_corad_implies_tt}) and $\phi(X) \in \Cat{F}$, 
$\hom_{\Cat{A}}(\kappa(X), Y) = 0$. Thus,
\[
\hom_{\Cat{A}}(\phi(X), Y) = \hom_{\Cat{F}}(\phi(X), Y) \simeq 
   \hom_{\Cat{A}}(X, Y)
\]
as desired.
\end{proof}

\begin{defn}
Let $(\Cat{T}, \Cat{F})$ be a torsion theory on $\Cat{A}$. We say 
that $(\Cat{T}, \Cat{F})$ is \DEF{hereditary} if $\Cat{T}$ is 
closed with respect to subobjects. That is, if $A \into B$ is an 
monomorphism in $\Cat{A}$ such that $B \in \Cat{T}$, then $A \in 
\Cat{T}$.
\end{defn}

\begin{thm}\label{thm_corad_equiv_htt}
Let $\Cat{A}$ be a well-powered Abelian category. Then there 
exists a one-to-one correspondence between coradicals of $\Cat{A}$ 
and \emph{hereditary} torsion theories on $\Cat{A}$.
\end{thm}
\begin{proof}

\pfitem{From coradicals to hereditary torsion theories} : Let 
$\phi$ be a coradical.

In view of Theorem \ref{thm_corad_implies_tt}, to associate a 
coradical with a hereditary torsion theory, we need only to show
that $\Cat{T}$ defined by $\{X : \phi(X) = 0\}$ is closed under
subobjects.
\end{proof}

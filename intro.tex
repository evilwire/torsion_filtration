\chapter{Introduction} 

The goal of this thesis is to show that the abelian categories
$\HI$ of homotopy invariant Nisnevich sheaves with transfers and 
$\CycMod$ of Rost's cycle modules each admits
three filtrations. Here, a (weak) filtration of a category roughly 
means a tower of subcategories together with reflection functors 
from the category to each of its subcategories. The filtrations 
are induced by the slice filtration on the tensor triangulated 
category $\DMeff$ of Voevodsky's derived category of motive. 
One of the key ingredients in constructing the three filtrations 
is a pair of adjoint functors from $\HI$ to itself, coming from the 
triangulated tensor structure of $\DMeff$. The other key 
ingredient is torsion theory.
 
We first revisit the basic definition and results of
classical torsion theory for well-powered abelian categories, as 
developed in \cite{BJV} or \cite{DTor}. However, instead of focusing
on the relationship between torsion theories and radicals, we
introduce the theory from the perspective of coradicals, which are 
radicals in the opposite category.  

In Chapter 3 and 4, we summarize the foundational theory in 
motivic cohomology necessary to understand the tensor triangulated 
structure on $\DMeff$, and the ``partial'' internal hom bifunctor. 
These are taken from early lectures in \cite{MVW}. The main 
results that we highlight in these two chapters are the 
Cancellation Theorem of Voevodsky and that there exists an object 
$\Z(1)$ of $\DMeff$ which gives rise to a pair of adjoint endofunctor
on $\DMeff$. These results provide the necessary scaffold to 
introduce the slice filtration on $\DMeff$. The slice filtration on
$\DMeff$ is based on the work of \cite{HuKa}, which in turn is
inspired by an analogous structure on the stable homotopy category
of motives (see \cite{VOP}). We extend the work of \cite{HuKa} by
extending the slice filtration on $\DMeff$ to the category $\DM$, 
which is obtained from $\DMeff$ by inverting the Tate motive.

In Chapter 6, we note that $\DMeff$ is equipped with a 
$t$-structure in the sense of \cite{BBD}, and $\HI$ is categorically
equivalent to the heart. An obvious question to ask is whether the
filtration structure on $\DMeff$ induces a similar structure on
$\HI$. In fact, the slice filtration on $\DMeff$ does induce two
filtrations on $\HI$. In addition, the reflection functors from
one of the filtrations define a sequence of coradicals. Applying
the results of Chapter 2, we obtain a third filtration, which has
the additional property that the filtration on $\HI$ induces a 
functorial filtration for each object of $\HI$. We then extend the 
filtrations on $\HI$ to the abelian category $\HI_*$ of homotopy 
modules (Chapter 6). Using the fact that $\HI_*$ is categorically 
equivalent to $\CycMod$, we conclude that these filtrations exist 
on $\CycMod$. 

In the last chapter, we summarize the results of the previous 
chapters by axiomatizing the conditions on a triangulated with a 
$t$-structure such that the heart is equipped with a sequence of 
coradicals whose associated torsion theories form two filtrations 
on the category. We have decided that the essential ingredient is 
for a tensor triangulated category with a $t$-structure to be 
equipped with a Tate object --- an object $S$ in the heart such 
that the functor given by tensoring with $S$ admits a right 
adjoint --- such that the Cancellation Theorem holds for $S$. We 
call such a triangulated category \emph{torsion monoidal}, and we 
show that the heart of any torsion monoidal category is equipped
with three filtrations, two of which is induced by a sequence of
coradicals.

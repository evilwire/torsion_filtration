\newpage
\bibliographystyle{plain}
\begin{thebibliography}{10}

\bibitem[SGA4]{SGA4}
M. Artin, A. Grothendieck, J. L. Verdier.
{\em S\'eminaire de G\'eom\'etrie Alg\'ebrique du Bois Marie,
Th\'eorie des topos et cohomologie \'etale des sch\'emas.}
Lecture notes in math. 270, Springer-Verlag, 1972.

\bibitem[BBD]{BBD}
J. Bernstein, A. Beilinson and P. Deligne,
Faisceaux pervers, {\em Ast\'erisque} 100 (1982) 1 - 172.

\bibitem[Bo]{Bo}
F. Borceaux,
{\em Handbook of Categorical Algebra, Vol. 3.}
Cambridge Univ. Press., 2008.

\bibitem[BJV]{BJV}
J. L. Bueso, P. Jara and A. Verschoren,
{\em Compatibility, Stability, and Sheaves.}
Monographs and textbooks in pure and applied math. vol. 185,
1995.

\bibitem[D\'eg06]{DegTransfert}
F. D\'eglise,
\emph{Transferts sur les groupes de Chow \`a coefficients.}
Mathematische Zeitschrift, 252. (2006), 315-343.

\bibitem[D\'eg08]{DegGenMot}
F. D\`eglise.
\emph{Motifs G\'en\'eriques.}
Rendiconti Sem. Mat. Univ. Padua, 119 (2008), 173 - 244.

\bibitem[D\'eg10]{DegModHom}
F. D\'eglise, 
\emph{Modules homotopiques.}
Documenta Math.,
16 (2011), 411 - 455.

\bibitem[Dic66]{DTor}
S. Dickson,
\emph{A torsion theory for abelian categories.}
Trans. of the Am. Math. Soc., 121(1) (1966), 223-223.

\bibitem[EGA3]{EGA3}
A. Grothendieck, J. Dieudonn\'e.
{\em \'Elements de g\'eom\'etrie alg\'ebrique: III. \'Etude 
cohomologique des faisceaux coh\'erents, Part I.} 
Pub. Math. de l'IH\'ES 11, 1961.

\bibitem[EGA4]{EGA4}
A. Grothendieck, J. Dieudonn\'e.
\emph{\'Elements de g\'eom\'etrie alg\'ebrique: IV. \'Etude locale 
des sch\'emas et des morphismes de sch\'emas, Part I.}
Pub. Math. de l'IH\'ES 11, 1964.

\bibitem[Ful84]{Ful84}
W. Fulton.
\emph{Intersection Theory},
Springer, 1984.

\bibitem[Hart77]{Hart}
R. Hartshorne.
\emph{Algebraic Geometry.}
Grad. Texts in Math.,
Springer, 1977.

\bibitem[HK06]{HuKa}
F. Huber, B. Kahn.
\emph{The slice filtration and mixed Tate motives.}
Compositio Math., 142 (2006), 907 - 936.

\bibitem[KaSu]{KaSu}
B. Kahn, R. Sujartha,
\emph{Birational motives, I: pure birational motives.}
In preparation.
http://arxiv.org/pdf/0902.4902.pdf.

\bibitem[MK]{MK}
G. M. Kelly.
\emph{Basic Concepts of Enriched Category Theory.}
London Math. Soc. Lecture Notes No. 64, 1982.

\bibitem[Lang02]{LangAlg}
S. Lang.
\emph{Algebra}, Grad. Texts in Math.,
Springer, 2002.

\bibitem[ML69]{MLCatThry}
S. Mac Lane.
Categories for the working mathematician, 
{\em Grad. Texts in Math.,} Springer, 1968.

\bibitem[Mat80]{MatsCA}
H. Matsumura.
\emph{Commutative ring theory.}
Cambridge studies in adv. math,
Cambridge University Press, 1980.

\bibitem[Milne]{Milne}
J. S. Milne.
\emph{\'Etale cohomology.} 
Princeton Univ. Press, 1980.

\bibitem[MVW]{MVW}
C. Mazza, V. Voevodsky, C. Weibel.
\emph{Lecture Notes on Motivic Cohomology}
Clay Mathematics Monograph, vol 2, 2006.

\bibitem[Mil70]{MilK}
J. Milnor.
\emph{Algebraic $K$-theory and quadratic forms.}
Invent. Math. 9 (1970), 318 - 344.

\bibitem[Ro96]{Rost96}
M. Rost.
\emph{Chow groups with coefficients.}
Doc. Math., 1 (1996), 319-394.

\bibitem[Swan]{Swan}
R. G. Swan,
\emph{Algebraic K-Theory.}
Lecture Notes in Math.,
Springer, 1968.

\bibitem[Tamme]{Tamme}
G. Tamme,
\emph{Introduction to \'Etale Cohomology},
Universitext,
Springer, 1991.

\bibitem[Verd96]{Verd96}
J.-L. Verdier, \emph{Des Cat\'egories d\'eriv\'ees des cat\'egories
ab\'eliennes},
Ast\'erisque, no. 239 (1996) 1-253.

\bibitem[Voe02]{V02}
V. Voevodsky,
\emph{Cancellation Theorem},
Doc. Math. (2010) 671-685.

\bibitem[Voe02b]{VOP}
V. Voevodsky,
{\em Open problems in the motivic stable homotopy theory I.}
Motives, Polylogarithms and Hodge Theory, Part I,
International Press (2002), 3-35.

\bibitem[Voe00]{TriCa}
V. Voevodsky,
\emph{Triangulated categories of motives over a field.}
Cycles, Transfers, and Motivic Homology Theories,
Princeton University Press, 2000.

\bibitem[Wei94]{WH}
C. Weibel,
\emph{An introduction to homological algebra.}
Cambridge Univ. Press, 1994.
\end{thebibliography}

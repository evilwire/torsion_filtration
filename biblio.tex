\newpage
\bibliographystyle{plain}
\begin{thebibliography}{10}

\bibitem[BBD]{BBD}
J. Bernstein, A. Beilinson and P. Deligne,
Faisceaux pervers, {\em Asterisque} 100 (1982).

\bibitem[Bo]{Bo}
F. Borceaux,
{\em Handbook of Categorical Algebra, Vol. 3.}
Cambridge Univ. Press., (2008).

\bibitem[BJV]{BJV}
J. L. Bueso, P. Jara and A. Verschoren,
{\em Compatibility, Stability, and Sheaves}, 
Monographs and textbooks in pure and applied math.,
1995.

\bibitem[D\'eg06]{DegTransfert}
F. D\'eglise,
Transferts sur les groupes de Chow \`a coefficients.
{\em Mathematische Zeitschrift.} 252. (2006), 315-343

\bibitem[D\'eg08]{DegGenMot}
F. D\'eglise.
Motifs Generiques.
\emph{Rendiconti Sem. Mat. Univ. Padua.}, 119 (2008), 173 - 244

\bibitem[D\'eg10]{DegModHom}
F. D\'eglise, 
Modules homotopiques.
{\em Documenta Math.}
16 (2011), 411 - 455

\bibitem[Dic66]{DTor}
S. Dickson,
A torsion theory for abelian categories.
{\em Trans. of the Am. Math. Soc.} 121(1) (1966), 223-223

\bibitem[EGA3]{EGA3}
A. Grothendieck,
{\em Elements de Geometrie Algebraique. 3}

\bibitem[EGA4]{EGA4}
A. Grothendieck,
{\em Elements de Geometrie Algebraique. 4}

\bibitem[Hart77]{Hart}
R. Hartshorne.
Algebraic Geometry, {\em Grad. Text in Math.}
Springer, (1977), 95 - 108

\bibitem[HK06]{HuKa}
F. Huber, B. Kahn, The slice filtration and mixed Tate motives,
{\em Compositio Math.} 142 (2006), 907 - 936.

\bibitem[Lang02]{LangAlg}
S. Lang.
Algebra, {\em Grad. Text in Math.}
Springer, (2002), 480 - 486

\bibitem[ML69]{MLCatThry}
S. Mac Lane.
Categories for the working mathematician, {\em Grad. Text in Math.}
Springer, (1968)

\bibitem[Mat80]{MatsCA}
H. Matsumura.
Commutative ring theory, {\em Cambridge studies in adv. math.}
Cambridge University Press, (1980), 71-86

\bibitem[Milne]{Milne}
J. S. Milne.
\'Etale cohomology, {\em Princeton Math.}
Princeton Univ. Press, \textbf{(1980)}

\bibitem[MVW]{MVW}
C. Mazza, V. Voevodsky, C. Weibel.
{\em Lecture Notes on Motivic Cohomology}
Clay Mathematics Monograph, vol 2.

\bibitem[Mil70]{MilK}
J. Milnor.
Algebraic $K$-theory and quadratic forms.
{\em Invent. Math.} 9 (1970), 318 - 344

\bibitem[Ro96]{Rost96}
M. Rost.
Chow groups with coefficients,
{\em Doc. Math.}, 1 (1996), 319-394

\bibitem[Swan]{Swan}
R. G. Swan,
Algebraic K-Theory
{\em Lecture Notes in Math},
Springer, 1968.

\bibitem[Wei94]{WH}
C. Weibel,
{\em An introduction to homological algebra},
Cambridge Univ. Press, 1994.
\end{thebibliography}
